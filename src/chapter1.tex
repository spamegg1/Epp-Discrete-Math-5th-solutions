\documentclass[14pt]{extarticle} 

\usepackage{amsmath,mathtools,amsfonts,amsthm,amssymb,hyperref}
\usepackage{wasysym,geometry,latexsym,parskip,bookmark}
\usepackage{mathtools,float}
%\usepackage{bussproofs}

\newtheorem{defn}{Definition}
\newtheorem{thm}{Theorem}
\newtheorem{claim}{Claim}
\newtheorem{lemma}{Lemma}
\newcommand{\dps}{\displaystyle}
\newcommand{\fillBlanks}{\underline{\hspace{1cm}}\,\,}

\hypersetup{colorlinks, allcolors=blue, linktoc=all}
\geometry{a4paper} 
\geometry{margin=0.5in}

\title{Chapter 1 Solutions, Susanna Epp Discrete Math 5th Edition}

\author{https://github.com/spamegg1}

\begin{document}
\maketitle
\tableofcontents

\section{Exercise Set 1.1}

{\bf In each of 1–6, fill in the blanks using a variable or variables to rewrite the given statement.}

\subsection{Problem 1}
Is there a real number whose square is $-1$?

\subsubsection{(a)}
Is there a real number $x$ such that \fillBlanks?

\begin{proof}
Is there a real number $x$ such that \underline{$x^2 = -1$}?
\end{proof}

\subsubsection{(b)}
Does there exist \fillBlanks such that $x^2 = -1$?

\begin{proof}
Does there exist \underline{a real number $x$} such that $x^2 = -1$?
\end{proof}

\subsection{Problem 2}
Is there an integer that has a remainder of 2 when it is divided by 5 and a remainder of 3 when it is divided by 6?

{\it Note: There are integers with this property. Can you
think of one?}

\subsubsection{(a)}
Is there an integer $n$ such that $n$ has \fillBlanks?

\begin{proof}
Is there an integer $n$ such that $n$ has \underline{a remainder of 2 when it is divided by 5} \underline{and a remainder of 3 when it is divided by 6}?
\end{proof}

\subsubsection{(b)}
Does there exist \fillBlanks such that if $n$ is divided by 5 the remainder is 2 and if \fillBlanks?

\begin{proof}
Does there exist \underline{an integer $n$} such that if $n$ is divided by 5 the remainder is 2 and if \underline{$n$ is divided by 6 the remainder is 3}?
\end{proof}

\subsection{Problem 3}

\subsubsection{(a)}

\begin{proof}
\end{proof}

\subsubsection{(b)}

\begin{proof}
\end{proof}

\subsection{Problem 3}

\subsubsection{(a)}

\begin{proof}
\end{proof}

\subsubsection{(b)}

\begin{proof}
\end{proof}

\end{document}
