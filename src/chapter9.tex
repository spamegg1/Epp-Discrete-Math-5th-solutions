\documentclass[14pt]{extarticle}

\usepackage[table]{xcolor} % colored lines for tables
%\usepackage[normalem]{ulem} % strike through text
\usepackage{amsmath,mathtools,amsfonts,amsthm,amssymb,hyperref}
\usepackage{parskip,geometry,latexsym,bookmark,mathtools,float,cancel}
\usepackage{tcolorbox,bm,minted}

\newtheorem{defn}{Definition}
\newtheorem{thm}{Theorem}
\newtheorem{claim}{Claim}
\newtheorem{lemma}{Lemma}

\newcommand{\dps}{\displaystyle}
\newcommand{\es}{\varnothing}
\newcommand{\fbl}{\underline{\hspace{1cm}}\,\,}
\newcommand{\R}{\mathbb{R}}
\newcommand{\Q}{\mathbb{Q}}
\newcommand{\Z}{\mathbb{Z}}
\newcommand{\from}{\leftarrow}
\newcommand{\true}{{\bf t}}
\newcommand{\false}{{\bf c}}
\newcommand{\bic}{\leftrightarrow}
\newcommand{\da}{\downarrow}
\newcommand{\cyda}{{\cy \(\downarrow\)}}
\newcommand{\fa}{\forall}
\newcommand{\te}{\exists}
\newcommand{\cy}{\color{cyan}}

\newcommand{\colsq}[1]{{\color{#1} $\blacksquare$}}

\newcommand{\base}[1]{{\cy #1}} % for log bases
\newcommand{\floor}[1]{{\left\lfloor#1\right\rfloor}}
\newcommand{\ceil}[1]{{\left\lceil#1\right\rceil}}
\newcommand\Ccancel[2][black]{\renewcommand\CancelColor{\color{#1}}\cancel{#2}}
\newcommand\Cbcancel[2][black]{\renewcommand\CancelColor{\color{#1}}\bcancel{#2}}
\newcommand\Cyancel[2][cyan]{\renewcommand\CancelColor{\color{#1}}\cancel{#2}}

\setlength{\extrarowheight}{10pt}

\hypersetup{colorlinks,allcolors=blue,linktoc=all}
\geometry{a4paper}
\geometry{margin=0.42in}

\title{Solutions to Chapter 9, Susanna Epp Discrete Math 5th Edition}

\author{https://github.com/spamegg1}

\begin{document}
\maketitle
\tableofcontents

\section{Exercise Set 9.1}

\subsection{Exercise 1}
Toss two coins 30 times and make a table showing the relative frequencies of 0, 1, and 2 heads. How do your
values compare with those shown in Table 9.1.1?

\begin{proof}
     \arrayrulecolor{cyan}
     \begin{tabular}{|c|c|c|}
          \hline
          {\bf\cy Event}   & {\bf\cy Freq.} & {\bf\cy Rel. freq.} \\
          \hline
          2 heads obtained & 7              & 23.33\%             \\
          \hline
          1 head obtained  & 16             & 53.33\%             \\
          \hline
          0 heads obtained & 7              & 23.33\%             \\
          \hline
     \end{tabular}
     \arrayrulecolor{black} % change it back!
\end{proof}

\subsection{Exercise 2}
In the example of tossing two quarters, what is the probability that at least one head is obtained? that coin
$A$ is a head? that coins $A$ and $B$ are either both
heads or both tails?

\begin{proof}
     3/4, 1/2, 1/2
\end{proof}

{\bf \cy In $3-6$ use the sample space given in example 9.1.1. Write each event as a set and compute its
probability.}

\subsection{Exercise 3}
The event that the chosen card is red and is not a face card.

\begin{proof}
     \(\{1\diamondsuit, 2\diamondsuit, 3\diamondsuit, 4\diamondsuit, 5\diamondsuit, 6\diamondsuit, 7\diamondsuit,
     8\diamondsuit, 9\diamondsuit, 10\diamondsuit, 1\heartsuit, 2\heartsuit, 3\heartsuit, 4\heartsuit, 5\heartsuit\),

     \(6\heartsuit, 7\heartsuit, 8\heartsuit, 9\heartsuit, 10\heartsuit\}\), probability \(= 20/52 \approx 38.5\%\)
\end{proof}

\subsection{Exercise 4}
The event that the chosen card is black and has an even number on it.

\begin{proof}
     \(\{2\spadesuit, 4\spadesuit, 6\spadesuit, 8\spadesuit, 10\spadesuit, 2\clubsuit, 4\clubsuit, 6\clubsuit,
     8\clubsuit, 10\clubsuit\}\), probability \(= 10/52 \approx 19.2\%\)
\end{proof}

\subsection{Exercise 5}
The event that the denomination of the chosen card is at least 10 (counting aces high).

\begin{proof}
     \(\{10\spadesuit, J\spadesuit, Q\spadesuit, K\spadesuit, A\spadesuit, 10\diamondsuit, J\diamondsuit, Q\diamondsuit,
     K\diamondsuit, A\diamondsuit, 10\heartsuit, J\heartsuit, Q\heartsuit, K\heartsuit, A\heartsuit\),

     \(10\clubsuit, J\clubsuit, Q\clubsuit, K\clubsuit, A\clubsuit\}\), probability \(= 20/52=5/13 \approx 38.5\%\)
\end{proof}

\subsection{Exercise 6}
The event that the denomination of the chosen card is at most 4 (counting aces high).

\begin{proof}
     \(\{2\clubsuit, 2\heartsuit, 2\spadesuit, 2\diamondsuit, 3\clubsuit, 3\heartsuit, 3\spadesuit, 3\diamondsuit,
     4\clubsuit, 4\heartsuit, 4\spadesuit, 4\diamondsuit\}\), probability \(= 12/52 \approx 23.0\%\)
\end{proof}

{\bf \cy In $7-10$, use the sample space given in example 9.1.2. Write each of the following events as a set and
compute its probability.}

\subsection{Exercise 7}
The event that the sum of the numbers showing face up is 8.

\begin{proof}
     \(\{26, 35, 44, 53, 62\}\), probability \(= 5/36 \approx 13.9\%\)
\end{proof}

\subsection{Exercise 8}
The event that the numbers showing face up are the same.

\begin{proof}
     \(\{11, 22, 33, 44, 55, 66\}\), probability \(= 6/36 \approx 16.6\%\)
\end{proof}

\subsection{Exercise 9}
The event that the sum of the numbers showing face up is at most 6.

\begin{proof}
     \(\{11, 12, 13, 14, 15, 21, 22, 23, 24, 31, 32, 33, 41, 42, 51\}\), prob. \(= 15/36 \approx 41.6\%\)
\end{proof}

\subsection{Exercise 10}
The event that the sum of the numbers showing face up is at least 9.

\begin{proof}
     \(\{36, 45, 46, 54, 55, 56, 63, 64, 65, 66\}\), probability \(= 10/36 \approx 27.7\%\)
\end{proof}

\subsection{Exercise 11}
Suppose that a coin is tossed three times and the side showing face up on each toss is noted. Suppose also that on
each toss heads and tails are equally likely. Let \(HHT\) indicate the outcome heads on the first two tosses and
tails on the third, \(THT\) the outcome tails on the first and third tosses and heads on the second, and so forth.

\subsubsection{(a)}
List the eight elements in the sample space whose outcomes are all the possible head-tail sequences obtained in the
three tosses.

\begin{proof}
     \(\{HHH, HHT, HTH, HTT, THH, THT, TTH, TTT\}\)
\end{proof}

\subsubsection{(b)}
Write each of the following events as a set and find its probability:

(i) The event that exactly one toss results in a head.

(ii) The event that at least two tosses result in a head.

(iii) The event that no head is obtained.

\begin{proof}
     (i) \(\{HTT, THT, TTH\}\), probability \(= 3/8 = 37.5\%\)

     (ii) \(\{HHT, HTH, THH, HHH\}\), probability \(= 4/8 = 50.0\%\)

     (iii) \(\{TTT\}\), probability \(= 1/8 = 12.5\%\)
\end{proof}

\subsection{Exercise 12}
Suppose that each child born is equally likely to be a boy or a girl. Consider a family with exactly three children.
Let \(BBG\) indicate that the first two children born are boys and the third child is a girl, let \(GBG\) indicate
that the first and third children born are girls and the second is a boy, and so forth.

\subsubsection{(a)}
List the eight elements in the sample space whose outcomes are all possible genders of the three children.

\begin{proof}
     \(\{BBB, BBG, BGB, BGG, GBB, GBG, GGB, GGG\}\)
\end{proof}

\subsubsection{(b)}
Write each of the events in the next column as a set and find its probability.

(i) The event that exactly one child is a girl.

(ii) The event that at least two children are girls.

(iii) The event that no child is a girl.

\begin{proof}
     (i) \(\{GBB, BGB, BBG\}\), probability \(= 3/8 = 37.5\%\)

     (ii) \(\{GGB, GBG, BGG, GGG\}\), probability \(= 4/8 = 50\%\)

     (iii) \(\{BBB\}\), probability \(= 1/8 = 12.5\%\)
\end{proof}

\subsection{Exercise 13}
Suppose that on a true/false exam you have no idea at all about the answers to three questions. You choose answers
randomly and therefore have a 50–50 chance of being correct on any one question. Let \(CCW\) indicate that you were
correct on the first two questions and wrong on the third, let \(WCW\) indicate that you were wrong on the first and
third questions and correct on the second, and so forth.

\subsubsection{(a)}
List the elements in the sample space whose outcomes are all possible sequences of correct and incorrect responses
on your part.

\begin{proof}
     \(\{CCC, CCW, CWC, CWW, WCC, WCW, WWC, WWW\}\)
\end{proof}

\subsubsection{(b)}
Write each of the following events as a set and find its probability:

(i) The event that exactly one answer is correct.

(ii) The event that at least two answers are correct.

(iii) The event that no answer is correct.

\begin{proof}
     (i) \(\{CWW, WCW, WWC\}\), probability \(= 3/8 = 37.5\%\)

     (ii) \(\{CCW, CWC, WCC, CCC\}\), probability \(= 4/8 = 50\%\)

     (iii) \(\{WWW\}\), probability \(= 1/8 = 12.5\%\)
\end{proof}

\subsection{Exercise 14}
Three people have been exposed to a certain illness. Once exposed, a person has a 50-50 chance of actually becoming ill.

\subsubsection{(a)}
What is the probability that exactly one of the people becomes ill?

\begin{proof}
     probability \(= 3/8 = 37.5\%\)
\end{proof}

\subsubsection{(b)}
What is the probability that at least two of the people become ill?

\begin{proof}
     probability \(= 4/8 = 50\%\)
\end{proof}

\subsubsection{(c)}
What is the probability that none of the three people becomes ill?

\begin{proof}
     probability \(= 1/8 = 12.5\%\)
\end{proof}

\subsection{Exercise 15}
When discussing counting and probability, we often consider situations that may appear frivolous or of little practical
value, such as tossing coins, choosing cards, or rolling dice. The reason is that these relatively simple examples
serve as models for a wide variety of more complex situations in the real world. In light of this remark,
comment on the relationship between your answer to exercise 11 and your answers to exercises $12-14$.

\begin{proof}
     The answers to exercises 11, 12, 13, 14 are the same, because all 4 situations are modeled exactly the same way.
     They each consist of 3 instances of the same event, each of which has 2 possible outcomes with 50\% probability (heads
     or tails, boy or girl, correct or incorrect, ill or not ill). They each have a sample space with 8 elements
     following the same pattern.
\end{proof}

\subsection{Exercise 16}
Two faces of a six-sided die are painted red, two are painted blue, and two are painted yellow. The die is rolled
three times, and the colors that appear face up on the first, second, and third rolls are recorded.


\subsubsection{(a)}
Let \(BBR\) denote the outcome where the color appearing face up on the first and second rolls is blue and the color
appearing face up on the third roll is red. Because there are as many faces of one color as of any other, the
outcomes of this experiment are equally likely. List all 27 possible outcomes.

\begin{proof}
     \(\{RRR, RRB, RRY, RBR, RBB, RBY, RYR, RYB, RYY,\)

     \(BRR, BRB, BRY, BBR, BBB, BBY, BYR, BYB, BYY,\)

     \(YRR, YRB, YRY, YBR, YBB, YBY, YYR, YYB, YYY\}\)
\end{proof}

\subsubsection{(b)}
Consider the event that all three rolls produce different colors. One outcome in this event is \(RBY\) and another
\(RYB\). List all outcomes in the event. What is the probability of the event?

\begin{proof}
     \(\{RBY, RYB, YBR, BRY, BYR, YRB\}\), probability \(= 6/27 = 2/9 \approx 22.2\%\)
\end{proof}

\subsubsection{(c)}
Consider the event that two of the colors that appear face up are the same. One outcome in this event is \(RRB\) and
another is \(RBR\). List all outcomes in the event. What is the probability of the event?

\begin{proof}
     \(\{RRB, RBR, BRR, RRY, RYR, YRR, BBR, BRB, RBB, BBY, BYB, YBB,\)

     \(YYR, YRY, RYY, YYB, YBY, BYY\}\), probability \(= 18/27 = 2/3 \approx 66.6\%\)
\end{proof}

\subsection{Exercise 17}
Consider the situation described in exercise 16.

\subsubsection{(a)}
Find the probability of the event that exactly one of the colors that appears face up is red.

\begin{proof}
     \(\{RBB, RBY, RYB, RYY, BRB, BRY, BBR, BYR, YRB, YRY, YBR, YYR\}\),

     probability \(= 12/27 = 4/9 \approx 44.4\%\)
\end{proof}

\subsubsection{(b)}
Find the probability of the event that at least one of the colors that appears face up is red.

\begin{proof}
     \(\{RRR, RRB, RRY, RBR, RBB, RBY, RYR, RYB, RYY, BRR, BRB, BRY,\)

     \(BBR, BYR, YRR, YRB, YRY, YBR, YYR\}\), probability \(= 20/27 \approx 74\%\)
\end{proof}

\subsection{Exercise 18}
An urn contains two blue balls (denoted \(B_1\) and \(B_2\)) and one white ball (denoted \(W\)). One ball is
drawn, its color is recorded, and it is replaced in the urn. Then another ball is drawn, and its color is recorded.

\subsubsection{(a)}
Let \(B_1 W\) denote the outcome that the first ball drawn is \(B_1\) and the second ball drawn is \(W\). Because the
first ball is replaced before the second ball is drawn, the outcomes of the experiment are equally likely. List all
nine possible outcomes of the experiment.

\begin{proof}
     \(\{B_1B_1, B_1B_2, B_1W, B_2B_1, B_2B_2, B_2W, WB_1, WB_2, WW\}\)
\end{proof}

\subsubsection{(b)}
Consider the event that the two balls that are drawn are both blue. List all outcomes in the event.
What is the probability of the event?

\begin{proof}
     \(\{B_1B_1, B_1B_2, B_2B_1, B_2B_2\}\), probability \(= 4/9 \approx 44.4\%\)
\end{proof}

\subsubsection{(c)}
Consider the event that the two balls that are drawn are of different colors. List all outcomes in the event.
What is the probability of the event?

\begin{proof}
     \(\{B_1W, B_2W, WB_1, WB_2\}\), probability \(= 4/9 \approx 44.4\%\)
\end{proof}

\subsection{Exercise 19}
An urn contains two blue balls (denoted \(B_1\) and \(B_2\)) and three white balls (denoted \(W_1, W_2\), and
\(W_3\)). One ball is drawn, its color is recorded, and it is replaced in the urn. Then another ball is drawn and its
color is recorded.

\subsubsection{(a)}
Let \(B_1 W_2\) denote the outcome that the first ball drawn is \(B_1\) and the second ball drawn is \(W_2\).
Because the first ball is replaced before the second ball is drawn, the outcomes of the experiment are equally
likely. List all 25 possible outcomes of the experiment.

\begin{proof}
     \(\{B_1B_1, B_1B_2, B_1W_1, B_1W_2, B_1W_3\),

     \(B_2B_1, B_2B_2, B_2W_1, B_2W_2, B_2W_3\),

     \(W_1B_1, W_1B_2, W_1W_1, W_1W_2, W_1W_3\),

     \(W_2B_1, W_2B_2, W_2W_1, W_2W_2, W_2W_3\),

     \(W_3B_1, W_3B_2, W_3W_1, W_3W_2, W_3W_3\}\)
\end{proof}

\subsubsection{(b)}
Consider the event that the first ball that is drawn is blue. List all outcomes in the event.
What is the probability of the event?

\begin{proof}
     \(\{B_1B_1, B_1B_2, B_1W_1, B_1W_2, B_1W_3, B_2B_1, B_2B_2, B_2W_1, B_2W_2, B_2W_3\}\),

     probability \(= 10/25 = 40\%\)
\end{proof}

\subsubsection{(c)}
Consider the event that only white balls are drawn. List all outcomes in the event.
What is the probability of the event?

\begin{proof}
     \(\{W_1W_1, W_1W_2, W_1W_3, W_2W_1, W_2W_2, W_2W_3, W_3W_1, W_3W_2, W_3W_3\}\)

     probability \(= 9/25 = 36\%\)
\end{proof}

\subsection{Exercise 20}
Refer to Example 9.1.3. Suppose you are appearing on a game show with a prize behind one of five closed doors: A, B, C,
D, and E. If you pick the correct door, you win the prize. You pick door A. The game show host then opens one of the
other doors and reveals that there is no prize behind it. Then the host gives you the option of staying with your
original choice of door A or switching to one of the other doors that is still closed.

\subsubsection{(a)}
If you stick with your original choice, what is the probability that you will win the prize?

\begin{proof}
     Either the prize is behind A or not.

     In Case 1, the prize is behind door A. In this case if I stick with door A, I would win.

     In the other case, the prize is behind one of the other 3 doors. In these cases if I stick with door A, I would lose.

     First case has 20\% probability (as given originally). If I stick with door A, I can only win in Case 1. Therefore the probability that I win the prize is 20\%.
\end{proof}

\subsubsection{(b)}
If you switch to another door, what is the probability that you will win the prize?

\begin{proof}
     Like above, if I switch to another door, I lose in Case 1, and I have a 4/5 chance of winning in the second case. But I have
     to pick one of the 3 doors, so the probability that I win the prize is \(\frac{4}{5} \cdot \frac{1}{3} = 4/15 \approx 26.7\%\).
\end{proof}

\subsection{Exercise 21}
\subsubsection{(a)}
How many positive two-digit integers are multiples of 3?

\begin{proof}
     Between 10 and 99 (inclusive), the multiples of 3 are: \(12, 15, 18, \ldots, 93, 96, 99\). Notice that
     \(12 = 3 \cdot {\cy 4}\) and \(99 = 3 \cdot {\cy 33}\). So there are as many positive two-digit integers that are
     multiples of 3 as there are integers from 4 to 33 inclusive. By Theorem 9.1.1 there are 33 - 4 + 1 = 30 such
     integers.
\end{proof}

\subsubsection{(b)}
What is the probability that a randomly chosen positive two-digit integer is a multiple of 3?

\begin{proof}
     There are \(99 - 10 + 1 = 90\) positive two-digit integers in all, and by part (a), 30 of these are multiples of 3. So
     the probability that a randomly chosen positive two-digit integer is a multiple of 3 is \(30/90=1/3 \approx 33.3\%\).
\end{proof}

\subsubsection{(c)}
What is the probability that a randomly chosen positive two-digit integer is a multiple of 4?

\begin{proof}
     Of the integers from 10 through 99 that are multiples of 4, the smallest is 12 \((= 4 \cdot 3)\) and the largest is 96
     \((= 4 \cdot 24)\). Thus there are \(24 - 3 + 1 = 22\) two-digit integers that are multiples of 4. Hence the
     probability that a randomly chosen two-digit integer is a multiple of 4 is \(22/90 \approx 36.6\%\).
\end{proof}

\subsection{Exercise 22}
\subsubsection{(a)}
How many positive three-digit integers are multiples of 6?

\begin{proof}
     They are \(102, 108, 114, \ldots, 984, 990, 996\). Notice \(102 = 6 \cdot 17\) and \(996 = 6 \cdot 166\). So there
     are \(166 - 17 + 1 = 150\) such integers.
\end{proof}

\subsubsection{(b)}
What is the probability that a randomly chosen positive three-digit integer is a multiple of 6?

\begin{proof}
     There are \(999 - 100 + 1 = 900\) positive three-digit integers. So the probability is, by part (a),
     \(150/900 = 1/6 \approx 16.6\%\).
\end{proof}

\subsubsection{(c)}
What is the probability that a randomly chosen positive three-digit integer is a multiple of 7?

\begin{proof}
     Multiples of 7 are \(7 \cdot 15 = 105, 7 \cdot 16 = 112, \ldots, 7 \cdot 141 = 987, 7 \cdot 142 = 994\). So there
     are \(142 - 15 + 1 = 128\) such integers, then the probability is \(128/900 = 32/225 \approx 14.22\%\).
\end{proof}

\subsection{Exercise 23}
Suppose \(A[1], A[2], A[3], \ldots, A[n]\) is a one-dimensional array and \(n > 50\).

\subsubsection{(a)}
How many elements are in the array?

\begin{proof}
     $n$ elements.
\end{proof}

\subsubsection{(b)}
How many elements are in the subarray \(A[4], A[5], \ldots, A[39]\)?

\begin{proof}
     \(39 - 4 + 1 = 36\) elements.
\end{proof}

\subsubsection{(c)}
If \(3 \leq m \leq n\), what is the probability that a randomly chosen array element is in the subarray
\(A[3], A[4], \ldots, A[m]\)?

\begin{proof}
     There are \(m - 3 + 1 = m-2\) elements in the subarray. There are $n$ elements in the array. So the probability is
     \(\frac{m-2}{n}\).
\end{proof}

\subsubsection{(d)}
What is the probability that a randomly chosen array element is in the subarray \\
\(A[\floor{n/2}], A[\floor{n/2}+1], \ldots, A[n]\) if \(n = 39\)?

\begin{proof}
     \(\floor{39/2} = \floor{19.5} = 19\), therefore there are \(39 - 19 + 1 = 21\) elements in the subarray, and there
     are 39 elements in the array, so the probability is \(21/39 \approx 53.84\%\).
\end{proof}

\subsection{Exercise 24}
Suppose \(A[1], A[2], \ldots, A[n]\) is a one-dimensional array and \(n \geq 2\). Consider the sub-array
\(A[1], A[2], \ldots, A[\floor{n/2}]\).

\subsubsection{(a)}
How many elements are in the sub-array (i) if $n$ is even? and (ii) if $n$ is odd?

\begin{proof}
     (i) There are \(\floor{\frac{n}{2}} = \frac{n}{2}\) elements in the sub-array.

     (ii) There are \(\floor{\frac{n}{2}} = \frac{n-1}{2}\) elements in the sub-array.
\end{proof}

\subsubsection{(b)}
What is the probability that a randomly chosen array element is in the sub-array (i) if $n$ is even? and
(ii) if $n$ is odd?

\begin{proof}
     There are $n$ elements in the array, so

     (i) The probability that an element is in the given sub-array is \(\frac{n/2}{n} = \frac{1}{2}\),

     (i) The probability that an element is in the given sub-array is \(\frac{(n-1)/2}{n} = \frac{n-1}{2n}\).
\end{proof}

\subsection{Exercise 25}
Suppose \(A[1], A[2], \ldots, A[n]\) is a one-dimensional array and \(n \geq 2\). Consider the sub-array
\(A[\floor{n/2}], A[\floor{n/2}+1], \ldots, A[n]\).

\subsubsection{(a)}
How many elements are in the sub-array (i) if $n$ is even? and (ii) if $n$ is odd?

\begin{proof}
     (i) There are \(n - \floor{\frac{n}{2}} + 1 = n - \frac{n}{2} + 1 = \frac{n+2}{2}\) elements in the sub-array.

     (ii) There are \(n-\floor{\frac{n}{2}}+1 = n-\frac{n-1}{2} + 1 = \frac{n+3}{2}\) elements in the sub-array.
\end{proof}

\subsubsection{(b)}
What is the probability that a randomly chosen array element is in the sub-array (i) if $n$ is even? and
(ii) if $n$ is odd?

\begin{proof}
     There are $n$ elements in the array, so

     (i) The probability that an element is in the given sub-array is \(\frac{(n+2)/2}{n} = \frac{n+2}{2n}\),

     (i) The probability that an element is in the given sub-array is \(\frac{(n+3)/2}{n} = \frac{n+3}{2n}\).
\end{proof}

\subsection{Exercise 26}
What is the 27th element in the one-dimensional array \(A[42], A[43], \ldots, A[100]\)?

\begin{proof}
     Let $k$ be the 27th element in the array. By Theorem 9.1.1, \(k - 42 + 1 = 27\), and so \(k = 42 + 27 - 1 = 68\).
     Thus the 27th element in the array is \(A[68]\).
\end{proof}

\subsection{Exercise 27}
What is the 62nd element in the one-dimensional array \(B[29], B[30], \ldots, B[100]\)?

\begin{proof}
     Let $k$ be the 62nd element in the array. By Theorem 9.1.1, \(k - 29 + 1 = 62\), and so \(k = 29 + 62 - 1 = 90\).
     Thus the 62th element in the array is \(A[90]\).
\end{proof}

\subsection{Exercise 28}
If the largest of 56 consecutive integers is 279, what is the smallest?

\begin{proof}
     Let $m$ be the smallest of the integers. By Theorem 9.1.1, \(279 - m + 1 = 56\), and so \(m = 279 - 56 + 1 = 224\).
     Thus the smallest of the integers is 224.
\end{proof}

\subsection{Exercise 29}
If the largest of 87 consecutive integers is 326, what is the smallest?

\begin{proof}
     Let $m$ be the smallest of the integers. By Theorem 9.1.1, \(326 - m + 1 = 87\), and so \(m = 326 - 87 + 1 = 240\).
     Thus the smallest of the integers is 240.
\end{proof}

\subsection{Exercise 30}
How many even integers are between 1 and 1,001?

\begin{proof}
     They are \(2 = 2 \cdot 1, 4 = 2 \cdot 2, \ldots, 998 = 2 \cdot 499, 1000 = 2 \cdot 500\). So there are 500 of them.
\end{proof}

\subsection{Exercise 31}
How many integers that are multiples of 3 are between 1 and 1,001?

\begin{proof}
     They are \(3 = 3 \cdot 1, 6 = 3 \cdot 2, \ldots, 996 = 3 \cdot 332, 999 = 3 \cdot 333\). So there are 333 of them.
\end{proof}

\subsection{Exercise 32}
A certain non-leap year has 365 days, and January 1 occurs on a Monday.

\subsubsection{(a)}
How many Sundays are in the year?

\begin{proof}
     Sundays occur on days 7, 14, 21, \(\ldots\), 364 of the year. Since \(7 = 7 \cdot 1\) and \(364 = 7 \cdot 52\),
     there are 52 Sundays in the year.
\end{proof}

\subsubsection{(b)}
How many Mondays are in the year?

\begin{proof}
     For each Sunday, there is a Monday in the same week. However there is also the 365th day, which comes directly
     after the last Sunday, which is the 364th day. Therefore there are \(52 + 1 = 53\) Mondays in the year.
\end{proof}

\subsection{Exercise 33}
Prove Theorem 9.1.1. (Let $m$ be any integer and prove the theorem by mathematical induction on $n$.)

\begin{proof}
     Let $m$ be any integer and let \(P(n)\) be the statement ``if \(m \leq n\), then there are \(n - m + 1\) integers
     from $m$ to $n$ inclusive.'' The base case is \(n=m\).

          {\bf Show $P(m)$ is true:} There is only one integer from $m$ to $m$ inclusive, namely $m$ itself. And \(n - m + 1
     = m - m + 1 = 1\), so $P(m)$ is true.

          {\bf Show that for any integer \(k \geq m\) if \(P(k)\) is true then \(P(k+1)\) is true:} Assume \(k \geq m\) and
     assume there are \(k-m+1\) integers from $m$ to $k$ inclusive. Then there is one more integer, namely \(k+1\),
     from $m$ to $k+1$ inclusive, thus there are \((k-m+1)+1 = (k+1)-m+1\) integers from $m$ to $k+1$ inclusive. So
     \(P(k+1)\) is true.
\end{proof}

\section{Exercise Set 9.2}
 {\bf \cy In $1-4$, use the fact that in baseball’s World Series, the first team to win four games wins the series.}

\subsection{Exercise 1}
Suppose team $A$ wins the first three games. How many ways can the World Series be completed? (Draw a tree.)

\begin{proof}
     \begin{figure}[ht!]
          \centering
          \includegraphics[scale=0.43]{../images/9.2.1.png}
     \end{figure}

     There are five ways to complete the series: $A$, \(B-A\), \(B-B-A\), \(B-B-B-A\), and \(B-B-B-B\).
\end{proof}

\subsection{Exercise 2}
Suppose team $A$ wins the first two games. How many ways can the World Series be completed? (Draw a tree.)

\begin{proof}
     \begin{figure}[ht!]
          \centering
          \includegraphics[scale=0.25]{../images/9.2.2.png}
     \end{figure}

     There are 15 ways: AA, ABA, ABBA, ABBBA, ABBBB, BAA, BABA, BABBA, BABBB, BBAA, BBABA, BBABB, BBBAA, BBBAB, BBBB.
\end{proof}

\subsection{Exercise 3}
How many ways can a World Series be played if team $A$ wins four games in a row?

\begin{proof}
     Four ways: \(A-A-A-A, B-A-A-A-A, B-B-A-A-A-A\), and \(B-B-B-A-A-A-A\)
\end{proof}

\subsection{Exercise 4}
How many ways can a World Series be played if no team wins two games in a row?

\begin{proof}
     Two ways: \(A-B-A-B-A-B-A\) and \(B-A-B-A-B-A-B\)
\end{proof}

\subsection{Exercise 5}
In a competition between players $X$ and $Y$, the first player to win three games in a row or a total of four games
wins. How many ways can the competition be played if $X$ wins the first game and $Y$ wins the second and third
games? (Draw a tree.)

\begin{proof}
     \begin{figure}[ht!]
          \centering
          \includegraphics[scale=0.31]{../images/9.2.5.png}
     \end{figure}

     There are seven ways: \(XXX, XXYX, XXYY, XYXX, XYXY, XYY, Y\).
\end{proof}

\subsection{Exercise 6}
One urn contains two black balls (labeled \(B_1\) and \(B_2\)) and one white ball. A second urn contains one black
ball and two white balls (labeled \(W_1\) and \(W_2\)). Suppose the following experiment is performed: One of the
two urns is chosen at random. Next a ball is randomly chosen from the urn. Then a second ball is chosen at random
from the same urn without replacing the first ball.

\subsubsection{(a)}
Construct the possibility tree showing all possible outcomes of this experiment.

\begin{proof}
     \begin{figure}[ht!]
          \centering
          \includegraphics[scale=0.5]{../images/9.2.6.a.png}
     \end{figure}
\end{proof}

\subsubsection{(b)}
What is the total number of outcomes of this experiment?

\begin{proof}
     12
\end{proof}

\subsubsection{(c)}
What is the probability that two black balls are chosen?

\begin{proof}
     \(2/12 = 1/6 \approx 16.6\%\)
\end{proof}

\subsubsection{(d)}
What is the probability that two balls of opposite color are chosen?

\begin{proof}
     \(8/12 = 2/3 \approx 66.6\%\)
\end{proof}

\subsection{Exercise 7}
One urn contains one blue ball (labeled \(B_1\)) and three red balls (labeled \(R_1, R_2\), and \(R_3\)). A second urn
contains two red balls (\(R_4\) and \(R_5\)) and two blue balls (\(B_2\) and \(B_3\)). An experiment is performed in
which one of the two urns is chosen at random and then two balls are randomly chosen from it, one after the other
without replacement.

\subsubsection{(a)}
Construct the possibility tree showing all possible outcomes of this experiment.

\begin{proof}
     \begin{figure}[ht!]
          \centering
          \includegraphics[scale=0.5]{../images/9.2.7.a.png}
     \end{figure}
\end{proof}

\subsubsection{(b)}
What is the total number of outcomes of this experiment?

\begin{proof}
     24
\end{proof}

\subsubsection{(c)}
What is the probability that two red balls are chosen?

\begin{proof}
     \(8/24 = 1/3 \approx 33.3\%\)
\end{proof}

\subsection{Exercise 8}
A person buying a personal computer system is offered a choice of three models of the basic unit, two models of
keyboard, and two models of printer. How many distinct systems can be purchased?

\begin{proof}
     By the multiplication rule, the answer is \(3 \cdot 2 \cdot 2 = 12\).
\end{proof}

\subsection{Exercise 9}
Suppose there are three roads from city $A$ to city $B$ and five roads from city $B$ to city $C$.

\subsubsection{(a)}
How many ways is it possible to travel from city $A$ to city $C$ via city $B$?

\begin{proof}
     In going from city $A$ to city $B$, one may take any of the 3 roads. In going from city $B$ to city $C$, one may take
     any of the 5 roads. So, by the multiplication rule, there are \(3 \cdot 5 = 15\) ways to travel from city $A$ to city
     $C$ via city $B$.
\end{proof}

\subsubsection{(b)}
How many different round-trip routes are there from city $A$ to $B$ to $C$ to $B$ and back to $A$?

\begin{proof}
     A round-trip journey can be thought of as a four-step operation: {\cy Step 1:} Go from $A$ to $B$. {\cy Step 2:}
     Go from $B$ to $C$. {\cy Step 3:} Go from $C$ to $B$. {\cy Step 4:} Go from $B$ to $A$.

     Since there are 3 ways to perform step 1, 5 ways to perform step 2, 5 ways to perform step 3, and 3 ways to perform
     step 4, by the multiplication rule, there are \(3 \cdot 5 \cdot 5 \cdot 3 = 225\) round-trip routes.
\end{proof}

\subsubsection{(c)}
How many different routes are there from cities $A$ to $B$ to $C$ to $B$ and back to $A$ in which no road is traversed twice?

\begin{proof}
     In this case the steps for making a round-trip journey are the same as in part (b), but since no route segment may be
     repeated, there are only 4 ways to perform step 3 and only 2 ways to perform step 4. So, by the multiplication rule,
     there are \(3 \cdot 5 \cdot 4 \cdot 2 = 120\) round-trip routes in which no road is traversed twice.
\end{proof}

\subsection{Exercise 10}
Suppose there are three routes from North Point to Boulder Creek, two routes from Boulder Creek to Beaver Dam, two
routes from Beaver Dam to Star Lake, and four routes directly from Boulder Creek to Star Lake. (Draw a sketch.)

\subsubsection{(a)}
How many routes from North Point to Star Lake pass through Beaver Dam?

\begin{proof}
     By the multiplication rule \(3 \cdot 2 \cdot 2 = 12\) routes.
\end{proof}

\subsubsection{(b)}
How many routes from North Point to Star Lake bypass Beaver Dam?

\begin{proof}
     By the multiplication rule \(3 \cdot 4 = 12\) routes.
\end{proof}

\subsection{Exercise 11}
\subsubsection{(a)}
A bit string is a finite sequence of 0’s and 1’s. How many bit strings have length 8?

\begin{proof}
     Imagine constructing a bit string of length 8 as an eight-step process:

     {\cy Step 1:} Choose either a 0 or a 1 for the left-most position,

     {\cy Step 2:} Choose either a 0 or a 1 for the next position to the right.

     \(\vdots\)

     {\cy Step 8:} Choose either a 0 or a 1 for the right-most position.

     Since there are 2 ways to perform each step, the total number of ways to accomplish the entire operation, which is
     the number of different bit strings of length 8, is \(2 \cdot 2 \cdot 2 \cdot 2 \cdot 2 \cdot 2 \cdot 2 \cdot 2 =
     2^8 = 256\).
\end{proof}

\subsubsection{(b)}
How many bit strings of length 8 begin with three 0’s?

\begin{proof}
     Imagine that there are three 0’s in the three left-most positions, and imagine filling in the remaining 5 positions
     as a five-step process, where step $i$ is to fill in the \((i + 3)\)rd position. Since there are 2 ways to perform
     each of the 5 steps, there are \(2^5\) ways to perform the entire operation. So there are \(2^5\), or 32, 8-bit
     strings that begin with three 0’s.
\end{proof}

\subsubsection{(c)}
How many bit strings of length 8 begin and end with a 1?

\begin{proof}
     \(2^6 = 64\)
\end{proof}

\subsection{Exercise 12}
Hexadecimal numbers are made using the sixteen hexadecimal digits 0, 1, 2, 3, 4, 5, 6, 7, 8, 9, A, B, C, D, E, F and
are denoted using the subscript 16. For example, 9A2D16 and BC5416 are hexadecimal numbers.

\subsubsection{(a)}
How many hexadecimal numbers begin with one of the digits 3 through B, end with one of the digits 5 through F, and are
5 digits long?

\begin{proof}
     Think of creating a hexadecimal number that satisfies the given requirements as a five-step process. \\
     {\cy Step 1:} Choose the left-most hexadecimal digits. It can be any of the 9 hexadecimal digits from 3 through B.

          {\cy Steps 2-4:} Choose the three hexadecimal digits for the middle three positions. Each can be any of the 16
     hexadecimal digits.

          {\cy Step 5:} Choose the right-most hexadecimal digit. It can be any of the 11 hexadecimal digits from 5 through F.

     There are 9 ways to perform step 1, 16 ways to perform each of steps 2 through 4, and 11 ways to perform step 5. Thus,
     the total number of specified hexadecimal numbers is \(9 \cdot 16 \cdot 16 \cdot 16 \cdot 11 = 405,504\).
\end{proof}

\subsubsection{(b)}
How many hexadecimal numbers begin with one of the digits 4 through D, end with one of the digits 2 through E, and are
6 digits long?

\begin{proof}
     There are 10 choices for the first digit: 4, 5, 6, 7, 8, 9, A, B, C, D.

     There are 14 choices for the first digit: 2, 3, 4, 5, 6, 7, 8, 9, A, B, C, D, E.

     There are 16 choices for each one of the 4 middle digits.

     So: \(10 \cdot 16 \cdot 16 \cdot 16 \cdot 16 \cdot 14 = 9175040\)
\end{proof}

\subsection{Exercise 13}
A coin is tossed four times. Each time the result H for heads or T for tails is recorded. An outcome of HHTT means that heads were obtained on the first two tosses and tails on the second two. Assume that heads and tails are equally likely on each toss.

\subsubsection{(a)}
How many distinct outcomes are possible?

\begin{proof}
     In each of the four tosses there are two possible results: Either a head (H) or a tail (T) is obtained. Thus, by the
     multiplication rule, the number of outcomes is \(2 \cdot 2 \cdot 2 \cdot 2 = 2^4 = 16\).
\end{proof}

\subsubsection{(b)}
What is the probability that exactly two heads occur?

\begin{proof}
     There are six outcomes with two heads: HHTT, HTHT, HTTH, THHT, THTH, TTHH. Thus the probability of obtaining exactly
     two heads is \(6/16 = 3/8\).
\end{proof}

\subsubsection{(c)}
What is the probability that exactly one head occurs?

\begin{proof}
     There are four outcomes with exactly one head: HTTT, THTT, TTHT, TTTH. Thus the probability of obtaining exactly two
     heads is \(4/16 = 1/4\).
\end{proof}

\subsection{Exercise 14}
Suppose that in a certain state, all automobile license plates have four uppercase letters followed by three
digits.

\subsubsection{(a)}
How many different license plates are possible?

\begin{proof}
     Think of creating license plates that satisfy the given conditions as the following seven-step process: In steps
     1–4 choose the letters to put in positions 1–4, and in steps 5–7, choose the digits to put in positions 5–7. Since
     there are 26 letters and 10 digits and since repetition is allowed, there are 26 ways to perform each of steps 1–4 and
     10 ways to perform each of steps 5–7. Thus the number of license plates is \(26 \cdot 26 \cdot 26 \cdot 26 \cdot 10
     \cdot 10 \cdot 10 = 456,976,000\).
\end{proof}

\subsubsection{(b)}
How many license plates could begin with A and end in 0?
\begin{proof}
     In this case there is only one way to perform step 1 (because the first letter must be an A) and only one way to
     perform step 7 (because the last digit must be a 0). Therefore, the number of license plates is
     \(26 \cdot 26 \cdot 26 \cdot 10 \cdot 10 = 1,757,600\).
\end{proof}

\subsubsection{(c)}
How many license plates could begin with TGIF?

\begin{proof}
     There are 3 digits left to be chosen. Each has 10 possibilities. So by the multiplication rule \(10^3=1000\)
     can begin with TGIF.
\end{proof}

\subsubsection{(d)}
How many license plates are possible in which all the letters and digits are distinct?

\begin{proof}
     In this case there are 26 ways to perform step 1, 25 ways to perform step 2, 24 ways to perform step 3, 23 ways to
     perform step 4, 10 ways to perform step 5, 9 ways to perform step 6, and 8 ways to perform step 7, so the number
     of license plates is \(26 \cdot 25 \cdot 24 \cdot 23 \cdot 10 \cdot 9 \cdot 8 = 258,336,000\).
\end{proof}

\subsubsection{(e)}
How many license plates could begin with AB and have all letters and digits distinct?

\begin{proof}
     24 choices for the third letter, 23 for the fourth letter. Then 10, 9, 8 choices for the 1st, 2nd, 3rd digits. So:
     \(24 \cdot 23 \cdot 10 \cdot 9 \cdot 8 = 397440\).
\end{proof}

\subsection{Exercise 15}
A combination lock requires three selections of numbers, each from 1 through 30.

\subsubsection{(a)}
How many different combinations are possible?

\begin{proof}
     \(30^3 = 27000\)
\end{proof}

\subsubsection{(b)}
Suppose the locks are constructed in such a way that no number may be used twice. How many different combinations
are possible?

\begin{proof}
     \(30 \cdot 29 \cdot 28 = 24360\)
\end{proof}

\subsection{Exercise 16}
\subsubsection{(a)}
How many integers are there from 10 through 99?

\begin{proof}
     Two solutions:

     (i) By the multiplication rule, the number of integers from 10 through 99 = (the number of ways to pick the first
     digit) \(\times\) (the number of ways to pick the second digit) = \(9 \cdot 10 = 90\).

     (ii) By Theorem 9.1.1, the number of integers from 10 through 99 = \(90 - 10 + 1 = 90\).
\end{proof}

\subsubsection{(b)}
How many odd integers are there from 10 through 99?

\begin{proof}
     Because odd integers end in 1, 3, 5, 7, or 9, the number of odd integers from 10 through 99 = (the number of ways to
     pick the first digit) \(\times\) (the number of ways to pick the second digit) = \(9 \cdot 5 = 45\).

     An alternative solution uses the listing method shown in the solution for Example 9.1.4.
\end{proof}

\subsubsection{(c)}
How many integers from 10 through 99 have distinct digits?

\begin{proof}
     (the number of ways to pick the first digit) \(\times\) (the number of ways to pick the second digit) =
     \(9 \cdot 9 = 81\). \\
     Another solution is to start with the solution to part (a), which is 90, and subtract the number of integers that have
     the same digit repeated, of which there are 9: 11, 22, 33, 44, 55, 66, 77, 88, 99.
\end{proof}

\subsubsection{(d)}
How many odd integers from 10 through 99 have distinct digits?

\begin{proof}
     (the number of ways to pick the second digit) \(\times\) (the number of ways to pick the first digit) = \(5 \cdot 8
     = 40\). Here the second digit can be 1,3,5,7,9; and the first digit cannot be 0 or the same as the first digit.
\end{proof}

\subsubsection{(e)}
What is the probability that a randomly chosen two-digit integer has distinct digits? has distinct digits and is odd?

\begin{proof}
     81/90 = 9/10, 40/90 = 4/9
\end{proof}

\subsection{Exercise 17}
\subsubsection{(a)}
How many integers are there from 1000 through 9999?

\begin{proof}
     By Theorem 9.1.1, \(9999-1000+1 = 9000\). Another solution: there are 9 choices for the first digit (cannot be 0), and
     10 choices for the other 3 digits, so \(9 \cdot 10 \cdot 10 \cdot 10 = 9000\).
\end{proof}

\subsubsection{(b)}
How many odd integers are there from 1000 through 9999?

\begin{proof}
     4500.

     One solution is that it's half of the answer to part (a), since half the integers are even, and the other odd.

     Another solution: there are 9 choices for the first digit (cannot be 0), 10 choices for the second and third digits,
     and 5 choices for the last (1,3,5,7,9), therefore \(9 \cdot 10 \cdot 10 \cdot 5 = 4500\).
\end{proof}

\subsubsection{(c)}
How many integers from 1000 through 9999 have distinct digits?

\begin{proof}
     There are 9 choices for the first digit (cannot be 0). There are 9 choices for the second digit (cannot be the same as the first digit, but can be 0). Then 8 and 7 choices for the third and fourth digits. So there are
     \(9 \cdot 9 \cdot 8 \cdot 7 = 4536\) such integers.
\end{proof}

\subsubsection{(d)}
How many odd integers from 1000 through 9999 have distinct digits?

\begin{proof}
     There are 5 choices for the last digit (1, 3, 5, 7, 9). There are 8 choices for the first digit (cannot be the
     same as the last digit, and cannot be 0). Then 8 choices for the second digit (cannot be the same as the first or
     the last, but can be 0) and 7 choices for the third digit. So there are \(5 \cdot 8 \cdot 8 \cdot 7 = 2240\) such
     integers.
\end{proof}

\subsubsection{(e)}
What is the probability that a randomly chosen four-digit integer has distinct digits? has distinct digits and is odd?

\begin{proof}
     4536/9000, 2240/9000
\end{proof}

\subsection{Exercise 18}
The following diagram shows the keypad for an automatic teller machine. As you can see, the same sequence of keys
represents a variety of different PINs. For instance, 2133, AZDE, and BQ3F are all keyed in exactly the same way.

\begin{figure}[ht!]
     \centering
     \includegraphics[scale=0.5]{../images/9.2.18.png}
\end{figure}

\subsubsection{(a)}
How many different PINs are represented by the same sequence of keys as 2133?

\begin{proof}
     Let step 1 be to choose either the number 2 or one of the letters corresponding to the number 2 on the keypad, let
     step 2 be to choose either the number 1 or one of the letters corresponding to the number 1 on the keypad, and
     let steps 3 and 4 be to choose either the number 3 or one of the letters corresponding to the number 3 on the keypad.
     There are 4 ways to perform step 1, 3 ways to perform step 2, and 4 ways to perform each of steps 3 and 4. So by the
     multiplication rule, there are \(4 \cdot 3 \cdot 4 \cdot 4 = 192\) ways to perform the entire operation. Thus there
     are 192 different PINs that are keyed the same as 2133. Note that on a computer keyboard, these PINs would not be
     keyed the same way.
\end{proof}

\subsubsection{(b)}
How many different PINs are represented by the same sequence of keys as 5031?

\begin{proof}
     \(4 \cdot 1 \cdot 4 \cdot 3 = 48\)
\end{proof}

\subsubsection{(c)}
How many different numeric sequences on the machine contain no repeated digit?

\begin{proof}
     \(10 \cdot 9 \cdot 8 \cdot 7 = 5040\)
\end{proof}

\subsection{Exercise 19}
Three officers - a president, a treasurer, and a secretary - are to be chosen from among four people: Ann, Bob, Cyd,
and Dan. Suppose that Bob is not qualified to be treasurer and Cyd’s other commitments make it impossible for her to
be secretary. How many ways can the officers be chosen? Can the multiplication rule be used to solve this problem?

\begin{proof}
     There are 14 different paths from “root” to “leaf” of this possibility tree, and so there are 14 ways the officers can
     be chosen. Because \(14 = 2\cdot7\), reordering the steps will not make it possible to use the multiplication rule
     alone to solve this problem.

     \begin{figure}[ht!]
          \centering
          \includegraphics[scale=0.45]{../images/9.2.19.png}
     \end{figure}
\end{proof}

\subsection{Exercise 20}
Modify Example 9.2.2 by supposing that a PIN must not begin with any of the letters A-M and must end with a digit. Continue to assume that no symbol may be used more than once and that the total number of PINs is to be determined.

\subsubsection{(a)}
Find the error in the following “solution.” “Constructing a PIN is a four-step process.

Step 1: Choose the left-most symbol.

Step 2: Choose the second symbol from the left.

Step 3: Choose the third symbol from the left.

Step 4: Choose the right-most symbol.

Because none of the thirteen letters from A through M may be chosen in step 1, there are \(36 - 13 = 23\) ways to
perform step 1. There are 35 ways to perform step 2 and 34 ways to perform step 3 because previously used symbols may
not be used. Since the symbol chosen in step 4 must be a previously unused digit, there are \(10 - 3 = 7\) ways to
perform step 4. Thus there are \(23 \cdot 35 \cdot 34 \cdot 7 = 191,590\) different PINs that satisfy the given
conditions.”

\begin{proof}
     The number of ways to perform step 4 is not constant; it depends on how the previous steps were performed. For
     instance, if 3 digits had been chosen in steps 1-3, then there would be \(10 - 3 = 7\) ways to perform step 4, but
     if 3 letters had been chosen in steps 1-3, then there would be 10 ways to perform step 4.
\end{proof}

\subsubsection{(b)}
Reorder steps 1-4 in part (a) as follows: {\cy Step 1:} Choose the right-most symbol. {\cy Step 2:} Choose the
left-most symbol. {\cy Step 3:} Choose the second symbol from the left. {\cy Step 4:} Choose the third symbol from
the left. Use the multiplication rule to find the number of PINs that satisfy the given conditions.

\begin{proof}
     \(10 \cdot 22 \cdot 34 \cdot 33 = 246840\)
\end{proof}

\subsection{Exercise 21}
Suppose $A$ is a set with $m$ elements and $B$ is a set with $n$ elements.

\subsubsection{(a)}
How many relations are there from $A$ to $B$? Explain.

\begin{proof}
     \(2^{mn}\). There are $mn$ pairs in \(A \times B\). A relation is simply any subset of \(A \times B\). So the
     number of relations from $A$ to $B$ is the same as the size of the power set of \(A \times B\), which is \(2^{mn}\).
\end{proof}

\subsubsection{(b)}
How many functions are there from $A$ to $B$? Explain.

\begin{proof}
     \(n^m\). For each input \(a \in A\) there are $n$ possible outputs \(b \in B\). Since there are $m$ inputs, by the
     multiplication rule there are \(n \cdot n \cdots n = n^m\) functions.
\end{proof}

\subsubsection{(c)}
What fraction of the relations from $A$ to $B$ are functions?

\begin{proof}
     \(n^m / 2^{mn}\)
\end{proof}

\subsection{Exercise 22}
\subsubsection{(a)}
How many functions are there from a set with three elements to a set with four elements?

\begin{proof}
     The answer is \(4 \cdot 4 \cdot 4 = 4^3 = 64\). Imagine creating a function from a 3-element set to a 4-element set
     as a three-step process: Step 1 is to send the first element of the 3-element set to an element of the 4-element
     set (there are four ways to perform this step); step 2 is to send the second element of the 3-element set to an
     element of the 4-element set (there are also four ways to perform this step); and step 3 is to send the third element
     of the 3-element set to an element of the 4-element set (there are four ways to perform this step). Thus the entire
     process can be performed in \(4 \cdot 4 \cdot 4\) different ways.
\end{proof}

\subsubsection{(b)}
How many functions are there from a set with five elements to a set with two elements?

\begin{proof}
     \(2^5 = 32\)
\end{proof}

\subsubsection{(c)}
How many functions are there from a set with $m$ elements to a set with $n$ elements, where $m$ and $n$ are positive
integers?

\begin{proof}
     \(n^m\).
\end{proof}

\subsection{Exercise 23}
In Section 2.5 we showed how integers can be represented by strings of 0’s and 1’s inside a digital computer. In fact,
through various coding schemes, strings of 0’s and 1’s can be used to represent all kinds of symbols. One commonly
used code is the Extended Binary-Coded Decimal Interchange Code (EBCDIC) in which each symbol has an 8-bit
representation. How many distinct symbols can be represented by this code?

\begin{proof}
     \(2^8 = 256\)
\end{proof}

{\bf \cy In each of $24-28$, determine how many times the innermost loop will be iterated when the algorithm segment
is implemented and run. (assume that \(m, n, p, a, b, c\), and $d$ are all positive integers.)}

\subsection{Exercise 24}
\begin{tabbing}
     {\bf for} \= \(i \coloneqq 1\) {\bf to} 30 \\
     \> {\bf for} \= \(j \coloneqq 1\) {\bf to} 15 \\
     \>           \> {\it [Statements in body of inner loop.]} \\
     \>           \> {\it [None contain branching statements that lead outside the loop.]} \\
     \> {\bf next j} \\
     {\bf next i}
\end{tabbing}

\begin{proof}
     The outer loop is iterated 30 times, and during each iteration of the outer loop there are 15 iterations of the
     inner loop. Hence, by the multiplication rule, the total number of iterations of the inner loop is
     \(30 \cdot 15 = 450\).
\end{proof}

\subsection{Exercise 25}
\begin{tabbing}
     {\bf for} \= \(j \coloneqq 1\) {\bf to} $m$ \\
     \> {\bf for} \= \(k \coloneqq 1\) {\bf to} $n$ \\
     \>           \> {\it [Statements in body of inner loop.]} \\
     \>           \> {\it [None contain branching statements that lead outside the loop.]} \\
     \> {\bf next k} \\
     {\bf next j}
\end{tabbing}

\begin{proof}
     \(mn\)
\end{proof}

\subsection{Exercise 26}
\begin{tabbing}
     {\bf for} \= \(i \coloneqq 1\) {\bf to} $m$ \\
     \> {\bf for} \= \(j \coloneqq 1\) {\bf to} $n$ \\
     \>           \> {\bf for} \= \(k \coloneqq 1\) {\bf to} $p$ \\
     \>           \>           \> {\it [Statements in body of inner loop.]} \\
     \>           \>           \> {\it [None contain branching statements that lead outside the loop.]} \\
     \>           \> {\bf next k} \\
     \> {\bf next j} \\
     {\bf next i}
\end{tabbing}

\begin{proof}
     \(mnp\)
\end{proof}

\subsection{Exercise 27}
\begin{tabbing}
     {\bf for} \= \(i \coloneqq 5\) {\bf to} 50 \\
     \> {\bf for} \= \(j \coloneqq 10\) {\bf to} 20 \\
     \>           \> {\it [Statements in body of inner loop.]} \\
     \>           \> {\it [None contain branching statements that lead outside the loop.]} \\
     \> {\bf next j} \\
     {\bf next i}
\end{tabbing}

\begin{proof}
     The outer loop is iterated \(50 - 5 + 1 = 46\) times, and during each iteration of the outer loop there are
     \(20 - 10 + 1 = 11\) iterations of the inner loop. Hence, by the multiplication rule, the total number of iterations
     of the inner loop is \(46 \cdot 11 = 506\).
\end{proof}

\subsection{Exercise 28}
Assume \(a \leq b\) and \(c \leq d\).
\begin{tabbing}
     {\bf for} \= \(i \coloneqq a\) {\bf to} $b$ \\
     \> {\bf for} \= \(j \coloneqq c\) {\bf to} $d$ \\
     \>           \> {\it [Statements in body of inner loop.]} \\
     \>           \> {\it [None contain branching statements that lead outside the loop.]} \\
     \> {\bf next j} \\
     {\bf next i}
\end{tabbing}

\begin{proof}
     \((b-a+1)(d-c+1)\)
\end{proof}

\subsection{Exercise 29}
Consider the numbers 1 through 99,999 in their ordinary decimal representations. How many contain exactly one of
each of the digits 2, 3, 4, and 5?

{\it Hint:} An efficient solution is to add leading zeros as needed to make each number five digits long. For
instance, write 1 as 00001. Then, instead of choosing digits for the positions, choose positions for the digits.
The answer is 720.

\begin{proof}
     Following the Hint, there are 5 possible positions to place 2, then 4 positions to place 3, 3 positions to place 4, 2
     positions to place 5, and 6 choices for the last position (1, 6, 7, 8, 9, 0). Therefore \(5 \cdot 4 \cdot 3 \cdot 2
     \cdot 6 = 720\). (Notice that there is no issue with the digit 0, since it can be placed in the first position.)
\end{proof}

\subsection{Exercise 30}
Let \(n = p_1^{k_1} p_2^{k_2} \cdots p_m^{k_m}\) where \(p_1, p_2, \ldots, p_m\) are distinct prime numbers and
\(k_1, k_2, \ldots, k_m\) are positive integers. How many ways can $n$ be written as a product of two positive
integers that have no common factors, assuming the following?

\subsubsection{(a)}
Order matters (that is, \(8 \cdot 15\) and \(15 \cdot 8\) are regarded as different).

\begin{proof}
     To split $n$ into two positive integers $a, b$ that have no common factors, we need to split the set of all the prime
     powers \(\{p_1^{k_1}, \cdots, p_m^{k_m}\) into two disjoint subsets $A, B$. Then $a$ is the product of the prime powers in $A$, and $b$ is the product of those in $B$.

     When one of the subsets is chosen, the other is automatically forced to be what's left. For example, if the
     set is \(p_1^{k_1}, p_2^{k_2}, p_3^{k_3}\) and we choose $A$ to be \(\{p_1^{k_1}, p_3^{k_3}\}\) then $B$ has to be
     \(\{p_2^{k_2}\}\).

     However, since order matters, choosing $A$ to be \(\{p_1^{k_1}, p_3^{k_3}\}\) and $B$ to be \(\{p_2^{k_2}\}\)
     is considered different than choosing $A$ to be \(\{p_2^{k_2}\}\) and $B$ to be \(\{p_1^{k_1}, p_3^{k_3}\}\).

     This means that the number of ways to write $n$ as the product of two positive integers with no common prime
     factors is equal to the number of ways we can choose $A$ to be a subset of \(\{p_1^{k_1}, \ldots, p_m^{k_m}\}\). This
     is a set with $m$ elements, so it has $2^m$ subsets. So there are $2^m$ ways to choose $A$, which is the answer.
\end{proof}

\subsubsection{(b)}
Order does not matter (that is, \(8 \cdot 15\) and \(15 \cdot 8\) are regarded as the same).

\begin{proof}
     The solution is the same as in part (a) except the choices where $A$ and $B$ are swapped do not count. For example,
     choosing $A$ to be \(\{p_1^{k_1}, p_3^{k_3}\}\) and $B$ to be \(\{p_2^{k_2}\}\) is considered the same as choosing $A$
     to be \(\{p_2^{k_2}\}\) and $B$ to be \(\{p_1^{k_1}, p_3^{k_3}\}\).

     So, in this case the answer is half the answer to part (a), namely, \(2^m / 2 = 2^{m-1}\).
\end{proof}

\subsection{Exercise 31}
\subsubsection{(a)}
If $p$ is a prime number and $a$ is a positive integer, how many distinct positive divisors does $p^a$ have?

\begin{proof}
     There are \(a + 1\) divisors: \(1, p, p^2, \ldots, p^a\).
\end{proof}

\subsubsection{(b)}
If $p$ and $q$ are distinct prime numbers and $a$ and $b$ are positive integers, how many distinct positive divisors
does $p^a q^b$ have?

\begin{proof}
     A divisor is a product of any one of the \(a + 1\) numbers listed in part (a) times any one of the \(b + 1\) numbers
     \(1, q, q^2, \ldots, q^b\). So, by the multiplication rule, there are \((a + 1)(b + 1)\) divisors in all.
\end{proof}

\subsubsection{(c)}
If $p, q$, and $r$ are distinct prime numbers and $a, b$, and $c$ are positive integers, how many distinct positive
divisors does $p^a q^b r^c$ have?

\begin{proof}
     \((a+1)(b+1)(c+1)\)
\end{proof}

\subsubsection{(d)}
If \(p_1, p_2, \ldots, p_m\) are distinct prime numbers and \(a_1, a_2, \ldots, a_m\) are positive integers, how many
distinct positive divisors does \(p_1^{a_1} p_2^{a_2} \cdots p_m^{a_m}\) have?

\begin{proof}
     \((a_1+1)(a_2+1) \cdots (a_m+1)\)
\end{proof}

\subsubsection{(e)}
What is the smallest positive integer with exactly 12 divisors?

\begin{proof}
     Assume \(p_1, p_2, \ldots, p_m\) are distinct prime numbers and \(a_1, a_2, \ldots, a_m\) are positive integers, and
     \(n = p_1^{a_1} p_2^{a_2} \cdots p_m^{a_m}\) is the integer we are looking for. Assume $n$ has exactly 12 divisors.

     By part (d), \(12 = 2 \cdot 2 \cdot 3\) so $m=3$ and \(2 \cdot 2 \cdot 3 = (a_1+1)(a_2+1)(a_3+1)\) so it follows
     \(a_1 = 1, a_2 = 1, a_3 = 2\).

     To make $n$ as small as possible, we can choose the smallest first three primes 2, 3, 5, and since \(a_3 = 2\)
     we can let \(p_1 = 3, p_2 = 5, p_3 = 2\). (This way the highest power 2 goes on top of the smallest base prime, 2.)

     Therefore \(n = p_1^{a_1} p_2^{a_2} p_3^{a_3} = 3^1 \cdot 5^1 \cdot 2^2 = 60\) is the smallest positive integer with
     exactly 12 divisors: 1, 2, 3, 4, 5, 6, 10, 12, 15, 20, 30, 60.
\end{proof}

\subsection{Exercise 32}
\subsubsection{(a)}
How many ways can the letters of the word ALGORITHM be arranged in a row?

\begin{proof}
     Since the nine letters of the word ALGORITHM are all distinct, there are as many arrangements of these letters
     in a row as there are permutations of a set with nine elements: \(9! = 362,880\).
\end{proof}

\subsubsection{(b)}
How many ways can the letters of the word ALGORITHM be arranged in a row if A and L must remain together (in
order) as a unit?

\begin{proof}
     In this case there are effectively eight symbols to be permuted (because AL may be regarded as a single symbol).
     So the number of arrangements is \(8! = 40,320\).
\end{proof}

\subsubsection{(c)}
How many ways can the letters of the word ALGORITHM be arranged in a row if the letters GOR must remain together
(in order) as a unit?

\begin{proof}
     In this case there are effectively seven symbols to be permuted (because GOR may be regarded as a single symbol).
     So the number of arrangements is \(7! = 5040\).
\end{proof}

\subsection{Exercise 33}
Six people attend the theater together and sit in a row with exactly six seats.

\subsubsection{(a)}
How many ways can they be seated together in the row?

\begin{proof}
     \(6! = 720\)
\end{proof}

\subsubsection{(b)}
Suppose one of the six is a doctor who must sit on the aisle in case she is paged. How many ways can the people be
seated together in the row with the doctor in an aisle seat?

\begin{proof}
     Excluding the doctor who must sit on the aisle seat, \(5! = 120\) ways. (However the question is a bit ambiguous, are
     there 2 aisle seats, one on each side of the row? In that case the answer would be 240.)
\end{proof}

\subsubsection{(c)}
Suppose the six people consist of three married couples and each couple wants to sit together with the older partner on
the left. How many ways can the six be seated together in the row?

\begin{proof}
     The couples can be treated as one person, so \(3! = 6\).
\end{proof}

\subsection{Exercise 34}
Five people are to be seated around a circular table. Two seatings are considered the same if one is a rotation of
the other. How many different seatings are possible?

\begin{proof}
     The same reasoning as in Example 9.2.9 gives an answer of \(4! = 24\).
\end{proof}

\subsection{Exercise 35}
Write all the 2-permutations of \(\{W, X, Y, Z\}\).

\begin{proof}
     \(WX, WY, WZ, XW, XY, XZ, YW, YX, YZ, ZW, ZX, ZY\)
\end{proof}

\subsection{Exercise 36}
Write all the 3-permutations of \(\{s, t, u, v\}\).

\begin{proof}
     There are \(P(4,3) = \dps\frac{4!}{(4-3)!} = 24\) 3-permutations of a 4-element set. So:

     \(stu, stv, suv, sut, svt, svu, tsu, tsv, tuv, tus, tvs, tvu,\)

     \(ust, usv, utv, uts, uvs, uvt, vst, vsu, vtu, vts, vus, vut.\)
\end{proof}

\subsection{Exercise 37}
Evaluate the following quantities.

\subsubsection{(a)}
\(P(6, 4)\)
\begin{proof}
     \(P(6,4) = \dps\frac{6!}{(6-4)!} = \frac{6!}{2!} = \frac{6 \cdot 5 \cdot 4 \cdot 3 \cdot \Ccancel[cyan]{2 \cdot 1}}
     {\Ccancel[cyan]{2 \cdot 1}} = 360\)
\end{proof}

\subsubsection{(b)}
\(P(6, 6)\)
\begin{proof}
     \(P(6,6) = \dps\frac{6!}{(6-6)!} = \frac{6!}{0!} = 720/1 = 720\)
\end{proof}

\subsubsection{(c)}
\(P(6, 3)\)
\begin{proof}
     \(P(6,3) = \dps\frac{6!}{(6-3)!} = \frac{6!}{3!} = \frac{6 \cdot 5 \cdot 4 \cdot \Ccancel[cyan]{3 \cdot 2 \cdot 1}}
     {\Ccancel[cyan]{3 \cdot 2 \cdot 1}} = 120\)
\end{proof}

\subsubsection{(d)}
\(P(6, 1)\)
\begin{proof}
     \(P(6,1) = \dps\frac{6!}{(6-1)!} = \frac{6!}{5!} = \frac{6 \cdot \Ccancel[cyan]{5 \cdot 4 \cdot 3 \cdot 2 \cdot 1}}
     {\Ccancel[cyan]{5 \cdot 4 \cdot 3 \cdot 2 \cdot 1}} = 6\)
\end{proof}

\subsection{Exercise 38}
\subsubsection{(a)}
How many 3-permutations are there of a set of five objects?

\begin{proof}
     \(P(5,3) = \dps\frac{5!}{(5-3)!} = \frac{5!}{2!} = \frac{5 \cdot 4 \cdot 3 \cdot \Ccancel[cyan]{2 \cdot 1}}
     {\Ccancel[cyan]{2 \cdot 1}} = 60\)
\end{proof}

\subsubsection{(b)}
How many 2-permutations are there of a set of eight objects?

\begin{proof}
     \(P(8,2) = \dps\frac{8!}{(8-2)!} = \frac{8!}{6!} = \frac{8 \cdot 7 \cdot \Ccancel[cyan]{6 \cdot 5 \cdot 4 \cdot 3
               \cdot 2 \cdot 1}}{\Ccancel[cyan]{6 \cdot 5 \cdot 4 \cdot 3
               \cdot 2 \cdot 1}} = 56\)
\end{proof}

\subsection{Exercise 39}
\subsubsection{(a)}
How many ways can three of the letters of the word ALGORITHM be selected and written in a row?

\begin{proof}
     \(P(9,3) = \dps\frac{9!}{(9-3)!} = \frac{9!}{6!} = \frac{9 \cdot 8 \cdot 7 \cdot \Ccancel[cyan]{6!}}
     {\Ccancel[cyan]{6!}} = 504\)
\end{proof}

\subsubsection{(b)}
How many ways can six of the letters of the word ALGORITHM be selected and written in a row?

\begin{proof}
     \(P(9,6) = \dps\frac{9!}{(9-6)!} = \frac{9!}{3!} = \frac{9 \cdot 8 \cdot 7 \cdot 6 \cdot 5 \cdot 4 \cdot
          \Ccancel[cyan]{3!}}{\Ccancel[cyan]{3!}} = 60480\)
\end{proof}

\subsubsection{(c)}
How many ways can six of the letters of the word ALGORITHM be selected and written in a row if the first letter must be A?

\begin{proof}
     \(P(8,5) = \dps\frac{8!}{(8-5)!} = \frac{8!}{3!} = \frac{8 \cdot 7 \cdot 6 \cdot 5 \cdot 4 \cdot \Ccancel[cyan]{3!}}
     {\Ccancel[cyan]{3!}} = 6720\)
\end{proof}

\subsubsection{(d)}
How many ways can six of the letters of the word ALGORITHM be selected and written in a row if the first two letters must be OR?

\begin{proof}
     \(P(7,4) = \dps\frac{7!}{(7-4)!} = \frac{7!}{3!} = \frac{7 \cdot 6 \cdot 5 \cdot 4 \cdot \Ccancel[cyan]{3!}}
     {\Ccancel[cyan]{3!}} = 840\)
\end{proof}

\subsection{Exercise 40}
Prove that for every integer \(n \geq 2\), \(P(n + 1, 3) = n^3 - n\).

\begin{proof}
     \(P(n+1,3) = \dps\frac{(n+1)!}{(n+1-3)!} = \frac{(n+1)!}{(n-2)!} = \frac{(n+1) \cdot n \cdot (n-1) \cdot
          \Ccancel[cyan]{(n-2)!}}{\Ccancel[cyan]{(n-2)!}}\)

     \(= (n+1)n(n-1) = (n^2-1)n = n^3-n\)
\end{proof}

\subsection{Exercise 41}
Prove that for every integer \(n \geq 2\), \(P(n + 1, 2) - P(n, 2) = 2P(n, 1)\).

\begin{proof}
     \(P(n+1,2) = \dps\frac{(n+1)!}{(n+1-2)!} = \frac{(n+1)!}{(n-1)!} = \frac{(n+1) \cdot n!}{(n-1)!} = (n+1)P(n,1)\)

     \(P(n,2) = \dps\frac{n!}{(n-2)!} = \frac{(n-1) \cdot n!}{(n-1) \cdot (n-2)!} = (n-1)\frac{n!}{(n-1)!}=(n-1)P(n,1)\)

     So \(P(n + 1, 2) - P(n, 2) = (n+1)P(n, 1) - (n-1)P(n, 1) = 2P(n,1)\).
\end{proof}

\subsection{Exercise 42}
Prove that for every integer \(n \geq 3\), \(P(n + 1, 3) - P(n, 3) = 3P(n, 2)\).

\begin{proof}
     \(P(n+1,3) = \dps\frac{(n+1)!}{(n+1-3)!} = \frac{(n+1)!}{(n-2)!} = \frac{(n+1) \cdot n!}{(n-2)!} = (n+1)P(n,2)\)

     \(P(n,3) = \dps\frac{n!}{(n-3)!} = \frac{(n-2) \cdot n!}{(n-2) \cdot (n-3)!} = (n-2)\frac{n!}{(n-2)!}=(n-2)P(n,2)\)

     So \(P(n + 1, 3) - P(n, 3) = (n+1)P(n, 2) - (n-2)P(n, 2) = 3P(n,2)\).
\end{proof}

\subsection{Exercise 43}
Prove that for every integer \(n \geq 2\), \(P(n, n) = P(n, n - 1)\).

\begin{proof}
     \(P(n,n)= \dps\frac{n!}{(n-n)!} = \frac{n!}{0!} = \frac{n!}{1} = \frac{n!}{1!} = \frac{n!}{(n-(n-1))!} = P(n,n-1)\)
\end{proof}

\subsection{Exercise 44}
Prove Theorem 9.2.1 by mathematical induction.

\begin{proof}
     Let \(P(k)\) be the statement: ``If an operation consists of $k$ steps and the first step can be performed in $n_1$
     ways, the second step can be performed in $n_2$ ways {\it [regardless of how the first step was performed]},
     \(\ldots\), the $k$th step can be performed in $n_k$ ways {\it [regardless of how the preceding steps were
                    performed]}, then the entire operation can be performed in \(n_1n_2 \cdots n_k\) ways.''

     {\bf Show that \(P(1)\) is true:} Assume an operation consists of 1 step, and the first step can be performed in
     \(n_1\) ways. Then the whole operation can be performed in \(n_1\) ways, so \(P(1)\) is true.

          {\bf Show that for any integer \(k \geq 1\) if \(P(k)\) is true then \(P(k+1)\) is true:} Assume an operation consists
     of $k+1$ steps, and assume steps \(1, 2, 3, \ldots, k+1\) can be performed in \(n_1, n_2, \ldots, n_{k+1}\) ways,
     respectively, regardless of how preceding steps are performed. Assume that the first $k$ steps can be performed
     in \(n_1n_2 \cdots n_k\) ways. {\it [This is the inductive hypothesis.]}

     Since the $k+1$st step can be performed in \(n_{k+1}\) ways regardless of how the preceding $k$ steps were performed,
     for each way of performing the preceding $k$ steps, there are \(n_{k+1}\) ways to perform step $k+1$. Thus there are
     \[
          \underbrace{n_{k+1} + n_{k+1} + \cdots + n_{k+1}}_{n_1n_2 \cdots n_k \text{ times}}
     \]
     ways to perform the whole task, where the sum has \(n_1n_2 \cdots n_k\) terms. And the sum equals
     \[
          n_{k+1} \cdot (n_1n_2 \cdots n_k) = n_1n_2 \cdots n_k n_{k+1}
     \]
     {\it [as was to be shown.]}
\end{proof}

\subsection{Exercise 45}
Prove Theorem 9.2.2 by mathematical induction.

\begin{proof}
     Let \(P(n)\) be the statement ``For any integer $n$ with \(n \geq 1\), the number of permutations of a set with $n$
     elements is \(n!\).''

     {\bf Show that \(P(1)\) is true:} There is only 1 permutation of a set with 1 element, and \(1! = 1\), so
     \(P(1)\) is true.

          {\bf Show that for any integer \(k \geq 1\) if \(P(k)\) is true then \(P(k+1)\) is true:} Assume \(k \geq 1\) and
     assume that the number of permutations of any set with $k$ elements is $k!$. {\it [This is the inductive hypothesis.]}
     Assume \(A = \{a_1, \ldots, a_{k+1}\}\) is a set with $k+1$ elements.

     Consider the subset \(A' = \{a_1, \ldots, a_k\}\) of $A$ with $k$ elements. By the inductive hypothesis $A'$ has
     $k!$ permutations. Now we need to find a way to write the permutations of $A$ in terms of the permutations of $A'$.

     Every permutation of $A$ can be thought of as taking a permutation of $A$ and then inserting \(a_{k+1}\) into a
     position in that permutation. For example, if \(a_3, a_1, a_{k-2}, \ldots, a_k, a_2, a_{k-1}\) is a permutation of
     $A'$, then there are $k+1$ positions in which \(a_{k+1}\) can be inserted to obtain a permutation of $A$:
     \begin{center}
          \begin{tabular}{ccccccccccccccc}
                            & \(a_3\) &              & \(a_1\) &              & \(a_{k-2}\) &              & \(\ldots\) &              & \(a_k\) &              & \(a_2\) &              & \(a_{k-1}\) &              \\
               \(\uparrow\) &         & \(\uparrow\) &         & \(\uparrow\) &             & \(\uparrow\) & \(\ldots\) & \(\uparrow\) &         & \(\uparrow\) &         & \(\uparrow\) &             & \(\uparrow\) \\
               here         &         & here         &         & here         &             & here         & \(\ldots\) & here         &         & here         &         & here         &             & here         \\
          \end{tabular}
     \end{center}
     So for each permutation of $A'$, there are $k+1$ permutations of $A$. Therefore the number of permutations
     of $A$ is \(k! \cdot (k+1) = (k+1)!\).
\end{proof}

\subsection{Exercise 46}
Prove Theorem 9.2.3 by mathematical induction.

\begin{proof}
     Let \(Q(n)\) be the statement: ``for every integer $r$ with \(1 \leq r \leq n\),
     \(P(n,r) = n(n - 1)(n - 2) \cdots (n - r + 1)\).''

     {\bf Show that \(Q(1)\) is true:} There is only one 1-permutation of a set of 1 element, and when \(n=r=1\)
     the formula \(n(n - 1)(n - 2) \cdots (n - r + 1)\) is equal to 1. Therefore \(Q(1)\) is true. \\
     {\bf Show that for any integer \(k \geq 1\) if \(Q(k)\) is true then \(Q(k+1)\) is true:}

     Assume \(k \geq 1\) and assume for every integer $r$ with \(1 \leq r \leq k\), \(P(k,r) = k(k - 1)(k - 2) \cdots (k-r+1)\).
          {\it [This is the inductive hypothesis.]}

     We want to show that for every integer $r$ with \(1 \leq r \leq k+1\), \(P(k+1,r) = (k+1)k(k-1) \cdots (k-r + 2)\).

     When \(r = 1\), there are $k+1$ ways to choose 1 element from a set of $k+1$ elements. So \(P(k+1,1) = k+1\), and
     the formula \((k+1)k(k-1) \cdots (k-r + 2)\) for \(r = 1\) becomes \((k+1)\), therefore the formula holds.

     Now assume \(2 \leq r \leq k+1\). We can think of choosing $r$ elements from a set of $k+1$ elements as follows: first
     choosing 1 element out of $k+1$ elements, then choosing $r-1$ elements from the remaining $k$ elements.

     There are $k+1$ ways to choose the first element. Then by the inductive hypothesis there are \(P(k, r-1) = k(k - 1)
     (k - 2) \cdots (k - (r-1) + 1) = k(k - 1)(k - 2) \cdots (k - r + 2)\) ways to choose $r-1$ elements from the remaining
     $k$ elements. Then by the multiplication rule, \(P(k+1,r) = (k+1) \cdot k(k - 1)(k - 2) \cdots (k - r + 2)\),
     {\it [as was to be shown.]}
\end{proof}

\subsection{Exercise 47}
A permutation on a set can be regarded as a function from the set to itself. For instance, one permutation of
\(\{1, 2, 3, 4\}\) is 2341. It can be identified with the function that sends each position number to the number
occupying that position. Since position 1 is occupied by 2, 1 is sent to 2 or \(1 \to 2\); since position 2 is occupied
by 3, 2 is sent to 3 or \(2 \to 3\); and so forth. The entire permutation can be written using arrows as follows:

\begin{center}
     \begin{tabular}{cccc}
          1     & 2     & 3     & 4     \\
          \cyda & \cyda & \cyda & \cyda \\
          2     & 3     & 4     & 1     \\
     \end{tabular}
\end{center}

\subsubsection{(a)}
Use arrows to write each of the six permutations of \(\{1, 2, 3\}\).

\begin{proof}
     \begin{center}
          \begin{tabular}{ccc|ccc|ccc|ccc|ccc|ccc}
               1     & 2     & 3     & 1     & 2     & 3     & 1     & 2     & 3     & 1     & 2     & 3     & 1     & 2     & 3 & 1 & 2 & 3 \\
               \cyda & \cyda & \cyda & \cyda & \cyda & \cyda & \cyda & \cyda & \cyda & \cyda & \cyda & \cyda & \cyda & \cyda &
               \cyda & \cyda & \cyda & \cyda                                                                                                 \\
               1     & 2     & 3     & 1     & 3     & 2     & 2     & 1     & 3     & 2     & 3     & 1     & 3     & 1     & 2 & 3 & 2 & 1
          \end{tabular}
     \end{center}
\end{proof}

\subsubsection{(b)}
Use arrows to write each of the permutations of \(\{1, 2, 3, 4\}\) that keep 2 and 4 fixed.

\begin{proof}
     \begin{center}
          \begin{tabular}{cccc|cccc}
               1     & 2     & 3     & 4     & 1     & 2     & 3     & 4     \\
               \cyda & \cyda & \cyda & \cyda & \cyda & \cyda & \cyda & \cyda \\
               1     & 2     & 3     & 4     & 3     & 2     & 1     & 4
          \end{tabular}
     \end{center}
\end{proof}

\subsubsection{(c)}
Which permutations of \(\{1, 2, 3\}\) keep no elements fixed?

\begin{proof}
     \begin{center}
          \begin{tabular}{ccc|ccc}
               1     & 2     & 3     & 1     & 2     & 3     \\
               \cyda & \cyda & \cyda & \cyda & \cyda & \cyda \\
               2     & 3     & 1     & 3     & 1     & 2     \\
          \end{tabular}
     \end{center}
\end{proof}

\subsubsection{(d)}
Use arrows to write all permutations of \(\{1, 2, 3, 4\}\) that keep no elements fixed.

\begin{proof}
     \begin{center}
          \begin{tabular}{cccc|cccc|cccc|cccc|cccc|cccc}
               1     & 2     & 3     & 4     & 1     & 2     & 3     & 4     & 1     & 2     & 3     & 4     & 1     & 2     & 3 & 4 & 1 & 2 & 3 & 4 & 1 & 2 & 3 & 4 \\
               \cyda & \cyda & \cyda & \cyda & \cyda & \cyda & \cyda & \cyda & \cyda & \cyda & \cyda & \cyda & \cyda & \cyda &
               \cyda & \cyda & \cyda & \cyda & \cyda & \cyda & \cyda & \cyda & \cyda & \cyda                                                                         \\
               2     & 1     & 4     & 3     & 2     & 3     & 4     & 1     & 2     & 4     & 1     & 3     & 3     & 1     & 4 & 2 & 3 & 4 & 1 & 2 & 3 & 4 & 2 & 1
          \end{tabular}
     \end{center}
     \begin{center}
          \begin{tabular}{cccc|cccc|cccc}
               1     & 2     & 3     & 4     & 1     & 2     & 3     & 4     & 1     & 2     & 3     & 4     \\
               \cyda & \cyda & \cyda & \cyda & \cyda & \cyda & \cyda & \cyda & \cyda & \cyda & \cyda & \cyda \\
               4     & 1     & 2     & 3     & 4     & 3     & 1     & 2     & 4     & 3     & 2     & 1
          \end{tabular}
     \end{center}
\end{proof}

\section{Exercise Set 9.3}

\subsection{Exercise 1}
\subsubsection{(a)}
How many bit strings consist of from one through four digits? (Strings of different lengths are considered
distinct. Thus 10 and 0010 are distinct strings.)

\begin{proof}
     Think of creating a bit string with $n$ bits as an $n$-step process where a general step $k$ is to place either a 0 or
     a 1 in the $k$th position. Since there are two ways to do this for each position, by the multiplication rule, the
     number of bit strings of length $k$ is $2^k$. Now the set of all bit strings consisting of from 1 through 4 bits can
     be broken into four disjoint subsets:

     \begin{figure}[ht!]
          \centering
          \includegraphics[scale=0.45]{../images/9.3.1.a.png}
     \end{figure}

     Applying the addition rule to the figure shows that there are \(2 + 2^2 + 2^3 + 2^4 = 30\) bit strings consisting of
     from one through four bits.
\end{proof}

\subsubsection{(b)}
How many bit strings consist of from five through eight digits?

\begin{proof}
     By reasoning similar to that of part (a), there are \(2^5 + 2^6 + 2^7 + 2^8 = 480\) bit strings of from five through
     eight bits.
\end{proof}

\subsection{Exercise 2}
\subsubsection{(a)}
How many strings of hexadecimal digits consist of from one through three digits? (Recall that hexadecimal numbers are
constructed using the 16 digits 0, 1, 2, 3, 4, 5, 6, 7, 8, 9, A, B, C, D, E, F.)

\begin{proof}
     \(16^1 + 16^2 + 16^3 = 16 + 256 + 4096 = 4368\)
\end{proof}

\subsubsection{(b)}
How many strings of hexadecimal digits consist of from two through five digits?

\begin{proof}
     \(16^2 + 16^3 + 16^4 + 16^5 = 256 + 4096 + 65536 + 1048576 = 1118464\)
\end{proof}

\subsection{Exercise 3}
\subsubsection{(a)}
How many integers from 1 through 999 do not have any repeated digits?

\begin{proof}
     The set of integers from 1 through 999 with no repeated digit can be broken into three disjoint subsets: those from
     1 through 9, those from 1 through 99, and those from 100 through 999. Now constructing an integer from 100 through
     999 with no repeated digit can be thought of as a three-step process.

          {\cy Step 1:} Choose a digit for the left-most position (where there are 9 choices because 0 cannot be chosen).

          {\cy Step 2:} Choose a digit for the middle position (where there are also 9 choices because the digit in the left-most
     position cannot be reused but 0 can be used).

          {\cy Step 3:} Choose a digit for the right-most position (where there are 8 choices because neither of the other two
     digits can be reused).

     Thus there are \(9 \cdot 9 \cdot 8\) integers from 100 through 999 with no repeated digit. Similar reasoning shows
     that there are \(9 \cdot 9\) integers from 10 through 99 with no repeated digit. Finally, there are clearly 9
     integers from 1 through 9 with no repeated digit. Hence, by the addition rule, the number of integers from 1 through
     999 with no repeated digit is \(9 + 9 \cdot 9 + 9 \cdot 9 \cdot 8 = 738\).
\end{proof}

\subsubsection{(b)}
How many integers from 1 through 999 have at least one repeated digit?

\begin{proof}
     Let

     $x$ = number of integers from 1 through 999 with at least one repeated digit,

     $y$ = total number of integers from 1 through 999,

     $z$ = number of integers from 1 through 999 with no repeated digits.

     Then \(x = y - z = 999 - 738 = 261\)
\end{proof}

\subsubsection{(c)}
What is the probability that an integer chosen at random from 1 through 999 has at least one repeated digit?

\begin{proof}
     The probability that an integer chosen at random has at least one repeated digit is \(261/999 \approx 26.1\%\).
\end{proof}

\subsection{Exercise 4}
How many arrangements in a row of no more than three letters can be formed using the letters of the word NETWORK
(with no repetitions allowed)?

\begin{proof}
     Use the multiplication rule to count the elements in each of the three sets containing 1, 2, and 3 letters, respectively.
     Then, because these sets are disjoint, use the addition rule to compute the total number of elements in the three sets
     taken together.

     Applying the addition rule to the figure below shows that there are \(7 + 7 \cdot 6 + 7 \cdot 6 \cdot 5 = 259\)
     arrangements of three letters of the word NETWORK if repetition of letters is not permitted.

     \begin{figure}[ht!]
          \centering
          \includegraphics[scale=0.5]{../images/9.3.4.png}
     \end{figure}
\end{proof}

\subsection{Exercise 5}
\subsubsection{(a)}
How many five-digit integers (integers from 10,000 through 99,999) are divisible by 5?

\begin{proof}
     Let

     $x$ = number of integers from 10,000 through 99,999 that are divisible by 5,

     $y$ = number of integers from 1 through 99,999 that are divisible by 5 = \(100,000 / 5 - 1 = 20,000 - 1 = 19,999\),

     $z$ = number of integers from 1 through 9999 that are divisible by 5 = \(10,000 / 5 - 1 = 2000 - 1 = 1999\).

     Then \(x = y - z = 19,999 - 1999 = 18,000\).
\end{proof}

\subsubsection{(b)}
What is the probability that a five-digit integer chosen at random is divisible by 5?

\begin{proof}
     There are \(99,999 - 10,000 + 1 = 90,000\) five-digit integers. So: \(18,000 / 90,000 = 1/5 = 20\%\).
\end{proof}

\subsection{Exercise 6}
In a certain state, all license plates consist of from four to six symbols chosen from the 26 uppercase letters of the
Roman alphabet together with the ten digits 0–9.

\subsubsection{(a)}
How many license plates are possible if repetition of symbols is allowed?

\begin{proof}
     In this exercise the 26 letters in the alphabet plus the 10 digits give a total of 36 symbols that can be used on a
     license plate.

     Imagine constructing a license plate with 4 symbols as a four-step process: {\cy step 1} is to fill in the first symbol, {\cy step 2} is to fill in the second symbol, {\cy step 3} is to fill in the third symbol, and {\cy step 4} is to fill in the fourth symbol. Because any one of the 36 symbols can be used in each step, by the multiplication rule, the number of license plates that use four symbols is \(36^4\). Similarly, the number that use 5 symbols is \(36^5\), and the number that use six symbols is \(36^6\). Thus because license plates can have anywhere from 4 to 6 symbols, the total number of plates with repeated symbols allowed is \(36^4 + 36^5 + 36^6 = 2,238,928,128\).
\end{proof}

\subsubsection{(b)}
How many license plates do not contain any repeated symbols?

\begin{proof}
     When repetition is not allowed, the number of license plates that use four symbols is \(36 \cdot 35 \cdot 34
     \cdot 33\). The reason is that in the second step the symbol used in the first step cannot be used, so there are
     only 35 choices for the second step. In the third step, neither of the symbols used in the first two steps can be
     used, and so there are only 34 choices for the third step. And in the fourth step, none of the symbols used in the
     first three steps can be used, and so there are only 33 choices for the fourth step. Similarly, the number of
     license plates that use 5 symbols is \(36 \cdot 35 \cdot 34 \cdot 33 \cdot 32\), and the number that use six symbols is
     \(36 \cdot 35 \cdot 34 \cdot 33 \cdot 32 \cdot 31\). Thus the total number of license plates is \(36 \cdot 35 \cdot
     34 \cdot 33 + 36 \cdot 35 \cdot 34 \cdot 33 \cdot 32 + 36 \cdot 35 \cdot 34 \cdot 33 \cdot 32 \cdot 31 =
     1,449,063,000\).
\end{proof}

\subsubsection{(c)}
How many license plates have at least one repeated symbol?

\begin{proof}
     Consider two sets: the set of plates with repetition not allowed and the set of plates that have a repeated symbol.
     Note that these two sets have no elements in common, and that since every license plate either has a repeated symbol
     or does not have a repeated symbol, every license plate considered in part (a) is in one of the two sets. In other
     words, the set of all license plates with repetition allowed is composed of two disjoint subsets: the set of
     plates with repetition not allowed and the set of plates that have a repeated symbol. Thus, by the difference rule,
     the number of license plates with a repeated symbol is the difference between the number of plates with repetition
     allowed minus the number of plates with repetition not allowed: \(2,238,928,128 - 1,449,063,000 = 789,865,128\).
\end{proof}

\subsubsection{(d)}
What is the probability that a license plate chosen at random has at least one repeated symbol?

\begin{proof}
     The probability that a license plate chosen at random has at least one repeated symbol is
     \(789,865,128 / 2,238,928,128 \approx 35.3\%\).
\end{proof}

\subsection{Exercise 7}
At a certain company, passwords must be from 3–5 symbols long and composed from the 26 uppercase letters of the
Roman alphabet, the ten digits 0–9, and the 14 symbols !, @, \#, \$, \%, \^\,, \&, *, (, ), -, +, \{, and \}.

\subsubsection{(a)}
How many passwords are possible if repetition of symbols is allowed?

\begin{proof}
     The 26 letters in the alphabet plus the 10 digits plus the 14 special characters give a total of 50 symbols that can
     be used. By the multiplication rule, the number of passwords with 3, 4, and 5 symbols is \(50^3, 50^4\), and
     \(50^5\). Since the sets consisting of these passwords are disjoint, by the addition rule, the number of passwords is

     \(50^3 + 50^4 + 50^5 = 318,875,000\).
\end{proof}

\subsubsection{(b)}
How many passwords contain no repeated symbols?

\begin{proof}
     There are \(50 \cdot 49 \cdot 48 = 117600\) passwords of length 3 with no repeated symbols.

     There are \(50 \cdot 49 \cdot 48 \cdot 47 = 5527200\) passwords of length 4 with no repeated symbols.

     There are \(50 \cdot 49 \cdot 48 \cdot 47 \cdot 46 = 254251200\) passwords of length 5 with no repeated symbols.

     So by the addition rule, the total is \(254251200 + 5527200 + 117600 = 259,896,000\).
\end{proof}

\subsubsection{(c)}
How many passwords have at least one repeated symbol?

\begin{proof}
     \(318,875,000 - 259,896,000 = 58,979,000\)
\end{proof}

\subsubsection{(d)}
What is the probability that a password chosen at random has at least one repeated symbol?

\begin{proof}
     \(58,979,000 / 318,875,000 \approx 18.5\% \)
\end{proof}

\subsection{Exercise 8}
In a certain country license plates consist of zero or one digit followed by four or five uppercase letters from the
Roman alphabet.

\subsubsection{(a)}
How many different license plates can the country produce?

{\it Hint:} One approach is to divide the license plates into four groups depending on the number of digits and
letters they contain. Another approach is to consider creating a license plate as a two-step process: step 1:
either choose one digit or do not choose a digit; and step 2: choose 4 or 5 letters.

\begin{proof}
     There are four groups: 1. No digit, 4 letters, 2. One digit, 4 letters, 3. No digit, 5 letters, 4. One digit, 5
     letters.

     First group has \(26^4\) possibilities. Second group has \(10 \cdot 26^4\) possibilities.

     Third group has \(26^5\) possibilities. Fourth group has \(10 \cdot 26^5\) possibilities.

     In total: \(26^4 + 10 \cdot 26^4 + 26^5 + 10 \cdot 26^5 = 11(26^4 + 26^5) = 11 \cdot 26^4 (1 + 26) = 11 \cdot 26^4
     \cdot 27 = 135,721,872\)
\end{proof}

\subsubsection{(b)}
How many license plates have no repeated letter?

\begin{proof}
     Choosing 4 non-repeated letters, no digit: \(26 \cdot 25 \cdot 24 \cdot 23 = 358,800\), with one digit: \(10 \cdot
     358,800 = 3,588,000\)

     Choosing 5 non-repeated letters, no digit: \(26 \cdot 25 \cdot 24 \cdot 23 \cdot 22 = 7,893,600\), with one digit:
     \(10 \cdot 7,893,600 = 78,936,000\)

     Total: \(358,800 + 3,588,000 + 7,893,600 + 78,936,000 = 90,776,400\)
\end{proof}

\subsubsection{(c)}
How many license plates have at least one repeated letter?

\begin{proof}
     \(135,721,872 - 90,776,400 = 44,945,472\)
\end{proof}

\subsubsection{(d)}
What is the probability that a license plate has a repeated letter?

\begin{proof}
     \(44,945,472 / 135,721,872 \approx 33.11\%\)
\end{proof}

\subsection{Exercise 9}
\subsubsection{(a)}
Consider the following algorithm segment:

\begin{tabbing}
     {\bf for} \= \(i \coloneqq 1\) {\bf to} 4 \\
     \> {\bf for} \= \(j \coloneqq 1\) {\bf to} $i$ \\
     \>           \> {\it [Statements in body of inner loop.]} \\
     \>           \> {\it [None contain branching statements that lead outside the loop.]} \\
     \> {\bf next} $j$ \\
     {\bf next} $i$
\end{tabbing}

How many times will the inner loop be iterated when the algorithm is implemented and run?

\begin{proof}
     Each column of the table below corresponds to a pair of values of $i$ and $j$ for which the inner loop will be iterated.

     \begin{figure}[ht!]
          \centering
          \includegraphics[scale=0.5]{../images/9.3.9.a.png}
     \end{figure}

     Since there are \(1 + 2 + 3 + 4 = 10\) columns, the inner loop will be iterated ten times.
\end{proof}

\subsubsection{(b)}
Let $n$ be a positive integer, and consider the following algorithm segment:

\begin{tabbing}
     {\bf for} \= \(i \coloneqq 1\) {\bf to} $n$ \\
     \> {\bf for} \= \(j \coloneqq 1\) {\bf to} $i$ \\
     \>           \> {\it [Statements in body of inner loop.]} \\
     \>           \> {\it [None contain branching statements that lead outside the loop.]} \\
     \> {\bf next} $j$ \\
     {\bf next} $i$
\end{tabbing}

How many times will the inner loop be iterated when the algorithm is implemented and run?

\begin{proof}
     \(1 + 2 + \cdots + n = n(n+1)/2\)
\end{proof}

\subsection{Exercise 10}
A calculator has an eight-digit display and a decimal point that is located at the extreme right of the number
displayed, or at the extreme left, or between any pair of digits. The calculator can also display a minus sign at the
extreme left of the number. How many distinct numbers can the calculator display? (Note that certain numbers are
equal, such as 1.9, 1.90, and 01.900, and should, therefore, not be counted twice.)

\begin{proof}
     Let's solve this problem for a 1-digit display, then 2-digit, and so on. Also, let's first solve the problem only
     for nonnegative numbers.

          {\bf 1-digit display:} First let's consider nonnegative numbers. A number $x$ can be in the range \(0 \leq x < 1\)
     (if the decimal point is on the left) or in the range \(1 \leq x < 10\) (if the decimal point is on the right).

     For the first range there are 10 possibilities: \(.0, .1, .2, .3, .4, .5, .6, .7, .8, .9\);

     for the second range we cannot allow the digit to be 0, otherwise \(x < 1\), so there are only 9 possibilities:
     \(1., 2., 3., 4., 5., 6., 7., 8., 9.\);

     so we have \(10^1 + (10^1 - 1) = 19\) nonnegative numbers.

          {\bf 2-digit display:} First let's consider nonnegative numbers. A number $x$ can be in the range \(0 \leq x < 1\)
     (if the decimal point is on the left) or in the range \(1 \leq x < 10\) (if the decimal point is in the middle) or in
     the range \(10 \leq x < 100\) (if the decimal point is on the right).

     For the first range there are \(10^2\) possibilities: \(.00, .01, \ldots, .09, .10, .11, \ldots, .99\);

     for the second range we cannot allow the left-most digit to be 0, otherwise \(x < 1\), so there are only 90
     possibilities: \(1.0, 1.1, 1.2, \ldots, 9.9\);

     for the third range there are also 90 possibilities: \(10., 11., 12., \ldots, 99.\) because, similarly, the first digit
     cannot be zero; so we have \(10^2 + 10^1 \cdot (10^1 - 1) + 10^1 \cdot (10^1 - 1) = 100 + 90 + 90 = 280\) nonnegative
     numbers.

          {\bf 3-digit display:} A number $x$ can be in one of the ranges \(0 \leq x < 1\) or \(1 \leq x < 10\) or \(10 \leq x
     < 100\) or \(100 \leq x < 1000\).

     Similar to the above, we have \(10^3 + 10^2 \cdot (10^1 - 1) + 10^2 \cdot (10^1 - 1) + 10^2 \cdot (10^1 - 1) =
     1000 + 900 + 900 + 900 = 3700\) nonnegative numbers.

     So the general formula for $n$ digits is: \(10^n + n \cdot 10^{n-1} \cdot 9\). For an $n=8$-digit display, there are
     \(10^8 + 8 \cdot 10^7 \cdot 9 = 100,000,000 + 10,000,000 \cdot 72 = 820,000,000\) unique nonnegative numbers.

          {\bf Conclusion:} 820,000,000 nonnegative and 820,000,000 nonpositive numbers (obtained by putting the negative sign
     in front of them) gives 1,640,000,000 numbers. However 0 and \(-0\) are the same, so it's 1,639,999,999 numbers.
\end{proof}

\subsection{Exercise 11}
\subsubsection{(a)}
How many ways can the letters of the word QUICK be arranged in a row?

\begin{proof}
     The answer is the number of permutations of the five letters in QUICK, which equals \(5! = 120\).
\end{proof}

\subsubsection{(b)}
How many ways can the letters of the word QUICK be arranged in a row if the Q and the U must remain next to each other
in the order QU?

\begin{proof}
     Because \boxed{QU} (in order) is to be considered as a single unit, the answer is the number of permutations of
     the four symbols \boxed{QU}, I, C, K. This is \(4! = 24\).
\end{proof}

\subsubsection{(c)}
How many ways can the letters of the word QUICK be arranged in a row if the letters QU must remain together but may be
in either the order QU or the order UQ?

\begin{proof}
     By part (b), there are 4! arrangements of \boxed{QU}, I, C, K. Similarly, there are 4! arrangements of \boxed{QU}, I,
     C, K. Therefore, by the addition rule, there are \(4! + 4! = 48\) arrangements in all.
\end{proof}

\subsection{Exercise 12}
\subsubsection{(a)}
How many ways can the letters of the word THEORY be arranged in a row?

\begin{proof}
     \(6! = 720\)
\end{proof}

\subsubsection{(b)}
How many ways can the letters of the word THEORY be arranged in a row if T and H must remain next to each other
as either TH or HT?

\begin{proof}
     \(5! + 5! = 240\)
\end{proof}

\subsection{Exercise 13}
A group of eight people are attending the movies together.

\subsubsection{(a)}
Two of the eight insist on sitting side-by-side. In how many ways can the eight be seated together in a row?

\begin{proof}
     Let $x$ = the number of ways to arrange \boxed{AB}, C, D, E, F, G, H,

     $y$ = the number of ways to arrange \boxed{BA}, C, D, E, F, G, H,

     $z$ = the number of ways to place eight people in a row keeping $A$ and $B$ together.

     Then \(z = x+y = 7!+7! = 5040+5040 = 10080\).
\end{proof}

\subsubsection{(b)}
Two of the people do not like each other and do not want to sit side-by-side. Now how many ways can the eight be seated
together in a row?

\begin{proof}
     Let $x$ = the number of ways to place eight people in a row,

     $y$ = the number of ways to place eight people in a row keeping $A$ and $B$ together,

     $z$ = the number of ways to place eight people in a row keeping $A$ and $B$ apart.

     By part (a) $y = 10080$. So \(z = x-y = 8!-10080 = 40320-10080 = 30240\).
\end{proof}

\subsection{Exercise 14}
An early compiler recognized variable names according to the following rules: Numeric variable names had to begin with a
letter, and then the letter could be followed by another letter or a digit or by nothing at all. String variable names
had to begin with the symbol \$ followed by a letter, which could then be followed by another letter or a digit or by
nothing at all. How many distinct variable names were recognized by this compiler?

\begin{proof}
     the number of variable names = the number of numeric variable names + the number of string variable names =
     \((26 + 26 \cdot 36) + (26 + 26 \cdot 36) = 1924\).
\end{proof}

\subsection{Exercise 15}
Identifiers in a certain database language must begin with a letter, and then the letter may be followed by other
characters, which can be letters, digits, or underscores (\_). However, 82 keywords (all consisting of 15 or fewer
characters) are reserved and cannot be used as identifiers. How many identifiers with 30 or fewer characters are
possible? (Write the answer using summation notation and evaluate it using a formula from Section 5.2.)

\begin{proof}
     For the first character there are 26 choices (letters).

     For other characters there are 37 choices (letters, digits, or underscore).

     So, there are \(26\) length 1 identifiers, \(26 \cdot 37\) length-2 identifiers, ..., and \(26 \cdot 37^{29}\)
     length-3 identifiers. So the number of identifiers with 30 or fewer characters is: \\
     \(26+26 \cdot 37 + \cdots + 26 \cdot 37^{29} = 26(1+37+37^2+\cdots+37^{29}) = 26 \cdot \frac{37^{30}-1}{37-1}\)
\end{proof}

\subsection{Exercise 16}
\subsubsection{(a)}
If any seven digits could be used to form a telephone number, how many seven-digit telephone numbers would not
have any repeated digits?

\begin{proof}
     \(10 \cdot 9 \cdot 8 \cdot 7 \cdot 6 \cdot 5 \cdot 4 = 604,800\)
\end{proof}

\subsubsection{(b)}
How many seven-digit telephone numbers would have at least one repeated digit?

\begin{proof}
     Let $x$ = the number of phone numbers with at least one repeated digit,

     $y$ = the total number of phone numbers,

     $z$ = the number of phone numbers with no repeated digits,

     then \(x = y - z = 10^7 - 604,800 = 9,395,200\).
\end{proof}

\subsubsection{(c)}
What is the probability that a randomly chosen seven-digit telephone number would have at least one repeated digit?

\begin{proof}
     \(9,395,200 / 10,000,000 \approx 93.95\%\)
\end{proof}

\subsection{Exercise 17}
\subsubsection{(a)}
How many strings of four hexadecimal digits do not have any repeated digits?

\begin{proof}
     \(16 \cdot 15 \cdot 14 \cdot 13 = 43680\)
\end{proof}

\subsubsection{(b)}
How many strings of four hexadecimal digits have at least one repeated digit?

\begin{proof}
     \(16^4 - 43680 = 65536 - 43680 = 21856\)
\end{proof}

\subsubsection{(c)}
What is the probability that a randomly chosen string of four hexadecimal digits has at least one repeated digit?

\begin{proof}
     \(21856 / 65536 \approx 33.34\)
\end{proof}

\subsection{Exercise 18}
Just as the difference rule gives rise to a formula for the probability of the complement of an event, so the addition
and inclusion/exclusion rules give rise to formulas for the probability of the union of mutually disjoint events and
for a general union of (not necessarily mutually exclusive) events.

\subsubsection{(a)}
Prove that for mutually disjoint events $A$ and $B$, \(P(A \cup B) = P(A) + P(B)\).

\begin{proof}
     Let $A$ and $B$ be mutually disjoint events in a sample space $S$. By the addition rule, \(N(A \cup B) = N(A) +
     N(B)\). Therefore, by the equally likely probability formula,
     \[
          P(A \cup B)=\frac{N(A\cup B)}{N(S)}=\frac{N(A)+N(B)}{N(S)}= \frac{N(A)}{N(S)} + \frac{N(B)}{N(S)} = P(A) + P(B).
     \]
\end{proof}

\subsubsection{(b)}
Prove that for any events $A$ and $B$, \(P(A \cup B) = P(A) + P(B) - P(A \cap B)\).

\begin{proof}
     Let $A$ and $B$ be any events in a sample space $S$. By the inclusion / exclusion rule, \(N(A \cup B) = N(A) + N(B) -
     N(A \cap B)\). Therefore, by the equally likely probability formula,
     \[
          P(A \cup B) = \frac{N(A \cup B)}{N(S)}=\frac{N(A) + N(B) - N(A \cap B)}{N(S)}= \frac{N(A)}{N(S)} + \frac{N(B)}{N(S)} -
          \frac{N(A \cap B)}{N(S)}
     \]
     \(= P(A) + P(B) - P(A \cap B)\).
\end{proof}

\subsection{Exercise 19}
A combination lock requires three selections of numbers, each from 1 through 39. Suppose the lock is constructed in
such a way that no number may be used twice in a row but the same number may occur both first and third. For
example, 20 13 20 would be acceptable, but 20 20 13 would not. How many different combinations are possible?

\begin{proof}
     There are 39 choices for the first selection. Then 38 choices for the second selection. Now there are 38 choices
     for the third selection: it cannot be the same as the second selection, but it can be the same as the first. So
     by the multiplication rule \(39 \cdot 38 \cdot 38\) possible combinations.
\end{proof}

\subsection{Exercise 20}
\subsubsection{(a)}
How many integers from 1 through 100,000 contain the digit 6 exactly once?

\begin{proof}
     Use strings of five digits to represent integers from 1 to 100,000 that contain the digit 6 exactly once. For example,
     use 00306 to represent 306. Strings of six digits are not needed because 100,000 does not contain a 6. Imagine
     constructing a five-digit string that contains exactly one 6 as a five-step operation to fill in five positions with
     five digits: \(\underset{1}{\fbl} \underset{2}{\fbl} \underset{3}{\fbl} \underset{4}{\fbl} \underset{5}{\fbl}\).

          {\cy Step 1:} Choose one of the five positions for the 6.

          {\cy Step 2:} Choose a digit for the left-most remaining position.

          {\cy Step 3:} Choose a digit for the next remaining position to the right.

          {\cy Step 4:} Choose a digit for the next remaining position to the right.

          {\cy Step 5:} Choose a digit for the right-most position.

     Since there are 5 choices for step 1 (any one of the five positions) and 9 choices for each of steps 2–5 (any digit
     except 6), by the multiplication rule, the number of ways to perform this operation is \(5 \cdot 9 \cdot 9 \cdot 9
     \cdot 9 = 32,805\). Hence there are 32,805 integers from 1 to 100,000 that contain the digit 6 exactly once.
\end{proof}

\subsubsection{(b)}
How many integers from 1 through 100,000 contain the digit 6 at least once?

\begin{proof}
     Let $x$ = the number of integers from 1 through 100,000,

     $y$ = the number of integers from 1 through 100,000, that do not contain the digit 6,

     $z$ = the number of integers from 1 through 100,000, that contain the digit 6 at least once.

     Then \(z = x - y = 100,000 - (9^5 + 1) = 40950\).
\end{proof}

\subsubsection{(c)}
If an integer is chosen at random from 1 through 100,000, what is the probability that it contains two or more occurrences of the digit 6?

\begin{proof}
     Let $x$ = the number of integers from 1 through 100,000, that contain the digit 6 at least once, $y$ = the number of
     integers from 1 through 100,000, that contain the digit 6 exactly once, $z$ = the number of integers from 1 through
     100,000, that contain the digit 6 two or more times. Then \(z = x - y = 40950 - 32805 = 8145\). The probability is
     \(8145 / 100000 = 8.145\%\)
\end{proof}

\subsection{Exercise 21}
Six new employees, two of whom are married to each other, are to be assigned six desks that are lined up in a row. If
the assignment of employees to desks is made randomly, what is the probability that the married couple will have
nonadjacent desks? (Hint: The event that the couple have nonadjacent desks is the complement of the event that they
have adjacent desks.)

\begin{proof}
     There are \(6! = 720\) possible assignments. The number of assignments where the married couple are adjacent is
     \(2 \cdot 5! = 240\) since we can think of the adjacent married couple as one, and with the remaining 4 employees,
     there are 5 people to assign desks, but we can also switch the married couple's positions. So \(720-240 = 480\)
     assignments have the married couple in nonadjacent desks. The probability is \(480/720 = 2/3\).
\end{proof}

\subsection{Exercise 22}
\subsubsection{(a)}
Consider strings of length $n$ over the set \(\{a, b, c, d\}\). How many such strings contain at least one pair of
adjacent characters that are the same?

\begin{proof}
     Let $x$ = the total number of all strings = \(4^n\), $y$ = the number of strings that have no pairs of adjacent characters
     that are the same = \(4 \cdot \underbrace{3 \cdot \cdots \cdot 3}_{n-1 \text{ times}} = 4 \cdot 3^{n-1}\), $z$ = the number
     of strings contain at least one pair of adjacent characters that are the same = \(x - y = 4^n - 4 \cdot 3^{n-1}\).
\end{proof}

\subsubsection{(b)}
If a string of length ten over \(\{a, b, c, d\}\) is chosen at random, what is the probability that it contains at
least one pair of adjacent characters that are the same?

\begin{proof}
     When $n=10$, the number of all strings is \(4^{10} = 1048576\) and the number of strings that contain at least
     one pair of adjacent characters that are the same is \(4^{10} - 4 \cdot 3^9 = 1048576 - 4 \cdot 19683 = 969844\)
     so the probability is \(969844 / 1048576 \approx 92.49\%\).
\end{proof}

\subsection{Exercise 23}
\subsubsection{(a)}
How many integers from 1 through 1,000 are multiples of 4 or multiples of 7?

\begin{proof}
     Let $A$ = the set of integers that are multiples of 4 and $B$ = the set of integers that are multiples of 7. Then
     \(A \cap B\) = the set of integers that are multiples of 28. Now \(N(A) = 250\) because
     \begin{tabular}{ccccccccccc}
          1 & 2 & 3 & 4                & 5 & 6 & 7 & 8                & \(\ldots\) & 999 & 1000             \\
            &   &   & \(\updownarrow\) &   &   &   & \(\updownarrow\) &            &     & \(\updownarrow\) \\
            &   &   & \(4 \cdot 1\)    &   &   &   & \(4 \cdot 2\)    & \(\ldots\) &     & \(4 \cdot 250\)  \\
     \end{tabular}

     or equivalently, because \(1000 = 4 \cdot 250\). Also \(N(B) = 142\) because


     \begin{tabular}{cccccccccccccc}
          1 & 2 & 3 & 4 & 5 & 6 & 7                & \(\ldots\) & 14               & \(\ldots\) & 994              & 995 & \(\ldots\) & 1000 \\
            &   &   &   &   &   & \(\updownarrow\) &            & \(\updownarrow\) &            & \(\updownarrow\) &     &                   \\
            &   &   &   &   &   & \(7 \cdot 1\)    &            & \(7 \cdot 2\)    &            & \(7 \cdot 142\)  &     &                   \\
     \end{tabular}

     or equivalently because \(1000 = 7 \cdot 142 + 6\). And \(N(A \cap B) = 35\) because

     \begin{tabular}{ccccccccccc}
          1 & 2 & 3 & \(\ldots\) & 28               & \(\ldots\) & 56               & \(\ldots\) & 980              & \(\ldots\) & 1000 \\
            &   &   &            & \(\updownarrow\) &            & \(\updownarrow\) &            & \(\updownarrow\) &            &      \\
            &   &   &            & \(28 \cdot 1\)   &            & \(28 \cdot 2\)   &            & \(28 \cdot 35\)  &            &      \\
     \end{tabular}

     or equivalently because \(1000 = 28 \cdot 35 + 20\). So \(N(A \cup B) = 250 + 142 - 35 = 357\).
\end{proof}

\subsubsection{(b)}
Suppose an integer from 1 through 1,000 is chosen at random. Use the result of part (a) to find the probability that the integer is a multiple of 4 or a multiple of 7.

\begin{proof}
     \(357/1000 = 35.7\%\)
\end{proof}

\subsubsection{(c)}
How many integers from 1 through 1,000 are neither multiples of 4 nor multiples of 7?

\begin{proof}
     \(1000-357 = 643\)
\end{proof}

\subsection{Exercise 24}
\subsubsection{(a)}
How many integers from 1 through 1,000 are multiples of 2 or multiples of 9?

\begin{proof}
     \(\floor{1000 / 2} = 500\) multiples of 2, \(\floor{1000 / 9} = 111\) multiples of 9, and \(\floor{1000 / 18} = 55\)
     multiples of 18 (which are double-counted). So: \(500 + 111 - 55 = 556\).
\end{proof}

\subsubsection{(b)}
Suppose an integer from 1 through 1,000 is chosen at random. Use the result of part (a) to find the probability that the integer is a multiple of 2 or a multiple of 9.

\begin{proof}
     \(556/1000=55.6\%\)
\end{proof}

\subsubsection{(c)}
How many integers from 1 through 1,000 are neither multiples of 2 nor multiples of 9?

\begin{proof}
     \(1000-556=444\)
\end{proof}

\subsection{Exercise 25}
{\it Counting Strings:}

\subsubsection{(a)}
Make a list of all bit strings of lengths 0, 1, 2, 3, and 4 that do not contain the bit pattern 111.

\begin{proof}
     \(\lambda\), 0, 1, 00, 01, 10, 11, 000, 001, 010, 100, 011, 101, 110, 0000, 0001, 0010, 0011, 0100, 0101, 0110, 1000, 1001, 1010, 1011, 1100, 1101
\end{proof}

\subsubsection{(b)}
For each integer \(n \geq 0\), let \(d_n =\) the number of bit strings of length $n$ that do not contain the bit
pattern 111. Find \(d_0, d_1, d_2, d_3\), and \(d_4\).

\begin{proof}
     \(d_0 = 1, d_1 = 2, d_2 = 4, d_3 = 7, d_4 = 13\)
\end{proof}

\subsubsection{(c)}
Find a recurrence relation for \(d_0, d_1, d_2, \ldots\).

\begin{proof}
     Let $k$ be an integer with \(k \geq 3\). Any string of length $k$ that does not contain the bit pattern 111 starts
     either with a 0 or with a 1. If it starts with a 0, this can be followed by any string of \(k - 1\) bits that does
     not contain the pattern 111. There are \(d_{k-1}\) of these. If the string starts with a 1, then the first two
     bits are 10 or 11. If the first two bits are 10, then these can be followed by any string of \(k - 2\) bits that does
     not contain the pattern 111. There are \(d_{k-2}\) of these. If the string starts with a 11, then the third bit
     must be 0 (because the string does not contain 111), and these three bits can be followed by any string of \(k - 3\)
     bits that does not contain the pattern 111. There are \(d_{k-3}\) of these. Therefore, for every integer
     \(k \geq 3, d_k = d_{k-1} + d_{k-2} + d_{k-3}\).
\end{proof}

\subsubsection{(d)}
Use the results of parts (b) and (c) to find the number of bit strings of length 5 that do not contain the pattern
111.

\begin{proof}
     \(d_5 = d_4 + d_3 + d_2 = 13+7+4 = 24\)
\end{proof}

\subsection{Exercise 26}
{\it Counting Strings:} Consider the set of all strings of
$a$’s, $b$’s, and $c$’s.

\subsubsection{(a)}
Make a list of all of these strings of lengths 0, 1, 2, and 3 that do not contain the pattern $aa$.

\begin{proof}
     Length 0: \(\lambda\), Length 1: \(a, b, c\), Length 2: \(\Ccancel{aa}, ab, ac, ba, bb, bc, ca, cb, cc\),

     Length 3: \(\Ccancel{aaa}, \Ccancel{aab}, \Ccancel{aac}, aba, abb, abc, aca, acb, acc\),
     \(\Ccancel{baa}, bab, bac, bba, bbb, bbc, bca, bcb, bcc\),
     \(\Ccancel{caa}, cab, cac, cba, cbb, cbc, cca, ccb, ccc\).
\end{proof}

\subsubsection{(b)}
For each integer \(n \geq 0\), let \(s_n =\) the number of strings of $a$’s, $b$’s, and $c$’s of length $n$ that do
not contain the pattern $aa$. Find \(s_0, s_1, s_2\), and \(s_3\).

\begin{proof}
     \(s_0 = 1, s_1 = 3, s_2 = 8, s_3 = 22\)
\end{proof}

\subsubsection{(c)}
Find a recurrence relation for \(s_0, s_1, s_2, \ldots\).

\begin{proof}
     Let $s$ be a string of length $k$ that does not contain the pattern $aa$.

     If $x$ starts with $b$ or $c$ then it can be followed by any string of length $k-1$ that does not contain the
     pattern $aa$. There are \(s_{k-1}\) of those, so there are \(2s_{k-1}\) such strings which $x$ can be: \(s_{k-1}\) of
     them starting with $b$ and \(s_{k-1}\) of them starting with $c$.

     If $x$ starts with $a$ then it can be followed by a $b$ or a $c$ and then a string of length $k-2$ that does not
     contain the pattern $aa$. There are \(s_{k-2}\) of those, so there are \(2s_{k-2}\) such strings that $x$ can be:
     \(s_{k-2}\) of them starting with $ab$ and \(s_{k-2}\) of them starting with $ac$.

     So \(s_k = 2s_{k-1} + 2s_{k-2}\).
\end{proof}

\subsubsection{(d)}
Use the results of parts (b) and (c) to find the number of strings of $a$’s, $b$’s, and $c$’s of length four that do
not contain the pattern $aa$.

\begin{proof}
     \(s_4 = 2s_3 + 2s_2 = 2(22+8) = 60\).
\end{proof}

\subsubsection{(e)}
Use the technique described in Section 5.8 to find an explicit formula for \(s_0, s_1, s_2, \ldots\).

\begin{proof}
     \(s_k = 2s_{k-1} + 2s_{k-2}\) so the characteristic equation is \(t^2 - 2t-2 = 0\) which has the solutions
     \[
          \frac{2 \pm \sqrt{(-2)^2 - 4(1)(-2)}}{2} = \frac{2 \pm \sqrt{4+8}}{2} = \frac{2 \pm 2\sqrt{3}}{2} = 1 \pm \sqrt{3}
     \]
     The general solution has the form \(s_k = A(1+\sqrt{3})^k + B(1-\sqrt{3})^k\). When $k=0$ we have
     \[
          s_0 = 1 = A(1+\sqrt{3})^0 + B(1-\sqrt{3})^0 = A+B
     \]
     which gives $B = 1-A$; and when $k=1$ we have
     \[
          s_1 = 3 = A(1+\sqrt{3})^1 + B(1-\sqrt{3})^1 = A+B + \sqrt{3}(A-B)
     \]
     so we get \(3 = 1 + \sqrt{3}(A - (1-A)) = 1+\sqrt{3}(2A-1)\), solving we get \(\frac{2}{\sqrt{3}} = 2A-1\) so
     \(\frac{1}{\sqrt{3}} + \frac{1}{2} = A\). Then \(B = \frac{1}{2} - \frac{1}{\sqrt{3}}\). So
     \[
          s_n = \left(\frac{1}{2} + \frac{1}{\sqrt{3}}\right)(1+\sqrt{3})^n + \left(\frac{1}{2} - \frac{1}{\sqrt{3}}\right)
          (1-\sqrt{3})^n
     \]
\end{proof}

\subsection{Exercise 27}
For each integer \(n \geq 0\), let \(a_k\) be the number of bit strings of length $n$ that do not contain the pattern
101.

\subsubsection{(a)}
Show that \(a_k = a_{k-1} + a_{k-3} + a_{k-4} + \cdots + a_0 + 2\), for every integer \(k \geq 3\).

\begin{proof}
     Assume \(k \geq 3\) and $s$ is a string of length $k$ that does not contain the pattern 101.

     If $s$ begins with a 0 then it can be followed by a string of length $k-1$ that does not contain the pattern 101.
     There are \(a_{k-1}\) of these.

     If $s$ begins with a 1 then consider the following. If it begins with:

     100 then it can be followed by a string of length $k-3$ that does not contain the pattern 101, and there are
     \(a_{k-3}\) of them,

     1100 then it can be followed by a string of length $k-4$ that does not contain the pattern 101, and there are
     \(a_{k-4}\) of them,

     11100 then it can be followed by a string of length $k-5$ that does not contain the pattern 101, and there are
     \(a_{k-5}\) of them,

     \(\underbrace{11\ldots 1}_{k-3}00\), then it can be followed by a string of length $k-(k-1) = 1$ that does not
     contain the pattern 101, and there are \(a_1\) of them,

     \(\underbrace{11\ldots 1}_{k-2}00\), then it can be followed by a string of length $k-k = 0$ that does not
     contain the pattern 101, and there are \(a_0\) of them,

     \(\underbrace{11\ldots 1}_{k-1}\), then it can be followed by 1 or 0, and there are \(2\) of them.

     So \(a_k= a_{k-1} + a_{k-3} + a_{k-4} + \cdots + a_0 + 2\).
\end{proof}

\subsubsection{(b)}
Use the result of part (a) to show that if \(k \geq 3\), then \(a_k = 2a_{k-1} - a_{k-2} + a_{k-3}\).

\begin{proof}
     By part (a) we have \(a_{k-1} = a_{k-2} + a_{k-4} + a_{k-5} + \cdots + a_0 + 2\).

     So \(a_{k-1} + a_{k-3} = a_{k-2} + a_{k-3} + a_{k-4} + a_{k-5} + \cdots + a_0 + 2\).

     So \(a_{k-1} - a_{k-2} + a_{k-3} = a_{k-3} + a_{k-4} + a_{k-5} + \cdots + a_0 + 2\).

     So \(2a_{k-1} - a_{k-2} + a_{k-3} = a_{k-1} + a_{k-3} + a_{k-4} + a_{k-5} + \cdots + a_0 + 2\).

     Notice the right-hand side is equal to \(a_k\) by part (a).
\end{proof}

\subsection{Exercise 28}
For each integer \(n \geq 2\) let \(a_n\) be the number of permutations of \(\{1, 2, 3, \ldots, n\}\) in which no
number is more than one place removed from its “natural” position. Thus \(a_1 = 1\) since the one permutation of
\(\{1\}\), namely, 1, does not move 1 from its natural position. Also, \(a_2 = 2\) since neither of the two
permutations of \(\{1, 2\}\), namely, 12 and 21, moves either number more than one place from its natural position.

\subsubsection{(a)}
Find \(a_3\).

\begin{proof}
     \(a_3 = 3\) (The three permutations that do not move more than one place from their “natural” positions are 213, 132,
     and 123.)
\end{proof}

\subsubsection{(b)}
Find a recurrence relation for \(a_1, a_2, a_3, \ldots\).

\begin{proof}
     Let's try to see a relationship between \(a_2 = 2\) and \(a_3 = 3\). We need to add the new element 3. Since we
     cannot move 3 more than one from its natural position, it can be in position 2 or 3. If it's in position 3, like
     \_\_3, then both of the permutations from \(a_2\) work: 123, 213. If it's in position 2, then 2 must be in position
     3: \_23, so there is only one possibility left: 123.

     Let's find $a_4$. There are \(4! = 24\) permutations of \(\{1,2,3,4\}\), but there are only 5 which do not move more
     more than one place: 1234, 2134, 1243, 2143, 1324. So \(a_4 = 5\).

     But how is it obtained from \(a_3\)? There are 3 permutations in $a_3$: 123, 132, 213. Now we need to add
     the element 4 to them. 4 can be either in position 3 or position 4. If it's in position 4, like \_\_\_4, then all
     the permutations from $a_3$ work: 1234, 2134, 1324. If 4 is in position 3, then 3 must be in position 4: \_\_43, so now
     there are $a_2$ ways to fill the two spots: 1243, 2143.

     So we can see that the general form of the recurrence is: \(a_n = a_{n-1} + a_{n-2}\).
\end{proof}

\subsection{Exercise 29}
Refer to Example 9.3.5.

\subsubsection{(a)}
Write the following IP address in dotted decimal form: 11001010 00111000 01101011 11101110

\begin{proof}
     \(11001010_2 = 2 + 23 + 26 + 27 = 202\)

     \(00111000_2 = 23 + 24 + 25 = 56\)

     \(01101011_2 = 1 + 2 + 23 + 25 + 26 = 107\)

     \(11101110_2 = 2 + 22 + 23 + 25 + 26 + 27 = 238\)

     So the answer is 202.56.107.238.
\end{proof}

\subsubsection{(b)}
How many Class A networks can there be?

\begin{proof}
     The network ID for a Class A network consists of 8 bits and begins with 0. If all possible combinations of eight 0’s
     and 1’s that start with a 0 were allowed, there would be 2 choices (0 or 1) for each of the 7 positions from the
     second through the eighth. This would give \(2^7 = 128\) possible ID’s. But because neither 00000000 nor 01111111 is
     allowed, the total is reduced by 2, so there are 126 possible Class A networks.
\end{proof}

\subsubsection{(c)}
What is the dotted decimal form of the IP address for a computer in a Class A network?

\begin{proof}
     Let \(w.x.y.z\) be the dotted decimal form of the IP address for a computer in a Class A network. Because the
     network IDs for a Class A network go from 00000001 (= 1) through 01111110 (= 126), $w$ can be any integer from 1
     through 126. In addition, each of \(x, y\), and \(z\) can be any integer from 0 (= 00000000) through 255 (=
     11111111), except that \(x, y\), and \(z\) cannot all be 0 simultaneously and cannot all be 255 simultaneously.
\end{proof}

\subsubsection{(d)}
How many host IDs can there be for a Class A network?

\begin{proof}
     Twenty-four positions are allocated for the host ID in a Class A network. If each could be either 0 or 1, there
     would be \(2^{24} = 16,777,216\) possible host IDs. But neither all 0’s nor all 1’s is allowed, which reduces the
     total by 2. Thus there are 16,777,214 possible host IDs in a Class A network.
\end{proof}

\subsubsection{(e)}
How many Class C networks can there be?

\begin{proof}
     The network ID for a Class C network consists of 24 bits and begins with 110. There are 2 choices (0 or 1) for each
     of the 21 positions from the fourth through the twenty-fourth. This gives \(2^{21}\) possible ID’s.
\end{proof}

\subsubsection{(f)}
What is the dotted decimal form of the IP address for a computer in a Class C network?

\begin{proof}
     Let \(w.x.y.z\) be the dotted decimal form of the IP address for a computer in a Class C network. The network IDs for a
     Class C network go from 110000000000000000000000 through 110111111111111111111111. If we break these down into chunks
     of 8 bits and convert them to binary, we get: \(11000000_2 = 192_{10}, 00000000_2 = 0_{10}, 00000000_2 = 0_{10}\),
     \(11011111_2 = 223_{10}, 11111111_2 = 255_{10}, 11111111_2 = 255_{10}\). As dotted decimals they range from 192.0 to
     223.255.

     So \(192 \leq w \leq 223\), \(0 \leq x \leq 255\), \(0 \leq y \leq 255\), \(0 \leq z \leq 255\).
\end{proof}

\subsubsection{(g)}
How many host IDs can there be for a Class C network?

\begin{proof}
     There are only 8 bits for host IDs in a Class C network, but a host ID may not consist of either all 0’s or all 1’s.
     Therefore \(2^8 - 2 = 254\) host IDs.
\end{proof}

\subsubsection{(h)}
How can you tell, by looking at the first of the four numbers in the dotted decimal form of an IP address, what
kind of network the address is from? Explain.

\begin{proof}
     For Class A networks \(0 \leq w \leq 126\), for Class B networks \(128 \leq w \leq 191\), for Class C networks
     \(192 \leq w \leq 223\). So we can tell from this info what kind of network an address is from.
\end{proof}

\subsubsection{(i)}
An IP address is 140.192.32.136. What class of network does it come from?

\begin{proof}
     Observe that \(140 = 128 + 8 + 4 = 10001100_2\), which begins with 10. Thus the IP address comes from a Class B
     network. An alternative solution uses the result of Example 9.3.5: Network IDs for Class B networks range from 128
     through 191. Thus, since \(128 \leq 140 \leq 191\), the given IP address is from a Class B network.
\end{proof}

\subsubsection{(j)}
An IP address is 202.56.107.238. What class of network does it come from?

\begin{proof}
     For Class C networks we have \(192 \leq w \leq 223\), and 202 is in this range, so it the address belongs to a Class
     C network.
\end{proof}

\subsection{Exercise 30}
A row in a classroom has $n$ seats. Let \(s_n\) be the number of ways nonempty sets of students can sit in the row
so that no student is seated directly adjacent to any other student. (For instance, a row of three seats could contain
a single student in any of the seats or a pair of students in the two outer seats. Thus \(s_3 = 4\).) Find a
recurrence relation for \(s_1, s_2, s_3, \ldots\).

\begin{proof}
     Let us denote students with a * and empty seats with a . .

     $n=1$: The seats . can be filled as *, so \(s_1 = 1\).

     $n=2$: The seats .. can be filled as *. or .*, so \(s_2 = 2\).

     $n=3$: The seats ... can be filled as *.., .*., ..*, or *.*, so \(s_3 = 4\).

     $n=4$: The seats .... can be filled as *..., .*.., ..*., ...*, *.*., *..*, or .*.*, so \(s_4 = 7\).

     $n=5$: The seats ..... can be filled as *...., .*..., ..*.., ...*., ....*, *.*.., *..*., *...*, .*.*., .*..*,
     ..*.*  or *.*.*, so \(s_5 = 12\).

     $n=6$: The seats ...... can be filled as *....., .*...., ..*..., ...*.., ....*., .....*, *.*..., *..*.., *...*., *....*, .*.*.., .*..*., .*...*, ..*.*., ..*..*, ...*.*, *.*.*., *.*..*, *..*.*, or .*.*.*, so \(s_6 = 20\).

     So the recurrence is \(s_n = s_{n-1}+s_{n-2}+1\). This makes sense as follows: we can take the $n-1$ seat patterns
     and add an empty seat . to the right end (there are \(s_{n-1}\) of them); we can take the $n-2$ seat patterns
     and add a .* at the right end (there are \(s_{n-2}\) of them), and we have \(\underbrace{.....}_{n-1}*\) finally.
\end{proof}

\subsection{Exercise 31}
Assume that birthdays are equally likely to occur in any one of the 12 months of the year.

\subsubsection{(a)}
Given a group of four people, A, B, C, and D, what is the total number of ways in which birth months could be
associated with A, B, C, and D? (For instance, A and B might have been born in May, C in September, and D in
February. As another example, A might have been born in January, B in June, C in March, and D in October.)

\begin{proof}
     There are 12 possible birth months for A, 12 for B, 12 for C, and 12 for D, so the total is \(12^4 = 20,736\).
\end{proof}

\subsubsection{(b)}
How many ways could birth months be associated with A, B, C, and D so that no two people would share the same birth
month?

\begin{proof}
     If no two people share the same birth month, there are 12 possible birth months for A, 11 for B, 10 for C, and 9 for
     D. Thus the total is \(12 \cdot 11 \cdot 10 \cdot 9 = 11,880\).
\end{proof}

\subsubsection{(c)}
How many ways could birth months be associated with A, B, C, and D so that at least two people would share the same
birth month?

\begin{proof}
     If at least two people share the same birth month, the total number of ways birth months could be associated with A, B, C, and D is \(20,736 - 11,880 = 8,856\).
\end{proof}

\subsubsection{(d)}
What is the probability that at least two people out of A, B, C, and D share the same birth month?

\begin{proof}
     The probability that at least two of the four people share the same birth month is \(8,856 / 20,736 \approx 42.7\%\).
\end{proof}

\subsubsection{(e)}
How large must $n$ be so that in any group of $n$ people, the probability that two or more share the same birth month
is at least 50\%?

\begin{proof}
     When there are five people, the probability that at least two share the same birth month is \(\frac{12^5 - 12 \cdot
          11 \cdot 10 \cdot 9 \cdot 8}{12^5} \approx 61.8\%\), and when there are more than five people, the probability is
     even greater. Thus, since the probability for four people is less than 50\%, the group must contain five or more
     people for the probability to be at least 50\% that two or more share the same birth month.
\end{proof}

\subsection{Exercise 32}
Assuming that all years have 365 days and all birthdays occur with equal probability, how large must $n$ be so that
in any randomly chosen group of $n$ people, the probability that two or more have the same birthday is at least 1/2?
(This is called the birthday problem. Many people find the answer surprising.)

\begin{proof}
     The probability that two out of $n$ people share the same birthday is
     \[
          \frac{365^n - 365 \cdot 364 \cdot \cdots \cdot (365-n+1)}{365^n}
     \]
     We want this to be \(\geq 50\%\). By trying values on the calculator I found that for $n=23$ it's \(\approx 50.73\%\)
     and for $n=22$ it's \(\approx 47.57\%\) so $n=23$ is the smallest value.
\end{proof}

\subsection{Exercise 33}
A college conducted a survey to explore the academic interests and achievements of its students. It asked
students to place checks beside the numbers of all the statements that were true of them. Statement \#1 was “I was
on the Dean’s list last term,” statement \#2 was “I belong to an academic club, such as the math club or the Spanish
club,” and statement \#3 was “I am majoring in at least two subjects.” Out of a sample of 100 students, 28 checked \#1,
26 checked \#2, and 14 checked \#3, 8 checked both \#1 and \#2, 4 checked both \#1 and \#3, 3 checked both \#2 and
\#3, and 2 checked all three statements.

\subsubsection{(a)}
How many students checked at least one of the statements?

\begin{proof}
     The number of students who checked at least one of the statements is \(N(H) + N(C) + N(D) - N(H \cap C) -
     N(H \cap D) - N(C \cap D) + N(H \cap C \cap D) = 28 + 26 + 14 - 8 - 4 - 3 + 2 = 55\).
\end{proof}

\subsubsection{(b)}
How many students checked none of the statements?

\begin{proof}
     By the difference rule, the number of students who checked none of the statements is the total number of students
     minus the number who checked at least one statement. This is \(100 - 55 = 45\).
\end{proof}

\subsubsection{(c)}
Let H be the set of students who checked \#1, C the set of students who checked \#2, and D the set of students who
checked \#3. Fill in the numbers for all eight regions of the diagram.

\begin{proof}
     \begin{figure}[ht!]
          \centering
          \includegraphics[scale=0.5]{../images/9.3.33.c.2.png}
     \end{figure}
\end{proof}

\subsubsection{(d)}
How many students checked \#1 and \#2 but not \#3?

\begin{proof}
     The number of students who checked \#1 and \#2 but not \#3 is \(N(H \cap D) - N(H \cap C \cap D) = 8 - 2 = 6\).
\end{proof}

\subsubsection{(e)}
How many students checked \#2 and \#3 but not \#1?

\begin{proof}
     The number of students who checked \#2 and \#3 but not \#1 is \(N(C \cap D) - N(H \cap C \cap D) = 3 - 2 = 1\).
\end{proof}

\subsubsection{(f)}
How many students checked \#2 but neither of the other two?

\begin{proof}
     The number of students who checked \#2 but not \#1 or \#3 is \(N(C) - N(C \cap (H \cup D)) = 17 - (6+2+1) = 8\).
\end{proof}

\subsection{Exercise 34}
A study was done to determine the efficacy of three different drugs: A, B, and C, in relieving headache pain.
Over the period covered by the study, 50 subjects were given the chance to use all three drugs. The following
results were obtained:

21 reported relief from drug A

21 reported relief from drug B

31 reported relief from drug C

9 reported relief from both drugs A and B

14 reported relief from both drugs A and C

15 reported relief from both drugs B and C

41 reported relief from at least one of the drugs.

Note that some of the 21 subjects who reported relief from drug A may also have reported relief from drugs B or C. A
similar occurrence may be true for the other data.

\subsubsection{(a)}
How many people got relief from none of the drugs?

\begin{proof}
     \(50-41=9\)
\end{proof}

\subsubsection{(b)}
How many people got relief from all three drugs?

\begin{proof}
     By the Inclusion-Exclusion Principle,
     \[
          N(A \cup B \cup C) = N(A) + N(B) + N(C) - N(A \cap B) - N(A \cap C) - N(B \cap C) + N(A \cap B \cap C).
     \]
     By using the values given in the problem, we get
     \[
          41 = 21 + 21 + 31 - 9 - 14 - 15 + N(A \cap B \cap C).
     \]
     Solving we get \(N(A \cap B \cap C) = 6\).
\end{proof}

\subsubsection{(c)}
\begin{figure}[ht!]
     \centering
     \includegraphics[scale=0.4]{../images/9.3.34.c.1.png}
\end{figure}

Let A be the set of all subjects who got relief from drug A, B the set of all subjects who got relief from drug B,
and C the set of all subjects who got relief from drug C. Fill in the numbers for all eight regions of the diagram.

\begin{proof}
     \begin{figure}[ht!]
          \centering
          \includegraphics[scale=0.4]{../images/9.3.34.c.2.png}
     \end{figure}
\end{proof}

\subsubsection{(d)}
How many subjects got relief from A only?

\begin{proof}
     4
\end{proof}

\subsection{Exercise 35}
An interesting use of the inclusion/exclusion rule is to check survey numbers for consistency. For example, suppose
a public opinion polltaker reports that out of a national sample of 1,200 adults, 675 are married, 682 are from 20 to
30 years old, 684 are female, 195 are married and are from 20 to 30 years old, 467 are married females, 318 are
females from 20 to 30 years old, and 165 are married females from 20 to 30 years old. Are the polltaker’s
figures consistent? Could they have occurred as a result of an actual sample survey?

\begin{proof}
     Let \(M\) = the set of married people in the sample,

     \(Y\) = the set of people between 20 and 30 in the sample, and

     \(F\) = the set of females in the sample.

     Then the number of people in the set \(M \cup Y \cup F\) is less than or equal to the size of the sample. And so

     \begin{tabular}{rcl}
          1,200 & \(\geq\) & \(N(M \cup Y \cup F)\)                             \\
                & =        & \(N(M) + N(Y) + N(F) - N(M \cap Y)\)               \\
                &          & \(-N(M \cap F) -N(Y \cap F) + N(M \cap Y \cap F)\) \\
                & =        & \(675 + 682 + 684 - 195 - 467 - 318 + 165\)        \\
                & =        & 1,226.
     \end{tabular}

     This is impossible since \(1,200 < 1,226\), so the polltaker’s figures are inconsistent. They could not have
     occurred as a result of an actual sample survey.
\end{proof}

\subsection{Exercise 36}
Fill in the reasons for each step below. If $A$ and $B$ are sets in a finite universe $U$, then
\begin{center}
     \begin{tabular}{rcll}
          \(N(A \cap B)\) & = & \(N(U) - N((A \cap B)^c)\)                     & {\cy (a) \fbl} \\
                          & = & \(N(U) - N(A^c \cup B^c)\)                     & {\cy (b) \fbl} \\
                          & = & \(N(U) - (N(A^c) + N(B^c) - N(A^c \cap B^c))\) & {\cy (c) \fbl}
     \end{tabular}
\end{center}
\begin{proof}
     (a) because
     \begin{center}
          \begin{tabular}{rcll}
               \(A \cap B\) & = & \((A \cap B) \cap U\)       & {\cy by identity law}          \\
                            & = & \(U \cap (A \cap B)\)       & {\cy by commutative law}       \\
                            & = & \(U \cap ((A \cap B)^c)^c\) & {\cy by double complement law} \\
                            & = & \(U - (A \cap B)^c\)        & {\cy by set difference law}    \\
          \end{tabular}
     \end{center}
     and because \(N(U - (A \cap B)^c) = N(U) - N((A \cap B)^c)\).

     (b) by De Morgan laws (c) by inclusion-exclusion principle for 2 sets
\end{proof}

{\bf \cy For each of exercises $37-39$, the number of elements in a certain set can be found by computing the
number in a larger universe that are not in the set and subtracting this from the total in the larger universe. In
each of these, as was the case for the solution to example 9.3.6(b), De Morgan’s laws and the inclusion / exclusion
rule can be used.}

\subsection{Exercise 37}
How many positive integers less than 1,000 have no common factors with 1,000?

\begin{proof}
     Let $A$ be the set of all positive integers less than 1,000 that are not multiples of 2, and let $B$ be the set of all
     positive integers less than 1,000 that are not multiples of 5. Since the only prime factors of 1,000 are 2 and 5, the
     number of positive integers that have no common factors with 1,000 is \(N(A \cap B)\). Let the universe $U$ be the
     set of all positive integers less than 1,000. Then \(A^c\) is the set of positive integers less than 1,000 that are
     multiples of 2, \(B^c\) is the set of positive integers less than 1,000 that are multiples of 5, and \(A^c \cap
     B^c\) is the set of positive integers less than 1,000 that are multiples of 10. By one of the procedures discussed in
     Section 9.1 or 9.2, it is easily found that \(N(A^c) = 499, N(B^c) = 199\), and \(N(A^c \cap B^c) = 99\). Thus, by the
     inclusion/exclusion rule, \(N(A^c \cup B^c) = N(A^c) + N(B^c) - N(A^c \cap B^c) = 499 + 199 - 99 = 599\). But by
     De Morgan’s law, \(N(A^c \cup B^c) = N((A \cap B)^c)\), and so (*) \(N((A \cap B)^c) = 599\). Now since \((A \cap B)^c
     = U - (A \cap B)\), by the difference rule we have (**) \(N((A \cap B)^c) = N(U) - N(A \cap B)\). Equating the
     right-hand sides of (*) and (**) gives \(N(U) - N(A \cap B) = 599\). And because \(N(U) = 999\), we conclude that \(999
     - N(A \cap B) = 599\), or, equivalently, \(N(A \cap B) = 999 - 599 = 400\). So there are 400 positive integers less
     than 1,000 that have no common factor with 1,000.
\end{proof}

\subsection{Exercise 38}
How many permutations of \(abcde\) are there in which the first character is \(a, b\), or $c$ and the last character is \(c, d\), or $e$?

\begin{proof}
     {\bf Case 1: first character is $a$.} For the last character there are 3 choices: \(c, d, e\). Then the middle
     3 characters can be permuted in \(3!\) ways. So \(3 \cdot 3! = 3 \cdot 6 = 18\) possibilities.

          {\bf Case 2: first character is $b$.} For the last character there are 3 choices: \(c, d, e\). Then the middle
     3 characters can be permuted in \(3!\) ways. So \(3 \cdot 3! = 3 \cdot 6 = 18\) possibilities.

          {\bf Case 3: first character is $c$.} For the last character there are 2 choices: \(d, e\). Then the middle
     3 characters can be permuted in \(3!\) ways. So \(2 \cdot 3! = 2 \cdot 6 = 12\) possibilities.

     In total, there are \(18+18+12 = 48\) permutations in which the first character is \(a, b\), or $c$ and the last
     character is \(c, d\), or $e$.
\end{proof}

\subsection{Exercise 39}
How many integers from 1 through 999,999 contain each of the digits 1, 2, and 3 at least once? ({\it Hint:} For each
\(i = 1, 2\), and 3, let \(A_i\) be the set of all integers from 1 through 999,999 that do not contain the digit $i$.)

\begin{proof}
     Let \(A_1, A_2, A_3\) be the sets of all integers from 1 through 999,999 that do not contain the digits 1, 2, 3,
     respectively.

     \(N(A_1) = 9^6 - 1\) because, excluding the digit 1, there are 9 choices for each digit, for a total of \(9^6\), but
     one of them, namely 000,000 corresponds to 0 which is out of the range 1 through 999,999.

     Similarly \(N(A_2) = N(A_3) = 9^6-1\).

     \(N(A_1 \cap A_2) = 8^6 - 1\) because, excluding the digits 1 and 2, there are 8 choices for each digit, for a total of
     \(8^6\), but one of them, namely 000,000 corresponds to 0 which is out of the range 1 through 999,999.

     Similarly \(N(A_1 \cap A_3) = N(A_2 \cap A_3) = 8^6 - 1\).

     Finally \(N(A_1 \cap A_2 \cap A_3) = 7^6-1\) by a similar argument to the above arguments.

     Now by the inclusion / exclusion principle, \(N(A_1 \cup A_2 \cup A_3) = N(A_1) + N(A_2) + N(A_3) - N(A_1 \cap A_2)
     - N(A_1 \cap A_3) - N(A_2 \cap A_3) + N(A_1 \cap A_2 \cap A_3) = 3(9^6-1) - 3(8^6-1) + 7^6-1 = 925,539\). \\
     So the number of integers that do not contain at least one of the digits 1,2,3 is 925,539. Then \(999,999 - 925,539 =
     74460\) integers contain all three of the digits 1,2,3. \\
     We can verify this with a computer:
     \begin{minted}{scala}
scala> var count = 0
var count: Int = 0
scala> for i <- 1 to 999999 do
     |   if "123".forall(c => i.toString.contains(c)) then
     |     count += 1
scala> count
val res0: Int = 74460
\end{minted}
\end{proof}

{\bf \cy For 40 and 41, use the definition of the Euler phi function \(\phi\) from Section 7.1, exercises $51-53$.}

\subsection{Exercise 40}
Use the inclusion/exclusion principle to prove the following: If \(n = pq\), where $p$ and $q$ are distinct prime numbers, then \(\phi(n) = (p - 1)(q - 1)\).

     {\it Hint:} Let $A$ and $B$ be the sets of all positive integers less than or equal to $n$ that are divisible by
$p$ and $q$, respectively. Then \(\phi(n) = n - N(A \cup B)\).

\begin{proof}
     (following the Hint)

     Notice \(N(A) = q\) since there are $q$ multiples of $p$ from 1 to $n$: \(p, 2p, 3p, \ldots, (q-1)p, qp = n\).
     Similarly \(N(B) = p\).

     Notice \(N(A \cap B) = 1\) since \(n = pq\) is the only number from 1 through $n$ that is divisible by both $p$ and $q$.

     By the inclusion / exclusion principle, \(N(A \cup B) = N(A) + N(B) - N(A \cap B) = q + p - 1\). So \(\phi(n) = n -
     (q+p-1) = pq-q-p+1 = q(p-1) - (p-1) = (p-1)(q-1)\).
\end{proof}

\subsection{Exercise 41}
Use the inclusion/exclusion principle to prove the following: If \(n = pqr\), where \(p, q\), and \(r\) are
distinct prime numbers, then \(\phi(n) = (p - 1)(q - 1)(r - 1)\).

\begin{proof}
     We can use an approach similar to Exercise 40. Let \(P, Q, R\) be the sets of positive integers less than or equal to
     $n$ that are divisible by \(p,q,r\) respectively.

     Then \(N(P) = qr, N(Q) = pr, N(R) = pq\) by similar arguments.

     Now \(N(P \cap Q) = r\) because: if an integer is divisible by both $p$ and $q$ then it is divisible by $pq$ since $p$
     and $q$ are both prime; and there are $r$ multiples of $pq$ from 1 through $n$: \(pq, 2pq, 3pq, \ldots, (r-1)pq, rpq = n\).

     By similar arguments \(N(P \cap R) = q\) and \(N(Q \cap R) = p\). Finally \(N(P \cap Q \cap R) = 1\) because \(n=pqr\)
     itself is the only integer divisible by all 3 primes.

     By the inclusion / exclusion principle, \(N(P \cup Q \cup R) = N(P) + N(Q) + N(R) - N(P \cap Q) -N(P \cap R) - N(Q
     \cap R) + N(P \cap Q \cap R) = qr + pr + pq - r - q - p + 1\).

     \(N(P \cup Q \cup R)\) is the number of integers that are divisible by at least one of \(p,q,r\), thus \(\phi(n) =
     n - N(P \cup Q \cup R) = pqr - qr + pr + pq - r - q - p + 1\), which, after factoring, equals \((p-1)(q-1)(r-1)\).
\end{proof}

\subsection{Exercise 42}
A gambler decides to play successive games of blackjack until he loses three times in a row. (Thus the gambler
could play five games by losing the first, winning the second, and losing the final three or by winning the first
two and losing the final three. These possibilities can be symbolized as LWLLL and WWLLL.) Let \(g_n\) be the number
of ways the gambler can play $n$ games.

\subsubsection{(a)}
Find \(g_3, g_4\), and \(g_5\).

\begin{proof}
     3 games: LLL so \(g_3 = 1\). \,\,\, 4 games: WLLL so \(g_4 = 1\).

     5 games: LWLLL, WWLLL so \(g_5 = 2\).
\end{proof}

\subsubsection{(b)}
Find \(g_6\).

\begin{proof}
     6 games: WWWLLL, LWWLLL, WLWLLL, LLWLLL so \(g_6 = 4\).
\end{proof}

\subsubsection{(c)}
Find a recurrence relation for \(g_3, g_4, g_5, \ldots\).

     {\it Hint:} If \(k \geq 6\), any sequence of $k$ games must begin with W, LW, or LLW.

\begin{proof}
     By the Hint we can see that for \(k \geq 6, g_k = g_{k-1} + g_{k-2} + g_{k-3}\).
\end{proof}

\subsection{Exercise 43}
A derangement of the set \(\{1, 2, \ldots, n\}\) is a permutation that moves every element of the set away from
its “natural” position. Thus 21 is a derangement of \(\{1, 2\}\), and 231 and 312 are derangements of \(\{1, 2, 3\}\).
For each positive integer $n$, let \(d_n\) be the number of derangements of the set \(\{1, 2, \ldots, n\}\).

\subsubsection{(a)}
Find \(d_1, d_2\), and \(d_3\).

\begin{proof}
     \(d_1 = 0\) because the only permutation of the set \(\{1\}\) is 1, which cannot move 1 from its natural position.

     \(d_2 = 1\) because the only two permutations of the set \(\{1,2\}\) are 12 and 21, and only 21 moves both 1 and 2
     away from their natural positions.

     \(d_3 = 2\) because there are 6 permutations of the set \(\{1,2,3\}\): 123, 132, 213, 231, 312, 321; and only 231
     and 312 move every element away from their natural positions.
\end{proof}

\subsubsection{(b)}
Find \(d_4\).

\begin{proof}
     There are 24 permutations: 1234, 1243, 1324, 1342, 1423, 1432, 2134, 2143, 2314, 2341, 2413, 2431, 3124, 3142, 3214,
     3241, 3412, 3421, 4123, 4132, 4213, 4231, 4312, 4321.

     The only ones that move every element away from their natural positions are: 2143, 2341, 2413, 3142, 3412, 3421,
     4123, 4312, 4321.

     So \(d_4 = 9\).
\end{proof}

\subsubsection{(c)}
Find a recurrence relation for \(d_1, d_2, d_3, \ldots\).

     {\it Hint:} Divide the set of all derangements into two subsets: one subset consists of all derangements in which
the number 1 changes places with another number, and the other subset consists of all derangements in which the
number 1 goes to position \(i \neq 1\) but $i$ does not go to position 1. The answer is \(d_k = (k - 1)d_{k-1} + (k-1)
d_{k-2}\). Can you justify it?

\begin{proof}
     (following the Hint)

     Consider a derangement in the first subset. Then 1 switches places with another number $i$. There are $k-1$ other
     numbers that 1 can switch places with. For each one of them, then the remaining \(k-2\) numbers have to also
     derange, and there are \(d_{k-2}\) ways to do that. Thus the number of derangements in the first subset is
     \((k-1)d_{k-2}\).

     Consider a derangement in the second subset. Then 1 moves to position $i$ for some number $i$. There are $k-1$
     choices for this $i$. For each one of these choices, there are \(d_{k-1}\) ways to derange the remaining $k-1$
     numbers. Why?

     Because number $i$ cannot go to position 1, but it can go to any one of the remaining $k-2$ positions, and similarly
     any other number $j$ (\(j \neq 1, j \neq i\)) can go to $k-2$ positions (any position except $j$ and $i$), so this
     situation is the same as derangements of $k-1$ elements.

     Thus the number of derangements in the second subset is \((k-1)d_{k-1}\). So the total is:
     \(d_k = (k - 1)d_{k-1} + (k - 1)d_{k-2}\).
\end{proof}

\subsection{Exercise 44}
Note that a product \(x_1x_2x_3\) may be parenthesized in two different ways: \((x_1x_2)x_3\) and \(x_1(x_2x_3)\).
Similarly, there are several different ways to parenthesize \(x_1x_2x_3x_4\). Two such ways are \((x_1x_2)(x_3x_4)\)
and \(x_1((x_2 x_3)x_4)\). Let \(P_n\) be the number of different ways to parenthesize the product \(x_1x_2 \ldots
x_n\). Show that if \(P_1 = 1\), then
\[
     P_n = \sum_{k=1}^{n-1} P_k P_{n-k} \text{ for every integer  } n \geq 2.
\]
(It turns out that the sequence \(P_1, P_2, P_3, \ldots\) is the same as the sequence of Catalan numbers:
\(P_n = C_{n-1}\) for every integer \(n \geq 1\). See Example 5.6.4.)

\begin{proof}
     Consider the product \(x_1x_2 \ldots x_n\). All the different ways to parenthesize it can be classified as
     follows: split it into two products \(x_1\) and \(x_2 \ldots x_n\), or into \(x_1x_2\) and \(x_3 \ldots x_n\),
     or into \(x_1x_2x_3\) and \(x_4 \ldots x_n\), \(\ldots\), or into \(x_1x_2 \ldots x_{n-1}\) and \(x_n\).

     By the addition rule, $P_n$ is the sum of the number of ways each one of these product pairs can be parenthesized.

     For each of the product pairs, by the product rule, the number of ways to parenthesize the product pair is the
     product of the numbers of ways to parenthesize each product.

     There are \(P_1\) ways to parenthesize \(x_1\) and \(P_{n-1}\) ways to parenthesize \(x_2 \ldots x_n\), so
     there are \(P_1P_{n-1}\) ways to parenthesize their product.

     Similarly there are \(P_2\) ways to parenthesize \(x_1x_2\) and \(P_{n-2}\) ways to parenthesize \(x_3 \ldots x_n\), so
     there are \(P_2P_{n-2}\) ways to parenthesize their product.

     And so on. Therefore \(P_n = \sum_{k=1}^{n-1} P_k P_{n-k}\) for every \(n \geq 2\).
\end{proof}

\subsection{Exercise 45}
Use mathematical induction to prove Theorem 9.3.1: ``Suppose a finite set $A$ equals the union of $k$ distinct
mutually disjoint subsets \(A_1, A_2, \ldots, A_k\). Then \(N(A) = N(A_1) + N(A_2) + \ldots + N(A_k)\).''

\begin{proof}
     Let \(P(n)\) be the statement ``if a finite set $A$ equals the union of $n$ distinct mutually disjoint subsets
     \(A_1, A_2, \ldots, A_n\) then \(N(A) = N(A_1) + N(A_2) + \cdots + N(A_n)\).''

     {\bf Show that \(P(1)\) is true:} In this case \(A = A_1\) therefore \(N(A) = N(A_1)\), so \(P(1)\) is true.

          {\bf Show that for any integer \(k \geq 1\) if \(P(k)\) is true then \(P(k+1)\) is true:} Assume \(P(k)\) is true and
     assume \(A\) is a finite set that equals the union of $k+1$ mutually disjoint subsets \(A_1, \ldots, A_k, A_{k+1}\).
          {\it [We want to show \(N(A) = N(A_1) + N(A_2) + \cdots + N(A_{k+1})\)]}.

     Let \(A' = A_1 \cup \cdots \cup A_k\). Then \(A'\) is a finite set that equals the union of $k$ mutually disjoint
     subsets \(A_1, \ldots, A_k\). By the inductive hypothesis \(N(A') = N(A_1) + N(A_2) + \cdots + N(A_k)\).

     Notice that since \(A_1, \ldots, A_{k+1}\) are mutually disjoint, \(A' = A_1 \cup \cdots \cup A_k\) and \(A_{k+1}\)
     are also disjoint. So \(N(A) = N(A' \cup A_{k+1}) = N(A') + N(A_{k+1})=N(A_1) +N(A_2) + \cdots + N(A_k) + N(A_{k+1})\),
     {\it [as was to be shown.]}
\end{proof}

\subsection{Exercise 46}
Prove the inclusion/exclusion rule for two sets $A$ and $B$ by showing that \(A \cup B\) can be partitioned into
\(A \cap B\), \(A - (A \cap B)\), and \(B - (A \cap B)\), and then using the addition and difference rules.
(See the hint for exercise 39 in Section 6.2.)

\begin{proof}
     {\bf Claim:} \(A \cup B = (A \cap B) \cup [A - (A \cap B)] \cup [B - (A \cap B)]\) and \((A \cap B), A-(A \cap B)\)
     and \(B-(A \cap B)\) are mutually disjoint sets.

          {\it Proof of Claim.}

          {\it First to prove the equality:}

     1. Assume \(x \in A \cup B\).

     2. By definition of union, \(x \in A\) or \(x \in B\).

     3. {\bf Case 1: \(x \in A\).}

     3.1 {\bf Case 1.1: \(x \in B\).} Then by definition of intersection, \(x \in A \cap B\). So by definition of union
     \(x \in (A \cap B) \cup [A - (A \cap B)] \cup [B - (A \cap B)]\).

     3.2 {\bf Case 1.2: \(x \notin B\).} Then by definition of intersection, \(x \notin A \cap B\). So by definition of
     difference \(x \in A - (A \cap B)\). So by definition of union, \(x \in (A \cap B) \cup [A - (A \cap B)] \cup
     [B - (A \cap B)]\).

     4. {\bf Case 2: \(x \in B\).} Similar to Case 1. We conclude \(x \in (A \cap B) \cup [A - (A \cap B)] \cup
     [B - (A \cap B)]\).

     5. By 3 and 4, \(x \in (A \cap B) \cup [A - (A \cap B)] \cup [B - (A \cap B)]\).

     6. By 1, 5 and definition of subset, \(A \cup B \subseteq (A \cap B) \cup [A - (A \cap B)] \cup [B - (A \cap B)]\).

     7. Assume \(x \in (A \cap B) \cup [A - (A \cap B)] \cup
     [B - (A \cap B)]\).

     8. By definition of union, \(x \in A \cap B\) or \(x \in A - (A \cap B)\) or \(x \in B - (A \cap B)\).

     9. {\bf Case 1: \(x \in A \cap B\).} By definition of intersection \(x \in A\) and \(x \in B\). So by definition
     of union, \(x \in A \cup B\).

     10. {\bf Case 2: \(x \in A - (A \cap B)\).} By definition of difference, \(x \in A\) and \(x \notin A \cap B\). So by
     definition of union, \(x \in A \cup B\).

     11. {\bf Case 3: \(x \in B - (A \cap B)\).} By definition of difference, \(x \in B\) and \(x \notin A \cap B\). So by
     definition of union, \(x \in A \cup B\).

     12. By 9, 10 and 11, \(x \in A \cup B\).

     13. By 7, 12 and definition of subset, \((A \cap B) \cup [A - (A \cap B)] \cup [B - (A \cap B)] \subseteq A \cup B\).

     14. By 6, 13 and definition of set equality \((A \cap B) \cup [A - (A \cap B)] \cup [B - (A \cap B)] = A \cup B\).

          {\it Now to prove that the three sets are mutually disjoint:}

     Assume \(x \in A \cap B\). We want to show \(x \notin A - (A \cap B)\) and \(x \notin B - (A \cap B)\).

     Argue by contradiction and assume \(x \in A - (A \cap B)\). Then by definition of difference \(x \in A\) and \(x \notin
     A \cap B\), contradicting \(x \in A\cap B\). Similarly we can prove \(x \notin B - (A \cap B)\).

     Assume \(x \in A - (A \cap B)\). We want to show \(x \notin A \cap B\) and \(x \notin B - (A \cap B)\).

     By definition of difference, \(x \in A\) and \(x \notin A \cap B\). Then by definition of intersection either
     \(x \notin A\) or \(x \notin B\). The first case \(x \notin A\) contradicts the fact that \(x\in A\), so \(x\notin B\).
     So by definition of intersection \(x \notin A \cap B\) (otherwise \(x \in B\), contradiction) and by definition of
     difference \(x \notin B - (A \cap B)\) either (otherwise \(x \in B\), contradiction).

     Assume \(x \in B - (A \cap B)\). We want to show \(x \notin A \cap B\) and \(x \notin A - (A \cap B)\). This proof is
     very similar to the case above.

          {\it [end of proof of the Claim.]}

     By the claim \(N(A \cup B) = N((A \cap B) \cup [A - (A \cap B)] \cup [B - (A \cap B)])\).

     Since these sets are disjoint by the claim, by the addition rule \(N(A \cup B) = N(A \cap B) + N(A - (A \cap B)) + N(B
     - (A \cap B)).\)

     By the difference rule \(N(A - (A \cap B)) = N(A) - N(A \cap B)\) and \(N(B - (A \cap B)) = N(B) - N(A \cap B)\).
     Substituting these into the equation above, we get
     \[
          N(A \cup B) = N(A \cap B) + N(A) - N(A \cap B) + N(B) - N(A \cap B) = N(A) + N(B) - N(A \cap B)
     \]
     {\it [which proves the Inclusion / Exclusion principle for two sets.]}
\end{proof}

\subsection{Exercise 47}
Prove the inclusion/exclusion rule for three sets.

\begin{proof}
     Let \(A,B,C\) be any sets. {\it [We want to show that]}
     \[
          N(A \cup B \cup C) = N(A) + N(B) + N(C) - N(A \cap B) - N(A \cap C) - N(B \cap C) + N(A \cap B \cap C).
     \]
     By the Inclusion / Exclusion Principle for two sets, applied to the two sets \(A \cup B\) and \(C\),
     \[
          N((A \cup B) \cup C) = N(A \cup B) + N(C) - N((A \cup B) \cap C) \hspace{2cm} (1)
     \]
     By the Inclusion / Exclusion Principle for two sets, applied to the two sets \(A\) and \(B\),
     \[
          N(A \cup B) = N(A) + N(B) - N(A \cap B) \hspace{2cm} (2)
     \]
     By the distributive law \((A \cup B) \cap C = (A \cap C) \cup (B \cap C)\). By Inclusion / Exclusion again,
     \[
          N((A \cap C) \cup (B \cap C)) = N(A \cap C) + N(B \cap C) - N((A \cap C) \cap (B \cap C)) \hspace{1cm} (3)
     \]
     Notice \((A \cap C) \cap (B \cap C) = A \cap B \cap C\). Substituting this together with (2) and (3) into (1) we get
     \(N((A \cup B) \cup C)\)
     \begin{center}
          \begin{tabular}{cl}
               = & \(N(A \cup B) + N(C) - N((A \cup B) \cap C)\)                                             \\
               = & \([N(A) + N(B) - N(A \cap B)] + N(C) - N((A \cap C) \cup (B \cap C))\)                    \\
               = & \([N(A) + N(B) - N(A \cap B)] + N(C) - [N(A \cap C) + N(B \cap C) - N(A \cap B \cap C)]\) \\
               = & \(N(A) + N(B) + N(C) - N(A \cap B) - N(A \cap C) - N(B \cap C) + N(A \cap B \cap C)\)     \\
          \end{tabular}
     \end{center}
     {\it [as was to be shown.]}
\end{proof}

\subsection{Exercise 48}
Use mathematical induction to prove the general inclusion/exclusion rule: If \(A_1, \ldots, A_n\) are finite sets,
then \(N(A_1 \cup A_2 \cup \cdots \cup A_n)\)
\begin{center}
     \begin{tabular}{l}
          = \(\dps \sum_{1 \leq i \leq n}N(A_i) - \sum_{1 \leq i < j \leq n} N(A_i \cap A_j) + \sum_{1 \leq i < j < k \leq n}
          N(A_i \cap A_j \cap A_k)\) \\
          \(\dps - \cdots + (-1)^{n+1} N(A_1 \cap A_2 \cap \cdots \cap A_n)\)
     \end{tabular}
\end{center}
(The notation \(\sum_{1 \leq i < j \leq n} N(A_i \cap A_j)\) means that quantities of the form \(N(A_i \cap A_j)\)
are to be added together for all integers $i$ and $j$ with \(1 \leq i < j \leq n\).)

{\it Hint:} Use the associative law for sets from Theorem 6.2.2 and the generalized distributive law for sets from
exercise 40, Section 6.2.

\begin{proof}
     Let \(P(n)\) be the statement: ``if \(A_1, \ldots, A_n\) are finite sets then the equation above is true.''

     {\bf Show that \(P(1)\) is true:} When \(n=1\) the equation becomes \(N(A_1) = \sum_{1 \leq i \leq 1} N(A_1)\), with
     all the other sums becoming empty sums, since \(1 \leq i < j \leq 1\) is impossible for integers $i,j$. Both sides are
     equal to \(N(A_1)\) therefore $P(1)$ is true.

          {\bf Show that for any integer \(n \geq 1\) if \(P(n)\) is true then \(P(n+1)\) is true:} Assume \(n \geq 1\) and
     assume \(P(n)\) is true. Assume \(A_1, \ldots, A_{n+1}\) are finite sets. {\it [We want to prove \(P(n+1)\).]}

     Let \(B = \bigcup_{i=1}^n A_i\). Notice \(\bigcup_{i=1}^{n+1}A_i = B \cup A_{n+1}\). Then by the inclusion / exclusion principle for two sets,
     \[
          N\left(\bigcup_{i=1}^{n+1}A_i\right) = N(B \cup A_{n+1}) = N(B) + N(A_{n+1}) - N(B \cap A_{n+1}) \hspace{2cm} (1)
     \]
     By the inductive hypothesis
     \[
          N(B) = \sum_{1 \leq i \leq n} N(A_i) - \sum_{1 \leq i < j \leq n} N(A_i \cap A_j) + \cdots + (-1)^{n+1} N(A_1 \cap A_2 \cap \cdots \cap A_n) \,\,\, (2)
     \]
     By the generalized distributive law,
     \[
          B \cap A_{n+1} = (A_1 \cap A_{n+1}) \cup (A_2 \cap A_{n+1}) \cup \cdots \cup (A_k \cap A_{n+1}) = \bigcup_{i=1}^{n}
          (A_i \cap A_{n+1})
     \]
     By the inductive hypothesis again, \(N(B \cap A_{n+1}) = N\left(\bigcup_{i=1}^{n} (A_i \cap A_{n+1}) \right)\)
     \begin{center}
          \begin{tabular}{l}
               = \(\dps \sum_{1 \leq i \leq n} N(A_i \cap A_{n+1}) - \sum_{1 \leq i < j \leq n} N((A_i \cap A_{n+1}) \cap
               (A_j \cap A_{n+1}))\) \\
               \(\dps + \cdots + (-1)^{n+1}N\left(\bigcap_{i=1}^n (A_i \cap A_{n+1})\right)\) \hspace{4cm} (3)
          \end{tabular}
     \end{center}
     Now let's try to substitute and combine all the terms.

     The term \(N(B)\) in (1) will be replaced by the right hand side of (2), and the term \(N(B \cap A_{n+1})\) in (1) will
     be replaced by the right hand side of (3). So let's imagine we've done that already, and start combining terms in (1).

     The term \(\dps \sum_{1 \leq i \leq n} N(A_i)\) on the right hand side of (2), and the term \(N(A_{n+1})\) in (1),
     can be combined into: \(\dps \sum_{1 \leq i \leq n+1} N(A_i)\).

     The term \(\dps\sum_{1 \leq i < j \leq n} N(A_i \cap A_j)\) in (2) and the term \(\dps\sum_{1 \leq i \leq n} N(A_i \cap
     A_{n+1})\) on the right side of (3) can be combined into \(\dps\sum_{1 \leq i < j \leq n+1} N(A_i \cap A_j)\). Why?
     The first sum has all the two-way intersections between all the \(A_i\) from 1 to $n$, and the second sum has all the
     two-way intersections of the \(A_i\) from 1 to $n$ with \(A_{n+1}\). Putting them together we end up with all
     the two-way intersections between all the \(A_i\) from 1 to $n+1$. Both of these sums have negative signs (the second
     sum gets a negative sign from \(N(B \cap A_{n+1})\)).

     Notice that the next term in (3) is, by the associative law and the idempotent law for intersection:
     \[
          \sum_{1 \leq i < j \leq n} N((A_i \cap A_{n+1}) \cap (A_j \cap A_{n+1})) = \sum_{1 \leq i < j \leq n} N(A_i \cap
          A_j \cap A_{n+1})
     \]

     So, similarly the term \(\dps \sum_{1 \leq i < j < k \leq n} N(A_i \cap A_j \cap A_k)\) in (2) and the term \(\dps
     \sum_{1 \leq i < j \leq n} N(A_i \cap A_j \cap A_{n+1})\) in (3) can be combined into \(\dps \sum_{1 \leq i < j < k
          \leq n+1} N(A_i \cap A_j \cap A_k)\) because the first sum has all the three-way intersections between \(A_i\) from 1
     to $n$, and the second sum has all the three-way intersections of \(A_{n+1}\) with the $A_i$ from 1 to $n$.
     Both sums have a positive sign (the second sum has a double negative sign, one from \(N(B \cap A_{n+1})\) in (1) and
     another from itself).

     All the other $i$-way intersection term sums from (2) and (3) can be combined in a similar way. So we end up with:
     \(\dps N\left(\bigcup_{i=1}^{n+1}A_i\right)\)
     \begin{center}
          \begin{tabular}{l}
               = \(\dps \sum_{1 \leq i \leq n+1}N(A_i) - \sum_{1 \leq i < j \leq n+1} N(A_i \cap A_j) + \sum_{1 \leq i < j < k \leq
               n+1} N(A_i \cap A_j \cap A_k)\) \\
               \(\dps - \cdots + (-1)^{n+2} N(A_1 \cap A_2 \cap \cdots \cap A_{n+1})\)
          \end{tabular}
     \end{center}
     which proves \(P(n+1)\), {\it [as was to be shown.]}
\end{proof}

\subsection{Exercise 49}
A circular disk is cut into $n$ distinct sectors, each shaped like a piece of pie and all meeting at the center
point of the disk. Each sector is to be painted red, green, yellow, or blue in such a way that no two adjacent sectors
are painted the same color. Let \(S_n\) be the number of ways to paint the disk.

\subsubsection{(a)}
Find a recurrence relation for \(S_k\) in terms of \(S_{k-1}\) and \(S_{k-2}\) for each integer \(k \geq 4\).

     {\it Hint:} Use the solution method described in Section 5.8. The answer is \(S_k = 2S_{k-1} + 3S_{k-2}\) for every
integer \(k \geq 4\).

\begin{proof}
     NOTE: I have to assume, without any extra information, that the disk is FIXED, in other words, rotation, flipping etc.
     are considered different ways to paint.

     For example, for \(n=3\) consider the clockwise paintings starting from the sector that occupies 12 o'clock:

     RGB and RBG: in both paintings, R is adjacent to G and B, G is adjacent to R and B, B is adjacent to R and G. In
     another way, they are equivalent if we traverse the second disk counter-clockwise.

     YBG and BGY: These would be equivalent if rotation is allowed, because it amounts to choosing a different starting point.

          {\it On to the problem:}

     Consider a disk cut into \(k\) sectors and painted as described in the problem. Number the sectors as \(s_1, s_2,
     \ldots, s_k\). So \(s_k\) and \(s_1\) are adjacent, therefore of different colors. There are two cases: either
     \(s_1\) and \(s_{k-1}\) are painted the same color, or not.

          {\bf Case 1: \(s_1\) and \(s_{k-1}\) are painted the same color.} In this case \(s_k\) can be painted one of the
     other 3 colors. Now remove \(s_k\) from the disk. The remaining \(k-1\) sectors can be considered a disk of
     \(k-2\) sectors, by merging \(s_1\) and \(s_{k-1}\) (since they are painted the same color). There are \(S_{k-2}\)
     ways to color this remaining disk. Therefore there are \(3S_{k-2}\) ways to color the original \(k\)-sector disk.

          {\bf Case 2: \(s_1\) and \(s_{k-1}\) are painted different colors.} In this case \(s_k\) can be painted with one of
     the remaining 2 colors. Now remove \(s_k\) from the disk. The remaining \(k-1\) sectors form a valid painting by
     making \(s_1\) and \(s_{k-1}\) adjacent (since they are painted with different colors). There are \(S_{k-1}\) ways
     to paint this remaining disk. Therefore there are \(2S_{k-1}\) ways to color the original \(k\)-sector disk.

     By the addition rule \(S_k = 2S_{k-1} + 3S_{k-2}\).
\end{proof}

\subsubsection{(b)}
Find an explicit formula for \(S_n\) for \(n \geq 2\).

\begin{proof}
     \(S_2 = 12\) because there are 4 colors to paint \(s_1\), 3 colors to paint \(s_2\).

     \(S_3 = 24\) because there are 4 colors to paint \(s_1\), 3 colors to paint \(s_2\), and, since \(s_3\) is adjacent to
     both \(s_1\) and \(s_2\), there are 2 colors to paint \(s_3\).

     The characteristic equation is \(t^2 - 2t - 3 = 0\) so \((t-3)(t+1) = 0\) which gives \(t = -1, 3\). So the general
     form of the solution is \(S_n = A \cdot (-1)^n + B \cdot 3^n\).

     Solving \(S_2 = 12 = A(-1)^2 + B \cdot 3^2\) we get \(12 = A + 9B\). Solving \(S_3 = 24 = A(-1)^3 + B \cdot 3^3\) we get
     \(24 = -A + 27B\). Adding these two equations together gives \(36 = 36B\) so \(B = 1\) and \(A = 3\).
     Thus \(S_n = 3\cdot(-1)^n + 3^n\) for all \(n \geq 2\).
\end{proof}

\section{Exercise Set 9.4}

\subsection{Exercise 1}
\subsubsection{(a)}
If 4 cards are selected from a standard 52-card deck, must at least 2 be of the same suit? Why?

\begin{proof}
     No. For instance, the aces of the four different suits could be selected.
\end{proof}

\subsubsection{(b)}
If 5 cards are selected from a standard 52-card deck, must at least 2 be of the same suit? Why?

\begin{proof}
     Yes. Let \(x_1,x_2,x_3,x_4,x_5\) be five cards. Consider the function \(S\) that sends each card to its suit.

     \begin{figure}[ht!]
          \centering
          \includegraphics[scale=0.5]{../images/9.4.1.b.png}
     \end{figure}

     By the pigeonhole principle, \(S\) is not one-to-one: \(S(x_i) = S(x_j)\) for some two cards \(x_i\) and \(x_j\).
     Hence at least two cards have the same suit.
\end{proof}

\subsection{Exercise 2}
\subsubsection{(a)}
If 13 cards are selected from a standard 52-card deck, must at least 2 be of the same denomination? Why?

\begin{proof}
     No, for example the 13 selected cards might have these denominations: 2, 3, 4, 5, 6, 7, 8, 9, 10, J, Q, K, A.
\end{proof}

\subsubsection{(b)}
If 20 cards are selected from a standard 52-card deck, must at least 2 be of the same denomination? Why?

\begin{proof}
     Yes, since \(20 > 13\) and there are only 13 distinct denominations, by the Pigeonhole Principle at least 2 of
     them must be the same.
\end{proof}

\subsection{Exercise 3}
A small town has only 500 residents. Must there be 2 residents who have the same birthday? Why?

\begin{proof}
     Yes. Denote the residents by \(x_1, x_2, \ldots, x_{500}\). Consider the function \(B\) from residents to birthdays
     that sends each resident to his or her birthday:

     \begin{figure}[ht!]
          \centering
          \includegraphics[scale=0.4]{../images/9.4.3.png}
     \end{figure}

     By the pigeonhole principle, \(B\) is not one-to-one: \(B(x_i) = B(x_j)\) for some two residents \(x_i\) and
     \(x_j\). Hence at least two residents have the same birthday.
\end{proof}

\subsection{Exercise 4}
In a group of 700 people, must there be 2 who have the same first and last initials? Why?

\begin{proof}
     There are 26 letters in the Roman Alphabet, so there are \(26 \cdot 26 = 676\) possibilities for a first and last
     initials combination. Since \(700 > 676\), by the Pigeonhole Principle at least 2 people must have the same initials.
\end{proof}

\subsection{Exercise 5}
\subsubsection{(a)}
Given any set of four integers, must there be two that have the same remainder when divided by 3? Why?

\begin{proof}
     Yes. There are only three possible remainders that can be obtained when an integer is divided by 3: 0, 1, and 2.
     Thus, by the pigeonhole principle, if four integers are each divided by 3, then at least two of them must have the
     same remainder. More formally, call the integers \(n_1, n_2, n_3\), and \(n_4\), and consider the function \(R\)
     that sends each integer to the remainder obtained when that integer is divided by 3:

     \begin{figure}[ht!]
          \centering
          \includegraphics[scale=0.4]{../images/9.4.5.a.png}
     \end{figure}

     By the pigeonhole principle, \(R\) is not one-to-one: \(R(n_i) = R(n_j)\) for some two integers \(n_i\) and
     \(n_j\). Hence at least two integers must have the same remainder.
\end{proof}

\subsubsection{(b)}
Given any set of three integers, must there be two that have the same remainder when divided by 3? Why?

\begin{proof}
     No. For instance, \(\{0, 1, 2\}\) is a set of three integers no two of which have the same remainder when
     divided by 3.
\end{proof}

\subsection{Exercise 6}
\subsubsection{(a)}
Given any set of seven integers, must there be two that have the same remainder when divided by 6? Why?

\begin{proof}
     Yes. The possible remainders are 0, 1, 2, 3, 4, 5 and there are 6 possibilities. Since \(7 > 6\), by the Pigeonhole
     Principle at least two of the seven integers must have the same remainder. More formally we can think of \(\mod 6\) as
     a function defined on the set of seven integers: \(M: \{x_1 ,x_2,x_3,x_4 , x_5 , x_6 , x_7\} \to \{0, 1, 2, 3, 4, 5\}\)
     and \(M\) is not one-to-one because the domain has 7 elements while the range only has 6.
\end{proof}

\subsubsection{(b)}
Given any set of seven integers, must there be two that have the same remainder when divided by 8? Why?

\begin{proof}
     No, for example \(\{0, 1, 2, 3, 4, 5, 6, 7\}\) is a set of seven integers no two of which have the same remainder.
\end{proof}

\subsection{Exercise 7}
Let \(S = \{3, 4, 5, 6, 7, 8, 9, 10, 11, 12\}\). Suppose six integers are chosen from \(S\). Must there be two
integers whose sum is 15? Why?

\begin{proof}
     Notice that \(3 + 12 = 15\), \(4 + 11 = 15\), \(5 + 10 = 15\), \(6 + 9 = 15\), and \(7 + 8 = 15\). We can divide \(S\) into
     five sets: \(\{3, 12\}\), \(\{4, 11\}\), \(\{5, 10\}\), \(\{6, 9\}\), \(\{7, 8\}\). By the Pigeonhole Principle, when 6
     integers are chosen from \(S\), two of them will belong to the same set among these 5 sets. Therefore when 6 integers are
     chosen from \(S\) there must be two integers whose sum is 15.
\end{proof}

\subsection{Exercise 8}
Let \(T = \{1, 2, 3, 4, 5, 6, 7, 8, 9\}\). Suppose five integers are chosen from \(T\). Must there be two integers
whose sum is 10? Why?

\begin{proof}
     No, for example \(\{1,2,3,4,5\}\) contains no two integers whose sum is 10.
\end{proof}

\subsection{Exercise 9}
\subsubsection{(a)}
If seven integers are chosen from between 1 and 12 inclusive, must at least one of them be odd? Why?

\begin{proof}
     Yes.

     {\it Solution 1:} Only six of the numbers from 1 to 12 are even (namely, 2, 4, 6, 8, 10, 12), so at most six even
     numbers can be chosen from between 1 and 12 inclusive. Hence if seven numbers are chosen, at least one must be odd.

          {\it Solution 2:} Partition the set of all integers from 1 through 12 into six subsets (the pigeonholes), each
     consisting of an odd and an even number: \(\{1, 2\}\), \(\{3, 4\}\), \(\{5, 6\}\), \(\{7, 8\}\), \(\{9, 10\}\),
     \(\{11, 12\}\). If seven integers (the pigeons) are chosen from among 1 through 12, then, by the pigeonhole principle,
     at least two must be from the same subset. But each subset contains one odd and one even number. Hence at least one of
     the seven numbers is odd.

          {\it Solution 3: (a formal version of Solution 2):} Let
     \(S = \{x_1, x_2, x_3, x_4, x_5, x_6, x_7\}\) be a set of
     seven numbers chosen from the set \(T = \{1, 2, 3, 4, 5, 6, 7, 8, 9, 10, 11, 12\}\), and let \(P\) be the following
     partition of \(T: \{1, 2\}, \{3, 4\}, \{5, 6\}, \{7, 8\}, \{9, 10\}\), and \(\{11, 12\}\). Since each element of
     \(S\) lies in exactly one subset of the partition, we can define a function \(F\) from \(S\) to \(P\) by letting
     \(F(x_i)\) be the subset that contains \(x_i\).

     \begin{figure}[ht!]
          \centering
          \includegraphics[scale=0.5]{../images/9.4.9.a.png}
     \end{figure}

     Since \(S\) has 7 elements and \(P\) has 6 elements, by the pigeonhole principle, \(F\) is not one-to-one. Thus two
     distinct numbers of the seven are sent to the same subset, which implies that these two numbers are the two distinct
     elements of the subset. Therefore, since each pair consists of one odd and one even integer, one of the seven numbers
     is odd.
\end{proof}

\subsubsection{(b)}
If ten integers are chosen from between 1 and 20 inclusive, must at least one of them be even? Why?

\begin{proof}
     No. For instance, none of the 10 numbers 1, 3, 5, 7,
     9, 11, 13, 15, 17, 19 is even.
\end{proof}

\subsection{Exercise 10}
If \(n + 1\) integers are chosen from the set \(\{1, 2, 3, \ldots, 2n\}\), where \(n\) is a positive integer, must at
least one of them be odd? Why?

\begin{proof}
     Yes. There are \(n\) even integers in the set \(\{1, 2, 3, \ldots, 2n\}\), namely, \(2(= 2 \cdot 1), 4(= 2 \cdot 2)\), \(6(= 2 \cdot 3), \ldots, 2n(= 2 \cdot n)\). So the maximum number of even integers that can be chosen is \(n\). Thus
     if \(n + 1\) integers are chosen, at least one of them must be odd.
\end{proof}

\subsection{Exercise 11}
If \(n + 1\) integers are chosen from the set \(\{1, 2, 3, \ldots, 2n\}\), where \(n\) is a positive integer, must at
least one of them be even? Why?

\begin{proof}
     Yes. The set contains \(n\) odd integers: \(1, 3, 5, \ldots, 2n-1\). Since \(n+1 > n\), by the Pigeonhole Principle at least one of the chosen integers must be even.
\end{proof}

\subsection{Exercise 12}
How many cards must you pick from a standard 52-card deck to be sure of getting at least 1 red card? Why?

\begin{proof}
     The answer is 27. There are only 26 black cards in a standard 52-card deck, so at most 26 black cards can be chosen. Hence
     if 27 are taken, at least one must be red.
\end{proof}

\subsection{Exercise 13}
Suppose six pairs of similar-looking boots are thrown together in a pile. How many individual boots must you pick to be sure
of getting a matched pair? Why?

\begin{proof}
     6 pairs means 12 boots. So we must pick at least 7 individual boots to get a matching pair. Because the 12 boots are split
     into 6 pairs, and since \(7>6\), by the Pigeonhole Principle the 7 picked boots must contain two boots from one of the
     pairs.
\end{proof}

\subsection{Exercise 14}
How many integers from 0 through 60 must you pick in order to be sure of getting at least one that is odd? at least one that
is even?

\begin{proof}
     There are 61 integers from 0 through 60. Of these, 31 are even \((0 = 2 \cdot 0, 2 = 2 \cdot 1, 4 = 2 \cdot 2,
     \ldots, 60 = 2 \cdot 30)\) and so 30 are odd. Hence if 32 integers are chosen, at least one must be odd, and if 31
     integers are chosen, at least one must be even.
\end{proof}

\subsection{Exercise 15}
If \(n\) is a positive integer, how many integers from 0
through \(2n\) must you pick in order to be sure of getting
at least one that is odd? at least one that is even?

\begin{proof}
     There are \(n+1\) even and \(n\) odd integers from 0 through \(2n\). Hence if \(n+2\) integers are chosen, at least one
     must be odd, and if \(n+1\) integers are chosen, at least one must be even.
\end{proof}

\subsection{Exercise 16}
How many integers from 1 through 100 must you pick in order to be sure of getting one that is divisible by 5?

\begin{proof}
     There are 20 that are divisible by 5: \(5 = 5 \cdot 1, \ldots, 100 = 5 \cdot 20\). The remaining 80 integers are
     not divisible by 5. So we must pick at least 81 integers.
\end{proof}

\subsection{Exercise 17}
How many integers must you pick in order to be sure that at least two of them have the same remainder when divided by 7?

\begin{proof}
     The answer is 8, since there are 7 possible remainders modulo 7: \(0, 1, 2, 3, 4, 5, 6\).
\end{proof}

\subsection{Exercise 18}
How many integers must you pick in order to be sure that at least two of them have the same remainder when divided by 15?

\begin{proof}
     The answer is 16, since there are 15 possible remainders modulo 15: \(0, 1, 2, \ldots, 14\).
\end{proof}

\subsection{Exercise 19}
How many integers from 100 through 999 must you pick in order to be sure that at least two of them have a digit in
common? (For example, 256 and 530 have the digit 5 in common.)

\begin{proof}
     Each integer we pick will use at least 1 of the 10 possible digits. The maximum number of integers we can pick without
     two integers having a digit in common is 9.
\end{proof}

\subsection{Exercise 20}
\subsubsection{(a)}
If repeated divisions by 20,483 are performed, how many distinct remainders can be obtained?

\begin{proof}
     The answer is 20,483 because the possible remainders are \(0, 1, 2, \ldots, 20482\).
\end{proof}

\subsubsection{(b)}
When 5/20483 is written as a decimal, what is the maximum length of the repeating section of the representation?

\begin{proof}
     The length of the repeating section of the decimal representation of 5/20483 is less than or equal to 20,482.
     The reason is that 20,482 is the number of nonzero remainders that can be obtained when a number is divided by
     20,483. Thus, in the long-division process of dividing 5 by 20,483, either some remainder is 0 and the decimal
     expansion terminates, or only nonzero remainders are obtained and at some point within the first 20,482
     successive divisions, a nonzero remainder is repeated. At that point the digits in the developing decimal expansion
     begin to repeat because the sequence of successive remainders repeats those previously obtained.
\end{proof}

\subsection{Exercise 21}
When 683/1493 is written as a decimal, what is the maximum length of the repeating section of the representation?

\begin{proof}
     The length of the repeating section of the decimal representation of 683/1493 is less than or equal to 1492.
     The reason is that 1492 is the number of nonzero remainders that can be obtained when a number is divided by 1493.
     Thus, in the long-division process of dividing 683 by 1493, either some remainder is 0 and the decimal expansion
     terminates, or only nonzero remainders are obtained and at some point within the first 1492 successive divisions, a
     nonzero remainder is repeated. At that point the digits in the developing decimal expansion begin to repeat because
     the sequence of successive remainders repeats those previously obtained.
\end{proof}

\subsection{Exercise 22}
Is \(0.101001000100001000001 \ldots\) (where each string of 0’s is one longer than the previous one) rational or
irrational?

\begin{proof}
     This number is irrational because the decimal expansion neither terminates nor repeats.
\end{proof}

\subsection{Exercise 23}
Is \(56.556655566655556666 \ldots\)  (where the strings of 5’s and 6’s become longer in each repetition) rational or
irrational?

\begin{proof}
     This number is irrational because the decimal expansion neither terminates nor repeats.
\end{proof}

\subsection{Exercise 24}
Show that within any set of thirteen integers chosen from 2 through 40, there are at least two integers with a common
divisor greater than 1.

\begin{proof}
     Let \(A\) be the set of the thirteen chosen numbers, and let \(B\) be the set of all prime numbers between 1 and 40.
     Note that \(B = \{2, 3, 5, 7, 11, 13, 17, 19, 23, 29, 31, 37\}\). For each \(x\) in \(A\), let \(F(x)\) be the
     smallest prime number that divides \(x\). Since \(A\) has 13 elements and \(B\) has 12 elements, by the pigeonhole
     principle \(F\) is not one-to-one. Thus \(F(x_1) = F(x_2)\) for some \(x_1 \neq x_2\) in \(A\). By definition of \(F\),
     this means that the smallest prime number that divides \(x_1\) equals the smallest prime number that divides
     \(x_2\). Therefore, two numbers in \(A\); namely, \(x_1\) and \(x_2\); have a common divisor greater than 1.
\end{proof}

\subsection{Exercise 25}
In a group of 30 people, must at least 3 have been born in the same month? Why?

\begin{proof}
     This follows from the generalized pigeonhole principle with 30 pigeons, 12 pigeonholes, and \(k = 2\), using the fact that \(30 > 2 \cdot 12\).
\end{proof}

\subsection{Exercise 26}
In a group of 30 people, must at least 4 have been born in the same month? Why?

\begin{proof}
     No. For instance, the birthdays of the 30 people could be distributed as follows: three birthdays in each of the six
     months January through June and two birthdays in each of the six months July through December.
\end{proof}

\subsection{Exercise 27}
In a group of 2,000 people, must at least 5 have the same birthday? Why?

\begin{proof}
     \(2000 = 365 \cdot 5 + 175\) so \(2000 > 365 \cdot 5\), so yes.
\end{proof}

\subsection{Exercise 28}
A programmer writes 500 lines of computer code in 17 days. Must there have been at least 1 day when the programmer
wrote 30 or more lines of code? Why?

\begin{proof}
     \(17 \cdot 29 = 493 < 500\) so yes.
\end{proof}

\subsection{Exercise 29}
A certain college class has 40 students. All the students in the class are known to be from 17 through 34 years of
age. You want to make a bet that the class contains at least \(x\) students of the same age. How large can you
make \(x\) and yet be sure to win your bet?

\begin{proof}
     The answer is \(x = 3\). There are 18 years from 17 through 34. Now \(40 > 18 \cdot 2\), so by the generalized
     pigeonhole principle, you can be sure that there are at least \(x = 3\) students of the same age. However, since
     \(18 \cdot 3 > 40\), you cannot be sure of having more than three students with the same age. (For instance, three
     students could be each of the ages 17 through 20, and two could be each of the ages from 21 through 34.) So \(x\)
     cannot be taken to be greater than 3.
\end{proof}

\subsection{Exercise 30}
A penny collection contains twelve 1967 pennies, seven 1968 pennies, and eleven 1971 pennies. If you are to pick some
pennies without looking at the dates, how many must you pick to be sure of getting at least five pennies from the
same year?

\begin{proof}
     The answer is \(4 \cdot 3 + 1 = 13\). Since there are 3 different years, with \(4 \cdot 3 = 12\) coins, we may get
     4 of each year. Then one more penny guarantees that we get at least 5 pennies from the same year.
\end{proof}

\subsection{Exercise 31}
A group of 15 executives are to share 5 assistants. Each executive is assigned exactly 1 assistant, and no assistant
is assigned to more than 4 executives. Show that at least 3 assistants are assigned to 3 or more executives.

     {\it Hint:} Use the same type of reasoning as in Example 9.4.6.

\begin{proof}
     Argue by contradiction and suppose not. Suppose that 2 (or fewer) assistants are assigned to 3 or more executives.
     In this case what is the maximum number of executives that are assigned an assistant? At most 2 assistants can be
     assigned to at most 4 executives for a maximum of \(2 \cdot 4 = 8\) executives, and the remaining 3 assistants can be
     assigned to at most 2 executives, for a maximum of \(3 \cdot 2 = 6\) executives, for a total maximum of \(8+6=14\)
     executives, which is a contradiction since there are 15 executives and \(15 > 14\).
\end{proof}

\subsection{Exercise 32}
Let \(A\) be a set of six positive integers each of which is less than 13. Show that there must be two distinct
subsets of \(A\) whose elements when added up give the same sum. (For example, if \(A = \{5, 12, 10, 1, 3, 4\}\), then
the elements of the subsets \(S_1 = \{1, 4, 10\}\) and \(S_2 = \{5, 10\}\) both add up to 15.)

{\it Hints:} (1) The number of subsets of the six integers
is \(2^6 = 64\). (2) Since each integer is less than 13,
the largest possible sum is 57. (Why? How is this sum obtained?)

\begin{proof}
     To clarify (2) in the Hint, the largest subset size is 6, and the largest possible 6 elements are 7, 8, 9, 10, 11, 12
     which add up to \(7+8+9+10+11+12 = 57\). The smallest possible sum is 0 for the empty subset of \(A\). \\
     Consider a function \(F\) from subsets of \(A\) to the set of positive integers from 0 to 57 defined by \(F(S) = \)
     the sum of all the elements of \(S\). Since the domain has 64 elements, and the co-domain has 58 elements (\(0,\ldots,
     57\)), and  \(64 > 57\), by the Pigeonhole Principle \(F\) is not one-to-one. Therefore \(F(S_1) = F(S_2)\) for two
     distinct subsets \(S_1, S_2\) of \(A\), which means the sums of elements of \(S_1\) and \(S_2\) are equal.
\end{proof}

\subsection{Exercise 33}
Let \(A\) be a set of six positive integers each of which is less than 15. Show that there must be two distinct
subsets of A whose elements when added up give the same sum.

     {\it Hint:} The power set of \(A\) has \(2^6 = 64\) elements, and so there are 63 nonempty subsets of \(A\).
Let \(k\) be the smallest number in \(A\). Then the sums over the elements in the nonempty subsets of \(A\) lie in
the range from \(k\) through \(k + 10 + 11 + 12 + 13 + 14 = k + 60\). How many numbers are in this range?

\begin{proof}
     Let \(k\) be the smallest integer in \(A\). To clarify the Hint, there are 61 numbers in the range from \(k\) through
     \(k+60\).

     Consider a function \(F\) from the nonempty subsets of \(A\) to the set of positive integers from \(k\) to \(k+60\)
     defined by \(F(S) = \) the sum of all the elements of \(S\). Since the domain has 63 elements, and the co-domain
     has 61 elements (\(k, k+1 \ldots, 60\)), and \(63 > 61\), by the Pigeonhole Principle \(F\) is not one-to-one.
     Therefore \(F(S_1) = F(S_2)\) for two distinct subsets \(S_1, S_2\) of \(A\), which means the sums of elements of
     \(S_1\) and \(S_2\) are equal.
\end{proof}

\subsection{Exercise 34}
Let \(S\) be a set of ten integers chosen from 1 through 50. Show that the set contains at least two different (but
not necessarily disjoint) subsets of four integers that add up to the same number. (For instance, if the ten numbers
are \(\{3, 8, 9, 18, 24, 34, 35, 41, 44, 50\}\), the subsets can be taken to be \(\{8, 24, 34, 35\}\) and
\(\{9, 18, 24, 50\}\). The numbers in both of these add up to 101.)

\begin{proof}
     There are \(\binom{10}{4} = \frac{10 \cdot 9 \cdot 8 \cdot 7}{4 \cdot 3 \cdot 2 \cdot 1} = 210\) subsets of \(S\) with
     four elements.

     The four element subset with the smallest possible sum is \(\{1,2,3,4\}\) which adds up to 10, and the four element
     subset with the largest possible sum is \(\{47,48,49,50\}\) which adds up to 194.

     Consider a function \(F\) from the four element subsets of \(S\) to the set of positive integers from \(10\) to
     \(194\) defined by \(F(A) = \) the sum of all the elements of \(A\). Since the domain has 210 elements, and the co-
     domain has \(194-10+1 = 185\) elements and \(210 > 185\), by the Pigeonhole Principle \(F\) is not one-to-one.
     Therefore \(F(S_1) = F(S_2)\) for two distinct subsets \(S_1, S_2\) of \(S\), which means the sums of elements of
     \(S_1\) and \(S_2\) are equal.
\end{proof}

\subsection{Exercise 35}
Given a set of 52 distinct integers, show that there must be 2 whose sum or difference is divisible by 100.

     {\it Hint:} Let \(X\) be the set consisting of the given 52 positive integers, and let \(Y\) be the set containing the
following elements: \(\{00, 50\}, \{01, 99\}, \{02, 98\}, \ldots, \{48, 52\}, \{49, 51\}\). Define a function \(F\)
from \(X\) to \(Y\) by the rule \(F(x) =\) the set containing the last two digits of \(x\). Use the pigeonhole
principle to argue that \(F\) is not one-to-one, and show how the desired conclusion follows.

\begin{proof}
     (following the Hint)

     The domain of \(F\) has 52 elements, while the co-domain has 50 elements. Since \(52 > 50\), by the Pigeonhole
     Principle, \(F\) is not one-to-one. So there are two distinct integers \(x,y\) such that \(F(x) = F(y)\).

     By definition of \(F\), \(x = 100 \cdot m + F(x)\) and \(y = 100 \cdot n + F(y)\) for some integers \(m,n\).

     Since every element of \(Y\) has its elements add up to 100, we have \(F(x) + F(y) = 100\).

     So \(x \pm y = 100m \pm 100n + (F(x) + F(y)) = 100(m \pm n) + 100 = 100(m \pm n+1)\). Therefore \(x \pm y\) is
     divisible by 100.
\end{proof}

\subsection{Exercise 36}
Show that if 101 integers are chosen from 1 to 200 inclusive, there must be 2 with the property that one is
divisible by the other.

     {\it Hint:} Write the 101 integers as \(x_1, x_2, x_3, \ldots, x_{101}\), and represent each \(x_i\) as
\(a_i \cdot 2^{k_i}\) where \(a_i\) is odd and \(k_i \geq 0\). Now \(1 < x_i \leq 200\), and so \(1 \leq a_i \leq
199\) for every \(i\). Use the fact that there are only 100 odd integers from 1 to 199 inclusive.

\begin{proof}
     (following the Hint)

     Since there are only 100 odd integers from 1 to 199 inclusive, there exist \(i, j\) such that \(1 \leq i \leq
     101, 1 \leq j \leq 101, i \neq j\) and \(a_i = a_j\). If \(k_i < k_j\) then \(x_j\) is divisible by \(x_i\) because
     \[
          x_j = a_j \cdot 2^{k_j} = a_i \cdot 2^{k_j} = a_i \cdot 2^{k_i} \cdot 2^{k_j-k_i} = x_i \cdot 2^{k_j-k_i},
     \]
     where \(k_j - k_i > 0\) so \(2^{k_j-k_i}\) is an integer. Similarly if \(k_j < k_i\) then \(x_i\) is divisible by
     \(x_j\). (Note that \(k_i = k_j\) is not possible since all the \(x_i\)s are distinct, so \(x_i \neq x_j\).)
\end{proof}

\subsection{Exercise 37}
\subsubsection{(a)}
Suppose \(a_1, a_2, \ldots, a_n\) is a sequence of \(n\) integers none of which is divisible by \(n\). Show that at
least one of the differences \(a_i - a_j\) (for \(i \neq j\)) must be divisible by \(n\).

\begin{proof}
     By the quotient-remainder theorem
     \[
          a_1 = q_1 \cdot n + r_1, \,\,\,\, a_2 = q_2 \cdot n + r_2,  \ldots, a_n = q_n \cdot n + r_n
     \]
     for some integers \(q_1, \ldots, q_n\) and \(r_1, \ldots, r_n\) where \(0 < r_i < n\) for all \(i = 1,\ldots,n\).
     Notice that \(r_i \neq 0\) because we are given that none of the \(a_i\) are divisible by \(n\).

     Since there are \(n\) integers \(r_1, \ldots, r_n\) from 1 through \(n-1\) and \(n > n-1\), by the Pigeonhole
     Principle two of the \(r_i\)'s are equal; i.e. there exist \(j, k\) such that \(j \neq k\) and \(r_j = r_k\).

     So \(a_j - a_k = q_j \cdot n + r_j - (q_k \cdot n + r_k) = n(q_j - q_k) + (r_j - r_k) = n(q_j-q_k)\), therefore
     \(a_j-a_k\) is divisible by \(n\).
\end{proof}

\subsubsection{(b)}
Show that every finite sequence \(x_1, x_2, \ldots, x_n\) of \(n\) integers has a consecutive subsequence
\(x_{i+1}, x_{i+2}, \ldots, x_j\) whose sum is divisible by \(n\). (For instance, the sequence 3, 4, 17, 7, 16 has the
consecutive subsequence 17, 7, 16 whose sum is divisible by 5.) (From: James E. Schultz and William F. Burger, “An
Approach to Problem-Solving Using Equivalence Classes Modulo \(n\),” College Mathematics Journal (15), No. 5,
1984, 401–405.)

{\it Hint:} For each \(k = 1, 2, \ldots, n\), let \(a_k = x_1 + x_2 + \cdots + x_k\). If some \(a_k\) is divisible by
\(n\), then the problem is solved: the consecutive subsequence is \(x_1, x_2, \ldots, x_k\). If no \(a_k\) is
divisible by \(n\), then \(a_1, a_2, a_3, \ldots, a_n\) satisfies the hypothesis of part (a). Hence \(a_j - a_i\)
is divisible by \(n\) for some integers \(i\) and \(j\) with \(j > i\). Write \(a_j - a_i\) in terms of the
\(x_i\)’s to derive the given conclusion.

\begin{proof}
     (following the Hint) For the \(i < j\) as mentioned above,
     \[
          a_j - a_i = (x_1 + \cdots + x_j) - (x_1 + \cdots x_i) = x_{i+1} + \cdots x_j
     \]
     is divisible by \(n\), so the desired consecutive subsequence is \(x_{i+1}, \ldots, x_j\).
\end{proof}

\subsection{Exercise 38}
Observe that the sequence 12, 15, 8, 13, 7, 18, 19, 11, 14, 10 has three increasing subsequences of length four: 12,
15, 18, 19; 12, 13, 18, 19; and 8, 13, 18, 19. It also has one decreasing subsequence of length four: 15, 13, 11, 10.
Show that in any sequence of \(n^2 + 1\) distinct real numbers, there must be a sequence of length \(n + 1\) that
is either strictly increasing or strictly decreasing.

     {\it Hint:} Let \(a_1, a_2, \ldots, a_{n^2+1}\) be any sequence of \(n^2 + 1\) distinct real numbers, and suppose
that this sequence contains neither a strictly increasing subsequence of length \(n + 1\) nor a strictly decreasing
subsequence of length \(n + 1\). Let \(S\) be the set of all ordered pairs of integers \((i, d)\), where \(1 \leq i
\leq n\) and \(1 \leq d \leq n\). For each term \(a_k\) in the sequence, let \(F(a_k) = (i_k, d_k)\), where \(i_k\) is
the length of the longest increasing sequence starting at \(a_k\), and \(d_k\) is the length of the longest decreasing
sequence starting at \(a_k\). Suppose that \(F\) is one-to-one and derive a contradiction.

\begin{proof}
     (following the Hint)

     The domain of \(F\) has \(n^2+1\) elements \(a_1, \ldots, a_{n^2+1}\) and the co-domain of \(F\) has \(n^2\) elements
     (because \(S\) is the Cartesian product of \(\{1, \ldots, n\}\) with itself). By the Pigeonhole Principle \(F\) is
     not one-to-one. So there exist \(j,l\) such that \(1 \leq j < l \leq n\) and \(F(a_j) = F(a_l)\). So \((i_j, d_j)\)
     equals \((i_l, d_l)\). So the lengths of the longest increasing sequences starting at \(a_j\) and \(a_l\) are
     equal, and the lengths of the longest decreasing sequences starting at \(a_j\) and \(a_l\) are equal. Since \(a_j \neq
     a_l\), this is a contradiction: if \(a_j < a_l\) then we can take the longest increasing sequence starting at
     \(a_l\), add \(a_j\) to the beginning of it, and get a new increasing sequence of length \(i_l + 1 = i_j + 1\)
     starting at \(a_j\), which is longer than \(i_j\).
\end{proof}

\subsection{Exercise 39}
What is the largest number of elements that a set of integers from 1 through 100 can have so that no one integer
in the set is divisible by another? ({\it Hint:} Imagine writing all the integers from 1 through 100 in the form
\(2^k \cdot m\), where \(k \geq 0\) and \(m\) is odd.)

\begin{proof}
     (following the Hint) If \(1 \leq 2^k \cdot m \leq 100\) then \(1 \leq m \leq 99\). There are 50 odd integers \(m\)
     from 1 through 99. For any set of 51 integers, since \(51 > 50\), by the Pigeonhole Principle there are two integers
     \(x = 2^j \cdot m_j\) and \(y = 2^k \cdot m_k\) such that \(m_j = m_k\) so either \(x \mid y\) or \(y \mid x\).

     Moreover the set \(\{51, 52, \ldots, 99, 100\}\) have no two integers where one divides the other, because
     \(51 \cdot 2 = 102 > 100, 52 \cdot 2 = 104 > 100, \ldots\), so no multiple of any of these integers stay within the
     range from 51 to 100.

     So we have an example of a set of integers with 50 elements with no two elements where one divides the other, and we
     have proof that in any 51-element set of integers, there must be two where one divides the other. Thus the largest
     number of elements is 50.
\end{proof}

\subsection{Exercise 40}
Suppose \(X\) and \(Y\) are finite sets, \(X\) has more elements than \(Y\), and \(F: X \to Y\) is a function. By
the pigeonhole principle, there exist elements \(a\) and \(b\) in \(X\) such that \(a \neq b\) and \(F(a) = F(b)\).
Write a computer algorithm to find such a pair of elements \(a\) and \(b\).

\begin{tcolorbox}[colframe=cyan]
     {\bf Input:} \(X\) {\it [a finite set represented by the array]} \(X[1], \ldots, X[n]\).

          {\bf Algorithm Body:}

     \(found \coloneqq false, i \coloneqq 0\)

     \begin{tabbing}
          {\bf wh}\={\bf ile} (\(i \leq n-1\) and \(found = false\)) \\
          \> \(j \coloneqq 0\) \\
          \> {\bf wh}\={\bf ile} (\(j \leq n\) and \(found = false\))\\
          \>         \> {\bf if} (\(F(X[i]) = F(X[j])\) and \(X[i] \neq X[j]\)) \\
          \>         \> {\bf then} \(found = true\) \\
          \>         \> \(j \coloneqq j+1\) \\
          \> {\bf end while} \\
          \> \(i \coloneqq i+1\) \\
          {\bf end while} \\
          {\bf Output:} \(a \coloneqq X[i-1], b \coloneqq X[j-1]\)
     \end{tabbing}
\end{tcolorbox}

\section{Exercise Set 9.5}

\subsection{Exercise 1}
\subsubsection{(a)}
List all 2-combinations for the set \(\{x_1, x_2, x_3\}\). Deduce the value of \(\binom{3}{2}\).

\begin{proof}
     2-combinations: \(\{x_1, x_2\}, \{x_1, x_3\}, \{x_2, x_3\}\). Hence \(\binom{3}{2} = 3\).
\end{proof}

\subsubsection{(b)}
List all unordered selections of four elements from the set \(\{a, b, c, d,e\}\). Deduce the value of \(\binom{5}{4}\).

\begin{proof}
     Unordered selections: \(\{a,b,c,d\}, \{a,b,c,e\}, \{a,b,d,e\}, \{a,c,d,e\}, \{b,c,d,e\}\).

     Hence \(\binom{5}{4} = 4\).
\end{proof}

\subsection{Exercise 2}
\subsubsection{(a)}
List all 3-combinations for the set \(\{x_1, x_2, x_3, x_4, x_5\}\). Deduce the value of \(\binom{5}{3}\).

\begin{proof}
     2-combinations: \(\{x_1, x_2, x_3\}, \{x_1, x_2, x_4\}, \{x_1, x_2, x_5\}, \{x_1, x_3, x_4\}, \{x_1, x_3, x_5\},\)

     \(\{x_1, x_4, x_5\}, \{x_2, x_3, x_4\}, \{x_2, x_3, x_5\}, \{x_2, x_4, x_5\}, \{x_3, x_4, x_5\}\).
     Hence \(\binom{5}{3} = 10\).
\end{proof}

\subsubsection{(b)}
List all unordered selections of two elements from the set \(\{x_1, x_2, x_3, x_4, x_5, x_6\}\).
Deduce the value of \(\binom{6}{2}\).

\begin{proof}
     \(\{x_1, x_2\}, \{x_1, x_3\}, \{x_1, x_4\}, \{x_1, x_5\}, \{x_1, x_6\}, \{x_2, x_3\}, \{x_2, x_4\}, \{x_2, x_5\}\),

     \(\{x_2, x_6\}, \{x_3, x_4\}, \{x_3, x_5\}, \{x_3, x_6\}, \{x_4, x_5\}, \{x_4, x_6\}, \{x_5, x_6\}\).
     Hence \(\binom{6}{2} = 15\).
\end{proof}

\subsection{Exercise 3}
Write an equation relating \(P(7,2)\) and \(\binom{7}{2}\).

\begin{proof}
     \(P(7,2) = \binom{7}{2} \cdot 2!\)
\end{proof}

\subsection{Exercise 4}
Write an equation relating \(P(8,3)\) and \(\binom{8}{3}\).

\begin{proof}
     \(P(8,3) = \binom{8}{3} \cdot 3!\)
\end{proof}

\subsection{Exercise 5}
Use Theorem 9.5.1 to compute each of the following.

\subsubsection{(a)}
\(\binom{6}{0}\)

\begin{proof}
     \(\dps \binom{6}{0} = \frac{6!}{0!(6-0)!} = \frac{\Ccancel[cyan]{6!}}{1 \cdot \Ccancel[cyan]{6!}} = 1\)
\end{proof}

\subsubsection{(b)}
\(\binom{6}{1}\)

\begin{proof}
     \(\dps \binom{6}{1} = \frac{6!}{1!(6-1)!} = \frac{6 \cdot \Ccancel[cyan]{5!}}{1 \cdot \Ccancel[cyan]{5!}} = 6\)
\end{proof}

\subsubsection{(c)}
\(\dps \binom{6}{2}\)

\begin{proof}
     \(\dps \binom{6}{2} = \frac{6!}{2!(6-2)!} = \frac{6 \cdot 5 \cdot \Ccancel[cyan]{4!}}{2 \cdot \Ccancel[cyan]{4!}} =
     \frac{6 \cdot 5}{2} = 15\)
\end{proof}

\subsubsection{(d)}
\(\dps \binom{6}{3}\)

\begin{proof}
     \(\dps \binom{6}{3} = \frac{6!}{3!(6-3)!} = \frac{6 \cdot 5 \cdot 4 \cdot \Ccancel[cyan]{3!}}{(3 \cdot 2 \cdot 1) \cdot
          \Ccancel[cyan]{3!}} = \frac{6 \cdot 5 \cdot 4}{6} = 20\)
\end{proof}

\subsubsection{(e)}
\(\dps \binom{6}{4}\)

\begin{proof}
     \(\dps \binom{6}{4} = \frac{6!}{4!(6-4)!} = \frac{6 \cdot 5 \cdot \Ccancel[cyan]{4!}}{\Ccancel[cyan]{4!} \cdot 2} =
     \frac{6 \cdot 5}{2} = 15\)
\end{proof}

\subsubsection{(f)}
\(\dps \binom{6}{5}\)

\begin{proof}
     \(\dps \binom{6}{5} = \frac{6!}{5!(6-5)!} = \frac{6 \cdot \Ccancel[cyan]{5!}}{\Ccancel[cyan]{5!} \cdot 1} = 6\)
\end{proof}

\subsubsection{(g)}
\(\binom{6}{6}\)

\begin{proof}
     \(\binom{6}{6} = \frac{6!}{6!(6-6)!} = \frac{\Ccancel[cyan]{6!}}{\Ccancel[cyan]{6!} \cdot 1} = 1\)
\end{proof}

\subsection{Exercise 6}
A student council consists of 15 students.

\subsubsection{(a)}
In how many ways can a committee of six be selected from the council?

\begin{proof}
     \(\binom{15}{6} = \frac{15!}{(15-6)!6!} = \frac{\Ccancel[red]{15} \cdot \Ccancel{14}^{{}^{{}^7}}
     \cdot 13 \cdot \Ccancel[green]{12} \cdot 11 \cdot \Ccancel{10}^{{}^{{}^5}} \cdot \Ccancel[cyan]{9!}}
     {\Ccancel[cyan]{9!} \cdot \Ccancel[green]{6} \cdot \Ccancel[red]{5} \cdot \Ccancel{4} \cdot \Ccancel[red]{3}
     \cdot \Ccancel[green]{2}} = 7 \cdot 13 \cdot 11 \cdot 5 = 5005\)
\end{proof}

\subsubsection{(b)}
Two council members have the same major and are not permitted to serve together on a committee. How many ways
can a committee of six be selected from the membership of
the council?

\begin{proof}
     Let \(w\) = the number of committees that don’t contain A and B together,

     \(x\) = the number of committees with A and five others, none of them B,

     \(y\) = the number of committees with B and five others, none of them A,

     \(z\) = the number of committees with neither A nor B,

     then \(w = x+y+z = \binom{13}{5} + \binom{13}{5} + \binom{13}{6} = 1287+1287+1716 = 4290\).

          {\it Alternative solution:} \(w\) = the total number of committees \(-\) the number of committees that contain both
     A and B = \(\binom{15}{6} - \binom{13}{4} = 5005 - 715 = 4290.\)
\end{proof}

\subsubsection{(c)}
Two council members always insist on serving on committees together. If they can’t serve together, they won’t serve at
all. How many ways can a committee of six be selected from the council membership?

\begin{proof}
     The number of committees with both A and B + the number of committees with neither A nor B = \(\binom{13}{4} +
     \binom{13}{6} = 715 + 1716 = 2431\)
\end{proof}

\subsubsection{(d)}
Suppose the council contains eight men and seven women.

(i) How many committees of six contain three men and three women?

(ii) How many committees of six contain at least one woman?

\begin{proof}
     (i) the number of subsets of three men chosen from eight \(\cdot\) the number of subsets of three women chosen from
     seven = \(\binom{8}{3} \cdot \binom{7}{3} = 56 \cdot 35 = 1960\)

     (ii) the number of committees with at least one woman

     = the total number of committees \(-\) the number of all-male committees

     = \(\binom{15}{6} - \binom{8}{6} = 5005 - 28 = 4977\)
\end{proof}

\subsubsection{(e)}
Suppose the council consists of three freshmen, four sophomores, three juniors, and five seniors. How many
committees of eight contain two representatives from each class?

\begin{proof}
     Let \(v\) = the number of committees of eight which contain two representatives from each class,

     \(w\) = the number of ways to choose 2 freshmen,

     \(x\) = the number of ways to choose 2 sophomores,

     \(y\) = the number of ways to choose 2 juniors,

     \(z\) = the number of ways to choose 2 seniors,

     then \(v = wxyz = \binom{3}{2}\binom{4}{2}\binom{3}{2}\binom{5}{2} = 540\)
\end{proof}

\subsection{Exercise 7}
A computer programming team has 13 members.

\subsubsection{(a)}
How many ways can a group of seven be chosen to work on a project?

\begin{proof}
     \(\binom{13}{7} = 1716\)
\end{proof}

\subsubsection{(b)}
Suppose seven team members are women and six are men.

(i) How many groups of seven can be chosen that contain four women and three men?

(ii) How many groups of seven can be chosen that contain at least one man?

(iii) How many groups of seven can be chosen that contain at most three women?

\begin{proof}
     (i) \(\binom{7}{4}\binom{6}{3} = 35 \cdot 20 = 700\), (ii) total number of groups of seven \(-\) the number of all-female
     groups of seven = \(\binom{13}{7} - \binom{7}{7} = 1716 - 1 = 1715\)

     (iii) Zero women: not possible, since we need a group of seven and there are only six men; One woman: \(\binom{7}{1}
     \binom{6}{6} = 7\); Two women: \(\binom{7}{2}\binom{6}{5} = 21 \cdot 6 = 126\); Three women: \(\binom{7}{3}\binom{6}{4}
     = 35 \cdot 15 = 525 126\); At most 3 women = \(7 + 126 + 525 = 658\)
\end{proof}

\subsubsection{(c)}
Suppose two team members refuse to work together on projects. How many groups of seven can be chosen to work on
a project?

\begin{proof}
     Teams that include A but not B: \(\binom{11}{6} = 462\);
     Teams that include B but not A: \(\binom{11}{6} = 462\);
     Teams that include neither A nor B: \(\binom{11}{7}=330\);
     \(462+462+330 = 1254\)
\end{proof}

\subsubsection{(d)}
Suppose two team members insist on either working together or not at all on projects. How many groups of seven can be
chosen to work on a project?

\begin{proof}
     The number of groups of seven that include both A and B + the number of groups of seven that include neither A nor B
     = \(\binom{11}{5} + \binom{11}{7} = 462 + 330 = 792\)
\end{proof}

\subsection{Exercise 8}
An instructor gives an exam with fourteen questions. Students are allowed to choose any ten to answer.

\subsubsection{(a)}
How many different choices of ten questions are there?

\begin{proof}
     \(\binom{14}{10} = \frac{14 \cdot 13 \cdot 12 \cdot 11 \cdot \Ccancel{10!}}{\Ccancel{10!} \cdot 4!} = 7 \cdot 13
     \cdot 11 = 1001\)
\end{proof}

\subsubsection{(b)}
Suppose six questions require proof and eight do not.

(i) How many groups of ten questions contain four that require proof and six that do not?

(ii) How many groups of ten questions contain at least one that requires proof?

(iii) How many groups of ten questions contain at most three that require proof?

\begin{proof}
     (i) \(\binom{6}{4} \binom{8}{6} = 15 \cdot 28 = 420\)

     (ii) all 1001 require proof, because there are only 8 non-proof questions, so 2 more questions are needed for 10
     questions, which have to be proof questions.

     (iii) There must be at least 2 proof questions, so: \(\binom{6}{2}\binom{8}{8} + \binom{6}{3}\binom{8}{7} =
     15 \cdot 1 + 20 \cdot 8 = 175\)
\end{proof}

\subsubsection{(c)}
Suppose the exam instructions specify that at most one of questions 1 and 2 may be included among the ten. How many
different choices of ten questions are there?

\begin{proof}
     Include Q1 but not Q2 = \(\binom{12}{9} = 220\),
     include Q2 but not Q1 = \(\binom{12}{9} = 220\),
     include neither Q1 nor Q2 = \(\binom{12}{10} = 66\),
     \(220 + 220 + 66 = 506\)
\end{proof}

\subsubsection{(d)}
Suppose the exam instructions specify that either both questions 1 and 2 are to be included among the ten or
neither is to be included. How many different choices of ten questions are there?

\begin{proof}
     Ten questions that include both Q1 and Q2 = \(\binom{12}{8} = 495\); Ten questions that include neither Q1 nor Q2 =
     \(\binom{12}{10} = 66\); so \(495 + 66 = 561\).
\end{proof}

\subsection{Exercise 9}
A club is considering changing its bylaws. In an initial straw vote on the issue, 24 of the 40 members of the club
favored the change and 16 did not. A committee of six is to be chosen from the 40 club members to devote further study
to the issue.

\subsubsection{(a)}
What is the total number of committees of six that can be formed from the club membership?

\begin{proof}
     \(\binom{40}{6} = \frac{40 \cdot 39 \cdot 38 \cdot 37 \cdot 36 \cdot 35}{6!} = 3,838,380\)
\end{proof}

\subsubsection{(b)}
How many of the total number of committees will contain at least three club members who, in the preliminary survey,
favored the change in the bylaws?

\begin{proof}
     \(\binom{24}{3}\binom{16}{3} + \binom{24}{4}\binom{16}{2} + \binom{24}{5}\binom{16}{1} + \binom{24}{6}\binom{16}{0} =
     3,223,220\)
\end{proof}

\subsection{Exercise 10}
Two new drugs are to be tested using a group of 60 laboratory mice, each tagged with a number for identification purposes.
Drug A is to be given to 22 mice, drug B is to be given to another 22 mice, and the remaining 16 mice are to be used as
controls. How many ways can the assignment of treatments to mice be made? (A single assignment involves specifying the
treatment for each mouse, whether drug A, drug B, or no drug.)

\begin{proof}
     \(\binom{60}{22}\binom{38}{22}\binom{16}{16} = \binom{60}{22}\binom{38}{22} = 3.1479 \times 10^{26}\)
\end{proof}

\subsection{Exercise 11}
Refer to Example 9.5.9. For each poker holding below, (1) find the number of five-card poker hands with that holding;
(2) find the probability that a randomly chosen set of five cards has that holding.

\subsubsection{(a)}
royal flush

\begin{proof}
     (1) 4 (because there are as many royal flushes as there are suits)

     (2) \(\dps \frac{4}{\binom{52}{5}} = \frac{4}{2,598,960} = 0.000001539 = 0.0001539\%\)
\end{proof}

\subsubsection{(b)}
straight flush

\begin{proof}
     (1) There are 10 straights: A-2-3-4-5, 2-3-4-5-6, 3-4-5-6-7, 4-5-6-7-8, 5-6-7-8-9, 6-7-8-9-10, 7-8-9-10-J, 8-9-10-J-Q,
     9-10-J-Q-K, 10-J-Q-K-A; and these can be flushes in 4 ways (since there are 4 suits). So \(10 \cdot 4 = 40\). But 4 of
     these are royal flushes, so \(40 - 4 = 36\) straight flushes.

     (2) \(\dps \frac{36}{\binom{52}{5}} = \frac{36}{2,598,960} = 0.000013852 = 0.0013852\%\)
\end{proof}

\subsubsection{(c)}
four of a kind

\begin{proof}
     (1) There are 13 possible four of a kinds. For each for of a kind, there are 48 ways to choose the fifth card.
     So \(13 \cdot 48 = 624\).

     (2) \(\dps \frac{624}{\binom{52}{5}} = \frac{624}{2,598,960} = 0.000240096 = 0.0240096\%\)
\end{proof}

\subsubsection{(d)}
full house

\begin{proof}
     (1) This is a two of a kind plus a three of a kind.

     There are 13 suits, and there are \(\binom{4}{2} = 6\) two of a kinds in each suit, for example:
     \(2\heartsuit 2\diamondsuit\), \(2\heartsuit 2\clubsuit\), \(2\heartsuit 2\spadesuit\), \(2\diamondsuit 2\clubsuit\),
     \(2\diamondsuit 2\spadesuit\), \(2\clubsuit 2\spadesuit\). So \(13 \cdot 6 = 78\) two of a kinds in total.

     Similarly there are \(\binom{4}{3} = 4\) three of a kinds in each suit, so \(13 \cdot 4 = 52\) three of a kinds in total.

     However, when a three of a kind is chosen, the two of a kind cannot be of the same value, because that would require 5
     cards of the same value, and there are only 4.

     So either choose a three of a kind (52 choices) then choose a two of a kind from the remaining \(78-6=72\) choices for a
     total of \(52 \cdot 72 = 3744\), or vice versa: choose a two of a kind (78 choices) then a three of a kind from the
     remaining \(52-4=48\) for a total of \(78 \cdot 48 = 3744\).

     (2) \(\frac{3744}{\binom{52}{5}} = \frac{3744}{2,598,960} = 0.001440576 = 0.1440576\%\)
\end{proof}

\subsubsection{(e)}
flush

\begin{proof}
     (1) There are 4 suits, for each suit there are \(\binom{13}{5} = 1287\) ways to choose 5 cards of that suit, for a total of
     \(4 \cdot 1287 = 5148\) flushes. 40 of them are royal or straight flushes, so \(5148-40 = 5108\) flushes. \,\,\,\,
     (2) \(\frac{5108}{\binom{52}{5}} = \frac{5108}{2,598,960} = 0.001965402 = 0.1965402\%\)
\end{proof}

\subsubsection{(f)}
straight

\begin{proof}
     (1) There are 10 straights. For each straight, there are \(4 \cdot 4 \cdot 4 \cdot 4 \cdot 4 = 1024\) ways to choose the
     cards (since each value has 4 copies). So \(10 \cdot 1024 = 10240\) straights. By (b), 40 of these are royal or straight
     flushes, so \(10240-40 = 10200\) non-royal, non-flush straights. \,\,\,\,
     (2) \(\frac{10200}{\binom{52}{5}} = \frac{10200}{2,598,960} = 0.003924647 = 0.3924647\%\)
\end{proof}

\subsubsection{(g)}
three of a kind

\begin{proof}
     (1) 13 values to make three of a kind, for each value \(\binom{4}{3} = 4\) ways to choose 3 out of 4 suits, for a
     total of \(13 \cdot 4 = 52\) three of a kinds. To fill the rest of the hand we need two more cards. These two cards may
     not include the same value, otherwise it becomes a four of a kind. So choose 2 values from 12 values in: \(\binom{12}{2} =
     66\) ways, and for each one, 4 ways to choose a suit. In total: \(52 \cdot 66 \cdot 4 \cdot 4 = 54912\). \\
     (2) \(\frac{54912}{\binom{52}{5}} = \frac{54912}{2,598,960} = 0.021128451 = 2.1128451\%\)
\end{proof}

\subsubsection{(h)}
one pair

\begin{proof}
     (1) 13 values to make one pair, for each value \(\binom{4}{2} = 6\) ways to choose 2 out of 4 suits, for a total of
     \(13 \cdot 6 = 78\) pairs. We need to choose 3 more cards from the remaining 50, but these three may not include the same
     value, otherwise it becomes three or four of a kind. So choose 3 values out of 12, which can be done in \(\binom{12}{3} =
     220\) ways, and for each value, 4 ways to choose a suit. In total: \(78 \cdot 220 \cdot 4 \cdot 4 \cdot 4 = 1,098,240\).\\
     (2) \(\frac{1,098,240}{\binom{52}{5}} = \frac{1,098,240}{2,598,960} = 0.422569028 = 42.2569028\%\)
\end{proof}

\subsubsection{(i)}
neither a repeated denomination nor five of the same suit nor five adjacent denominations

\begin{proof}
     (1) Total number of all poker hands is 2,598,960. There are 4 royal flushes, 36 straight flushes, 624 four of a kinds, 5108
     flushes, 10200 straights, 3744 full houses, 54912 three of a kinds, 123,552 two pairs, 1,098,240 one pairs.

     So we simply subtract these from the total number of all hands:
     \[
          2,598,960 - (4 + 36 + 624 + 5108 + 10200 + 3744 + 54912 + 123,552 + 1,098,240) = 1,302,540
     \]
     (2) \(\frac{1,302,540}{\binom{52}{5}} = \frac{1,302,540}{2,598,960} = 0.501177394 = 50.1177394\%\)
\end{proof}

\subsection{Exercise 12}
How many pairs of two distinct integers chosen from the set \(\{1, 2, 3, \ldots, 101\}\) have a sum that is even?

\begin{proof}
     Even sum means: either both even, or both odd integers. There are 51 odd and 50 even integers in the set. There are
     \(\binom{51}{2} = 1275\) pairs of distinct odd integers, and there are \(\binom{50}{2} = 1225\) pairs of distinct even
     integers, for a total of \(2500\) pairs.
\end{proof}

\subsection{Exercise 13}
A coin is tossed ten times. In each case the outcome H (for heads) or T (for tails) is recorded. (One possible outcome
of the ten tosses is denoted T H H T T T H T T H.)

\subsubsection{(a)}
What is the total number of possible outcomes of the coin-tossing experiment?

\begin{proof}
     \(2^{10} = 1024\)
\end{proof}

\subsubsection{(b)}
In how many of the possible outcomes are exactly five heads obtained?

\begin{proof}
     \(\binom{10}{5}\binom{5}{5} = 252\)
\end{proof}

\subsubsection{(c)}
In how many of the possible outcomes are at least eight heads obtained?

\begin{proof}
     \(\binom{10}{8} + \binom{10}{9} + \binom{10}{10} = 45 + 10 + 1 = 56\)
\end{proof}

\subsubsection{(d)}
In how many of the possible outcomes is at least one head obtained?

\begin{proof}
     All outcomes \(-\) outcomes with no heads = \(1024 - 1 = 1023\)
\end{proof}

\subsubsection{(e)}
In how many of the possible outcomes is at most one head obtained?

\begin{proof}
     \(\binom{10}{0} + \binom{10}{1} = 1 + 10 = 11\)
\end{proof}

\subsection{Exercise 14}
\subsubsection{(a)}
How many 16-bit strings contain exactly seven 1’s?

\begin{proof}
     \(\dps\binom{16}{7}\binom{9}{9} = 11440\)
\end{proof}

\subsubsection{(b)}
How many 16-bit strings contain at least thirteen 1’s?

\begin{proof}
     \(\dps\binom{16}{13} + \binom{16}{14} + \binom{16}{15} + \binom{16}{16} = 560 + 120 + 16 + 1 = 697\)
\end{proof}

\subsubsection{(c)}
How many 16-bit strings contain at least one 1?

\begin{proof}
     All 16-bit strings \(-\) strings with no 1's = \(2^{16} - 1 = 16384 - 1 = 16383\)
\end{proof}

\subsubsection{(d)}
How many 16-bit strings contain at most one 1?

\begin{proof}
     \(\binom{16}{0} + \binom{16}{1} = 1 + 16 = 17\)
\end{proof}

\subsection{Exercise 15}
\subsubsection{(a)}
How many even integers are in the set \(\{1, 2, 3, \ldots, 100\}\)?

\begin{proof}
     50
\end{proof}

\subsubsection{(b)}
How many odd integers are in the set \(\{1, 2, 3, \ldots, 100\}\)?

\begin{proof}
     50
\end{proof}

\subsubsection{(c)}
How many ways can two integers be selected from the set \(\{1, 2, 3, \ldots, 100\}\) so that their sum is even?

\begin{proof}
     If the sum of two integers is even, then either both are even or both are odd. We can choose 2 even integers in \(\binom{50}
     {2} = 1225\) ways, or choose 2 odd integers in \(\binom{50}{2}= 1225\) ways, for a total of \(1225 + 1225 = 2450\) ways.
\end{proof}

\subsubsection{(d)}
How many ways can two integers be selected from the set \(\{1, 2, 3, \ldots, 100\}\) so that their sum is odd?

\begin{proof}
     If the sum of two integers is odd, then one is even and the other is odd. We can choose 1 even integer in \(\binom{50}{1}
     = 50\) ways, and choose 1 odd integer in \(\binom{50}{1} = 50\) ways, for a total of \(50 \cdot 50 = 2500\) ways.

     Alternatively, note that the answer equals the total number of subsets of two integers chosen from 1 through 100 minus the number of such subsets for which the sum of the elements is 100 even. Thus the answer is \(\binom{100}{2} - 2450 = 2500\).
\end{proof}

\subsection{Exercise 16}
Suppose that three microchips in a production run of forty are defective. A sample of five is to be selected to be checked
for defects.

\subsubsection{(a)}
How many different samples can be chosen?

\begin{proof}
     \(\binom{40}{5} = 658,008\)
\end{proof}

\subsubsection{(b)}
How many samples will contain at least one defective chip?

\begin{proof}
     All samples \(-\) samples with no defective chips = \(\binom{40}{5} - \binom{37}{5} = 658,008 - 435,897 = 222,111\)
\end{proof}

\subsubsection{(c)}
What is the probability that a randomly chosen sample of five contains at least one defective chip?

\begin{proof}
     \(\frac{222,111}{658,008} = 0.337550607\)
\end{proof}

\subsection{Exercise 17}
Ten points labeled A, B, C, D, E, F, G, H, I, J are arranged in a plane in such a way that no three lie on the same
straight line.

\subsubsection{(a)}
How many straight lines are determined by the ten points?

\begin{proof}
     Two points determine a line, so \(\binom{10}{2} = 45\)
\end{proof}

\subsubsection{(b)}
How many of these straight lines do not pass through point A?

\begin{proof}
     Excluding A there are 9 points, so \(\binom{9}{2} = 36\)
\end{proof}

\subsubsection{(c)}
How many triangles have three of the ten points as vertices?

\begin{proof}
     Three points determine a triangle, so \(\binom{10}{3} = 120\)
\end{proof}

\subsubsection{(d)}
How many of these triangles do not have point A as a vertex?

\begin{proof}
     Excluding A there are 9 points, so \(\binom{9}{3} = 84\)
\end{proof}

\subsection{Exercise 18}
Suppose that you placed the letters in Example 9.5.11 into positions in the following order: first the M, then the
I’s, then the S’s, and then the P’s. Show that you would obtain the same answer for the number of distinguishable
orderings.

\begin{proof}
     The answer in the exercise was 34650. Now:

     Choose a position for the M: \(\binom{11}{1} = 11\),
     Choose a position for the I's: \(\binom{10}{4} = 210\)

     Choose a position for the S's: \(\binom{6}{4} = 15\),
     Choose a position for the P's: \(\binom{2}{2} = 1\)

     Result = \(11 \cdot 210 \cdot 15 \cdot 1 = 34650\)
\end{proof}

\subsection{Exercise 19}
\subsubsection{(a)}
How many distinguishable ways can the letters of the word HULLABALOO be arranged in order?

\begin{proof}
     HULLABALOO contains: 1 H, 1 U, 3 L's, 2 A's, 1 B, 2 O's, a total of \(1+1+3+2+1+2 = 10\) letters. By Theorem 9.5.2,
     \(\dps\frac{10!}{1!1!3!2!1!2!} = 151,200\).
\end{proof}

\subsubsection{(b)}
How many distinguishable orderings of the letters of HULLABALOO begin with U and end with L?

\begin{proof}
     Now we have 1 H, 2 L's, 2 A's, 1 B, 2 O's, so \(\dps\frac{8!}{1!2!2!1!2!} = 5040\)
\end{proof}

\subsubsection{(c)}
How many distinguishable orderings of the letters of HULLABALOO contain the two letters HU next to each other in
order?

\begin{proof}
     Consider HU as a single character, say X. Now we have 1 X, 3 L's, 2 A's, 1 B, 2 O's, so \(\frac{9!}{1!3!2!1!2!} = 15120\)
\end{proof}

\subsection{Exercise 20}
\subsubsection{(a)}
How many distinguishable ways can the letters of the word MILLIMICRON be arranged in order?

\begin{proof}
     MILLIMICRON has 2 M's, 3 I's, 2 L's, 1 C, 1 R, 1 O and 1 N, and a total of 11 letters. By Theorem 9.5.2
     \(\frac{11!}{2!3!2!1!1!1!1!} = 1,663,200\)
\end{proof}

\subsubsection{(b)}
How many distinguishable orderings of the letters of MILLIMICRON begin with M and end with N?

\begin{proof}
     Now we have 1 M's, 3 I's, 2 L's, 1 C, 1 R, and 1 O, and a total of 9 letters. By Theorem 9.5.2
     \(\frac{9!}{1!3!2!1!1!1!} = 30,240\)
\end{proof}

\subsubsection{(c)}
How many distinguishable orderings of the letters of MILLIMICRON contain the letters CR next to each other in
order and also the letters ON next to each other in order?

\begin{proof}
     Consider CR to be a single character, say X, and ON to be a single character, say Y. Now we have 1 M's, 3 I's, 2 L's, 1 X,
     and 1 Y, and a total of 8 letters. By Theorem 9.5.2
     \(\frac{8!}{1!3!2!1!1!} = 3360\)
\end{proof}

\subsection{Exercise 21}
In Morse code, symbols are represented by variable-length sequences of dots and dashes. (For example, \(A = \cdot -,
1 = \cdot - - - -, ? = \cdot \cdot - - \cdot \cdot\).) How many different symbols can be represented by sequences of
seven or fewer dots and dashes?

\begin{proof}
     \(2^1 + 2^2 + 2^3 + 2^4 + 2^5 + 2^6 + 2^7 = 2 + 4 + 8 + 16 + 32 + 64 + 128 = 254\)
\end{proof}

\subsection{Exercise 22}
Each symbol in the Braille code is represented by a rectangular arrangement of six dots, each of which may be raised or flat against a smooth background. For instance, when the word Braille is spelled out, it looks like this:

\begin{figure}[ht!]
     \centering
     \includegraphics[scale=0.5]{../images/9.5.22.png}
\end{figure}

Given that at least one of the six dots must be raised, how many symbols can be represented in the Braille code?

\begin{proof}
     \(2^6 = 64\) possibilities, except the case where no dot is raised, so \(64-1=63\).
\end{proof}

\subsection{Exercise 23}
On an \(8 \times 8\) chessboard, a rook is allowed to move any number of squares either horizontally or vertically.
How many different paths can a rook follow from the bottom-left square of the board to the top-right square of the
board if all moves are to the right or upward?

\begin{proof}
     The rook must move seven squares to the right and seven squares up, so let \(x\) = the number of paths the rook can
     take, \(y\) =  the number of of orderings of seven R’s and seven U’s, where R stands for “right” and U stands for “up”,
     then \(x = y\) and by Theorem 9.5.2 \(x = y = \frac{14!}{7!7!} = 3432\).
\end{proof}

\subsection{Exercise 24}
The number 42 has the prime factorization \(2 \cdot 3 \cdot 7\). Thus 42 can be written in four ways as a product of
two positive integer factors (without regard to the order of the factors): \(1 \cdot 42, 2 \cdot 21, 3 \cdot 14\),
and \(6 \cdot 7\). Answer a–d below without regard to the order of the factors.

\subsubsection{(a)}
List the distinct ways the number 210 can be written as a product of two positive integer factors.

\begin{proof}
     \(210 = 2 \cdot 3 \cdot 5 \cdot 7\). Now we want to divide these 4 factors into two disjoint subsets \(A, B\). Now, if
     \(A\) is chosen, then \(B\) is automatically chosen since \(A \cup B = \{2,3,5,7\}\). So the problem is equivalent to
     finding the number of ways to choose \(A\). However notice that each choice is repeated: for example, choosing \(A = \{
     2,3\}\) and \(B = \{5,7\}\) is the same as choosing \(A = \{ 5,7\}\) and \(B = \{2,3\}\).

     There are \(2^4 = 16\) ways to choose \(A\). Each choice of the \(A,B\) pair is repeated as \(B,A\), so there are 8 ways
     to write 210 as a product of two positive integer factors:

     \(1 \cdot 210\) (where \(A = \es, B = \{2,3,5,7\}\)),
\end{proof}

\subsubsection{(b)}
If \(n = p_1 p_2 p_3 p_4\), where the \(p_i\) are distinct prime numbers, how many ways can \(n\) be written as a
product of two positive integer factors?

\begin{proof}
     {\it Solution 1:} One factor can be 1, and the other factor can be the product of all the primes. (This gives 1
     factorization.) One factor can be one of the primes, and the other factor can be the product of the other 4 three. (This
     gives \(\binom{4}{1} = 4\) factorizations.) One factor can be a product of two of the primes, and the other factor can be a
     product of the two other primes. The number \(\binom{4}{2} = 6\) counts all possible sets of two primes chosen from the
     four primes, and each set of two primes corresponds to a factorization. Note, however, that the set \(\{p_1, p_2\}\)
     corresponds to the same factorization as the set \(\{p_3, p_4\}\), namely, \(p_1 p_2 p_3 p_4\) (just written in a
     different order). In general, each choice of two primes corresponds to the same factorization as one other choice of
     two primes. Thus the number of factorizations in which each factor is a product of two primes is \(\dps\frac{\binom{4}{2}}
     {2} = 3\) (This gives 3 factorizations.) The foregoing cases account for all the possibilities, so the answer is
     \(4 + 3 + 1 = 8\).

          {\it Solution 2:} Let \(S = \{p_1, p_2, p_3, p_4\}\). Let \(p_1 p_2 p_3 p_4 = P\), and let \(f_1 \cdot f_2\) be any
     factorization of \(P\). The product of the numbers in any subset \(A \subseteq S\) can be used for \(f_1\), with the
     product of the numbers in \(A^c\) being \(f_2\). There are as many ways to write \(f_1\) as there are subsets of \(S\),
     namely, \(2^4 = 16\) (by Theorem 6.3.1).

     However, because \(f_1 \cdot f_2 = f_2 \cdot f_1\), and because two factorizations are considered the same regardless of the
     order in which the factors are written, the number of ways to write \(P\) as a product of two factors is half the 16 number
     of subsets of \(S\). So the answer is \(\frac{16}{2} = 8\).
\end{proof}

\subsubsection{(c)}
If \(n = p_1 p_2 p_3 p_4 p_5\), where the \(p_i\) are distinct prime numbers, how many ways can \(n\) be written
as a product of two positive integer factors?

\begin{proof}
     By similar reasoning \(2^5 / 2 = 16\) ways.
\end{proof}

\subsubsection{(d)}
If \(n = p_1p_2 \cdots p_k\), where the \(p_i\) are distinct prime numbers, how many ways can \(n\) be written
as a product of two positive integer factors?

\begin{proof}
     By similar reasoning \(2^k / 2 = 2^{k-1}\) ways.
\end{proof}

\subsection{Exercise 25}
\subsubsection{(a)}
How many one-to-one functions are there from a set with three elements to a set with four elements?

\begin{proof}
     There are four choices for where to send the first element of the domain (any element of the co-domain may be chosen), three
     choices for where to send the second (since the function is one-to-one, the second element of the domain must go to a
     different element of the co-domain from the one to which the first element went), and two choices for where to send the
     third element (again since the function is one-to-one). Thus the answer is \(4 \cdot 3 \cdot 2 = 24\).
\end{proof}

\subsubsection{(b)}
How many one-to-one functions are there from a set with three elements to a set with two elements?

\begin{proof}
     None, because \(3 > 2\) so by the Pigeonhole Principle such a function cannot be one-to-one.
\end{proof}

\subsubsection{(c)}
How many one-to-one functions are there from a set with three elements to a set with three elements?

\begin{proof}
     \(3 \cdot 2 \cdot 1 = 6\) one-to-one functions.
\end{proof}

\subsubsection{(d)}
How many one-to-one functions are there from a set with three elements to a set with five elements?

\begin{proof}
     \(5 \cdot 4 \cdot 3 = 60\) one-to-one functions.
\end{proof}

\subsubsection{(e)}
How many one-to-one functions are there from \(H\) a set with \(m\) elements to a set with \(n\) elements, where
\(m \leq n\)?

\begin{proof}
     \(n\) choices for the first element of the domain, \(n-1\) choices for the second element of the domain, and so on.
     Therefore \(n(n-1)(n-2)\cdots(n-m+1)\) one-to-one functions.
\end{proof}

\subsection{Exercise 26}
\subsubsection{(a)}
How many onto functions are there from a set with three elements to a set with two elements?

\begin{proof}
     Let the elements of the domain be called \(a, b\), and \(c\) and the elements of the co-domain be called \(u\) and \(v\).
     In order for a function from \(\{a, b, c\}\) to \(\{u, v\}\) to be onto, two elements of the domain must be sent to \(u\)
     and one to \(v\), or two elements must be sent to \(v\) and one to \(u\). There are as many ways to send two elements of
     the domain to \(u\) and one to \(v\) as there are ways to choose which elements of \(\{a, b, c\}\) to send to \(u\),
     namely, \(\binom{3}{2} = 3\). Similarly, there are \(\binom{3}{2} = 3\) ways to send two elements of the domain to \(v\) and
     one to \(u\). Therefore, there are \(3 + 3 = 6\) onto functions from a set with three elements to a set with two elements.
\end{proof}

\subsubsection{(b)}
How many onto functions are there from a set with three elements to a set with five elements?

\begin{proof}
     None, because \(3 < 5\) and sending the 3 elements of the domain to the co-domain can cover at most 3 elements. At least
     2 remaining elements of the co-domain are left uncovered, so the function cannot be onto.
\end{proof}

\subsubsection{(c)}
How many onto functions are there from a set with three elements to a set with three elements?

\begin{proof}
     There are 3 ways to cover the first element of the co-domain. This consumes one element of the domain, leaving 2. Then there
     are 2 ways to cover the second element of the co-domain. This uses up the second element of the domain. So there is only 1
     way left to cover the last element of the co-domain, for the function to be onto. Therefore \(3 \cdot 2 \cdot 1 = 6\) onto
     functions.
\end{proof}

\subsubsection{(d)}
How many onto functions are there from a set with four elements to a set with two elements?

\begin{proof}
     Consider functions from a set with four elements to a set with two elements. Denote the set of four elements by \(X =
     \{a, b, c, d\}\) and the set of two elements by \(Y = \{u, v\}\).

     Divide the set of all onto functions from \(X\) to \(Y\) into two categories. The first category consists of all those that
     send the three elements in \(\{a, b, c\}\) onto \(\{u, v\}\) and that send \(d\) to either \(u\) or \(v\). The functions in
     this category can be defined by the following two-step process:

     {\cy Step 1:} Construct an onto function from \(\{a, b, c\}\) to \(\{u, v\}\).

          {\cy Step 2:} Choose whether to send \(d\) to \(u\) or to \(v\).

     By part (a), there are six ways to perform step 1, and, because there are two choices for where to send \(d\), there
     are two ways to perform step 2. Thus, by the multiplication rule, there are \(6 \cdot 2 = 12\) ways to define the
     functions in the first category.

     The second category consists of all the other onto functions from \(X\) to \(Y\): those that send all three elements in
     \(\{a, b, c\}\) to either \(u\) or \(v\) and that send \(d\) to whichever of \(u\) or \(v\) is not the image of \(a, b\),
     and \(c\). Because there are only two choices for where to send the elements in \(\{a, b, c\}\), and because \(d\) is
     simply sent to wherever \(a, b\), and \(c\) do not go, there are just two functions in the second category.

     Every onto function from \(X\) to \(Y\) either sends at least two elements of \(X\) to the image of \(d\) or it does not. If
     it does, then it is in the first category. If it does not, then it is in the second category. Therefore, all onto
     functions from \(X\) to \(Y\) are in one of the two categories and no function is in both categories. So the total number of
     onto functions is \(12 + 2 = 14\).
\end{proof}

\subsubsection{(e)}
How many onto functions are there from a set with four elements to a set with three elements?

\begin{proof}
     Let \(X = \{a, b, c, d\}\) and \(Y = \{u, v, w\}\). Similar to part (d), we categorize onto functions into two categories.

     The first category consists of functions that map \(\{a,b,c\}\) onto \(\{u,v,w\}\) and then send \(d\) to one of \(u,v\) or
     \(w\). The functions in this category can be defined by the following two-step process:

     {\cy Step 1:} Construct an onto function from \(\{a, b, c\}\) to \(\{u, v, w\}\).

          {\cy Step 2:} Choose whether to send \(d\) to \(u\), to \(v\) or to \(w\).

     By part (c) there are 6 ways to perform Step 1, and there are 3 ways to perform Step 2. So there are \(6 \cdot 3 = 18\) onto
     functions in the first category.

     The second category consists of functions that send \(\{a,b,c\}\) onto either \(\{u,v\}\), or onto \(\{u,w\}\), or
     onto \(\{v,w\}\), and send \(d\) to whichever element in the co-domain is left over. This can be thought of as a three step
     process:

     {\cy Step 1:} Choose a 2-element subset of the co-domain.

     {\cy Step 2:} Construct an onto function from \(\{a,b,c\}\) to the 2-element set from Step 1.

          {\cy Step 3:} Send \(d\) to whichever element is left over.

     There are \(\binom{3}{2} = 3\) ways to perform Step 1, then by part (a) there are 6 ways to perform Step 2. There is only one
     way to perform Step 3 since there is no choice. So there are \(6 \cdot 3 \cdot 1 = 18\) onto functions in this category.

     Again notice that the two categories are disjoint. Every onto function \(X \to Y\) either sends at least two elements to the
     image of \(d\) or it does not. If it does, it belongs to the first category. If it does not, then it belongs to the second category.

     So there are \(18+18 = 36\) onto functions from a set of 3 elements to a set of 3 elements.
\end{proof}

\subsubsection{(f)}
Let \(c_{m,n}\) be the number of onto functions from a set of \(m\) elements to a set of \(n\) elements, where
\(m \geq n \geq 1\). Find a formula relating \(c_{m,n}\) to \(c_{m-1,n}\) and \(c_{m-1,n-1}\).

\begin{proof}
     We can generalize the arguments of parts (d) and (e).

     There are \(c_{m-1,n} \cdot n\) onto functions in the first category: \(c_{m-1,n}\) ways to send the first \(m-1\)
     elements of the domain onto the \(n\) elements of the co-domain, then sending the last domain element to one of the
     \(n\) elements in the co-domain.

     There are \(c_{m-1,n-1} \cdot \binom{n}{n-1}\) onto functions in the second category: \(\binom{n}{n-1} = n\) ways to choose
     an \(n-1\) element subset of the co-domain, then \(c_{m-1,n-1}\) ways to send the first \(m-1\) domain elements onto that
     subset, and 1 way to send the remaining domain element to the remaining co-domain element.

     By the same argument, the two categories are disjoint. So \(c_{m,n} = n(c_{m-1,n} + c_{m-1,n-1})\).

     We can confirm the formula: in parts (a), (c), (d), (e) we found \(c_{3,2} = 6\), \(c_{3,3} = 6\), \(c_{4,2} = 14\) and
     \(c_{4,3} = 36\). By the formula \(36 = c_{4,3} = 3(c_{3,3} + c_{3,2}) = 3(6+6)\), true.
\end{proof}

\subsection{Exercise 27}
Let \(A\) be a set with eight elements.

\subsubsection{(a)}
How many relations are there on \(A\)?

\begin{proof}
     A relation on \(A\) is any subset of \(A \times A\), and \(A \times A\) has \(8^2 = 64\) elements. So there are \(2^{64}\)
     relations on \(A\).
\end{proof}

\subsubsection{(b)}
How many relations on \(A\) are reflexive?

\begin{proof}
     A reflexive relation is any subset of \(A \times A\) that contains ``the diagonal'' \(\{(a,a) \,|\, a \in A\}\). Since
     there are \(8^2 = 64\) elements in \(A \times A\) and the diagonal has 8 elements, there are as many reflexive relations
     as there are subsets of a \(64-8 = 56\) element set. So there are \(2^{56}\) reflexive relations.
\end{proof}

\subsubsection{(c)}
How many relations on \(A\) are symmetric?

\begin{proof}
     Form a relation that is symmetric by a two-step process:

     (1) pick a set of elements of the form \((a, a)\) (there are eight such elements, so \(2^8\) sets);

     (2) pick a set of pairs of elements of the form \((a, b)\) and \((b, a)\) where \(a \neq b\) (there are \((64 - 8)/2 = 28\)
     such pairs, so \(2^{28}\) such sets). The answer is therefore \(2^8 \cdot 2^{28} = 2^{36}\).
\end{proof}

\subsubsection{(d)}
How many relations on \(A\) are both reflexive and symmetric?

\begin{proof}
     We can follow the same procedure as in part (c), but to ensure reflexivity, the only possibility in Step 1 is to pick the
     entire diagonal \(\{(a,a) \,|\, a \in A\}\), so there is only 1 set. This gives us \(2^{28}\) relations that are both
     reflexive and symmetric.
\end{proof}

\subsection{Exercise 28}
A student council consists of three freshmen, four sophomores, four juniors, and five seniors. How many committees of eight
members of the council contain at least one member from each class?

\begin{proof}
     Choose one freshman: \(\binom{3}{1} = 3\), leaving 2.

     Choose one sophomore: \(\binom{4}{1} = 4\), leaving 3.

     Choose one junior: \(\binom{4}{1} = 4\), leaving 3.

     Choose one senior: \(\binom{5}{1} = 5\), leaving 4.

     This gives us 4 people chosen. We need to choose 4 more members to complete a committee of eight members. There are
     \(2+3+3+4 = 12\) students left to choose from, so \(\binom{12}{4} = 495\).

     Result = \(3 \cdot 4 \cdot 4 \cdot 5 \cdot 495 = 118,800\)
\end{proof}

\subsection{Exercise 29}
An alternative way to derive Theorem 9.5.1 uses the following division rule: Let \(n\) and \(k\) be integers so that \(k\)
divides \(n\). If a set consisting of \(n\) elements is divided into subsets that each contain \(k\) elements, then
the number of such subsets is \(n/k\). Explain how Theorem 9.5.1 can be derived using the division rule.

\begin{proof}
     Consider the set \(P\) of all permutations of an \(n\) element set \(A = \{a_1, \ldots, a_n\}\). \(P\) has \(n!\) elements.

     Let us divide \(P\) into subsets. For each \(r\)-element subset \(B\) of \(A\), let \(P_B \subseteq P\) be the set of
     permutations where, the first \(r\) elements are a permutation of \(B\), and the remaining \(n-r\) elements are a permutation
     of the remaining elements \(A - B\).

     To give an example, take \(n = 4, r = 2, B = \{a_2, a_4\}\), so \(P_B = \{a_2a_4a_1a_3, a_2a_4a_3a_1\), \\ \(a_4a_2a_1a_3,
     a_4a_2a_3a_1\}\).

     There are \(r!\) ways to permute the \(r\) elements of \(B\), and there are \((n-r)!\) ways to permute the remaining
     \(n-r\) elements of \(A - B\).

     Then each subset \(P_B \subseteq P\) contains \(r!(n-r)!\) permutations. By the division rule given in the problem, there
     are \(\dps \frac{n!}{r!(n-r)!}\) such subsets of \(P\).
     Each one of these subsets \(P_B \subseteq P\) corresponds one-to-one to an \(r\)-element subset \(B\) of \(A\). So there are
     as many ways of choosing an \(r\)-element subset of \(A\) as there are subsets \(P_B\) of \(P\).
\end{proof}

\subsection{Exercise 30}
Find the error in the following reasoning: “Consider forming a poker hand with two pairs as a five-step process.

     {\cy Step 1:} Choose the denomination of one of the pairs.

     {\cy Step 2:} Choose the two cards of that denomination.

     {\cy Step 3:} Choose the denomination of the other of the pairs.

     {\cy Step 4:} Choose the two cards of that second denomination.

     {\cy Step 5:} Choose the fifth card from the remaining denominations.

There are \(\binom{13}{1}\) ways to perform step 1, \(\binom{4}{2}\) ways to perform step 2, \(\binom{12}{1}\) ways
to perform step 3, \(\binom{4}{2}\) ways to perform step 4, and \(\binom{44}{1}\) ways to perform step 5. Therefore, the
total number of five-card poker hands with two pairs is \(13 \cdot 6 \cdot 12 \cdot 6 \cdot 44 = 247,104\).”

\begin{proof}
     The reasoning above leads to double-counting. First choosing the denomination of the first pair, let's say 4, then choosing
     the denomination of the second pair, let's say Q, is the same as first picking Q then picking 4.

     So the correct answer is \(247,104 / 2 = 123,552\). One way to see it is this: first choose TWO denominations out of 13 for
     the two pairs: \(\binom{13}{2}\). Then choose the two pairs: \(\binom{4}{2}\binom{4}{2}\). Then choose a fifth card from the
     remaining 44 cards: \(\binom{44}{1}\). The result is
     \[
          \binom{13}{2}\binom{4}{2}\binom{4}{2}\binom{44}{1} = 78 \cdot 6 \cdot 6 \cdot 44 = 123,552.
     \]
\end{proof}

\section{Exercise Set 9.6}
\subsection{Exercise 1}
\subsubsection{(a)}
According to Theorem 9.6.1, how many 5-combinations with repetition allowed can be chosen from a set of three elements?

\begin{proof}
     \(\dps \binom{5+3-1}{5} = \binom{7}{5} = \frac{7 \cdot 6}{2} = 21\)
\end{proof}

\subsubsection{(b)}
List all of the 5-combinations that can be chosen with repetition allowed from the set \(\{1, 2, 3\}\).

\begin{proof}
     The three elements of the set are 1, 2, and 3. The 5-combinations are \([1, 1, 1, 1, 1]\), \([1, 1, 1, 1, 2]\),
     \([1, 1, 1, 1, 3]\), \([1, 1, 1, 2, 2]\), \([1, 1, 1, 2, 3]\), \([1, 1, 1, 3, 3]\), \([1, 1, 2, 2, 2]\), \([1, 1, 2, 2, 3]\),
     \([1, 1, 3, 3, 3]\), \([1, 2, 2, 2, 2]\), \([1, 2, 2, 2, 3]\), \([1, 2, 2, 3, 3]\), \([1, 2, 3, 3, 3]\), \([1, 3, 3, 3, 3]\),
     \([2, 2, 2, 2, 2]\), \([2, 2, 2, 2, 3]\), \([2, 2, 2, 3, 3]\), \([2, 2, 3, 3, 3]\), \([2, 3, 3, 3, 3]\), and
     \([3, 3, 3, 3, 3]\).
\end{proof}

\subsection{Exercise 2}
\subsubsection{(a)}
According to Theorem 9.6.1, how many multisets of size four can be chosen from a set of three elements?

\begin{proof}
     \(\binom{4+3-1}{4} = \binom{6}{4} = \frac{6 \cdot 5}{2} = 15\)
\end{proof}

\subsubsection{(b)}
List all of the multisets of size four that can be chosen from the set \(\{x, y, z\}\).

\begin{proof}
     \(\{x, x, x, x\}\), \(\{x, x, x, y\}\), \(\{x, x, x, z\}\),
     \(\{x, x, y, y\}\), \(\{x, x, y, z\}\), \(\{x, x, z, z\}\),

     \(\{x, y, y, y\}\), \(\{x, y, y, z\}\), \(\{x, y, z, z\}\),
     \(\{x, z, z, z\}\), \(\{y, y, z, z\}\), \(\{y, y, y, z\}\),

     \(\{y, z, z, z\}\), \(\{y, y, y, y\}\), \(\{z, z, z, z\}\).
\end{proof}

\subsection{Exercise 3}
A bakery produces six different kinds of pastry, one of which is éclairs. Assume there are at least 20 pastries of each
kind.

\subsubsection{(a)}
How many different selections of twenty pastries are there?

\begin{proof}
     \(\binom{20+6-1}{20} = \binom{25}{20} = 53,130\)
\end{proof}

\subsubsection{(b)}
How many different selections of twenty pastries are there if at least three must be éclairs?

\begin{proof}
     If at least three are éclairs, then 17 additional pastries are selected from six kinds. The number of selections is
     \(\binom{17+6-1}{17} = \binom{22}{17} = 26,334\).

          {\it Note:} In parts (a) and (b), it is assumed that the selections being counted are unordered.
\end{proof}

\subsubsection{(c)}
How many different selections of twenty pastries contain at most two éclairs?

\begin{proof}
     Let \(T\) be the set of selections of pastry that may be any one of the six kinds, let \(E_{\geq 3}\) be the set of
     selections containing three or more éclairs, and let \(E_{\geq 2}\) be the set of selections containing two or fewer éclairs.
     Notice that \(T = E_{\geq 2} \cup E_{\geq 3}\) and \(E_{\geq 2} \cap E_{\geq 3} = \es\). So by the inclusion / exclusion
     principle \(N(T) = N(E_{\geq 2}) + N(E_{\geq 3}) - N(\es)\), and by parts (a) and (b) \(53,130=N(E_{\geq 2}) + 26,334 - 0\)
     and solving, we get \(N(E_{\geq 2}) = 26,976\).
\end{proof}

\subsection{Exercise 4}
A camera shop stocks eight different types of batteries, one of which is type A76. Assume there are at least 30 batteries
of each type.

\subsubsection{(a)}
How many ways can a total inventory of 30 batteries be distributed among the eight different types?

\begin{proof}
     \(\binom{30+8-1}{30} = \binom{37}{30} = 10,295,472\)
\end{proof}

\subsubsection{(b)}
How many ways can a total inventory of 30 batteries be distributed among the eight different types if the inventory must include at least four A76 batteries?

\begin{proof}
     We need to choose 26 more batteries: \(\binom{26+8-1}{26} = \binom{33}{26} = 4,272,048\)
\end{proof}

\subsubsection{(c)}
How many ways can a total inventory of 30 batteries be distributed among the eight different types if the inventory
includes at most three A76 batteries?

\begin{proof}
     Same reasoning as in Exercise 3(c), \(10,295,472 - 4,272,048 = 6,023,424\)
\end{proof}

\subsection{Exercise 5}
If \(n\) is a positive integer, how many 4-tuples of integers from 1 through \(n\) can be formed in which the elements of
the 4-tuple are written in increasing order but are not necessarily distinct? In other words, how many 4-tuples of
integers \((i, j, k, m)\) are there with \(1 \leq i \leq j \leq k \leq m \leq n\)?

\begin{proof}
     The answer equals the number of 4-combinations with repetition allowed that can be formed from a set of \(n\) elements. It is
     \(\binom{4+n-1}{4} = \binom{n+3}{4} = \frac{(n+3)(n+2)(n+1)n}{4 \cdot 3 \cdot 2 \cdot 1}\).
\end{proof}

\subsection{Exercise 6}
If \(n\) is a positive integer, how many 5-tuples of integers from 1 through \(n\) can be formed in which the elements of
the 5-tuple are written in decreasing order but are not
necessarily distinct? In other words, how many 5-tuples of
integers \((h, i, j, k, m)\) are there with \(n \geq h \geq i
\geq j \geq k \geq m \geq 1\)?

\begin{proof}
     The answer equals the number of 5-combinations with repetition allowed that can be formed from a set of \(n\) elements. It is
     \(\binom{5+n-1}{5} = \binom{n+4}{5} = \frac{(n+4)(n+3)(n+2)(n+1)n}{5 \cdot 4 \cdot 3 \cdot 2 \cdot 1}\).
\end{proof}

\subsection{Exercise 7}
Another way to count the number of nonnegative integral solutions to an equation of the form \(x_1 + x_2 + \ldots +
x_n = m\) is to reduce the problem to one of finding the number of \(n\)-tuples \((y_1, y_2, \ldots, y_n)\) with
\(0 \leq y_1 \leq y_2 \leq \ldots \leq y_n \leq m\). The reduction results from letting \(y_i = x_1 + x_2 + \ldots +
x_i\) for each \(i = 1, 2, \ldots, n\). Use this approach to derive a general formula for the number of nonnegative
integral solutions to \(x_1 + x_2 + \ldots + x_n = m\).

\begin{proof}
     Let the \(y_i\) be defined as above. If \(x_1 + x_2 + \ldots + x_n = m\) then \(y_n = m\). So we need to find the number of
     ways to choose \(n-1\) integers \(0 \leq y_1 \leq \cdots \leq y_{n-1} \leq m\). There are \(m+1\) integers to choose from:
     \(0, 1, \ldots, m\). So this can be done in \(\binom{(m+1) + (n-1) - 1}{n-1} = \binom{n+m-1}{n-1}\) ways.
\end{proof}

{\bf \cy In 8 and 9, how many times will the innermost loop be iterated when the algorithm segment is implemented and run?
Assume \(n, m, k\), and \(j\) are positive integers.}

\subsection{Exercise 8}
\begin{tabbing}
     {\bf for} \=\(m \coloneqq 1\) to \(n\) \\
     \>{\bf for} \=\(k \coloneqq 1\) to \(m\) \\
     \>         \>{\bf for} \=\(j\coloneqq 1\) to \(k\)\\
     \>         \>         \>{\bf for} \=\(i \coloneqq 1\) to \(j\) \\
     \>         \>         \>          \>{\it [Statements in the body of the inner loop, none} \\
     \>         \>         \>          \>{\it containing branching statements that lead outside the loop]} \\
     \>         \>         \>{\bf next} \(i\) \\
     \>         \>{\bf next} \(j\) \\
     \>{\bf next} \(k\) \\
     {\bf next} \(m\)
\end{tabbing}

\begin{proof}
     As in Example 9.6.4, the answer is the same as the number of quadruples of integers \((i, j, k, m)\) for which \(1 \leq i
     \leq j \leq k \leq m \leq n\). By exercise 5, this number is \(\binom{n+3}{4} = \frac{(n+3)(n+2)(n+1)n}{24}\)
\end{proof}

\subsection{Exercise 9}
\begin{tabbing}
     {\bf for} \=\(k \coloneqq 1\) to \(n\) \\
     \>{\bf for} \=\(j\coloneqq k\) to \(n\) \\
     \>         \>{\bf for} \=\(i \coloneqq j\) to \(n\) \\
     \>         \>          \>{\it [Statements in the body of the inner loop, none} \\
     \>         \>          \>{\it containing branching statements that lead outside the loop]} \\
     \>         \>{\bf next} \(i\) \\
     \>{\bf next} \(j\) \\
     {\bf next} \(k\)
\end{tabbing}

\begin{proof}
     As in Example 9.6.4, the answer is the same as the number of triples of integers \((i, j, k)\) for which \(1 \leq k \leq j
     \leq i \leq n\). By Example 9.6.4 this number is \(\binom{n+2}{3} = \frac{(n+2)(n+1)n}{6}\).
\end{proof}

{\bf \cy In 10–14, find how many solutions there are to the given equation that satisfy the given condition.}

\subsection{Exercise 10}
\(x_1 + x_2 + x_3 = 20\), each \(x_i\) is a nonnegative integer.

\begin{proof}
     Think of the number 20 as divided into 20 individual units and the variables \(x_1, x_2\), and \(x_3\) as three categories
     into which these units are placed. The number of units in category \(x_i\) indicates the value of \(x_i\), in a solution
     of the equation. By Theorem 9.6.1, the number of ways to select 20 objects from the three categories is
     \(\binom{20+3-1}{20} = \binom{22}{20} = 231\), so there are 231 non-negative integer solutions to the equation.
\end{proof}

\subsection{Exercise 11}
\(x_1 + x_2 + x_3 = 20\), each \(x_i\) is a positive integer.

\begin{proof}
     The analysis for this exercise is the same as for exercise 10 except that since each \(x_i \geq 1\), we can imagine taking 3
     of the 20 units, placing one in each category \(x_1, x_2\), and \(x_3\), and then distributing the remaining 17 units
     among the three categories. The number of ways to do this is \(\binom{17+3-1}{17} = \binom{19}{17} = \frac{19 \cdot 18}{2} =
     171\), so there are 171 positive integer solutions to the equation.
\end{proof}

\subsection{Exercise 12}
\(y_1 + y_2 + y_3 + y_4 = 30\), each \(y_i\) is a nonnegative integer.

\begin{proof}
     Like Exercise 10, the answer is \(\binom{30+4-1}{30} = \binom{33}{30} = \binom{33}{3} = \frac{33 \cdot 32 \cdot 31}{3
          \cdot 2 \cdot 1} = 11 \cdot 16 \cdot 31 = 5456\).
\end{proof}

\subsection{Exercise 13}
\(y_1 + y_2 + y_3 + y_4 = 30\), each \(y_i\) is an integer that is at least 2.

\begin{proof}
     Like in Exercise 11, we can think of taking 8 units and placing 2 units into each one of \(y_1,y_2,y_3,y_4\) and then
     distributing the remaining 22 units among the 4 categories. So \(\binom{22+4-1}{22} = \binom{25}{22} = \binom{25}{3} =
     \frac{25 \cdot 24 \cdot 23}{3 \cdot 2 \cdot 1} = 2300\).
\end{proof}

\subsection{Exercise 14}
\(a + b + c + d + e = 500\), each of \(a, b, c, d\), and \(e\) is an integer that is at least 10.

\begin{proof}
     Like Exercise 13, place 10 units into each one of the 5 categories, then distribute the remaining 450 units. The
     answer is \(\binom{450+5-1}{450} = \binom{454}{450} = 1,746,858,751\).
\end{proof}

\subsection{Exercise 15}
For how many integers from 1 through 99,999 is the sum of their digits equal to 10?

\begin{proof}
     Pad integers with zeros on the left. So think of single digit numbers as 00001, 00002, etc., think of two-digit numbers as
     00012, 00064 etc., and so on.

     So we need to distribute 10 units into 5 buckets, or digits. There are \(\binom{10+5-1}{10} = \binom{14}{10} = 1001\) ways
     to do that.

     However, some of the possibilities are not allowed. We cannot distribute all 10 units into a single digit, like 0-0-0-0-10,
     0-0-0-10-0, 0-0-10-0-0, 0-10-0-0-0, 10-0-0-0-0. There are 5 violations, so the answer is \(1001-5 = 996\).

     We can verify this with some code:
     \begin{minted}{scala}
scala> var count = 0
var count: Int = 0
scala> for i <- 1 to 999999 do
     |   if i.toString.map(_.asDigit).sum == 10 then
     |     count += 1
     |
scala> count
val res0: Int = 996
\end{minted}
\end{proof}

\subsection{Exercise 16}
Consider the situation in Example 9.6.2. (5 different types of soft drinks.)

\subsubsection{(a)}
Suppose the store has only six cans of lemonade but at least 15 cans of each of the other four types of soft drink. In how
many different ways can fifteen cans of soft drink be selected?

\begin{proof}
     Let \(T\) be the set of all possible selections assuming that there are at least 15 cans of each type. Then \(N(T) =
     \binom{15+5-1}{15} = \binom{19}{15} = 3876\).

     Let \(L_{\geq 7}\) be the set of all possible selections with at least 7 cans of lemonade. We can think of these selections
     as selecting \(15-7=8\) cans of soft drink from 5 types. So \(N(L_{\geq 7}) = \binom{8+5-1}{8} = \binom{12}{8} = 495\).

     Let \(L_{\leq 6}\) be the set of all possible selections with at most 6 cans of lemonade. Then \(L_{\leq 6} \cap L_{\geq 7}
     = \es\) and \(T = L_{\leq 6} \cup L_{\geq 7}\), thus \(N(L_{\leq 6}) = N(T) - N(L_{\geq 7}) = 3876 - 495 = 3381\).
\end{proof}

\subsubsection{(b)}
Suppose that the store has only five cans of root beer and only six cans of lemonade but at least 15 cans of each of the
other three types of soft drink. In how many different ways can fifteen cans of soft drink be selected?

\begin{proof}
     Let \(T, L_{\leq 6}, L_{\geq 7}\) be as in the solution to part (a). We know \(T = L_{\leq 6}\cup L_{\geq 7}\) and
     \(L_{\leq 6} \cap L_{\geq 7} = \es\).

     Let \(R_{\leq 5}, R_{\geq 6}\) be the set of all possible selections with, at most 5, and at least 6, cans of root beer,
     respectively. Similarly we know \(T = R_{\leq 5}\cup R_{\geq 6}\) and \(R_{\leq 5} \cap R_{\geq 6} = \es\).

     We want to find \(N(R_{\leq 5} \cap L_{\leq 6})\). In part (a) we found \(N(L_{\geq 7}) = 495\) and \(N(L_{\leq 6}) = 3381\).

     Since \(T = R_{\leq 5} \cup R_{\geq 6}\) and \(L_{\leq 6} \subseteq T\), we have by De Morgan laws
     \[
          L_{\leq 6} = L_{\leq 6} \cap T = L_{\leq 6} \cap (R_{\leq 5} \cup R_{\geq 6}) = (L_{\leq 6} \cap R_{\leq 5}) \cup
          (L_{\leq 6} \cap R_{\geq 6})
     \]
     Notice that \((L_{\leq 6} \cap R_{\leq 5})\) and \((L_{\leq 6} \cap R_{\geq 6})\) are disjoint, since \(R_{\leq 5}\) and
     \(R_{\geq 6}\) are disjoint. So by the Inclusion / Exclusion Principle and above,
     \[
          N(L_{\leq 6}) = N(L_{\leq 6} \cap R_{\leq 5}) + N(L_{\leq 6} \cap R_{\geq 6}).
     \]
     Then \(N(L_{\leq 6} \cap R_{\leq 5}) = 3381 - N(L_{\leq 6} \cap R_{\geq 6})\). We need to calculate \(N(L_{\leq 6} \cap
     R_{\geq 6})\).

     First, let's find \(N(R_{\geq 6})\). This is similar to how we found \(N(L_{\geq 7})\) in part (a). Think of having chosen 6
     cans of root beer, then distributing the remaining \(15-6=9\) cans among the 5 types: \(\binom{9+5-1}{9} = 715\).

     Since \(T = L_{\leq 6} \cup L_{\geq 7}\) and \(R_{\geq 6} \subseteq T\), we have by De Morgan laws
     \[
          R_{\geq 6} = R_{\geq 6} \cap T = R_{\geq 6} \cap (L_{\leq 6} \cup L_{\geq 7}) = (R_{\geq 6} \cap L_{\leq 6}) \cup
          (R_{\geq 6} \cap L_{\geq 7})
     \]
     Notice that \((R_{\geq 6} \cap L_{\leq 6})\) and \((R_{\geq 6} \cap L_{\geq 7})\) are disjoint, since \(L_{\leq 6}\) and
     \(L_{\geq 7}\) are disjoint. So by the Inclusion / Exclusion Principle and above,
     \[
          N(R_{\geq 6}) = N(R_{\geq 6} \cap L_{\leq 6}) + N(R_{\geq 6} \cap L_{\geq 7}).
     \]
     Then \(N(R_{\geq 6} \cap L_{\leq 6}) = 715 - N(R_{\geq 6} \cap L_{\geq 7})\).

     Let's calculate \(N(R_{\geq 6} \cap L_{\geq 7})\). We can think of it as having 6 cans of root beer and 7 cans of
     lemonade, then distributing the remaining \(15-(6+7) = 2\) cans among the 5 types of drinks. This can be done in
     \(\binom{2+5-1}{2} = \binom{6}{2} = 15\) ways.

     Then \(N(R_{\geq 6} \cap L_{\leq 6}) = 715 - 15 = 700\), so \(N(L_{\leq 6} \cap R_{\leq 5}) = 3381 - 700 = 2681\).
\end{proof}

\subsection{Exercise 17}
\subsubsection{(a)}
A store sells 8 colors of balloons with at least 30 of each color. How many different combinations of 30 balloons can be
chosen?

\begin{proof}
     \(\binom{30+8-1}{30} = \binom{37}{30} = 10,295,472\)
\end{proof}

\subsubsection{(b)}
If the store has only 12 red balloons but at least 30 of each other color of balloon, how many combinations of balloons can
be chosen?

\begin{proof}
     Using a notation similar to the previous exercise, \(T = R_{\leq 12} \cup R_{\geq 13}\) and \(R_{\leq 12} \cap
     R_{\geq 13} = \es\), so by Inclusion / Exclusion \(N(T) = N(R_{\leq 12}) + N(R_{\geq 13})\). By part (a) \(N(T) =
     10,295,472\) and by an argument similar to previous exercises \(N(R_{\geq 13}) = \binom{17+8-1}{17} = \binom{24}{17} =
     346,104\) so \(N(R_{\leq 12}) = 10,295,472 - 346,104 = 9,949,368\).
\end{proof}

\subsubsection{(c)}
If the store has only 8 blue balloons but at least 30 of each other color of balloon, how many combinations of balloons can
be chosen?

\begin{proof}
     Again by a similar reasoning \(N(T) = N(B_{\leq 8}) + N(B_{\geq 9})\) and \(N(B_{\geq 9}) = \binom{21 + 8 - 1}{21}
     = \binom{28}{21} = 1,184,040\), so \(N(B_{\leq 8}) = 10,295,472 - 1,184,040 = 9,111,432\).
\end{proof}

\subsubsection{(d)}
If the store has only 12 red balloons and only 8 blue balloons but at least 30 of each other color of balloon, how many
combinations of balloons can be chosen?

\begin{proof}
     Similar to part (c) of the previous exercise. We need to find \(N(R_{\leq 12} \cap B_{\leq 8})\). We have \(T = B_{\leq 8}
     \cup B_{\geq 9}\) and \(B_{\leq 8} \cap B_{\geq 9} = \es\), so
     \[
          R_{\leq 12} = R_{\leq 12} \cap T = R_{\leq 12} \cap (B_{\leq 8} \cup B_{\geq 9}) = (R_{\leq 12} \cap B_{\leq 8}) \cup
          (R_{\leq 12} \cap B_{\geq 9})
     \]
     Again \((R_{\leq 12} \cap B_{\leq 8})\) and \((R_{\leq 12} \cap B_{\geq 9})\) are disjoint. By the Inclusion / Exclusion
     Principle
     \[
          N(R_{\leq 12}) = N(R_{\leq 12} \cap B_{\leq 8}) + N(R_{\leq 12} \cap B_{\geq 9})
     \]
     So by part (b), \(N(R_{\leq 12} \cap B_{\leq 8}) = 9,949,368 - N(R_{\leq 12} \cap B_{\geq 9})\).

     Since \(T = R_{\leq 12} \cup R_{\geq 13}\) and \(B_{\geq 9} \subseteq T\) we have by De Morgan laws
     \[
          B_{\geq 9} = B_{\geq 9} \cap T = B_{\geq 9} \cap (R_{\leq 12} \cup R_{\geq 13}) = (B_{\geq 9} \cap R_{\leq 12}) \cup
          (B_{\geq 9} \cap R_{\geq 13})
     \]
     We can find \(N(B_{\geq 9} \cap R_{\geq 13})\) as follows: choose 13 red balloons and 9 blue balloons, then distribute
     the remaining \(30 - (13+9) = 8\) balloons among 8 colors, which can be done in \(\binom{8+8-1}{8} = \binom{15}{8} =
     6435\) ways.

     So \(N(B_{\geq 9}) = N(B_{\geq 9} \cap R_{\leq 12}) + N(B_{\geq 9} \cap R_{\geq 13})\) which gives \(1,184,040 =
     N(B_{\geq 9} \cap R_{\leq 12}) + 6435\), so \(N(B_{\geq 9} \cap R_{\leq 12}) = 1,184,040 - 6435 = 1,177,605\).

     Then \(N(R_{\leq 12} \cap B_{\leq 8}) = 9,949,368 - 1,177,605 = 8,771,763\).
\end{proof}

\subsection{Exercise 18}
A large pile of coins consists of pennies, nickels, dimes, and quarters. (4 types of coins)

\subsubsection{(a)}
How many different collections of 30 coins can be chosen if there are at least 30 of each kind of coin?

\begin{proof}
     \(\binom{30+4-1}{30} = \binom{33}{30} = 5456\)
\end{proof}

\subsubsection{(b)}
If the pile contains only 15 quarters but at least 30 of each other kind of coin, how many collections of 30 coins can be
chosen?

\begin{proof}
     Similar notation as before: \(N(T) = N(Q_{\leq 15}) + N(Q_{\geq 16})\). We can find \(N(Q_{\geq 16})\) as follows:
     choose 16 quarters, then distribute the remaining \(30-16=14\) coins among 4 types, which can be done in \(\binom{14+4-1}{14}
     = \binom{17}{14} = 680\) ways. By part (a) \(N(T) = 5456\) so \(N(Q_{\leq 15}) = 5456 - 680 = 4776\).
\end{proof}

\subsubsection{(c)}
If the pile contains only 20 dimes but at least 30 of each other kind of coin, how many collections of 30 coins can be
chosen?

\begin{proof}
     Similar to part (b). Choose 21 dimes then distribute the remaining \(30-21=9\) coins among 4 types in \(N(D_{\geq 21})
     = \binom{9+4-1}{9} = \binom{12}{9} = 220\) ways. So \(N(D_{\leq 20}) = 5456 - 220 = 5236\).
\end{proof}

\subsubsection{(d)}
If the pile contains only 15 quarters and only 20 dimes but at least 30 of each other kind of coin, how many collections of
30 coins can be chosen?

\begin{proof}
     Again similar arguments, so skipping the details. We want \(N(Q_{\leq 15} \cap D_{\leq 20})\). We can find \(N(Q_{\geq
               16})\) as before: \(\binom{(30-16)+4-1}{30-16} = \binom{17}{14} = 680\) and \(N(D_{\geq 21}) = \binom{(30-21)+4-1}{30-21}
     = \binom{12}{9} = 220\).

     So \(N(Q_{\leq 15}) = N(T) - N(Q_{\geq 16}) = 5456 - 680 = 4776\) and \(N(D_{\leq 20}) = N(T) - N(D_{\geq 21}) = 5456 -
     220 = 5236\). Using similar arguments as before:
     \[
          Q_{\leq 15} = Q_{\leq 15} \cap T = Q_{\leq 15} \cap (D_{\leq 20} \cup D_{\geq 21}) = (Q_{\leq 15} \cap D_{\leq 20}) \cup
          (Q_{\leq 15} \cap D_{\geq 21})
     \]
     So \(N(Q_{\leq 15}) = N(Q_{\leq 15} \cap D_{\leq 20}) + N(Q_{\leq 15} \cap D_{\geq 21})\) and \(N(Q_{\leq 15} \cap
     D_{\leq 20}) = N(Q_{\leq 15}) - N(Q_{\leq 15} \cap D_{\geq 21}) = 4776 - N(Q_{\leq 15} \cap D_{\geq 21})\).
     \[
          D_{\geq 21} = D_{\geq 21} \cap T = D_{\geq 21} \cap (Q_{\leq 15} \cup Q_{\geq 16}) = (D_{\geq 21} \cap Q_{\leq 15}) \cup
          (D_{\geq 21} \cap Q_{\geq 16}).
     \]
     Notice that \(N(D_{\geq 21} \cap Q_{\geq 16}) = 0\) because this would require at least 37 coins. So \(N(D_{\geq 21}) =
     N(D_{\geq 21} \cap Q_{\leq 15}) + N(D_{\geq 21} \cap Q_{\geq 16})\) and \(N(D_{\geq 21} \cap Q_{\leq 15}) = N(D_{\geq 21})
     - N(D_{\geq 21} \cap Q_{\geq 16}) = 220 - 0 = 220\).

     Finally \(N(Q_{\leq 15} \cap D_{\leq 20}) = 4776 - 220 = 4556\).
\end{proof}

\subsection{Exercise 19}
Suppose the bakery in exercise 3 has only ten éclairs but has at least twenty of each of the other kinds of pastry. (6 kinds
of pastry.)

\subsubsection{(a)}
How many different selections of twenty pastries are there?

\begin{proof}
     \(N(E_{\leq 10}) = N(T) - N(E_{\geq 10}) = 53130 - \binom{(20-10)+6-1}{20-10} = 53130 - \binom{15}{10} = 53130 -
     3003 = 50127\).
\end{proof}

\subsubsection{(b)}
Suppose in addition to having only ten éclairs, the bakery has only eight napoleon slices. How many different selections of twenty pastries are there?

\begin{proof}
     We need \(N(E_{\leq 10} \cap S_{\leq 8})\).
     \[
          E_{\leq 10} = E_{\leq 10} \cap T = E_{\leq 10} \cap (S_{\leq 8} \cup S_{\geq 9}) = (E_{\leq 10} \cap S_{\leq 8}) \cup (E_{\leq 10} \cap S_{\geq 9})
     \]
     So \(N(E_{\leq 10} \cap S_{\leq 8}) = N(E_{\leq 10}) - N(E_{\leq 10} \cap S_{\geq 9}) = 50127 - N(E_{\leq 10} \cap
     S_{\geq 9})\).
     \[
          S_{\geq 9} = S_{\geq 9} \cap T = S_{\geq 9} \cap (E_{\leq 10} \cup E_{\geq 11}) = (S_{\geq 9} \cap E_{\leq 10}) \cup
          (S_{\geq 9} \cap E_{\geq 11})
     \]
     So \(N(S_{\geq 9} \cap E_{\leq 10}) = N(S_{\geq 9}) - N(S_{\geq 9} \cap E_{\geq 11}) = \binom{(20-9)+6-1}{20-9} -
     \binom{(20-(9+11))+6-1}{20-(9+11)} = \binom{16}{11} - \binom{5}{0} = 4368-1 = 4367\). Finally \(N(E_{\leq 10} \cap
     S_{\leq 8}) = 50127 - 4367 = 45760\).
\end{proof}

\subsection{Exercise 20}
Suppose the camera shop in exercise 4 can obtain at most ten A76 batteries but can get at least 30 of each of the other
types. (8 different types)

\subsubsection{(a)}
How many ways can a total inventory of 30 batteries be distributed among the eight different types?

\begin{proof}
     Similar notation to previous exercises. By Exercise 4, \(N(T) = 10,295,472\). We can find \(N(A76_{\geq 11})\) as follows:
     choose 11 A76 batteries, then distribute the remaining \(30-11=19\) batteries among 8 types: \(\binom{19+8-1}{19} =
     \binom{26}{19} = 657,800\). So \(N(A76_{\leq 10}) = 10,295,472 - 657,800 = 9,637,672\).
\end{proof}

\subsubsection{(b)}
Suppose that in addition to being able to obtain only ten A76 batteries, the store can get only six of type D303. How many
ways can a total inventory of 30 batteries be distributed among the eight different types?

\begin{proof}
     We need to find \(N(A76_{\leq 10} \cap D303_{\leq 6})\). Now \(A76_{\leq 10} = \)
     \[
          A76_{\leq 10} \cap T = A76_{\leq 10} \cap (D303_{\leq 6} \cup D303_{\geq 7}) = (A76_{\leq 10} \cap
          D303_{\leq 6}) \cup (A76_{\leq 10} \cap D303_{\geq 7})
     \]
     So \(N(A76_{\leq 10} \cap D303_{\leq 6}) = N(A76_{\leq 10}) - N(A76_{\leq 10} \cap D303_{\geq 7}) = 9,637,672 -
     N(A76_{\leq 10} \cap D303_{\geq 7})\). Now \(D303_{\geq 7} = D303_{\geq 7} \cap T = \)
     \[
          D303_{\geq 7} \cap (A76_{\leq 10} \cup A76_{\geq 11}) = (D303_{\geq 7} \cap A76_{\leq 10}) \cup (D303_{\geq 7} \cap
          A76_{\geq 11})
     \]
     So \(N(D303_{\geq 7} \cap A76_{\leq 10}) = N(D303_{\geq 7}) - N(D303_{\geq 7} \cap A76_{\geq 11})\).

     Now by arguments similar to before, \(N(D303_{\geq 7}) = \binom{23+8-1}{23} = \binom{30}{23} = 2,035,800\) and
     \(N(D303_{\geq 7} \cap A76_{\geq 11}) = \binom{(30-18)+8-1}{30-18} = \binom{19}{12} = 50,388\), thus \(N(D303_{\geq 7}
     \cap A76_{\leq 10}) = 2,035,800 - 50,388 = 1,985,412\).

     Finally \(N(A76_{\leq 10} \cap D303_{\leq 6}) = 9,637,672 -
     1,985,412 = 7,652,260\).
\end{proof}

\subsection{Exercise 21}
Observe that the number of columns in the trace table for Example 9.6.4 can be expressed as the sum \(1 + (1 + 2) + (1 +
2 + 3) + \cdots + (1 + 2 + \cdots + n)\). Explain why this is so, and show how this sum simplifies to the same expression
given in the solution of Example 9.6.4. (Hint: A formula from exercise 13 in Section 5.2 will be helpful.)

\begin{proof}
     When \(k=1\), \(j\) ranges from 1 to \(k=1\), and \(i\) ranges from 1 to \(j=1\), so there is 1 column.

     When \(k=2\), \(j\) ranges first from 1 to \(k=1\), and then from 1 to \(k=2\); and \(i\) ranges first from 1 to \(j=1\),
     then again from 1 to \(j=1\), and then from 1 to \(j=2\), so there are 1 + (1+2) columns.

     When \(k=3\), \(j\) ranges first from 1 to \(k=1\), and then from 1 to \(k=2\) and then from 1 to \(k=3\); and \(i\) ranges
     first from 1 to \(j=1\), then again from 1 to \(j=1\) and then from 1 to \(j=2\), and then again from 1 to \(j=1\) then again
     from 1 to \(j=2\) and then from 1 to \(j=3\), so there are 1 + (1+2) + (1+2+3) columns.

     And so on. So the overall sum has the form \(\dps\sum_{i=1}^{n}(1+\cdots+i)\).

     Since \(1+\cdots+i = \frac{i(i+1)}{2}\),
     it becomes \(\dps \sum_{i=1}^{n}\frac{i(i+1)}{2} = \frac{1}{2}\sum_{i=1}^{n}i(i+1)\). Now by Exercise 13 in 5.2, we have
     \(\dps \frac{1}{2}\sum_{i=1}^{n}i(i+1) = \frac{1}{2} \cdot \frac{(n+1)(n+1-1)(n+1+1)}{3} = \frac{(n+1)n(n+2)}{6}\).
\end{proof}

\section{Exercise Set 9.7}
 {\bf \cy In \(1-4\), use theorem 9.5.1 to compute the values of the indicated quantities. (assume \(n\) is an integer.)}

\subsection{Exercise 1}
\(\binom{n}{0}\), for \(n \geq 0\)
\begin{proof}
     \(\frac{n!}{0!(n-0)!} = \frac{\Cyancel{n!}}{1 \cdot \Cyancel{n!}} = 1\)
\end{proof}

\subsection{Exercise 2}
\(\binom{n}{1}\), for \(n \geq 1\)
\begin{proof}
     \(\dps \frac{n\Cyancel{(n-1)!}}{1!\Cyancel{(n-1)!}} = n\)
\end{proof}

\subsection{Exercise 3}
\(\binom{n}{2}\), for \(n \geq 2\)
\begin{proof}
     \(\dps \frac{n(n-1)\Cyancel{(n-2)!}}{2!\Cyancel{(n-2)!}} = \frac{n(n-1)}{2}\)
\end{proof}

\subsection{Exercise 4}
\(\binom{n}{3}\), for \(n \geq 3\)
\begin{proof}
     \(\dps \frac{n(n-1)(n-2)\Cyancel{(n-3)!}}{3!\Cyancel{(n-3)!}} = \frac{n(n-1)(n-2)}{6}\)
\end{proof}

\subsection{Exercise 5}
Use Theorem 9.5.1 to prove algebraically that \(\binom{n}{r} = \binom{n}{n-r}\), for integers \(n,r\) with \(0\leq r\leq n\).
(This can be done by direct calculation; it is not necessary to use mathematical induction.)

\begin{proof}
     Suppose \(n\) and \(r\) are nonnegative integers and \(r \leq n\). Then
     \begin{center}
          \begin{tabular}{rcll}
               \(\dps\binom{n}{r}\) & = & \(\dps\frac{n!}{r!(n-r)!}\)         & {\cy by Theorem 9.5.1}      \\
                                    & = & \(\dps\frac{n!}{(n-(n-r))!(n-r)!}\) & {\cy since \(n-(n-r) = r\)} \\
                                    & = & \(\dps\frac{n!}{(n-r)!(n-(n-r))!}\) & {\cy by commutativity}      \\
                                    & = & \(\dps\binom{n}{n-r}\)              & {\cy by Theorem 9.5.1}
          \end{tabular}
     \end{center}
\end{proof}

{\bf \cy Justify the equations in \(6-9\) either by deriving them from formulas in Example 9.7.1 or by direct computation
from Theorem 9.5.1. Assume \(m, n, k\), and \(r\) are integers.}

\subsection{Exercise 6}
\(\dps\binom{m+k}{m+k-1} = m+k\), for \(m+k \geq 1\)
\begin{proof}
     {\it Solution 1:} Apply formula (9.7.2) with \(m + k\) in place of \(n\). This is legal because \(m + k \geq 1\).

          {\it Solution 2:}
     \begin{center}
          \begin{tabular}{rcll}
               \(\dps\binom{m+k}{m+k-1}\) & = & \(\dps\frac{(m+k)!}{(m+k-1)![(m+k)-(m+k-1)]!}\)       & {\cy by Theorem 9.5.1} \\
                                          & = & \(\dps\frac{(m+k)(m+k-1)!}{(m+k-1)![m+k-m-k+1]!}\)    & {\cy by algebra}       \\
                                          & = & \(\dps\frac{(m+k)(m+k-1)!}{(m+k-1)! \cdot 1!} = m+k\) & {\cy by algebra}
          \end{tabular}
     \end{center}
\end{proof}

\subsection{Exercise 7}
\(\dps\binom{n+3}{n+1} = \frac{(n+3)(n+2)}{2}\), for \(n \geq -1\)

\begin{proof}
     \begin{center}
          \begin{tabular}{rcll}
               \(\dps\binom{n+3}{n+1}\) & = & \(\dps\frac{(n+3)!}{(n+1)![(n+3)-(n+1)]!}\)      & {\cy by Theorem 9.5.1} \\
                                        & = & \(\dps\frac{(n+3)(n+2)(n+1)!}{(n+1)! \cdot 2!}\) & {\cy by algebra}       \\
                                        & = & \(\dps\frac{(n+3)(n+2)}{2}\)                     & {\cy by algebra}
          \end{tabular}
     \end{center}
\end{proof}

\subsection{Exercise 8}
\(\binom{k-r}{k-r} = 1\), for \(k-r \geq 0\)
\begin{proof}
     \(\binom{k-r}{k-r} = \frac{(k-r)!}{(k-r)![(k-r)-(k-r)]!} = \frac{\Cyancel{(k-r)!}}{\Cyancel{(k-r)!} \cdot 0!} = 1\)
\end{proof}

\subsection{Exercise 9}
\(\binom{2(n+1)}{2n} = (n+1)(2n+1)\), for \(n \geq 0\)
\begin{proof}
     \(\binom{2(n+1)}{2n} = \frac{(2n+2)!}{(2n)!(2n+2-2n)!} = \frac{(2n+2)(2n+1)(2n)!}{(2n)!\cdot 2!}=\frac{(2n+2)(2n+1)}{2}
     = (n+1)(2n+1)\)
\end{proof}

\subsection{Exercise 10}
\subsubsection{(a)}
Use Pascal’s triangle given in Table 9.7.1 to compute the values of \(\binom{6}{2}, \binom{6}{3}, \binom{6}{4},
\binom{6}{5}\).

\begin{proof}
     \(\binom{6}{2} = \binom{5}{2} + \binom{5}{1} = 10 + 5 = 15\),
     \(\binom{6}{3} = \binom{5}{3} + \binom{5}{2} = 10 + 10 = 20\)

     \(\binom{6}{4} = \binom{5}{4} + \binom{5}{3} = 5 + 10 = 15\),
     \(\binom{6}{5} = \binom{5}{5} + \binom{5}{4} = 1 + 5 = 6\)
\end{proof}

\subsubsection{(b)}
Use the result of part (a) and Pascal’s formula to compute
\(\binom{7}{3}, \binom{7}{4}, \binom{7}{5}\).

\begin{proof}
     \(\binom{7}{3} = \binom{6}{3} + \binom{6}{2} = 20 + 15 = 35\),
     \(\binom{7}{4} = \binom{6}{4} + \binom{6}{3} = 15 + 20 = 35\)

     \(\binom{7}{5} = \binom{6}{5} + \binom{6}{4} = 6 + 15 = 21\)
\end{proof}

\subsubsection{(c)}
Complete the row of Pascal’s triangle that corresponds to \(n = 7\).

\begin{proof}
     1 7 21 35 35 21 7 1
\end{proof}

\subsection{Exercise 11}
The row of Pascal’s triangle that corresponds to \(n = 8\) is as follows: 1 8 28 56 70 56 28 8 1. What is the row that
corresponds to \(n = 9\)?

\begin{proof}
     1 9 26 84 126 126 84 26 9 1
\end{proof}

\subsection{Exercise 12}
Use Pascal’s formula repeatedly to derive a formula for \(\binom{n+3}{r}\) in terms of values of \(\binom{n}{k}\) with
\(k \leq r\). (Assume \(n\) and \(r\) are integers with \(n \geq r \geq 3\).)

\begin{proof}
     \(\binom{n+3}{r} = \binom{n+2}{r} + \binom{n+2}{r-1} = \left[\binom{n+1}{r} + \binom{n+1}{r-1}\right] +
     \left[\binom{n+1}{r-1} + \binom{n+1}{r-2}\right]\)
     \newpage
     \(= \left[\left(\binom{n}{r} + \binom{n}{r-1}\right) + \left(\binom{n}{r-1} + \binom{n}{r-2}\right)\right] +
     \left[\left(\binom{n}{r-1} + \binom{n}{r-2}\right) + \left(\binom{n}{r-2} + \binom{n}{r-3}\right)\right]\)
\end{proof}

\subsection{Exercise 13}
Use Pascal’s formula to prove by mathematical induction that if \(n\) is an integer and \(n \geq 1\), then
\[
     \sum_{i=2}^{n+1}\binom{i}{2} = \binom{2}{2} + \cdots + \binom{n+1}{2} = \binom{n+2}{3}.
\]
\begin{proof}
     Let \(P(n)\) be the statement of the formula above.

          {\bf Show that \(P(1)\) is true:} The left hand side is \(\sum_{i=2}^{1+1}\binom{i}{2} = \binom{2}{2} = 1\) and the
     right side is \(\binom{1+2}{3} = \binom{3}{3} = 1\). So \(P(1)\) is true.

          {\bf Show that for any integer \(k \geq 1\) if \(P(k)\) is true then \(P(k+1)\) is true:} Assume \(k \geq 1\) is any
     integer such that \(\sum_{i=2}^{k+1} \binom{i}{2} = \binom{k+2}{3}\). {\it [This is the inductive hypothesis.]}
     \begin{center}
          \begin{tabular}{rcll}
               \(\dps \sum_{i=2}^{k+2} \binom{i}{2}\) & = & \(\dps \sum_{i=2}^{k+1} \binom{i}{2} + \binom{k+2}{2}\) & {\cy separating last
               term}                                                                                                                                \\
                                                      & = & \(\dps \binom{k+2}{3} + \binom{k+2}{2}\)                & {\cy by inductive hypothesis} \\
                                                      & = & \(\dps \binom{(k+2)+1}{3}\)                             & {\cy by Pascal's formula}     \\
                                                      & = & \(\dps \binom{k+3}{3}\)                                 & {\cy by algebra}
          \end{tabular}
     \end{center}
     which is the right-hand side of \(P(k + 1)\) {\it [as was to be shown]}. {\it [Since we have proved the basis step and the
          inductive step, we conclude that \(P(n)\) is true for all \(n \geq 1\).]}
\end{proof}

\subsection{Exercise 14}
Prove that if \(n\) is an integer and \(n \geq 1\), then \(1 \cdot 2 + 2 \cdot 3 + \cdots + n(n + 1) = 2 \binom{n+2}{3}\).

\begin{proof}
     By Exercise 3, \(\binom{i}{2} = \frac{i(i-1)}{2}\).
     So \(1 \cdot 2 + 2 \cdot 3 + \cdots + n(n + 1) = 2 \left( \frac{1 \cdot 2}{2} + \cdots + \frac{n(n+1)}{2}\right) =
     2\left(\binom{2}{2} + \cdots + \binom{n+1}{2}\right)\).
     By Exercise 13, the last expression is equal to \(2 \binom{n+2}{3}\).
\end{proof}

\subsection{Exercise 15}
Prove the following generalization of exercise 13: Let \(r\) be a fixed nonnegative integer. For every integer \(n\) with
\(n \geq r\),
\[
     \sum_{i=r}^{n}\binom{i}{r} = \binom{n+1}{r+1}
\]
\begin{proof}
     Let \(P(n)\) be the statement of the equation above.

          {\bf Show that \(P(r)\) is true:} The left hand side is
     \(\sum_{i=r}^r \binom{i}{r} = \binom{r}{r} = 1\), the right
     hand side is \(\binom{r+1}{r+1} = 1\). {\it [So \(P(r)\) is true.]}

          {\bf Show that for any integer \(k \geq r\) if \(P(k)\) is true then \(P(k+1)\) is true:} Assume \(k \geq r\) and assume
     \(\sum_{i=r}^{k}\binom{i}{r} = \binom{k+1}{r+1}\). {\it [This is the inductive hypothesis.]} Then
     \begin{center}
          \begin{tabular}{rcll}
               \(\dps \sum_{i=r}^{k+1} \binom{i}{r}\) & = & \(\dps \sum_{i=r}^{k} \binom{i}{r} + \binom{k+1}{r}\) & {\cy separating last
               term}                                                                                                                              \\
                                                      & = & \(\dps \binom{k+1}{r+1} + \binom{k+1}{r}\)            & {\cy by inductive hypothesis} \\
                                                      & = & \(\dps \binom{(k+1)+1}{r+1}\)                         & {\cy by Pascal's formula}     \\
                                                      & = & \(\dps \binom{k+2}{r+1}\)                             & {\cy by algebra}
          \end{tabular}
     \end{center}
     which is the right-hand side of \(P(k + 1)\) {\it [as was to be shown]}. {\it [Since we have proved the basis step and the
          inductive step, we conclude that \(P(n)\) is true for all \(n \geq r\).]}
\end{proof}

\subsection{Exercise 16}
Think of a set with \(m + n\) elements as composed of two parts, one with \(m\) elements and the other with \(n\)
elements. Give a combinatorial argument to show that
\[
     \binom{m+n}{r} = \binom{m}{0}\binom{n}{r} + \binom{m}{1}\binom{n}{r-1} + \cdots + \binom{m}{r}\binom{n}{0}
\]
where \(m\) and \(n\) are positive integers and \(r\) is an integer that is less than or equal to both \(m\) and \(n\).
This identity gives rise to many useful additional identities involving the quantities \(\binom{n}{k}\). Because Alexander
Vandermonde published an influential article about it in 1772, it is generally called the Vandermonde convolution. However, it was known at least in the 1300s in China by Chu Shih-chieh.

\begin{proof}
     This follows from the addition and multiplication rules. We want to choose \(r\) elements from a set \(K\) of \(m+n\)
     elements. We can think of \(K\) as the union of two disjoint sets \(M\) and \(N\) of \(m\) and \(n\) elements respectively.
     The possibilities for an \(r\) element subset are:

     0 elements from \(M\) and \(r\) elements from \(N\) (\(\binom{m}{0}\binom{n}{r}\) ways to choose),

     1 element from \(M\) and \(r-1\) elements from \(N\) (\(\binom{m}{1}\binom{n}{r-1}\) ways to choose),

     \(\vdots\)

     \(r\) elements from \(M\) and 0 elements from \(N\) (\(\binom{m}{r}\binom{n}{0}\) ways to choose).

     These possibilities are disjoint, then the formula follows by the addition rule.
\end{proof}

\subsection{Exercise 17}
Prove that for every integer \(n \geq 0\),
\[
     \binom{n}{0}^2 + \binom{n}{1}^2 + \cdots + \binom{n}{n}^2 = \binom{2n}{n}.
\]
\begin{proof}
     By the previous exercise, with \(m=n\) and \(r=n\), we get
     \[
          \binom{n+n}{n} = \binom{n}{0}\binom{n}{n} + \binom{n}{1}\binom{n}{n-1} + \cdots + \binom{n}{n}\binom{n}{0}
     \]
     Since \(\binom{n}{i} = \binom{n}{n-i}\) for all \(i = 0, 1, \ldots, n\), we have \(\binom{n}{n} = \binom{n}{0}\) and
     \(\binom{n}{n-1} = \binom{n}{1}\) and so on. Therefore
     \[
          \binom{2n}{n} = \binom{n}{0}\binom{n}{0} + \binom{n}{1}\binom{n}{1} + \cdots + \binom{n}{n}\binom{n}{n}
     \]
     {\it [as was to be shown.]}
\end{proof}

\subsection{Exercise 18}
Let \(m\) be any nonnegative integer. Use mathematical induction and Pascal’s formula to prove that for every integer
\(n \geq 0\),
\[
     \binom{m}{0} + \binom{m+1}{1} + \cdots + \binom{m+n}{n} = \binom{m+n+1}{n}
\]
\begin{proof}
     Let \(P(n)\) be the statement above.

          {\bf Show that \(P(0)\) is true:} The left hand side is \(\binom{m}{0} = 1\) and the right hand side is \(\binom{m+1}
     {0} = 1\), so \(P(0)\) is true.

          {\bf Show that for any integer \(k \geq 0\), if \(P(k)\) is true then \(P(k+1)\) is true:} Assume \(k \geq 0\) and assume
     \[
          \binom{m}{0} + \binom{m+1}{1} + \cdots + \binom{m+k}{k} = \binom{m+k+1}{k} \,\,\, {\cy \from P(k) \text{(inductive hypothesis)}}
     \]
     Then \(\binom{m}{0} + \binom{m+1}{1} + \cdots + \binom{m+k}{k}+ \binom{m+k+1}{k+1} = \binom{m+k+1}{k} + \binom{m+k+1}{k+1}\)
     by the inductive hypothesis. By Pascal's formula \(\binom{m+k+1}{k} + \binom{m+k+1}{k+1} = \binom{m+k+2}{k+1}\),
     which proves \(P(k+1)\).
\end{proof}

{\bf \cy Use the binomial theorem to expand the expressions in \(19-27\).}

\subsection{Exercise 19}
\((1+x)^7\)
\begin{proof}
     \(1 + 7x + 21x^2 + 35x^3 + 35x^4 + 21x^5 + 7x^6 + x^7\)
\end{proof}

\subsection{Exercise 20}
\((p+q)^6\)
\begin{proof}
     \(p^6 + 6p^5q + 15p^4q^2 + 20p^3q^3 + 15p^2q^4 + 6pq^5 + q^6\)
\end{proof}

\subsection{Exercise 21}
\((1-x)^6\)
\begin{proof}
     \(1 - 6x + 15x^2 - 20x^3 + 15x^4 - 6x^5 + x^6\)
\end{proof}

\subsection{Exercise 22}
\((u-v)^5\)
\begin{proof}
     \(u^5 -5u^4v +10u^3v^2 -10u^2v^3 +5uv^4 -v^5\)
\end{proof}

\subsection{Exercise 23}
\((p-2q)^4\)
\begin{proof}
     \(p^4 - 8p^3q + 24p^2q^2 - 32pq^3 + 16q^4\)
\end{proof}

\subsection{Exercise 24}
\((u-3v)^4\)
\begin{proof}
     \(u^4 - 12u^3v + 54u^2v^2 - 108uv^3 + 81v^4\)
\end{proof}

\subsection{Exercise 25}
\(\left(x+\frac{1}{x}\right)^5\)
\begin{proof}
     \(x^5 + 5x^3 + 10x + \frac{10}{x} + \frac{5}{x^3} + \frac{1}{x^5}\)
\end{proof}

\subsection{Exercise 26}
\(\left(\frac{3}{a}-\frac{a}{3}\right)^5\)
\begin{proof}
     \(\frac{243}{a^5} - \frac{135}{a^3} + \frac{30}{a} - \frac{10a}{3} + \frac{5a^3}{27} - \frac{a^5}{243}\)
\end{proof}

\subsection{Exercise 27}
\(\left(x^2+\frac{1}{x}\right)^5\)
\begin{proof}
     \(x^{10} + 5x^7 + 10x^4 + 10x + \frac{5}{x^2} + \frac{1}{x^5}\)
\end{proof}

\subsection{Exercise 28}
In Example 9.7.5 it was shown that \((a + b)^5 = a^5 + 5a^4b + 10a^3b^2 + 10a^2b^3 + 5ab^4 + b^5\). Evaluate \((a + b)^6\) by
substituting the expression above into the equation \((a + b)^6 = (a + b)(a + b)^5\) and then multiplying out and
combining like terms.

\begin{proof}
     \((a + b)^6\) = \((a + b)(a + b)^5\) = \((a+b)(a^5 + 5a^4b + 10a^3b^2 + 10a^2b^3 + 5ab^4 + b^5)\) = \((a^6 + 5a^5b +
     10a^4b^2 + 10a^3b^3 + 5a^2b^4 + ab^5)\) + \((a^5b + 5a^4b^2 + 10a^3b^3 + 10a^2b^4 + 5ab^5 + b^6)\) = \(a^6 + 6a^5b + 15a^4b^2 + 20a^3b^3 + 15a^2b^4 + 6ab^5 + b^6\)
\end{proof}

{\bf \cy In \(29-34\), find the coefficient of the given term when the expression is expanded by the binomial theorem.}

\subsection{Exercise 29}
\(x^6y^3\) in \((x+y)^9\)
\begin{proof}
     The term is \(\binom{9}{3}x^6y^3 = 84x^6y^3\), so the coefficient is 84.
\end{proof}

\subsection{Exercise 30}
\(x^7\) in \((2x+3)^{10}\)
\begin{proof}
     The term is \(\binom{10}{3}(2x)^7 3^3 = 120(128x^7)(27)\), so the coefficient is \(120 \cdot 128 \cdot 27 = 414,720\).
\end{proof}

\subsection{Exercise 31}
\(a^5b^7\) in \((a-2b)^{12}\)
\begin{proof}
     The term is \(\dps \binom{12}{7}a^5(-2b)^7 = 792a^5(-128b^7)\), so the coefficient is \(792 \cdot (-128) = -101,376\).
\end{proof}

\subsection{Exercise 32}
\(u^{16}v^4\) in \((u^2-v^2)^{10}\)
\begin{proof}
     The term is \(\dps \binom{10}{2}(u^2)^8(-v^2)^2 = 45u^{16}v^4\), so the coefficient is 45.
\end{proof}

\subsection{Exercise 33}
\(p^{16}q^7\) in \((3p^2-2q)^{15}\)
\begin{proof}
     The term is \(\dps \binom{15}{7}(3p^2)^8(-2q)^7 = 6435(3^8)(-2)^7p^{16}q^7\), so the coefficient is \(-6435 \cdot 3^8
     \cdot 2^7 = -5,404,164,480\).
\end{proof}

\subsection{Exercise 34}
\(x^9y^{10}\) in \((2x-3y^2)^{14}\)
\begin{proof}
     The term is \(\binom{14}{5}(2x)^9(-3y)^5 = 2002(512)(-243)x^9y^5\), so the coefficient is \(-2002 \cdot 512 \cdot
     243 = -249,080,832\).
\end{proof}

\subsection{Exercise 35}
As in the proof of the binomial theorem, transform the summation
\[
     \sum_{k=0}^{n}\binom{m}{k}a^{m-k}b^{k+1}
\]
by making the change of variable \(j = k + 1\).
\begin{proof}
     \[
          \sum_{k=0}^{n}\binom{m}{k}a^{m-k}b^{k+1} = \sum_{j=1}^{n+1}\binom{m}{j-1}a^{m-(j-1)}b^j
     \]
\end{proof}

{\bf \cy Use the binomial theorem to prove each statement in \(36-41\).}

\subsection{Exercise 36}
For every integer \(n \geq 1\),
\[
     \binom{n}{0} - \binom{n}{1} + \binom{n}{2} - \cdots + (-1)^n \binom{n}{n} = 0.
\]
({\it Hint:} Use the fact that \(1 + (-1) = 0\).)

\begin{proof}
     By the binomial theorem
     \[
          0 = 1-1 = (1-1)^n = \sum_{i=0}^{n}\binom{n}{i}1^{n-i}(-1)^i = \binom{n}{0} - \binom{n}{1} + \cdots + (-1)^n \binom{n}{n}
     \]
\end{proof}

\subsection{Exercise 37}
For every integer \(n \geq 0\), \( 3^n = \binom{n}{0} + 2\binom{n}{1} + 2^2\binom{n}{2} + \cdots + 2^n\binom{n}{n}.\)

\begin{proof}
     By the binomial theorem \(\dps 3^n = (1+2)^n = \sum_{i=0}^{n}\binom{n}{i}1^{n-i}2^i = \binom{n}{0} + 2\binom{n}{1} + \cdots
     + 2^n\binom{n}{n}\).
\end{proof}

\subsection{Exercise 38}
For every integer \(m \geq 0\), \(\dps \sum_{i=0}^{m} (-1)^i \binom{m}{i}2^{m-i} = 1\).

\begin{proof}
     By the binomial theorem \(1 = 1^m = (2-1)^m = \sum_{i=0}^{m} (-1)^i \binom{m}{i}2^{m-i}\).
\end{proof}

\subsection{Exercise 39}
For every integer \(n \geq 0\), \(\dps \sum_{i=0}^{n} (-1)^i \binom{n}{i}3^{n-i} = 2^n\).

\begin{proof}
     By the binomial theorem \(2^n = (3-1)^n = \sum_{i=0}^{n} (-1)^i \binom{n}{i}3^{n-i}.\)
\end{proof}

\subsection{Exercise 40}
For every integer \(n \geq 0\) and for every nonnegative real number \(x\), \(1 + nx \leq (1 + x)^n\).

\begin{proof}
     By the binomial theorem
     \[
          (1+x)^n = \sum_{i=0}^{n} \binom{n}{i}1^{n-i}x^i = 1 + nx + \binom{n}{2}x^2 + \cdots + x^n
     \]
     Since \(x \geq 0\) we have \(\binom{n}{2}x^2 + \cdots + x^n \geq 0\), therefore \((1+x)^n \geq 1 + nx\).
\end{proof}

\subsection{Exercise 41}
For every integer \(n \geq 1\),
\[
     \binom{n}{0} - \frac{1}{2}\binom{n}{1} + \frac{1}{2^2}\binom{n}{2} - \frac{1}{2^3}\binom{n}{3} + \cdots + (-1)^{n-1}
     \frac{1}{2^{n-1}}\binom{n}{n-1} =
     \left\{
     \begin{tabular}{ll}
          \(0\)                 & if \(n\) is even \\
          \(\frac{1}{2^{n-1}}\) & if \(n\) is odd
     \end{tabular}
     \right.
\]
\begin{proof}
     By the binomial theorem,
     \[
          \frac{1}{2^n} = (1-\frac{1}{2})^n = \binom{n}{0}- \frac{1}{2}\binom{n}{1} + \frac{1}{2^2} \binom{n}{2} - \cdots +
          (-1)^{n-1} \frac{1}{2^{n-1}} \binom{n}{n-1} + (-1)^{n} \frac{1}{2^{n}}\binom{n}{n}
     \]
     Notice that \(\binom{n}{n} = 1\), so moving the last term of the right hand side to the left, we get
     \[
          \frac{1}{2^n} - (-1)^{n} \frac{1}{2^{n}} = \binom{n}{0}- \frac{1}{2}\binom{n}{1} + \frac{1}{2^2} \binom{n}{2} - \cdots
          + (-1)^{n-1} \frac{1}{2^{n-1}} \binom{n}{n-1}
     \]
     Finally notice that the left hand side is 0 when \(n\) is even, and \(2 \cdot \frac{1}{2^n} = \frac{1}{2^{n-1}}\) when
     \(n\) is odd, {\it [as was to be shown.]}
\end{proof}

\subsection{Exercise 42}
Use mathematical induction to prove that for every integer \(n \geq 1\), if \(S\) is a set with \(n\) elements, then \(S\)
has the same number of subsets with an even number of elements as with an odd number of elements. Use this fact to give a
combinatorial argument to justify the identity of Exercise 36.

\begin{proof}
     Let \(P(n)\) be the statement ``if \(S\) is a set with \(n\) elements, then \(S\) has the same number of subsets with an
     even number of elements as with an odd number of elements.''

     {\bf Show that \(P(1)\) is true:} If \(S = \{s\}\) has only 1 element, then there is only 1 subset with an even number of
     elements, which is \(\es\), and there is only 1 subset with an odd number of elements, which is \(\{s\}\). Since \(1=1\),
     \(P(1)\) is true.

          {\bf Show that for any integer \(k \geq 1\) if \(P(k)\) is true then \(P(k+1)\) is true:} Assume \(k \geq 1\) and assume
     \(P(k)\) is true. Assume \(S = \{s_1, \ldots, s_{k+1}\}\) is a set with \(k+1\) elements.

     Define \(S' = \{s_1, \ldots, s_k\}\). Then \(S'\) has \(k\) elements. Let \(S'_{even}\) be the set of subsets of \(S'\)
     with an even number of elements, and let \(S'_{odd}\) be the set of subsets of \(S'\) with an odd number of elements. By
     the inductive hypothesis, \(S'_{even}\) and \(S'_{odd}\) have the same number of elements, say \(N\).

     Let \(S_{even}\) be the set of subsets of \(S\) with an even number of elements, and let \(S_{odd}\) be the set of subsets
     of \(S\) with an odd number of elements.

     \(S_{even}\) is the disjoint union of two sets: \(S_{even} = S'_{even} \cup A\) where \(A\) is the set of subsets of \(S\) with an even number of elements that contain \(s_{k+1}\).

     Similarly \(S_{odd}\) is the disjoint union of two sets: \(S_{odd} = S'_{odd} \cup B\) where \(B\) is the set of subsets
     of \(S\) with an odd number of elements that contain \(s_{k+1}\).

     Notice that there is a 1-1 correspondence between \(A\) and \(S'_{odd}\): if \(X \in A\) then \((X - \{s_{k+1}\}) \in
     S'_{odd}\), and vice versa. So \(A\) has \(N\) elements too.

     Similarly notice that there is a 1-1 correspondence between \(B\) and \(S'_{even}\): if \(X \in B\) then \((X - \{s_{k+1}\}) \in S'_{even}\), and vice versa. So \(B\) has \(N\) elements too.

     This shows that both \(S_{even}\) and \(S_{odd}\) each have \(2N\) elements, {\it [as was to be shown.]}

     The identity of Exercise 36 can be interpreted as follows: the positive terms \(\binom{n}{0},\binom{n}{2},\ldots\) correspond
     to the number of ways to choose the subsets with an even number of elements; and the negative terms \(\binom{n}{1},
     \binom{n}{3}, \ldots\) correspond to the number of ways to choose the subsets with an odd number of elements. By this
     exercise, the sums of these two quantities are equal, so if we subtract them, we get 0, which proves the identity.
\end{proof}

{\bf \cy Express each of the sums in \(43-54\) in closed form (without using a summation symbol and without using an
ellipsis \(\ldots\)).}

\subsection{Exercise 43}
\(\sum_{k=0}^{n} \binom{n}{k} 5^{k}\)
\begin{proof}
     By the binomial theorem this is \((1+5)^n = 6^n\).
\end{proof}

\subsection{Exercise 44}
\(\dps \sum_{i=0}^{m} \binom{m}{i} 4^{i}\)
\begin{proof}
     By the binomial theorem this is \((1+4)^m = 5^m\).
\end{proof}

\subsection{Exercise 45}
\(\dps \sum_{i=0}^{n} \binom{n}{i} x^{i}\)
\begin{proof}
     By the binomial theorem this is \((1+x)^n\).
\end{proof}

\subsection{Exercise 46}
\(\dps \sum_{k=0}^{m} \binom{m}{k} 2^{m-k}x^k\)
\begin{proof}
     By the binomial theorem this is \((2+x)^m\).
\end{proof}

\subsection{Exercise 47}
\(\dps \sum_{j=0}^{2n} (-1)^j \binom{2n}{j} x^{j}\)
\begin{proof}
     By the binomial theorem this is \((1-x)^{2n}\).
\end{proof}

\subsection{Exercise 48}
\(\dps \sum_{r=0}^{n} \binom{n}{r} x^{2r}\)
\begin{proof}
     By the binomial theorem this is \((1+x^2)^n\).
\end{proof}

\subsection{Exercise 49}
\(\dps \sum_{i=0}^{m} \binom{m}{i} p^{m-i}q^{2i}\)
\begin{proof}
     By the binomial theorem this is \((p+q^2)^m\).
\end{proof}

\subsection{Exercise 50}
\(\dps \sum_{k=0}^{n} \binom{n}{k} \frac{1}{2^k}\)
\begin{proof}
     By the binomial theorem this is \((1+\frac{1}{2})^n = \frac{3^n}{2^n}\).
\end{proof}

\subsection{Exercise 51}
\(\dps \sum_{i=0}^{m} (-1)^{i} \binom{m}{i} \frac{1}{2^i}\)
\begin{proof}
     By the binomial theorem this is \((1-\frac{1}{2})^m = \frac{1}{2^m}\).
\end{proof}

\subsection{Exercise 52}
\(\dps \sum_{k=0}^{n} \binom{n}{k} 3^{2n-2k}2^{2k}\)
\begin{proof}
     By the binomial theorem this is \((3^2+2^2)^n = 13^n\).
\end{proof}

\subsection{Exercise 53}
\(\dps \sum_{i=0}^{n} (-1)^i \binom{n}{i} 5^{n-i}2^i\)
\begin{proof}
     By the binomial theorem this is \((5-2)^n = 3^n\).
\end{proof}

\subsection{Exercise 54}
\(\dps \sum_{k=0}^{n} (-1)^k \binom{n}{k} 3^{2n-2k}2^{2k}\)
\begin{proof}
     By the binomial theorem this is \((3^2-2^2)^n = 5^n\).
\end{proof}

\subsection{Exercise 55}
(For students who have studied calculus.)

\subsubsection{(a)}
Explain how the equation below follows from the binomial theorem:
\[
     (1+x)^n = \sum_{k=0}^{n} \binom{n}{k} x^k
\]
\begin{proof}
     By the binomial theorem \((1+x)^n = \sum_{k=0}^{n} \binom{n}{k} 1^{n-k}x^k\) but \(1^{n-k}=1\) for all \(k= 0, \ldots, n\)
     so the equation follows.
\end{proof}

\subsubsection{(b)}
Write the formula obtained by taking the derivative of both sides of the equation in part (a) with respect to \(x\).

\begin{proof}
     \(n(1+x)^{n-1} = \sum_{k=1}^n \binom{n}{k}kx^{k-1}\)
\end{proof}

\subsubsection{(c)}
Use the result of part (b) to derive the formulas below.

(i) \(\dps 2^{n-1} = \frac{1}{n}\left[\binom{n}{1} + 2\binom{n}{2} + \cdots + n\binom{n}{n}\right]\)

(ii) \(\dps \sum_{k=0}^{n} k \binom{n}{k} (-1)^k = 0\)

\begin{proof}
     (i) Let \(x = 1\) in the result of part (b): \(n(1+1)^{n-1} = \sum_{k=1}^n \binom{n}{k}k \cdot 1^{k-1}\) which gives
     \(n2^{n-1} = \sum_{k=1}^n \binom{n}{k}k = \binom{n}{1} + \cdots + n\binom{n}{n}\). Dividing by \(n\) gives us the
     desired result.

     (ii) Let \(x = -1\) in the result of part (b): \(n(1-1)^{n-1} = \sum_{k=1}^n \binom{n}{k}k (-1)^{k-1}\) which gives
     \(0 = \sum_{k=1}^n \binom{n}{k}k (-1)^{k-1}\). Multiplying by \(-1\) gives us the desired result.
\end{proof}

\subsubsection{(d)}
Express \(\dps \sum_{k=1}^{n} k \binom{n}{k} 3^k\) in closed form (without using a summation sign or ellipsis).

\begin{proof}
     \(\dps \sum_{k=1}^{n} k \binom{n}{k} 3^k = 3 \sum_{k=1}^{n} k \binom{n}{k} 3^{k-1} = 3n(1+3)^{n-1} = 3n4^{n-1}\)
\end{proof}

\section{Exercise Set 9.8}
\subsection{Exercise 1}
In any sample space \(S\), what is \(P(\es)\)?
\begin{proof}
     0
\end{proof}

\subsection{Exercise 2}
Suppose \(A, B\), and \(C\) are mutually exclusive events in a sample space \(S\), \(A \cup B \cup C = S\), and \(A\) and
\(B\) have probabilities 0.3 and 0.5, respectively.

\subsubsection{(a)}
What is \(P(A \cup B)\)?
\begin{proof}
     By probability axiom 3, \(P(A \cup B) = P(A) + P(B) = 0.3 + 0.5 = 0.8\).
\end{proof}

\subsubsection{(b)}
What is \(P(C)\)?
\begin{proof}
     Because \(A \cup B \cup C = S\) and because \(A, B\), and \(C\) are mutually exclusive events, \(C = S - (A \cup B)\).
     Thus, by the formula for the probability of the complement of an event, \(P(C) = P((A \cup B)^c) = 1 - P(A \cup B) = 1 - 0.8
     = 0.2\).
\end{proof}

\subsection{Exercise 3}
Suppose \(A\) and \(B\) are mutually exclusive events in a sample space \(S\), \(C\) is another event in \(S\),
\(A \cup B \cup C = S\), and \(A\) and \(B\) have probabilities 0.4 and 0.2, respectively.

\subsubsection{(a)}
What is \(P(A \cup B)\)?
\begin{proof}
     Mutually exclusive means \(A \cap B = \es\). So \(P(A \cup B) = P(A) + P(B) - P(A \cap B) = 0.4+0.2-0 = 0.6\)
\end{proof}

\subsubsection{(b)}
Is it possible that \(P(C) = 0.2\)? Explain.
\begin{proof}
     Since \(A \cup B \cup C = S\), we have \(P(A \cup B \cup C) = P(S) = 1\). We also have \(P(A \cup B \cup C) = P(A \cup B) +
     P(C) - P((A \cup B) \cap C)\). Using these together we get \(1 = 0.6 + 0.2 - P((A \cup B) \cap C)\) which is impossible,
     since the right hand side is \(\leq 0.8\).
\end{proof}

\subsection{Exercise 4}
Suppose \(A\) and \(B\) are events in a sample space \(S\) with probabilities 0.8 and 0.7, respectively. Suppose also
that \(P(A \cap B) = 0.6\). What is \(P(A \cup B)\)?

\begin{proof}
     \(P(A \cup B) = P(A) + P(B) - P(A \cap B) = 0.8+0.7-0.6 = 0.9\)
\end{proof}

\subsection{Exercise 5}
Suppose \(A\) and \(B\) are events in a sample space \(S\) and suppose that \(P(A) = 0.6, P(B^c) = 0.4\), and \(P(A \cap B)
= 0.2\). What is \(P(A \cup B)\)?

\begin{proof}
     \(P(A \cup B) = P(A) + P(B) - P(A \cap B) = 0.6 + (1-P(B^c)) - 0.2 = 0.6 + 1 - 0.4 - 0.2 = 1\)
\end{proof}

\subsection{Exercise 6}
Suppose \(U\) and \(V\) are events in a sample space \(S\) and suppose that \(P(U^c) = 0.3, P(V) = 0.6\), and \(P(U^c \cup
V^c) = 0.4\). What is \(P(U \cup V)\)?

\begin{proof}
     \(P(U \cup V) = P(U) + P(V) - P(U \cap V) = (1-P(U^c)) + 0.6 - (1 - P((U \cap V)^c)) = (1-0.3) + 0.6 - (1-P(U^c \cup V^c))\)
     \(= 0.7 + 0.6 - (1-0.4) = 0.7\)
\end{proof}

\subsection{Exercise 7}
Suppose a sample space \(S\) consists of three outcomes: 0, 1, and 2. Let \(A = \{0\}, B = \{1\}\), and \(C = \{2\}\), and
suppose \(P(A) = 0.4\) and \(P(B) = 0.3\). Find each of the following:

\subsubsection{(a)}
\(P(A \cup B)\)
\begin{proof}
     \(P(A \cup B) = 0.4 + 0.3 = 0.7\)
\end{proof}

\subsubsection{(b)}
\(P(C)\)
\begin{proof}
     \(P(C) = P((A \cup B)^c) = 1 - P(A \cup B) = 1 - 0.7 = 0.3\)
\end{proof}

\subsubsection{(c)}
\(P(A \cup C)\)
\begin{proof}
     \(P(A \cup C) = 0.4 + 0.3 = 0.7\)
\end{proof}

\subsubsection{(d)}
\(P(A^c)\)
\begin{proof}
     \(P(A^c) = 1 - P(A) = 1 - 0.4 = 0.6\)
\end{proof}

\subsubsection{(e)}
\(P(A^c \cap B^c)\)
\begin{proof}
     \(P(A^c \cap B^c) = P((A \cup B)^c) = 1 - P(A \cup B) = 1 - 0.7 = 0.3\)
\end{proof}

\subsubsection{(f)}
\(P(A^c \cup B^c)\)
\begin{proof}
     \(P(A^c \cup B^c) = P((A \cap B)^c) = P(\es^c) = P(S) = 1\)
\end{proof}

\subsection{Exercise 8}
Redo exercise 7 assuming that \(P(A) = 0.5\) and \(P(B) = 0.4\).

\begin{proof}
     \(P(A \cup B) = 0.5 + 0.4 = 0.9\)

     \(P(C) = P((A \cup B)^c) = 1 - P(A \cup B) = 1 - 0.9 = 0.1\)

     \(P(A \cup C) = 0.5 + 0.1 = 0.6\)

     \(P(A^c) = 1 - P(A) = 1 - 0.5 = 0.5\)

     \(P(A^c \cap B^c) = P((A \cup B)^c) = 1 - P(A \cup B) = 1 - 0.9 = 0.1\)

     \(P(A^c \cup B^c) = P((A \cap B)^c) = P(\es^c) = P(S) = 1\)
\end{proof}

\subsection{Exercise 9}
Let \(A\) and \(B\) be events in a sample space \(S\), and let \(C = S - (A \cup B)\). Suppose \(P(A) = 0.4, P(B) = 0.5\),
and \(P(A \cap B) = 0.2\). Find each of the following:

\subsubsection{(a)}
\(P(A \cup B)\)
\begin{proof}
     \(P(A \cup B) = P(A) + P(B) - P(A \cap B) = 0.4 + 0.5 - 0.2 = 0.7\)
\end{proof}

\subsubsection{(b)}
\(P(C)\)
\begin{proof}
     \(P(C) = P(S - (A \cup B)) = P((A \cup B)^c) = 1 - P(A \cup B) = 1 - 0.7 = 0.3\)
\end{proof}

\subsubsection{(c)}
\(P(A^c)\)
\begin{proof}
     \(P(A^c) = 1 - P(A) = 1 - 0.4 = 0.6\)
\end{proof}

\subsubsection{(d)}
\(P(A^c \cap B^c)\)
\begin{proof}
     \(P(A^c \cap B^c) = P((A \cup B)^c) = 1 - P(A \cup B) = 1 - 0.7 = 0.3\)
\end{proof}

\subsubsection{(e)}
\(P(A^c \cup B^c)\)
\begin{proof}
     \(P(A^c \cup B^c) = P((A \cap B)^c) = 1 - P(A \cap B) = 1 - 0.2 = 0.8\)
\end{proof}

\subsubsection{(f)}
\(P(B^c \cap C)\)
\begin{proof}
     \(B^c \cap C = B^c \cap (S - (A \cup B)) = B^c \cap (A \cup B)^c\) = \(B^c \cap (A^c \cap B^c) = B^c \cap A^c\), so \(P(B^c \cap C) = P(B^c \cap A^c) = 0.3\)
\end{proof}

\subsection{Exercise 10}
Redo exercise 9 assuming that \(P(A) = 0.7, P(B) = 0.3\), and \(P(A \cap B) = 0.1\).

\begin{proof}
     \(P(A \cup B) = P(A) + P(B) - P(A \cap B) = 0.7 + 0.3 - 0.1 = 0.9\)

     \(P(C) = P(S - (A \cup B)) = P((A \cup B)^c) = 1 - P(A \cup B) = 1 - 0.9 = 0.1\)

     \(P(A^c) = 1 - P(A) = 1 - 0.7 = 0.3\)

     \(P(A^c \cap B^c) = P((A \cup B)^c) = 1 - P(A \cup B) = 1 - 0.9 = 0.1\)

     \(P(A^c \cup B^c) = P((A \cap B)^c) = 1 - P(A \cap B) = 1 - 0.1 = 0.9\)

     \(B^c \cap C = B^c \cap (S - (A \cup B)) = B^c \cap (A \cup B)^c\) = \(B^c \cap (A^c \cap B^c) = B^c \cap A^c\), so \(P(B^c \cap C) = P(B^c \cap A^c) = 0.1\)
\end{proof}

\subsection{Exercise 11}
Prove that if \(S\) is any sample space and \(U\) and \(V\) are events in \(S\) with \(U \subseteq V\), then \(P(U) \leq
P(V)\).

\begin{proof}
     \(U \subseteq V\) implies \(V = U \cup (V - U)\). Notice \(U \cap (V - U) = \es\). Thus \(P(V) = P(U) + P(V-U) - P(U \cap
     (V-U)) = P(U) + P(V-U) - 0 = P(U) + P(V-U) \geq P(U)\) because \(P(V-U) \geq 0\).
\end{proof}

\subsection{Exercise 12}
Prove that if \(S\) is any sample space and \(U\) and \(V\) are any events in \(S\), then \(P(V-U) = P(V) - P(U \cap V)\).

\begin{proof}
     Notice \(U \cup V = U \cup (V - U)\) and \(U \cap (V-U) = \es\). We can write \(P(U \cup V)\) in two ways:

     \(P(U \cup V) = P(U) + P(V) - P(U \cap V)\), and

     \(P(U \cup V) = P(U) + P(V-U) - P(U \cap (V-U)) = P(U) + P(V-U) - P(\es) = P(U) + P(V-U) - 0 = P(U) + P(V-U)\)

     The left hand sides are equal, so we set the right hand sides equal to each other:

     \(P(U) + P(V) - P(U \cap V) = P(U) + P(V-U)\).

     Canceling \(P(U)\) on both sides we get \(P(V) - P(U \cap V) = P(V-U)\), {\it [as was to be shown.]}
\end{proof}

\subsection{Exercise 13}
Use the axioms for probability and mathematical induction to prove that for each integer \(n \geq 2\), if \(A_1, A_2, A_3,
\ldots, A_n\) are any mutually disjoint events in a sample space \(S\), then \(P(A_1 \cup A_2 \cup A_3 \cup \cdots \cup
A_n) = \sum_{k=1}^n P(A_k)\).

\begin{proof}
     Let \(Q(n)\) be the statement: ``if \(A_1, A_2, A_3, \ldots, A_n\) are any mutually disjoint events in a sample space
     \(S\), then \(P(A_1 \cup A_2 \cup A_3 \cup \cdots \cup A_n) = \sum_{k=1}^n P(A_k)\).''

     {\bf Show that \(Q(2)\) is true:} Assume \(A_1, A_2\) are mutually disjoint. Then \(A_1 \cap A_2 = \es\). So
     \(P(A_1 \cup A_2) = P(A_1) + P(A_2) - P(A_1 \cap A_2) = P(A_1) + P(A_2) - P(\es) = P(A_1) + P(A_2) - 0 = P(A_1) + P(A_2)\).
     This is the right hand side \(\sum_{k=1}^2 P(A_k)\), so \(Q(2)\) is true.

          {\bf Show that for any integer \(k \geq 2\), if \(Q(k)\) is true then \(Q(k+1)\) is true:} Assume \(k \geq 2\) and assume
     \(Q(k)\) is true. Assume \(A_1, \ldots, A_{k+1}\) are mutually disjoint events.

     Let \(A = A_1 \cup \cdots \cup A_k\). Notice that \(A \cap A_{k+1} = \es\) since the sets are all mutually disjoint. So
     \(P(A \cap A_{k+1})= P(\es) = 0\) and thus \(P(A \cup A_{k+1}) = P(A) + P(A_{k+1}) - P(A \cap A_{k+1}) = P(A) + P(A_{k+1})\).

     By the induction hypothesis \(P(A) = \sum_{i=1}^k P(A_i)\), so \(P(A \cup A_{k+1}) = P(A) + P(A_{k+1}) = \sum_{i=1}^k P(A_i)
     + P(A_{k+1}) = \sum_{i=1}^{k+1} P(A_i)\), which proves \(Q(k+1)\).
\end{proof}

\subsection{Exercise 14}
A lottery game offers \$2 million to the grand prize winner, \$20 to each of 10,000 second prize winners, and \$4 to each of
50,000 third prize winners. The cost of the lottery is \$2 per ticket. Suppose that 1.5 million tickets are sold. What is the
expected gain or loss of a ticket?

\begin{proof}
     {\it Solution 1:} The net gain of grand prize winner is \(\$2,000,000 - \$2 = \$1,999,998\). Each of the 10,000 second
     prize winners has a net gain of \(\$20 - \$2 = \$18\), and each of the 50,000 third prize winners has a net gain of \(\$4
     - \$2 = \$2\). The number of people who do not win anything is \(1,500,000 - 1 - 10,000 - 50,000 = 1,439,999\), and each of
     these people has a net loss of \(\$2\). Because all of the 1,500,000 tickets have an equal chance of winning a prize, the
     expected gain or loss of a ticket is
     \[
          \frac{1}{1500000}(\$1,999,998 \cdot 1 + \$18 \cdot 10000 + \$2 \cdot 50000 + (-\$2) \cdot 1,439,999) = -\$0.40.
     \]
     {\it Solution 2:} The total income to the lottery organizer is \$2 (per ticket) \(\cdot\) 1,500,000 (tickets) = \$3,000,000.
     The payout the lottery organizer must make is \(\$2,000,000 + (\$20)(10,000) + (\$4)(50,000) = \$2,400,000\), so the net
     gain to the lottery organizer is \$600,000, which amounts to \(\frac{\$600,000}{1,500,000} = \$0.40\) per ticket. Thus the
     expected net loss to a purchaser of a ticket is \$0.40.
\end{proof}

\subsection{Exercise 15}
A company offers a raffle whose grand prize is a \$40,000 new car. Additional prizes are a \$1,000 television and a \$500
computer. Tickets cost \$20 each. Ticket income over the cost of the prizes will be donated to charity. If 3,000 tickets are
sold, what is the expected gain or loss of each ticket?

\begin{proof}
     Each of the 3,000 tickets has the same chance as any other of  winning the raffle, and so \(p_k = \frac{1}{3,000}\) for each
     \(k = 1, 2, 3, \cdots, 3,000\).

     Let \(a_i\) be the net gain for an individual ticket \(a_i\), where \(a_1 = 39,980\) (the net gain for the grand-prize
     ticket, which is 40,000 dollars minus the \$20 cost of the winning ticket), \(a_2 = 980\) (the net gain for the
     television), \(a_3 = 480\) (the net gain for the computer). Since the remaining tickets lose \$20 each, \(a_4 = \cdots =
     a_{3000} = -20\).

     The expected value of a ticket is therefore
     \[
          \sum_{k=1}^{3000}a_kp_k = \frac{1}{3000}\sum_{k=1}^{3000}a_k =  \frac{1}{3000}(39,980+980+480+(-20) \cdot 2997) \approx -6.17
     \]
\end{proof}

\subsection{Exercise 16}
An urn contains four balls numbered 2, 2, 5, and 6. If a person selects a set of two balls at random, what is the
expected value of the sum of the numbers on the balls?

\begin{proof}
     Let \(2_1\) and \(2_2\) denote the two balls with the number 2, and let 5 and 6 denote the other two balls. There are
     \(\binom{4}{2} = 6\) subsets of 2 balls that can be chosen from the urn. The following table shows the sums of the
     numbers on the balls in each set and the corresponding probabilities:
     \begin{center}
          \arrayrulecolor{cyan}
          \begin{tabular}{|l|c|c|}
               \hline
               {\bf Subset}             & {\bf Sum \(s\)} & {\bf Probability that the sum = \(s\)} \\
               \hline
               \(\{2_1, 2_2\}\)         & 4               & 1/6                                    \\
               \hline
               \(\{2_1,5\}, \{2_2,5\}\) & 7               & 2/6                                    \\
               \hline
               \(\{2_1,6\}, \{2_2,6\}\) & 8               & 2/6                                    \\
               \hline
               \(\{5,6\}\)              & 11              & 1/6                                    \\
               \hline
          \end{tabular}
          \arrayrulecolor{black} % change it back!
     \end{center}
     So the expected value is \(4\cdot\frac{1}{6} + 7\cdot \frac{2}{6} + 8 \cdot \frac{2}{6} + 11 \cdot \frac{1}{6} = 7.5\)
\end{proof}

\subsection{Exercise 17}
An urn contains five balls numbered 1, 2, 2, 8, and 8. If a person selects a set of two balls at random, what is the
expected value of the sum of the numbers on the balls?

\begin{proof}
     Let \(2_1\) and \(2_2\) denote the two balls with the number 2, let \(8_1, 8_2\) denote the two balls with the number 8,
     and let 1 denote the other ball. There are \(\binom{5}{2} = 10\) subsets of 2 balls that can be chosen from the urn. The
     following table shows the sums of the numbers on the balls in each set and the corresponding probabilities:
     \begin{center}
          \arrayrulecolor{cyan}
          \begin{tabular}{|l|c|c|}
               \hline
               {\bf Subset}                                           & {\bf Sum \(s\)} & {\bf Probability that the sum = \(s\)} \\
               \hline
               \(\{2_1, 2_2\}\)                                       & 4               & 1/10                                   \\
               \hline
               \(\{2_1,1\}, \{2_2,1\}\)                               & 3               & 2/10                                   \\
               \hline
               \(\{2_1,8_1\}, \{2_2,8_1\}, \{2_2,8_1\}, \{2_2,8_2\}\) & 10              & 4/10                                   \\
               \hline
               \(\{1,8_1\}, \{1,8_2\}\)                               & 9               & 2/10                                   \\
               \hline
          \end{tabular}
          \arrayrulecolor{black} % change it back!
     \end{center}
     So the expected value is \(4 \cdot \frac{1}{10} + 3 \cdot \frac{2}{10} + 10 \cdot \frac{4}{10} + 9 \cdot \frac{2}{10} =
     6.8\)
\end{proof}

\subsection{Exercise 18}
An urn contains five balls numbered 1, 2, 2, 8, and 8. If a person selects a set of three balls at random, what is the
expected value of the sum of the numbers on the balls?

\begin{proof}
     There are \(\binom{5}{3} = 10\) subsets of 3 balls that can be chosen from the urn. The following table shows the sums of the
     numbers on the balls in each set and the corresponding probabilities:
     \begin{center}
          \arrayrulecolor{cyan}
          \begin{tabular}{|l|c|c|}
               \hline
               {\bf Subset}                                                   & {\bf Sum \(s\)} & {\bf Probability that sum = \(s\)} \\
               \hline
               \(\{1, 2_1, 2_2\}\)                                            & 5               & 1/10                               \\
               \hline
               \(\{1,2_1,8_1\}, \{1,2_2,8_1\}, \{1,2_1,8_2\}, \{1,2_2,8_2\}\) & 11              & 4/10                               \\
               \hline
               \(\{1,8_1,8_2\}\)                                              & 17              & 1/10                               \\
               \hline
               \(\{2_1,2_2,8_1\}, \{2_1,2_2,8_2\}\)                           & 12              & 2/10                               \\
               \hline
               \(\{2_1,8_1,8_2\}, \{2_2,8_1,8_2\}\)                           & 18              & 2/10                               \\
               \hline
          \end{tabular}
          \arrayrulecolor{black} % change it back!
     \end{center}
     So the expected value is \(5 \cdot \frac{1}{10} + 11 \cdot \frac{4}{10} + 17 \cdot \frac{1}{10} + 12 \cdot \frac{2}{10} +
     18 \cdot \frac{2}{10} = 12.6\)
\end{proof}

\subsection{Exercise 19}
When a pair of balanced dice are rolled and the sum of the numbers showing face up is computed, the result can be any
number from 2 to 12, inclusive. What is the expected value of the sum?

\begin{proof}
     The following table displays the sum of the numbers showing face up on the dice. The top row represents one of the dice,
     and the leftmost column represents the other die. Each cell in the remaining \(6 \times 6\) part of the table represents an
     outcome whose probability is 1/36. We can count the number of ways a particular sum can be obtained by looking at the
     bottom-left-to-top-right diagonals. For example there are 6 ways to obtain a sum of 7: 1+6, 2+5, 3+4, 4+3, 5+2, 6+1. Thus
     the expected value of the sum is \(\frac{1}{36}(2 \cdot 1 + 3 \cdot 2 + 4 \cdot 3 + 5 \cdot 4 + 6 \cdot 5 + 7 \cdot 6 + 8 \cdot 5 + 9 \cdot 4 + 10 \cdot 3 + 11 \cdot 2 + 12 \cdot 1) = \frac{252}{36} = 7\).
     \begin{center}
          \arrayrulecolor{cyan}
          \begin{tabular}{c|c|c|c|c|c|c|}
               Dice & 1 & 2 & 3 & 4  & 5  & 6  \\
               \hline
               1    & 2 & 3 & 4 & 5  & 6  & 7  \\
               \hline
               2    & 3 & 4 & 5 & 6  & 7  & 8  \\
               \hline
               3    & 4 & 5 & 6 & 7  & 8  & 9  \\
               \hline
               4    & 5 & 6 & 7 & 8  & 9  & 10 \\
               \hline
               5    & 6 & 7 & 8 & 9  & 10 & 11 \\
               \hline
               6    & 7 & 8 & 9 & 10 & 11 & 12 \\
               \hline
          \end{tabular}
          \arrayrulecolor{black} % change it back!
     \end{center}
\end{proof}

\subsection{Exercise 20}
Suppose a person offers to play a game with you. In this game, when you draw a card from a standard 52-card deck, if the card
is a face card (J, Q, K) you win \$3, and if the card is anything else you lose \$1. If you agree to play the game,
what is your expected gain or loss?

\begin{proof}
     There are 12 face cards in a 52 card deck. So the expected value is \(3 \cdot \frac{12}{52} + (-1) \cdot \frac{40}{52}
     = \frac{36-40}{52} \approx -0.07\), or, in other words, a 7.7 cent loss.
\end{proof}

\subsection{Exercise 21}
A person pays \$1 to play the following game: The person tosses a fair coin four times. If no heads occur, the person
pays an additional \$2, if one head occurs, the person pays an additional \$1, if two heads occur, the person just loses the
initial dollar, if three heads occur, the person wins \$3, and if four heads occur, the person wins \$4. What is the person’s
expected gain or loss?

\begin{proof}
     There are 16 possible outcomes in the sample space. \\
     No heads can occur in 1 way, in which case the person gets \(-1-2 = \$-3\). \\
     One head can occur in 4 ways, in which case the person gets \(-1-1 = \$-2\). \\
     Two heads can occur in 6 ways, in which case the person gets \(\$-1\). \\
     Three heads can occur in 4 ways, in which case the person gets \(-1+3 = \$2\). \\
     Four heads can occur in 1 way, in which case the person gets \(-1+4= \$3\). \\
     So the expected value is \((-3) \cdot \frac{1}{16} + (-2) \cdot \frac{4}{16} + (-1) \cdot \frac{6}{16} + 2 \cdot
     \frac{4}{16} + 3 \cdot \frac{1}{16} = \$-6/16.\)
\end{proof}

\subsection{Exercise 22}
A fair coin is tossed until either a head comes up or four tails are obtained. What is the expected number of tosses?

\begin{proof}
     The possibilities are: H, TH, TTH, TTTH, TTTT, with expected number of tosses 1, 2, 3, 4, 4; and probabilities 1/2, 1/4,
     1/8, 1/16, 1/16, respectively. So \(1 \cdot \frac{1}{2} + 2 \cdot \frac{1}{4} + 3 \cdot \frac{1}{8} + 4 \cdot \frac{1}{16}
     + 4 \cdot \frac{1}{16} = \frac{30}{16} = 1.875\) is the expected number of tosses.
\end{proof}

\subsection{Exercise 23}
A gambler repeatedly bets that a die will come up 6 when rolled. Each time the die comes up 6, the gambler wins \$1;
each time it does not, the gambler loses \$1. He will quit playing either when he is ruined or when he wins \$300. If
\(P_n\) is the probability that the gambler is ruined when he begins play with \(\$n\), then \(P_{k-1} = \frac{1}{6}P_k +
\frac{5}{6} P_{k-2}\) for every integer \(k\) with \(2 \leq k \leq 300\). Also \(P_0 = 1\) and \(P_{300} = 0\). Find an
explicit formula for \(P_n\) and use it to calculate \(P_{20}\). (Exercise 33 in Section 9.9 asks you to derive the
recurrence relation for this sequence.)

\begin{proof}
     Solving, we get \(P_k = 6P_{k-1} - 5P_{k-2}\). The characteristic equation is \(r^2 - 6r + 5 = 0\) with roots
     \(r = 1,5\). The general solution is \(P_n = A+B \cdot 5^n\). Applying initial conditions \(P_0 = 1 = A + B\) and \(P_{300}
     = 0 = A + B \cdot 5^{300}\) and solving both for \(A\) we get \(A = -B \cdot 5^{300} = 1-B\), solving for \(B\) we get
     \(\dps B = \frac{1}{1-5^{300}}\) and \(\dps A = 1 - \frac{1}{1-5^{300}}\). So
     \[
          P_n = 1 - \frac{1}{1-5^{300}} + \frac{1}{1-5^{300}} \cdot 5^n
     \]
     Then \(\dps P_{20}=1-\frac{1}{1-5^{300}} + \frac{1}{1-5^{300}} \cdot 5^{20} = \frac{5^{20}-5^{300}}{1-5^{300}} \approx 1\).
\end{proof}

\section{Exercise Set 9.9}
\subsection{Exercise 1}
Suppose \(P(A|B) = 1/2\) and \(P(A \cap B) = 1/6\). What is \(P(B)\)?

\begin{proof}
     \(P(B) = \frac{P(A \cap B)}{P(A | B)} = \frac{1/6}{1/2} = 1/3\)
\end{proof}

\subsection{Exercise 2}
Suppose \(P(X|Y) = 1/3\) and \(P(Y) = 1/4\). What is \(P(X \cap Y)\)?

\begin{proof}
     \(P(X \cap Y) = P(X|Y)P(Y) = (1/3)(1/4) = 1/12\)
\end{proof}

\subsection{Exercise 3}
The instructor of a discrete mathematics class gave two tests. Twenty-five percent of the students received an A on the first
test and 15\% of the students received A’s on both tests. What percent of the students who received A’s on the first test
also received A’s on the second test?

\begin{proof}
     Let \(A_1, A_2\) be the percentage of students who got an A in tests 1,2 respectively. Then \(P(A_1) = 0.25, P(A_1 \cap A_2)
     = 0.15\). Getting an A on test 1 is independent of getting an A on test 2. Thus \(P(A_2 | A_1) = \frac{P(A_1 \cap A_2)}
     {P(A_1)} = \frac{0.15}{0.25} = 3/5 = 60\%.\)
\end{proof}

\subsection{Exercise 4}
\subsubsection{(a)}
Prove that if \(A\) and \(B\) are any events in a sample space \(S\), with \(P(B) \neq 0\), then \(P(A^c | B) = 1 - P(A|B)\).

\begin{proof}
     Suppose \(S\) is any sample space and \(A\) and \(B\) are any events in \(S\) such that \(P(B) \neq 0\). Note that \\
     (1) \(A \cup A^c = S\) by the complement law for \(\cup\). \\
     (2) \(B \cap S = B\) by the identity law for \(\cap\). \\
     (3) \(B \cap (A \cup A^c) = (A \cap B) \cup (A^c \cap B)\) by the distributive law and commutative laws for sets. \\
     (4) \((A \cap B) \cap (A^c \cap B) = \es\) by the complement law for \(\cap\) and the commutative and associative laws for
     sets.

     Thus \(B = (A \cap B) \cup (A^c \cap B)\), and, by probability axiom 3, \(P(B) = P (A \cap B) + P (A^c \cap B)\). Therefore,
     \(P(A^c \cap B) = P(B) - P(A \cap B)\). By definition of conditional probability, it follows that \(P(A^c | B) =
     \frac{P(A^c \cap B)}{P(B)} = \frac{P(B) - P(A \cap B)}{P(B)} = 1 - \frac{P(A \cap B)}{P(B)} = 1 - P(A | B)\).
\end{proof}

\subsubsection{(b)}
Explain how the result in part (a) justifies the following statements:

(1) If the probability of a false positive on a test for a condition is 4\%, then there is a 96\% probability that a
person who does not have the condition will have a negative test result.

(2) If the probability of a false negative on a test for a condition is 1\%, then there is a 99\% probability that a
person who does have the condition will test positive for it.

\begin{proof}
     (1) Let \\
     \(A\): the event that a test for the condition is positive, \\ \(B\): the event that a person does not have the condition. \\
     Then \\
     \(A|B\): the event that a person tests positive, given that they do not have the condition; in other words, {\it a false
               positive}, therefore \(P(A|B) = 0.04\), \\
     \(A^c|B\): the event that a person tests negative, given that they do not have the condition. \\
     Assume \(P(B) \neq 0\). (This makes sense, otherwise all people in the world would have the condition!) \\
     Then by part (a) \(P(A^c | B) = 1 - P(A|B) = 1-0.04 = 0.96\). By this result and above explanation, there is a 96\% chance
     that someone who does not have the condition will test negative.

     (2) Let \\
     \(A\): the event that a test for the condition is negative, \\ \(B\): the event that a person has the condition. \\
     Then \\
     \(A|B\): the event that a person tests negative, given that they have the condition; in other words, {\it a false
               negative}, therefore \(P(A|B) = 0.01\), \\
     \(A^c|B\): the event that a person tests positive, given that they have the condition. \\
     Assume \(P(B) \neq 0\). (This makes sense, otherwise nobody in the world would have the condition!) \\
     Then by part (a) \(P(A^c | B) = 1 - P(A|B) = 1-0.01 = 0.99\). By this result and above explanation, there is a 99\% chance
     that someone who has the condition will test positive.
\end{proof}

\subsection{Exercise 5}
Suppose that \(A\) and \(B\) are events in a sample space \(S\) and that \(P(A), P(B)\), and \(P(A | B)\) are known.
Derive a formula for \(P(A | B^c)\).

\begin{proof}
     1. By previous exercises about sets we have \(A = (A \cap B) \cup (A \cap B^c)\), and \((A \cap B) \cap (A \cap B^c) =
     \es\). So \(P((A \cap B) \cap (A \cap B^c)) = 0\).

     2. By 1 and probability axioms, \(P(A) = P(A \cap B) + P(A \cap B^c) - P((A \cap B) \cap (A \cap B^c)) = P(A \cap B) +
     P(A \cap B^c) - 0 = P(A \cap B) + P(A \cap B^c)\). So \(P(A \cap B^c) = P(A) - P(A \cap B)\).

     3. By definition of conditional probability, \(P(A | B^c) = \frac{P(B^c \cap A)}{P(B^c)} = \frac{P(A \cap B^c)}{1-P(B)}\).
     Then by 2, \(P(A | B^c) = \frac{P(A) - P(A \cap B)}{1-P(B)}\).

     4. Again by definition of conditional probability, \(P(A|B) = \frac{P(B \cap A)}{P(B)}\) so \(P(B \cap A) = P(A|B)P(B)\).

     5. Substituting 4 into 3 we get \(P(A | B^c) = \frac{P(A) - P(A|B)P(B)}{1-P(B)}\).
\end{proof}

\subsection{Exercise 6}
An urn contains 25 red balls and 15 blue balls. Two are chosen at random, one after the other, without replacement.

\subsubsection{(a)}
Use a tree diagram to help calculate the following probabilities: the probability that both balls are red, the
probability that the first ball is red and the second is not, the probability that the first ball is not red and the second
is red, the probability that neither ball is red.

\begin{proof}
     Let \(R_1\) be the probability that the first ball is red, and let \(R_2\) be the probability that the second ball is red.
     Then \(R_1^c\) is the probability that the first ball is not red, and \(R_2^c\) is the probability that the second ball is
     not red. The tree diagram shows the various relations among the probabilities.

     \begin{figure}[ht!]
          \centering
          \includegraphics[scale=0.5]{../images/9.9.6.a.png}
     \end{figure}
\end{proof}

\subsubsection{(b)}
What is the probability that the second ball is red?
\begin{proof}
     This is the event \((R_1 \cap R_2) \cup (R_1^c \cap R_2)\) and notice that these two events are disjoint, so it is \(\frac{5}
     {8} \cdot \frac{8}{13} + \frac{3}{8} \cdot \frac{25}{39} = \frac{5}{13} + \frac{25}{104} = \frac{65}{104}= \frac{5}{8}\).
\end{proof}

\subsubsection{(c)}
What is the probability that at least one of the balls is red?
\begin{proof}
     This is the event \((R_1 \cap R_2) \cup (R_1 \cap R_2^c) \cup (R_1^c \cap R_2)\) and notice that these three events are
     mutually disjoint, so it's \(\frac{5}{8} \cdot \frac{8}{13} + \frac{5}{8}\cdot\frac{5}{13} + \frac{3}{8} \cdot \frac{25}{39}
     = \frac{5}{13} + \frac{25}{104} + \frac{25}{104} = \frac{90}{104} = \frac{45}{52}\).
\end{proof}

\subsection{Exercise 7}
Redo exercise 6 assuming that the urn contains 30 red balls and 40 blue balls.

\begin{proof}
     (a) \(P(R_1) = 30 / (30+40) = 3/7, P(R_1^c) = 40 / (30+40) = 4/7, P(R_2 | R_1) = 29/69, P(R_2^c | R_1) = 40/69, P(R_2 |
     R_1^c) = 30/69, P(R_2^c | R_1^c) = 39/69\)

     (b) \(P(R_2) = P(R_2 | R_1) + P(R_2 | R_1^c) = \frac{29}{69} + \frac{30}{69} = \frac{59}{69}\)

     (c) \(P(R_1 \cap R_2) + P(R_1^c \cap R_2) + P(R_1 \cap R_2^c) = P(R_1)P(R_2|R_1) + P(R_1^c)P(R_2|R_1^c) + P(R_1)P(R_2^c|R_1)
     = \frac{3}{7}\frac{29}{69} + \frac{4}{7}\frac{30}{69} + \frac{3}{7}\frac{40}{69} = \frac{3 \cdot 29 + 4 \cdot 30 +
          3 \cdot 40}{7 \cdot 69} = 327/483 = 0.677018634\)
\end{proof}

\subsection{Exercise 8}
A pool of 10 semifinalists for a job consists of 7 men and 3 women. Because all are considered equally qualified, the names
of two of the semifinalists are drawn, one after the other, at random, to become finalists for the job.

\subsubsection{(a)}
What is the probability that both finalists are women?
\begin{proof}
     Let \(W_1\) be the event that the first semifinalist chosen is a women, \\
     Let \(W_2\) be the event that a woman is chosen on the second draw. \\
     \(P(W_1 \cap W_2) = P(W_1)P(W_2 | W_1) = \frac{3}{10} \cdot \frac{2}{9} = \frac{1}{15}\).
\end{proof}

\subsubsection{(b)}
What is the probability that both finalists are men?
\begin{proof}
     Let \(M_1\) be the event that a man is chosen in the first draw, \\
     Let \(M_2\) be the event that a man is chosen in the second drawn. \\
     \(P(M_1 \cap M_2) = P(M_1)P(M_2 | M_1) = \frac{7}{10} \cdot \frac{6}{9} = \frac{7}{15}\).
\end{proof}

\subsubsection{(c)}
What is the probability that one finalist is a woman and the other is a man?

\begin{proof}
     Notice the events \(W_1 \cap M_2\) and \(W_2 \cap M_1\) are disjoint. So \(P((W_1 \cap M_2) \cup (W_2 \cap M_1)) =
     P(W_1 \cap M_2) + P(W_2 \cap M_1)\).

     \(P(W_1 \cap M_2) = P(W_1)P(M_2 | W_1) = \frac{3}{10} \cdot \frac{7}{9} = \frac{7}{30}\)

     \(P(W_2 \cap M_1) = P(M_1 \cap W_2) = P(M_1)P(W_2 | M_1) = \frac{7}{10} \cdot \frac{3}{9} = \frac{7}{30}\)

     So the answer is \(\frac{7}{30} + \frac{7}{30} = \frac{7}{15}\).
\end{proof}

\subsection{Exercise 9}
Prove Bayes’ theorem for \(n = 2\). That is, prove that if a sample space \(S\) is a union of mutually disjoint events
\(B_1\) and \(B_2\), if \(A\) is an event in \(S\) with \(P(A) \neq 0\), and if \(k = 1\) or \(k = 2\), then
\[
     P(B_k | A) = \frac{P(A| B_k)P(B_k)}{P(A | B_1) P(B_1) + P(A | B_2) P(B_2)}
\]
{\it Hint:} Use the facts that \(P(B_k | A) = \frac{P(B_k \cap A)}{P(A)}\) and \((A \cap B_1) \cup (A \cap B_2) = A\).

\begin{proof}
     (following the Hint)

     1. We have \(P(B_k | A) = \frac{P(B_k \cap A)}{P(A)}\).

     2. By definition of conditional probability \(P(B_k \cap A) = P(A|B_k)P(B_k)\).

     3. Since \(B_1\) and \(B_2\) are disjoint, \(A \cap B_1\) and \(A \cap B_2\) are disjoint.

     4. By the fact \((A \cap B_1) \cup (A \cap B_2) = A\), we have \(P(A) = P((A \cap B_1) \cup (A \cap B_2))\).

     5. By 3 and 4, \(P(A) = P(A \cap B_1) + P(A \cap B_2)\).

     6. By 1,2,3,4,5,
     \[
          P(B_k | A) = \frac{P(B_k \cap A)}{P(A)} = \frac{P(A|B_k)P(B_k)}{P(A \cap B_1) + P(A \cap B_2)}
          = \frac{P(A|B_k)P(B_k)}{P(A|B_1)P(B_1) + P(A|B_2)P(B_2)}.
     \]
\end{proof}

\subsection{Exercise 10}
Prove the full version of Bayes’ theorem.
\begin{proof}
     The proof is the same as in Exercise 9, except now in Step 5 we have
     \[
          P(A) = P(A \cap B_1) + P(A \cap B_2) + \cdots + P(A \cap B_n)
     \]
     and in Step 6, the denominator becomes
     \[
          P(A|B_1)P(B_1) + P(A|B_2)P(B_2) + \cdots + P(A|B_n)P(B_n).
     \]
     The proof does not require mathematical induction.
\end{proof}

\subsection{Exercise 11}
One urn contains 12 blue balls and 7 white balls, and a second urn contains 8 blue balls and 19 white balls. An urn is
selected at random, and a ball is chosen from the urn.

\subsubsection{(a)}
What is the probability that the chosen ball is blue?
\begin{proof}
     Let \(U_1\) be the event that the first urn is chosen, \(U_2\) the event that the second urn is chosen, and \(B\) the event
     that the chosen ball is blue. Then \(P(B|U_1) = \frac{12}{19}\) and \(P(B|U_2) = \frac{8}{27}\). So \(P(B \cap U_1) = P(B|
     U_1)P(U_1) = \frac{12}{19}\frac{1}{2} = \frac{6}{19}\), and \(P(B \cap U_2) = P(B|U_2)P(U_2) = \frac{8}{27}\frac{1}{2} =
     \frac{4}{27}\). Now \(B\) is the disjoint union of \(B \cap U_1\) and \(B \cap U_2\). So \(P(B)P(U_1|B) = P(B \cap U_1) +
     P(B \cap U_2) = \frac{6}{19} + \frac{4}{27} \approx 46.4\%\).
\end{proof}

\subsubsection{(b)}
If the chosen ball is blue, what is the probability that it came from the first urn?

\begin{proof}
     Given that the chosen ball is blue, the probability that it came from the first urn is \(P(U_1|B)\). By Bayes’ theorem and
     the computations in part (a),
     \[
          P(U_1|B) = \frac{P(B|U_1)P(U_1)}{P(B|U_1)P(U_1) + P(B|U_2)P(U_2)} = \frac{(12/19)(1/2)}{(12/19)(1/2) + (8/27)(1/2)}
          \approx 68.1\%
     \]
\end{proof}

\subsection{Exercise 12}
Redo exercise 11 assuming that the first urn contains 4 blue balls and 16 white balls and the second urn contains 10 blue
balls and 9 white balls.

\begin{proof}
     (a) \(P(B|U_1) = \frac{4}{20}\) and \(P(B|U_2) = \frac{10}{19}\). So \(P(B \cap U_1) = P(B|U_1)P(U_1) = \frac{4}{20} \cdot
     \frac{1}{2} = \frac{1}{10}\), and \(P(B \cap U_2) = P(B|U_2) P(U_2) = \frac{10}{19} \cdot \frac{1}{2} = \frac{5}{19}\). Now
     \(B\) is the disjoint union of \(B \cap U_1\) and \(B \cap U_2\). So \(P(B)P(U_1|B) = P(B \cap U_1) + P(B \cap U_2) =
     \frac{1}{10} + \frac{5}{19} = \frac{69}{190} \approx 36.3\%\).

     (b) Given that the chosen ball is blue, the probability that it came from the first urn is \(P(U_1|B)\). By Bayes’ theorem
     and the computations in part (a),
     \[
          P(U_1|B) = \frac{P(B|U_1)P(U_1)}{P(B|U_1)P(U_1) + P(B|U_2)P(U_2)} = \frac{1/10}{1/10 + 5/19} = \frac{19}{69}
          \approx 27.5\%
     \]
\end{proof}

\subsection{Exercise 13}
One urn contains 10 red balls and 25 green balls, and a second urn contains 22 red balls and 15 green balls. A ball is chosen
as follows: First an urn is selected by tossing a loaded coin with probability 0.4 of landing heads up and probability 0.6
of landing tails up. If the coin lands heads up, the first urn is chosen; otherwise, the second urn is chosen. Then a ball is
picked at random from the chosen urn.

\subsubsection{(a)}
What is the probability that the chosen ball is green? ({\it Hint:} 52.9\%)

\begin{proof}
     Let \(U_1, U_2\) be the events where the chosen urn is 1,2 respectively. So \(P(U_1) = 0.4, P(U_2) = 0.6 \).

     Let \(G\) be the event where the chosen ball is green. Then \(P(G|U_1) = \frac{25}{10+25} = \frac{5}{7}, P(G|U_2) =
     \frac{15}{22+15} = \frac{15}{37}\).

     Then \(P(G) = P(G|U_1)P(U_1) + P(G|U_2)P(U_2) = \frac{5}{7} \cdot 0.4 + \frac{15}{37} \cdot 0.6 = \frac{2}{7} + \frac{9}
     {37} = \frac{137}{259} \approx 52.9\%\)
\end{proof}

\subsubsection{(b)}
If the chosen ball is green, what is the probability that it was picked from the first urn? ({\it Hint:} 54\%)

\begin{proof}
     \(P(U_1|G) = \frac{P(G|U_1)P(U_1)}{P(G|U_1)P(U_1) + P(G|U_2)P(U_2)} = \frac{2/7}{(2/7) + (9/37)} = \frac{2}{7} \cdot
     \frac{259}{137} = \frac{518}{959} \approx 54\%\)
\end{proof}

\subsection{Exercise 14}
A drug-screening test is used in a large population of people of whom 4\% actually use drugs. Suppose that the false
positive rate is 3\% and the false negative rate is 2\%. Thus a person who uses drugs tests positive for them 98\% of the
time, and a person who does not use drugs tests negative for
them 97\% of the time.

\subsubsection{(a)}
What is the probability that a randomly chosen person who tests positive for drugs actually uses drugs?

\begin{proof}
     Let \(pos, neg, dr, nodr\) be the events that a randomly chosen person tests positive, negative, uses drugs, and does
     not use drugs, respectively.

     We're given: \(P(dr) = 0.04, P(nodr) = 1 - 0.04 = 0.96, P(pos|nodr) = 0.03, P(neg|dr) = 0.02, P(pos|dr) = 1 - 0.02 =
     0.98, P(neg|nodr) = 1 - 0.03 = 0.97\).

     Then \(P(dr|pos) = \frac{P(pos|dr)P(dr)}{P(pos|dr)P(dr) + P(pos|nodr)P(nodr)} = \frac{0.98 \cdot 0.04}{0.98 \cdot 0.04
          + 0.03 \cdot 0.96} = 0.576470588 \approx 57.6\%\)
\end{proof}

\subsubsection{(b)}
What is the probability that a randomly chosen person who tests negative for drugs does not use drugs?

\begin{proof}
     \(P(nodr|neg) = \frac{P(neg|nodr)P(nodr)}{P(neg|nodr) P(nodr) + P(neg|dr)P(dr)} = \frac{0.97 \cdot 0.96}{0.97 \cdot 0.96 + 0.02 \cdot 0.04} = 0.999141631 \approx 99.9\%\)
\end{proof}

\subsection{Exercise 15}
Two different factories both produce a certain automobile part. The probability that a component from the first factory
is defective is 2\%, and the probability that a component from the second factory is defective is 5\%. In a supply of 180 of
the parts, 100 were obtained from the first factory and 80 from the second factory.

\subsubsection{(a)}
What is the probability that a part chosen at random from the 180 is from the first factory?

\begin{proof}
     Let \(A,B,D\) be the events that a part chosen at random came from the first factory, the second factory, or is defective,
     respectively. We are given \(P(D|A) = 0.02, P(D|B) = 0.05, P(A) = \frac{100}{180} = \frac{5}{9}, P(B) = \frac{80}{180} =
     \frac{4}{9}\). The answer is \(P(A) = 5/9\).
\end{proof}

\subsubsection{(b)}
What is the probability that a part chosen at random from the 180 is from the second factory?

\begin{proof}
     4/9.
\end{proof}

\subsubsection{(c)}
What is the probability that a part chosen at random from the 180 is defective?

\begin{proof}
     Since \(A\) and \(B\) are disjoint events, \(D|A\) and \(D|B\) are disjoint events. So \(P(D) = P(D|A)P(A) + P(D|B)P(B) =
     0.02 \cdot \frac{5}{9} + 0.05 \cdot \frac{4}{9} = \frac{1}{30} \approx 3.3\%\).
\end{proof}

\subsubsection{(d)}
If the chosen part is defective, what is the probability that it came from the first factory?

\begin{proof}
     \(P(A|D) = \frac{P(D|A)P(A)}{P(D|A)P(A) + P(D|B)P(B)} = \frac{0.02 \cdot \frac{5}{9}}{0.02 \cdot \frac{5}{9} + 0.05
          \cdot \frac{4}{9}} = \frac{1/90}{1/90 + 2/90} = 1/3 \approx 33.3\%\).
\end{proof}

\subsection{Exercise 16}
Three different suppliers X, Y, and Z provide produce for a grocery store. Twelve percent of produce from X is superior
grade, 8\% of produce from Y is superior grade, and 15\% of produce from Z is superior grade. The store obtains 20\% of
its produce from X, 45\% from Y, and 35\% from Z.

\subsubsection{(a)}
If a piece of produce is purchased, what is the probability that it is superior grade?

\begin{proof}
     \(\dps \frac{20}{100} \cdot \frac{12}{100} + \frac{45}{100} \cdot \frac{8}{100} + \frac{35}{100} \cdot \frac{15}{100} =
     \frac{240+360+525}{10000} = 0.1125 = 11.25\%\)
\end{proof}

\subsubsection{(b)}
If a piece of produce in the store is superior grade, what is the probability that it is from X?

\begin{proof}
     Let X, Y, Z, S be the events that the produce is from X, Y, Z or superior grade, respectively. Then \(P(X|S)\)
     \[
          \begin{array}{cl}
               = & \dps \frac{P(S|X)P(X)}{P(S|X)P(X) + P(S|Y)P(Y) + P(S|Z)P(Z)} \\
               = & \dps \frac{\frac{20}{100} \cdot \frac{12}{100}}
               {\frac{20}{100} \cdot \frac{12}{100} + \frac{45}{100} \cdot \frac{8}{100} + \frac{35}{100} \cdot \frac{15}{100}}
               = \dps \frac{240/10000}{1125/10000} = 240/1125 \approx 21.3\%
          \end{array}
     \]
\end{proof}

\subsection{Exercise 17}
Prove that if \(A\) and \(B\) are events in a sample space \(S\) with the property that \(P(A|B) = P(A)\) and
\(P(A) \neq 0\), then \(P(B|A) = P(B)\).

\begin{proof}
     Suppose \(A\) and \(B\) are events in a sample space \(S\), and \(P(A|B) = P(A) \neq 0\). Then
     \[
          P(B|A) = \frac{P(B\cap A)}{P(A)} = \frac{P(A|B)P(B)}{P(A)} = \frac{P(A)P(B)}{P(A)} = P(B).
     \]
\end{proof}

\subsection{Exercise 18}
Prove that if \(P(A \cap B) = P(A) \cdot P(B), P(A) \neq 0\), and \(P(B) \neq 0\), then \(P(A|B) = P(A)\) and
\(P(B|A) = P(B)\).

\begin{proof}
     \(P(A|B) = \frac{P(A \cap B)}{P(B)} = \frac{P(A)P(B)}{P(B)} = P(A)\). Then by Exercise 17, \(P(B|A) = P(B)\).
\end{proof}

\subsection{Exercise 19}
A pair of fair dice, one blue and the other gray, are rolled. Let \(A\) be the event that the number face up on the blue die
is 2, and let \(B\) be the event that the number face up on the gray die is 4 or 5. Show that \(P(A|B) = P(A)\) and
\(P(B|A) = P(B)\).

\begin{proof}
     As in Example 6.9.1, the sample space is the set of all 36 outcomes obtained from rolling the two dice and noting the
     numbers showing face up on each. Then \(A = \{21, 22, 23, 24, 25, 26\}, B = \{14, 24, 34, 44, 54, 64, 15, 25, 35, 45, 55,
     65\}\), and \(A \cap B = \{24, 25\}\). Since the dice are fair (so all outcomes are equally likely), \(P(A) = 6/36, P(B) =
     12/36\), and \(P(A \cap B) = 36\). By definition of conditional probability, \(P(A|B) = \frac{P(A \cap B)}{P(B)} =
     \frac{2/36}{12/36} = 1/6\) and \(P(B|A) = \frac{P(A \cap B)}{P(A)} = \frac{2/36}{6/36} = 1/3\). Hence \(P(A|B) = P(A)\)
     and \(P(B|A) = P(B)\).
\end{proof}

\subsection{Exercise 20}
Suppose a fair coin is tossed three times. Let \(A\) be the event that a head appears on the first toss, and let \(B\) be
the event that an even number of heads is obtained. Show that \(P(A|B) = P(A)\) and \(P(B|A) = P(B)\).

\begin{proof}
     The sample space is \(S = \{HHH, HHT, HTH, THH, HTT, THT, TTH, TTT\}\). Then \(A = \{HHH, HHT, HTH, HTT\}\) and \(P(A) = 4/8
     = 1/2\); and \(B = \{HHT, HTH\), \(THH, TTT\}\) and \(P(B) = 4/8 = 1/2\). Now \(A \cap B = \{HHT, HTH\}\) so \(P(A|B) =
     \frac{P(A \cap B)}{P(B)} = \frac{2/8}{4/8} = 1/2 = P(A)\) and \(P(B|A) = \frac{P(A \cap B)}{P(A)} = \frac{2/8}{4/8} = 1/2 =
     P(B)\).
\end{proof}

\subsection{Exercise 21}
If \(A\) and \(B\) are events in a sample space \(S\) and \(A \cap B = \es\), what must be true in order for \(A\) and \(B\)
to be independent? Explain.

\begin{proof}
     For independence we need \(P(A \cap B) = P(A)P(B)\). Since \(P(A \cap B) = P(\es) = 0\) either \(P(A) = 0\) or \(P(B) =
     0\), or both.
\end{proof}

\subsection{Exercise 22}
Prove that if \(A\) and \(B\) are independent events in a sample space \(S\), then \(A^c\) and \(B\) are also
independent, and so are \(A^c\) and \(B^c\).

\begin{proof}
     Assume \(A\) and \(B\) are independent.

     1. By definition of independence \(P(A \cap B) = P(A)P(B)\).

     2. By Demorgan's law, \(A^c \cap B^c = (A \cup B)^c\) so \(P(A^c \cap B^c) = P(A \cup B)^c = 1 - P(A \cup B)\).

     3. By 2, \(P(A^c \cap B^c) = 1 - (P(A) + P(B) - P(A \cap B)) = 1 - P(A) - P(B) + P(A \cap B)\).

     4. By 1 and 3, \(P(A^c \cap B^c) = 1 - P(A) - P(B) + P(A)P(B) = (1-P(A))(1-P(B))\).

     5. By 4, \(P(A^c \cap B^c) = P(A^c)P(B^c)\). So \(A^c\) and \(B^c\) are independent.

     6. Similarly \(B = (A^c \cap B) \cup (A \cap B)\) where \(A^c \cap B\) and \(A \cap B\) are disjoint.

     7. By 6 and independence, \(P(B) = P(A^c \cap B) + P(A \cap B) = P(A^c \cap B) + P(A)P(B)\).

     8. By 7, \(P(B) - P(A)P(B) = P(A^c \cap B)\) so \(P(A^c \cap B) = P(B)(1-P(A)) = P(B)P(A^c)\) so \(A^c\) and \(B\) are
     independent.
\end{proof}

\subsection{Exercise 23}
A student taking a multiple-choice exam does not know the answers to two questions. All have five choices for the
answer. For one of the two questions, the student can eliminate two answer choices as incorrect but has no idea
about the other answer choices. For the other question, the student has no clue about the correct answer at all. Assume
that whether the student chooses the correct answer on one of the questions does not affect whether the student chooses the
correct answer on the other question.

\subsubsection{(a)}
What is the probability that the student will answer both questions correctly?

\begin{proof}
     Let \(A\) be the event that the student answers the first question correctly, and let \(B\) be the event that the
     student answers the second question correctly. Because two choices can be eliminated on the first question,
     \(P(A) = 1/3\), and because no choices can be eliminated on the second question, \(P(B) = 1/5\). Thus \(P(A^c) = 2/3\) and
     \(P(B^c) = 4/5\).

     \(P(A \cap B) = P(A)P(B) = 1/3 \cdot 1/5 = 1/15 \approx 6.7\%\)
\end{proof}

\subsubsection{(b)}
What is the probability that the student will answer exactly one of the questions correctly?

\begin{proof}
     The probability that the student answers exactly one question correctly is \(P((A \cap B^c) \cup (A^c \cap B)) = P(A \cap
     B^c) + P(A^c \cap B) = P(A)P(B^c) + P(A^c)P(B) = 1/3 \cdot 4/5 + 2/3 \cdot 1/5 = 6/15 = 2/5 = 40\%\)
\end{proof}

\subsubsection{(c)}
What is the probability that the student will answer neither question correctly?

\begin{proof}
     One solution is to say that the probability that the student answers both questions incorrectly is \(P(A^c \cap B^c)\), and
     \(P(A^c \cap B^c) = P(A^c)P(B^c)\) by the result of exercise 22. Thus the answer is \(P(A^c)P(B^c) = 2/3 \cdot 4/5 = 8/15
     \approx 53.3\%\)

     Another solution uses the fact that the event that the student answers both questions incorrectly is the complement of the
     event that the student answers at least one question correctly. Thus, by the results of parts (a) and (b), the
     answer is \(1-(1/15+2/5) = 1-7/15 = 8/15 \approx 53.3\%\)
\end{proof}

\subsection{Exercise 24}
A software company uses two quality assurance (QA) checkers X and Y to check an application for bugs. X misses 12\% of the bugs and Y misses 15\%. Assume that the QA checkers work independently.

\subsubsection{(a)}
What is the probability that a randomly chosen bug will be missed by both QA checkers?

\begin{proof}
     Let \(X, Y\) be the events that a randomly chosen bug is missed by proofreader X, Y, respectively. So \(P(X) = 0.12\),
     \(P(Y) = 0.15\), and by independence \(P(X \cap Y) = P(X)P(Y) = 0.12 \cdot 0.15 = 0.018 = 1.8\%\).
\end{proof}

\subsubsection{(b)}
If the program contains 1,000 bugs, what number can be expected to be missed?

\begin{proof}
     \(0.018 \cdot 1000 = 18\) bugs.
\end{proof}

\subsection{Exercise 25}
A coin is loaded so that the probability of heads is 0.7 and the probability of tails is 0.3. Suppose that the coin is
tossed twice and that the results of the tosses are independent.

\subsubsection{(a)}
What is the probability of obtaining exactly two heads?
\begin{proof}
     HH = \(0.7 \cdot 0.7 = 0.49\)
\end{proof}

\subsubsection{(b)}
What is the probability of obtaining exactly one head?
\begin{proof}
     HT = \(0.7 \cdot 0.3 = 0.21\) + TH = \(0.3 \cdot 0.7 = 0.21\) = 0.42
\end{proof}

\subsubsection{(c)}
What is the probability of obtaining no heads?
\begin{proof}
     TT = \(0.3 \cdot 0.3 = 0.09\)
\end{proof}

\subsubsection{(d)}
What is the probability of obtaining at least one head?

\begin{proof}
     HT + TH + HH = 0.21 + 0.21 + 0.49 = 0.91, or 1 - TT = 1 - 0.09 = 0.91.
\end{proof}

\subsection{Exercise 26}
Describe a sample space and events \(A, B\), and \(C\), where \(P(A \cap B \cap C) = P(A) \cdot P(B) \cdot P(C)\) but
\(A, B\), and \(C\) are not pairwise independent.

\begin{proof}
     Let \(S = \{a, b, c\}, A = \{a,b\}, B = \{b,c\}, C = \es\).

     Then \(P(A) = 2/3, P(B) = 2/3, P(C) = 0\), \(A \cap B = \{b\}\) and \(A \cap B \cap C = \es\), so
     \[
          P(A \cap B \cap C) = 0 = (2/3) \cdot (2/3) \cdot 0 = P(A) \cdot P(B) \cdot P(C);
     \]
     but \(P(A \cap B) = 1/3 \neq (2/3) \cdot (2/3) = P(A) \cdot P(B)\) so \(A\) and \(B\) are not independent, hence \(A,B\)
     and \(C\) are not pairwise independent.
\end{proof}

\subsection{Exercise 27}
The example used to introduce conditional probability described a family with two children each of whom was equally
likely to be a boy or a girl. The example showed that if it is known that one child is a boy, the probability that the other
child is a boy is 1/3. Now imagine the same kind of family—two children each of whom is equally likely to be a boy or a girl.
Suppose you meet one of the children and that child is a boy. What is the probability that the other child is a boy?
Explain. (Be careful. The answer may surprise you.)

\begin{proof}
     1/2 but why? I don't get it. It might be hidden in the vague meanings of the words ``meeting'' and ``known''. This is the
     best explanation I can come up with:

     Consider the 4 possibilities BB, BG, GB, GG in terms of ``the order of meeting them''. This means that, in the case of BG,
     we first meet a boy, then we meet a girl. So if we meet the first child, and it's a boy, then the possibilities are BB and
     BG (GB is not possible because we first met a boy). Then the second child has 1/2 chance to be a boy.

     Whereas, in the original example of ``it is known that one child is a boy'', this meeting order is disregarded, so there
     are three possibilities: BG, GB, BB.

     If you ask me, this is really stupid and playing with the vague meanings of words. Because once we meet a boy, we know
     there is at least one boy, so we are back in the ``same knowledge state'' as the original example.

          {\it EDIT:} Turns out this is a well-known ``paradox'' called Sisters Paradox (where the question is asked with girl instead
     of boy). Here is an interesting excerpt from Jukka Corander:

     I have earlier done an experiment by asking this question from two separate audiences; one consisting of people with rigorous
     mathematical background, the other one consisting of scientists in general (ranging from economics to history).
     Without exceptions, members of the mathematical audience answered that the probability equals 1/3 and members of the
     other audience answered that it equals 1/2. The general audience referred to common sense when motivating their
     answer, but the mathematically oriented audience backed up their reasoning with a calculation referring to the very
     definition of conditional probability.

     Most maths sources present this problem as an example of the mathematically untrained people have concerning understanding
     conditional probabilities. In roughly 99 out of 100 sources, the solution given to the problem equals 1/3, and these
     sources typically claim that people who claim the answer to equal 1/2, are blatantly wrong. In fact, more rarely any
     sources (one is Dr. Math at the web) provide an analysis of the problem at more depth, and show that both 1/3 and 1/2 are
     correct answers, and moreover, that 1/2 can be a more reasonable answer under certain circumstances.

     The ambiguity arises from the fact that the problem formulation doesn’t state exactly in probabilistic terms how
     you gained the information about the existence of the girl in this particular family. To really solve the problem in a more
     realistic way, one needs to assume a probabilistic model which generates your observation.

     End Quote.

     So my thoughts were right and the problem is ambiguously stated. Unfortunately probability is full of such ill-posed,
     poorly structured / stated problems.

     The trick here is how ``knowing'' and ``meeting'' work in detail. In the ``meeting order'' (1/2) explanation we are
     assuming that parents put their two children in some fixed order, and made an agreement that the ``first'' child must
     meet strangers before the ``second''. So, if we meet a boy first, it's either BB or BG. But it might easily as well be
     that we are in the case GB, if the parents did no such ordering / agreement, or the way we acquire our information /
     make our observation / ``meet'' one of the children is not affected by such arrangements; then we'd get 1/3.
\end{proof}

\subsection{Exercise 28}
A coin is loaded so that the probability of heads is 0.7 and the probability of tails is 0.3. Suppose that the coin is
tossed ten times and that the results of the tosses are mutually independent.

\subsubsection{(a)}
What is the probability of obtaining exactly seven heads?
\begin{proof}
     \(\binom{10}{7} \cdot (0.7)^7 \cdot (0.3)^3 = 0.26682792 \approx 26.7\%\)
\end{proof}

\subsubsection{(b)}
What is the probability of obtaining exactly ten heads?
\begin{proof}
     \((0.7)^{10} = 0.028247525 \approx 2.8\%\)
\end{proof}

\subsubsection{(c)}
What is the probability of obtaining no heads?
\begin{proof}
     \((0.3)^{10} = 0.000005905 \approx 0.0006\%\)
\end{proof}

\subsubsection{(d)}
What is the probability of obtaining at least one head?
\begin{proof}
     \(1 - 0.000005905 \approx 99.999\%\)
\end{proof}

\subsection{Exercise 29}
Suppose that ten items are chosen at random from a large batch delivered to a company. The manufacturer claims that just 3\%
of the items in the batch are defective. Assume that the batch is large enough so that even though the selection is made
without replacement, the number 0.03 can be used to approximate the probability that any one of the ten items is
defective. In addition, assume that because the items are chosen at random, the outcomes of the choices are mutually
independent. Finally, assume that the manufacturer’s claim is correct.

\subsubsection{(a)}
What is the probability that none of the ten is defective?
\begin{proof}
     \(\binom{10}{0} \cdot (0.03)^0 \cdot (1 - 0.03)^{10} \approx 73.7\%\)
\end{proof}

\subsubsection{(b)}
What is the probability that at least one of the ten is defective?

\begin{proof}
     \(1 - 0.737424127 = 0.262575873 \approx 26.2\%\)
\end{proof}

\subsubsection{(c)}
What is the probability that exactly four of the ten are defective?

\begin{proof}
     \(\binom{10}{4} \cdot (0.03)^4 \cdot (1 - 0.03)^{10-4} = 210 \cdot 0.00000081 \cdot 0.832972005 = 0.000141689 \approx
     0.014\%\)
\end{proof}

\subsubsection{(d)}
What is the probability that at most two of the ten are defective?

\begin{proof}
     Exactly zero defective: \(\approx 73.7\%\)

     Exactly one defective: \(\binom{10}{1} \cdot (0.03)^1 \cdot (1 - 0.03)^{10-1} = 0.228069318 \approx 22.8\%\)

     Exactly two defective: \(\binom{10}{2} \cdot (0.03)^2 \cdot (1 - 0.03)^{10-2} = 0.031741606 \approx 3.2\%\)

     At most two defective: \(\approx 73.7+22.8+3.2 = 99.7\%\)
\end{proof}

\subsection{Exercise 30}
Suppose the probability of a false positive result on a mammogram is 4\% and that radiologists’ interpretations of
mammograms are mutually independent in the sense that whether or not a radiologist finds a positive result on one mammogram
does not influence whether or not a radiologist finds a positive result on another mammogram. Assume that a woman has
a mammogram every year for ten years.

\subsubsection{(a)}
What is the probability that she will have no false positive results during that time?

\begin{proof}
     \((0.96)^{10} = 0.664832636 \approx 66.5\%\)
\end{proof}

\subsubsection{(b)}
What is the probability that she will have at least one false positive result during that time?

\begin{proof}
     The probability that a woman will have at least one false positive result over a period of ten years is
     \(1 - (0.96)^{10} \approx 33.5\%\).
\end{proof}

\subsubsection{(c)}
What is the probability that she will have exactly two false positive results during that time?

\begin{proof}
     \(\binom{10}{2} \cdot (0.04)^{2} \cdot (0.96)^{8} = 0.05194005 \approx 5.2\%\)
\end{proof}

\subsubsection{(d)}
Suppose that the probability of a false negative result on a mammogram is 2\%, and assume that the probability that a
randomly chosen woman has breast cancer is 0.0002.

(i) If a woman has a positive test result one year, what is the probability that she actually has breast cancer?

(ii) If a woman has a negative test result one year, what is the probability that she actually has breast cancer?

\begin{proof}
     Let \(A\) be the event that the result is positive, \(B\) be the event that the result is negative and \(C\) be the event
     that a woman has breast cancer. So \(P(C) = 0.0002\) and \(P(C^c) = 0.9998\).

     (i) \(P(A|C)\) is the probability that the result is positive given that a woman has breast cancer. If it's for 1 year, then
     there is a true positive 1 year, and false positives for the other 9 years. Therefore \(P(A|C) = \binom{10}{1} \cdot
     (0.96)^1 \cdot (0.04)^9 = 10 \cdot \frac{96}{100} \cdot \frac{4^9}{10^{18}} = \frac{25,165,824}{10^{19}} =
     0.000000000002516582 \approx 0\%\).

     \(P(A|C^c)\) is the probability that the result is positive given that a woman does not have breast cancer. If it's for 1
     year, then there is a false positive for 1 year, and true positives for the other 9 years. Therefore \(P(A|C^c) =
     \binom{10}{1} \cdot (0.04)^1 \cdot (0.96)^9 = 0.277013598\).

     We want to find \(P(C|A)\). By Bayes' theorem, \(P(C|A) =\)
     \[
          \frac{P(A|C)P(C)}{P(A|C)P(C) + P(A|C^c)P(C^c)} = \frac{0 \cdot 0.0002}{0 \cdot 0.0002 + 0.277013598 \cdot 0.9998} \approx 0\%
     \]
     (ii) \(P(B|C)\) is the probability that the result is negative given that a woman has breast cancer. If it's for 1 year, then
     there is a false negative 1 year, and no false negatives for the other 9 years. Therefore \(P(B|C) = \binom{10}{1} \cdot
     (0.02)^1 \cdot (0.98)^9 = 0.166749552\).

     \(P(B|C^c)\) is the probability that the result is negative given that a woman does not have breast cancer. If it's for 1
     year, then there is a true negative for 1 year, and false negatives for the other 9 years. Therefore \(P(B|C^c) =
     \binom{10}{1} \cdot (0.98)^1 \cdot (0.02)^9 = 10 \cdot \frac{98}{100} \cdot \frac{512}{10^{18}} =
     0.0000000000000050176 \approx 0\%\).

     We want to find \(P(C|B)\). By Bayes' theorem, \(P(C|B) =\)
     \[
          \frac{P(B|C)P(C)}{P(B|C)P(C) + P(B|C^c)P(C^c)} = \frac{0.166749552 \cdot 0.0002}{0.166749552 \cdot 0.0002 + 0
               \cdot 0.9998} \approx 100\%
     \]
\end{proof}

\subsection{Exercise 31}
Empirical data indicate that approximately 103 out of every 200 children born are male. Hence the probability of a newborn
being male is about 51.5\%. Suppose that a family has six children, and suppose that the genders of all the children are
mutually independent.

\subsubsection{(a)}
What is the probability that none of the children is male?

\begin{proof}
     Female: 48.5\%, so \((0.485)^6 = 0.013015188 \approx 1.3\%\)
\end{proof}

\subsubsection{(b)}
What is the probability that at least one of the children is male?

\begin{proof}
     \(1 - 0.013015188 = 0.986984812 \approx 98.7\%\)
\end{proof}

\subsubsection{(c)}
What is the probability that exactly five of the children are male?

\begin{proof}
     \(\binom{6}{5} \cdot (0.485)^1 \cdot (0.515)^5 = 0.105421486 \approx 10.54\%\)
\end{proof}

\subsection{Exercise 32}
A person takes a multiple-choice exam in which each question has four possible answers. Suppose that the person has no idea
about the answers to three of the questions and simply chooses randomly for each one.

\subsubsection{(a)}
What is the probability that the person will answer all three questions correctly?

\begin{proof}
     \(\binom{3}{3} \cdot (1/4)^3 \cdot (3/4)^0 = 1/64\)
\end{proof}

\subsubsection{(b)}
What is the probability that the person will answer exactly two questions correctly?

\begin{proof}
     \(\binom{3}{2} \cdot (1/4)^2 \cdot (3/4)^1 = 9/64\)
\end{proof}

\subsubsection{(c)}
What is the probability that the person will answer exactly one question correctly?

\begin{proof}
     \(\binom{3}{1} \cdot (1/4)^1 \cdot (3/4)^2 = 27/64\)
\end{proof}

\subsubsection{(d)}
What is the probability that the person will answer no questions correctly?

\begin{proof}
     \(\dps \binom{3}{0} \cdot (1/4)^0 \cdot (3/4)^3 = 27/64\)
\end{proof}

\subsubsection{(e)}
Suppose that the person gets one point of credit for each correct answer and that 1/3 point is deducted for each
incorrect answer. What is the expected value of the person’s score for the three questions?

\begin{proof}
     All 3 correct: 3 points, exactly 2 correct: \(2-1/3 = 5/3\) points, exactly 1 correct: \(1 - 2/3 = 1/3\) points, no
     correct: \(-3 \cdot (1/3) = -1\) points. So the expected value is
     \[
          \frac{1}{64} \cdot 3 + \frac{9}{64} \cdot \frac{5}{3} + \frac{27}{64} \cdot \frac{1}{3} + \frac{27}{64} \cdot (-1) =
          \frac{9 + 45 + 27 - 81}{192} = 0
     \]
\end{proof}

\subsection{Exercise 33}
In exercise 23 of Section 9.8, let \(C_k\) be the event that the gambler has \(k\) dollars, wins the next roll of the die,
and is eventually ruined, let \(D_k\) be the event that the gambler has \(k\) dollars, loses the next roll of the die, and
is eventually ruined, and let \(P_k\) be the probability that the gambler is eventually ruined if he has \(k\) dollars. Use
the probability axioms and the definition of conditional probability to derive the equation
\(P_{k-1} = \frac{1}{6}P_k + \frac{5}{6}P_{k-2}\).

\begin{proof}
     Notice that \(C_k\) and \(D_k\) are disjoint events for every \(k\), since one requires winning and the other losing. Also
     note that \(E_k = C_k \cup D_k\) is the event of being eventually ruined from \(k\) dollars. So by axioms of
     probability \(P_k = P(E_k) = P(C_k) + P(D_k)\). Let's try to find \(P_{k-1} = P(C_{k-1}) + P(D_{k-1})\).

     \(C_{k-1}\) is the event that the gambler starts with \(\$k-1\), wins \$1 (which brings him up to \(\$k\)), and is
     eventually ruined. Notice that this is the same as the event of winning and then \(E_k\). So by the multiplication rule
     \(P(C_{k-1}) = \frac{1}{6} \cdot P(E_k) = \frac{1}{6}P_k\).

     \(D_{k-1}\) is the event that the gambler starts with \(\$k-1\), loses \$1 (which brings him to \(\$k-2\)), and is
     eventually ruined. Notice that this is the same as the event of losing and then \(E_{k-2}\). So by the multiplication rule
     \(P(D_{k-1}) = \frac{5}{6} \cdot P(E_{k-2}) = \frac{5}{6}P_{k-2}\).

     Using the results above, \(P_{k-1} = \frac{1}{6}P_k + \frac{5}{6}P_{k-2}\).
\end{proof}

\subsection{Exercise 34}
Use conditional probability to analyze exercise 20 in Section 9.1. Let X be the event that the prize is not behind door A,
and let Y be the event that you switch and choose the door with the prize. Should you switch? Explain why or why not.

     {\it Hint:} \(P(Y) = P(Y \cap X) + P(Y \cap X^c)\)

\begin{proof}
     Sticking with door A, I have 1/5 chance to win the prize (this is the original probability that the prize is behind door A).

     What about switching? \(P(Y) = P(Y \cap X) + P(Y \cap X^c)\), but \(P(Y \cap X^c) = 0\) because \(Y \cap X^c = \es\),
     because if the prize is behind door A then I cannot switch and choose A (because I switch from A). So \(P(Y) = P(Y \cap X)\).

     By above and definition of conditional probability, \(P(Y) = P(Y \cap X) = P(Y|X)P(X) = \frac{4}{5}P(Y|X)\). Now if the
     prize is not behind door A, and I switch, I have 1/3 chance of picking the right door. So \(P(Y|X) = 1/3\) and therefore
     \(P(Y) = 4/15\).

     Notice \(\frac{4}{15} > \frac{1}{5}\) therefore I should switch.
\end{proof}

\end{document}
