\documentclass[14pt]{extarticle}

\usepackage[table]{xcolor} % colored lines for tables
\usepackage[normalem]{ulem} % strike through text
\usepackage{amsmath,mathtools,amsfonts,amsthm,amssymb,hyperref}
\usepackage{parskip,geometry,latexsym,bookmark,mathtools,float,cancel}
\usepackage{minted,tcolorbox,bm}

\usepackage[mathscr]{euscript}
\let\euscr\mathscr
\let\mathscr\relax
\usepackage[scr]{rsfso}
\newcommand{\ps}{\mathscr{P}} % for the Power set P

\newtheorem{defn}{Definition}
\newtheorem{thm}{Theorem}
\newtheorem{claim}{Claim}
\newtheorem{lemma}{Lemma}

\newcommand{\dps}{\displaystyle}
\newcommand{\es}{\varnothing}
\newcommand{\fbl}{\underline{\hspace{1cm}}\,\,}
\newcommand{\R}{\mathbb{R}}
\newcommand{\Q}{\mathbb{Q}}
\newcommand{\Z}{\mathbb{Z}}
\newcommand{\from}{\leftarrow}
\newcommand{\true}{{\bf t}}
\newcommand{\false}{{\bf c}}
\newcommand{\bic}{\leftrightarrow}
\newcommand{\da}{\downarrow}
\newcommand{\fa}{\forall}
\newcommand{\te}{\exists}
\newcommand{\cy}{\color{cyan}}

\newcommand{\colsq}[1]{{\color{#1} $\blacksquare$}}

\newcommand{\base}[1]{{\cy #1}} % for log bases
\newcommand{\floor}[1]{{\left\lfloor#1\right\rfloor}}
\newcommand{\ceil}[1]{{\left\lceil#1\right\rceil}}
\newcommand\Ccancel[2][black]{\renewcommand\CancelColor{\color{#1}}\cancel{#2}}
\newcommand\Cbcancel[2][black]{\renewcommand\CancelColor{\color{#1}}\bcancel{#2}}

\renewcommand{\arraystretch}{1.2}

\hypersetup{colorlinks,allcolors=blue,linktoc=all}
\geometry{a4paper}
\geometry{margin=0.42in}

\title{Solutions to Chapter 8, Susanna Epp Discrete Math 5th Edition}

\author{https://github.com/spamegg1}

\begin{document}
\maketitle
\tableofcontents


\section{Exercise Set 8.1}

\subsection{Exercise 1}
As in Example 8.1.2, the {\bf congruence modulo 2} relation $E$ is defined from $\Z$ to $\Z$ as follows: For every
ordered pair \((m, n) \in \Z \times \Z\), $m$ $E$ $n$ \(\iff m - n\) is even.

\subsubsection{(a)}
Is $0$ $E$ $0$? Is $5$ $E$ $2$? Is $(6, 6) \in E$? Is $(21, 7) \in E$?

\begin{proof}
        $0$ $E$ $0$ because \(0 - 0 = 0 = 2 \cdot 0\), so \(2 \mid (0 - 0)\). $5$ $\Ccancel{E}$ $2$ because \(5 - 2 = 3\) and
        \(3 \neq 2k\) for any integer $k$, so \(2 \nmid (5 - 2)\). \((6, 6) \in E\) because \(6 - 6 = 0 = 2 \cdot 0\), so \(2
        \mid (6 - 6). (-1, 7) \in E\) because \(-1 - 7 = -8 = 2 \cdot (-4)\), so \(2 \mid (-1 - 7)\).
\end{proof}

\subsubsection{(b)}
Prove that for any even integer $n$, $n$ $E$ $0$.

\begin{proof}
        Assume $n$ is even. By definition of even, $n=2k$ for some integer $k$. Then \(n - 0 = 2k - 0 = 2k\) is also even.
        Therefore by definition of $E$, $n$ $E$ 0.
\end{proof}

\subsection{Exercise 2}
Prove that for all integers $m$ and $n$, $m - n$ is even if, and only if, both $m$ and $n$ are even or both $m$ and
$n$ are odd.

\begin{proof}
        \(\bm{\implies}\): Assume $m-n$ is even. {\it [We want to prove that both $m$ and $n$ are even or both $m$ and $n$
                                are odd.]} By definition of even, \(m - n = 2k\) for some integer $k$. There are 4 cases:

        {\bf Case 1: both $m$ and $n$ are even:} Nothing to prove.

                {\bf Case 2: both $m$ and $n$ are odd:} Nothing to prove.

                {\bf Case 3: $m$ is even, $n$ is odd:} By definitions of even and odd, \(m=2k, n=2l+1\) for some integers $k,l$.
        So \(m-n = 2k-2l-1 = 2(k-l-1) + 1\) where \(k-l-1\) is an integer. So by definition of odd, \(m-n\) is odd, a
        contradiction. So this case is impossible.

                {\bf Case 4: $m$ is odd, $n$ is even:} By definitions of even and odd, \(m=2k+1, n=2l\) for some integers $k,l$.
        So \(m-n = 2k+1-2l = 2(k-l) + 1\) where \(k-l\) is an integer. So by definition of odd, \(m-n\) is odd, a
        contradiction. So this case is impossible.

        \(\bm{\impliedby}\): Assume both $m$ and $n$ are even or both $m$ and $n$ are odd. {\it [We want to prove that $m-n$
                                is even.]} There are 2 cases:

        {\bf Case 1: both $m$ and $n$ are even:} By definition of even, \(m = 2k, n = 2l\) for some integers $k,l$. Then
        \(m-n = 2k-2l = 2(k-l)\) where $k-l$ is an integer. So by definition, $m-n$ is even.

                {\bf Case 2: both $m$ and $n$ are odd:} By definition of even, \(m = 2k+1, n = 2l+1\) for some integers $k,l$. Then
        \(m-n = 2k+1-2l-1 = 2(k-l)\) where $k-l$ is an integer. So by definition, $m-n$ is even.
\end{proof}

\subsection{Exercise 3}
The congruence modulo 3 relation, $T$, is defined from $\Z$ to $\Z$ as follows: For all integers $m$ and $n$, $m$ $T$
$n$ \(\iff 3 \mid (m - n)\).

\subsubsection{(a)}
Is $10$ $T$ $1$? Is $1$ $T$ $10$? Is \((2, 2) \in T\)? Is \((8, 1) \in T\)?

\begin{proof}
        10 $T$ 1 because \(10 - 1 = 9 = 3 \cdot 3\), and so \(3 \mid (10 - 1)\).

        1 $T$ 10 because \(1 - 10 = -9 = 3 \cdot (-3)\), and so \(3 \mid (1 - 10)\).

        2 $T$ 2 because \(2 - 2 = 0 = 3 \cdot 0\), and so \(3 \mid (2 - 2)\).

        8 $\Ccancel{T}$ 1 because \(8 - 1 = 7 \neq 3k\), for any integer $k$. So \(3 \nmid (8 - 1)\).
\end{proof}

\subsubsection{(b)}
List five integers $n$ such that $n$ $T$ $0$.

\begin{proof}
        One possible answer: 3, 6, 9, $-3$, $-6$
\end{proof}

\subsubsection{(c)}
List five integers $n$ such that $n$ $T$ 1.

\begin{proof}
        One possible answer: 4, 7, 10, $-2$, $-5$
\end{proof}

\subsubsection{(d)}
List five integers $n$ such that $n$ $T$ $2$.

\begin{proof}
        One possible answer: 5, 8, 11, $-1$, $-4$
\end{proof}

\subsubsection{(e)}
Make and prove a conjecture about which integers are related by $T$ to $0$, which integers are related by $T$ to
1, and which integers are related by $T$ to 2.

All integers of the form \(3k + 1\), for some integer $k$, are related by $T$ to $1$.

\begin{proof}
        All integers of the form \(3k\), for some integer $k$, are related by $T$ to 0.

        All integers of the form \(3k+1\), for some integer $k$, are related by $T$ to 1.

        All integers of the form \(3k+2\), for some integer $k$, are related by $T$ to 2.
\end{proof}

\subsection{Exercise 4}
Define a relation $P$ on $\Z$ as follows: For every ordered pair \((m, n) \in \Z \times \Z\), $m$ $P$ $n$ \(\iff\) $m$
and $n$ have a common prime factor.

\subsubsection{(a)}
Is 15 $P$ 25?

\begin{proof}
        Yes, because 15 and 25 are both divisible by 5, which is prime.
\end{proof}

\subsubsection{(b)}
Is 22 $P$ 27?

\begin{proof}
        No, because 22 and 27 have no common prime factor.
\end{proof}

\subsubsection{(c)}
Is 0 $P$ 5?

\begin{proof}
        Yes, because 0 and 5 are both divisible by 5, which is prime.
\end{proof}

\subsubsection{(d)}
Is 8 $P$ 8?

\begin{proof}
        Yes, because 8 and 8 are both divisible by 2, which is prime.
\end{proof}

\subsection{Exercise 5}
Let \(X = \{a, b, c\}\). Recall that \(\ps(X)\) is the power set of $X$. Define a relation {\bf S} on \(\ps(X)\)
as follows: For all sets $A$ and $B$ in \(\ps(X)\), $A$ {\bf S} $B$ $\iff$ $A$ has the same number of elements as $B$.

\subsubsection{(a)}
Is \(\{a, b\}\) {\bf S} \(\{b, c\}\)?

\begin{proof}
        Yes, because both \(\{a, b\}\) and \(\{b, c\}\) have two elements.
\end{proof}

\subsubsection{(b)}
Is \(\{a\}\) {\bf S} \(\{a,b\}\)?

\begin{proof}
        No, one has 1 element, the other has 2 elements.
\end{proof}

\subsubsection{(c)}
Is \(\{c\}\) {\bf S} \(\{b\}\)?

\begin{proof}
        Yes, because both \(\{c\}\) and \(\{b\}\) have one element.
\end{proof}

\subsection{Exercise 6}
Let \(X = \{a, b, c\}\). Define a relation {\bf J} on \(\ps(X)\) as follows: For all sets $A$ and $B$ in $\ps(X)$,
$A$ {\bf J} $B$ $\iff$ \(A \cap B \neq \es\).

\subsubsection{(a)}
Is $\{a\}$ {\bf J} $\{c\}$?

\begin{proof}
        No, because \(\{a\} \cap \{c\} = \es\).
\end{proof}

\subsubsection{(b)}
Is $\{a, b\}$ {\bf J} $\{b, c\}$?

\begin{proof}
        Yes, because \(\{a,b\} \cap \{b,c\} = \{b\} \neq \es\).
\end{proof}

\subsubsection{(c)}
Is $\{a, b\}$ {\bf J} $\{a, b, c\}$?

\begin{proof}
        Yes, because \(\{a,b\} \cap \{a,b,c\} = \{a,b\} \neq \es\).
\end{proof}

\subsection{Exercise 7}
Define a relation $R$ on $\Z$ as follows: For all integers $m$ and $n$, $m$ $R$ $n$ $\iff$ \(5 \mid (m^2 - n^2)\).

\subsubsection{(a)}
Is 1 $R$ $(-9)$?

\begin{proof}
        Yes. 1 $R$ \((-9) \iff 5 \mid (1^2 - (-9)^2)\). But \(1^2 - (-9)^2 = 1 - 81 = -80\), and \(5 \mid (-80)\) because
        \(-80 = 5 \cdot (-16)\).
\end{proof}

\subsubsection{(b)}
Is 2 $R$ 13?

\begin{proof}
        Yes, \(2^2 - (13)^2 = 4 - 169 = -165 = 5 \cdot (-33)\). So \(5 \mid 2^2 - (13)^2\).
\end{proof}

\subsubsection{(c)}
Is 2 $R$ $(-8)$?

\begin{proof}
        Yes, \(2^2 - (-8)^2 = 4 - 64 = -60 = 5 \cdot (-12)\). So \(5 \mid 2^2 - (-8)^2\).
\end{proof}

\subsubsection{(d)}
Is $(-8)$ $R$ 2?

\begin{proof}
        Yes, \((-8)^2 - 2^2 = 64 - 4 = 60 = 5 \cdot 12\). So \(5 \mid (-8)^2 - 2^2\).
\end{proof}

\subsection{Exercise 8}
Let $A$ be the set of all strings of $a$’s and $b$’s of length 4. Define a relation $R$ on $A$ as follows: For
every \(s, t \in A\), $s$ $R$ $t$ $\iff$ $s$ has the same first two characters as $t$.

\subsubsection{(a)}
Is $abaa$ $R$ $abba$?

\begin{proof}
        Yes, because both $abaa$ and $abba$ have the same first two characters $ab$.
\end{proof}

\subsubsection{(b)}
Is $aabb$ $R$ $bbaa$?

\begin{proof}
        No, because the first two characters of $aabb$ are different from the first two characters of $bbaa$.
\end{proof}

\subsubsection{(c)}
Is $aaaa$ $R$ $aaab$?

\begin{proof}
        Yes, because both $aaaa$ and $aaab$ have the same first two characters $aa$.
\end{proof}

\subsubsection{(d)}
Is $baaa$ $R$ $abaa$?

\begin{proof}
        No, because the first two characters of $baaa$ are different from the first two characters of $abaa$.
\end{proof}

\subsection{Exercise 9}
Let $A$ be the set of all strings of 0’s, 1’s, and 2’s of length 4. Define a relation $R$ on $A$ as follows: For
every \(s, t \in A\), $s$ $R$ $t$ $\iff$ the sum of the characters in $s$ equals the sum of the characters in $t$.

\subsubsection{(a)}
Is 0121 $R$ 2200?

\begin{proof}
        Yes, because the sum of the characters in 0121 is 4 and the sum of the characters in 2200 is also 4.
\end{proof}

\subsubsection{(b)}
Is 1011 $R$ 2101?

\begin{proof}
        No, because the sum of the characters in 1011 is 3, whereas the sum of the characters in 2101 is 4.
\end{proof}

\subsubsection{(c)}
Is 2212 $R$ 2121?

\begin{proof}
        No, because the sum of the characters in 2212 is 7, whereas the sum of the characters in 2121 is 6.
\end{proof}

\subsubsection{(d)}
Is 1220 $R$ 2111?

\begin{proof}
        Yes, because the sum of the characters in 1220 is 5 and the sum of the characters in 2111 is also 5.
\end{proof}

\subsection{Exercise 10}
Let \(A = \{3, 4, 5\}\) and \(B = \{4, 5, 6\}\) and let $R$ be the “less than” relation. That is, for every ordered
pair \((x, y) \in A \times B, x \, R \, y \iff x < y\). State explicitly which ordered pairs are in $R$ and $R^{-1}$.

\begin{proof}
        \(R = \{(3, 4), (3, 5), (3, 6), (4, 5), (4, 6), (5, 6)\}\)

        \(R^{-1} = \{(4, 3), (5, 3), (6, 3), (5, 4), (6, 4), (6, 5)\}\)
\end{proof}

\subsection{Exercise 11}
Let \(A = \{3, 4, 5\}\) and \(B = \{4, 5, 6\}\) and let $S$ be the “divides” relation. That is, for every ordered pair
\((x, y) \in A \times B, x \, S \, y \iff x \mid y\). State explicitly which ordered pairs are in $S$ and $S^{-1}$.

\begin{proof}
        \(S = \{(3, 6), (4, 4), (5, 5)\}, S^{-1} = \{(6, 3), (4, 4), (5, 5)\}\)
\end{proof}

\subsection{Exercise 12}

\subsubsection{(a)}
Suppose a function \(F: X \to Y\) is one-to-one but not onto. Is \(F^{-1}\) (the inverse relation for $F$) a
function? Explain your answer.

\begin{proof}
        No. If \(F: X \to Y\) is not onto, then $F$ fails to be defined on all of $Y$. In other words, there is an element
        $y$ in $Y$ such that \((y, x) \notin F^{-1}\) for any \(x \in X\). Consequently, \(F^{-1}\) does not satisfy property
        (1) of the definition of function.
\end{proof}

\subsubsection{(b)}
Suppose a function \(F: X \to Y\) is onto but not one-to-one. Is \(F^{-1}\) (the inverse relation for $F$) a
function? Explain your answer.

\begin{proof}
        No. If \(F: X \to Y\) is not one-to-one, then $F$ for some $y$ in $Y$, there will be multiple potential values for
        \(F^{-1}(y)\). In other words, there is an element $y$ in $Y$ and elements \(x_1, x_2 \in X\) such that \((y, x_1)
        \in F^{-1}\) and \((y, x_2) \in F^{-1}\). Consequently, \(F^{-1}\) does not satisfy property (2) of the definition
        of function.
\end{proof}

{\bf \cy Draw the directed graphs of the relations defined in $13-18$.}

\subsection{Exercise 13}
Define a relation $R$ on \(A = \{0, 1, 2, 3\}\) by \(R = \{(0, 0), (1, 2), (2, 2)\}\).

\begin{proof}
        \begin{figure}[ht!]
                \centering
                \includegraphics[scale=0.4]{../images/8.1.13.png}
        \end{figure}
\end{proof}

\subsection{Exercise 14}
Define a relation $S$ on \(B = \{a, b, c, d\}\) by \(S = \{(a, b), (a, c), (b, c), (d, d)\}\).

\begin{proof}
        \begin{figure}[ht!]
                \centering
                \includegraphics[scale=0.4]{../images/8.1.14.png}
        \end{figure}
\end{proof}

\subsection{Exercise 15}
Let \(A = \{2, 3, 4, 5, 6, 7, 8\}\) and define a relation $R$ on $A$ as follows: For every \(x, y \in A, x \, R\, y
\iff x \mid y\).

\begin{proof}
        \begin{figure}[ht!]
                \centering
                \includegraphics[scale=0.4]{../images/8.1.15.png}
        \end{figure}
\end{proof}

\subsection{Exercise 16}
Let \(A = \{5, 6, 7, 8, 9, 10\}\) and define a relation $S$ on $A$ as follows: For every \(x, y \in A, x \, S \, y \iff
2 \mid (x - y)\).

\begin{proof}
        \begin{figure}[ht!]
                \centering
                \includegraphics[scale=0.4]{../images/8.1.16.png}
        \end{figure}
\end{proof}

\subsection{Exercise 17}
Let \(A = \{2, 3, 4, 5, 6, 7, 8\}\) and define a relation $T$ on $A$ as follows: For every \(x, y \in A, x \, T \, y
\iff 3 \mid (x - y)\).

\begin{proof}
        \begin{figure}[ht!]
                \centering
                \includegraphics[scale=0.4]{../images/8.1.17.png}
        \end{figure}
\end{proof}

\subsection{Exercise 18}
Let \(A = \{0, 1, 3, 4, 5, 6\}\) and define a relation $V$ on $A$ as follows: For every \(x, y \in A, x \, V \, y \iff
5 \mid (x^2 - y^2)\).

\begin{proof}
        \begin{figure}[ht!]
                \centering
                \includegraphics[scale=0.4]{../images/8.1.18.png}
        \end{figure}
\end{proof}

\subsection{Exercise 19}
Let \(A = \{2, 4\}\) and \(B = \{6, 8, 10\}\) and define relations $R$ and $S$ from $A$ to $B$ as follows: For every
\((x, y) \in A \times B, x \, R \, y \iff x \mid y\) and \( x \, S \, y \iff y - 4 = x\). State explicitly which
ordered pairs are in \(A \times B, R, S, R \cup S\), and \(R \cap S\).

\begin{proof}
        \(A \times B = \{(2, 6), (2, 8), (2, 10), (4, 6), (4, 8), (4, 10)\}\)

        \(R = \{(2, 6), (2, 8), (2, 10), (4, 8)\}\), \(S = \{(2, 6), (4, 8)\}\), \(R \cup S = R, R \cap S = S\)
\end{proof}

\subsection{Exercise 20}
Let \(A = \{-1, 1, 2, 4\}\) and \(B = \{1, 2\}\) and define relations $R$ and $S$ from $A$ to $B$ as follows: For every
\((x, y) \in A \times B, x \, R \, y \iff |x| \mid |y|\) and \(x \, S \, y \iff x - y\) is even. State explicitly
which ordered pairs are in \(A \times B, R, S, R \cup S\), and \(R \cap S\).

\begin{proof}
        \(A \times B = \{(-1, 1), (-1, 2), (1, 1), (1, 2), (2, 1), (2, 2), (4, 1), (4, 2)\}\)

        \(R = \{(-1, 1), (1, 1), (2, 2)\}\), \(S = \{(-1, 1), (1, 1), (2, 2), (4, 2)\}\), \(R \cup S = S, R \cap S = R\)
\end{proof}

\subsection{Exercise 21}
Define relations $R$ and $S$ on $R$ as follows: \(R = \{(x, y) \in \R \times \R \, | \, x < y\}\) and

\(S = \{(x, y) \in \R \times \R \, | \, x = y\}\). That is, $R$ is the “less than” relation and $S$ is the “equals”
relation on $\R$. Graph \(R, S, R \cup S\), and \(R \cap S\) in the Cartesian plane.

\begin{proof}
        The graph of the intersection of $R$ and $S$ is obtained by finding the set of all points common to both graphs. But
        there are no points for which both \(x < y\) and \(x = y\). Hence \(R \cap S = \es\) and the graph consists of no
        points at all.

        \begin{figure}[ht!]
                \centering
                \includegraphics[scale=0.35]{../images/8.1.21.png}
        \end{figure}
\end{proof}

\subsection{Exercise 22}
Define relations $R$ and $S$ on $R$ as follows: \(R = \{(x, y) \in \R \times \R \, | \, x^2 + y^2 = 4\}\) and
\(S = \{(x, y) \in \R \times \R \, | \, x = y\}\). Graph \(R, S, R \cup S\), and \(R \cap S\) in the Cartesian plane.

\begin{proof}
        \begin{figure}[ht!]
                \centering
                \includegraphics[scale=0.3]{../images/8.1.22.png}
        \end{figure}
\end{proof}

\subsection{Exercise 23}
Define relations $R$ and $S$ on $R$ as follows: \(R = \{(x, y) \in \R \times \R \, | \, y = |x|\}\) and
\(S = \{(x, y) \in \R \times \R \, | \, y = 1\}\). Graph \(R, S, R \cup S\), and \(R \cap S\) in the Cartesian plane.

\begin{proof}
        \begin{figure}[ht!]
                \centering
                \includegraphics[scale=0.35]{../images/8.1.23.png}
        \end{figure}
\end{proof}

\subsection{Exercise 24}
In Example 8.1.7 consider the query SELECT Patient\_ID\#, Name FROM S WHERE Primary\_Diagnosis = $X$. The response to
the query is the projection onto the first two coordinates of the intersection of the database with the set
\(A_1 \times A_2 \times A_3 \times \{X\}\).

\subsubsection{(a)}
Find the result of the query SELECT Patient\_ID\#, Name FROM S WHERE Primary\_Diagnosis = pneumonia.

\begin{proof}
        574329 Tak Kurosawa, 011985 John Schmidt
\end{proof}

\subsubsection{(b)}
Find the result of the query SELECT Patient\_ID\#, Name FROM S WHERE Primary\_Diagnosis = appendicitis.

\begin{proof}
        466581 Mary Lazars, 778400 Jamal Baskers
\end{proof}

\section{Exercise Set 8.2}

 {\bf \cy In $1-8$, a number of relations are defined on the set \(A = \{0, 1, 2, 3\}\). For each relation:

  a. Draw the directed graph.

  b. Determine whether the relation is reflexive.

  c. Determine whether the relation is symmetric.

  d. Determine whether the relation is transitive.

  Give a counterexample in each case in which the relation does not satisfy one of the properties.}

\subsection{Exercise 1}
\(R_1 = \{(0, 0), (0, 1), (0, 3), (1, 1), (1, 0), (2, 3), (3, 3)\}\)

\subsubsection{(a)}

\begin{proof}
        \begin{figure}[ht!]
                \centering
                \includegraphics[scale=0.35]{../images/8.2.1.a.png}
        \end{figure}
\end{proof}

\subsubsection{(b)}
\begin{proof}
        $R_1$ is not reflexive: \(2 \,\Ccancel{R_1} \, 2\).
\end{proof}

\subsubsection{(c)}
\begin{proof}
        $R_1$ is not symmetric: \(2 \,R_1\, 3\) but \(3 \, \Ccancel{R_1} \,2\).
\end{proof}

\subsubsection{(d)}
\begin{proof}
        $R_1$ is not transitive: \(1 \,R_1\, 0\) and \(0 \,R_1\, 3\) but \(1 \,\Ccancel{R_1}\, 3\).
\end{proof}

\subsection{Exercise 2}
\(R_2 = \{(0, 0), (0, 1), (1, 1), (1, 2), (2, 2), (2, 3)\}\)

\subsubsection{(a)}
\begin{proof}
        \begin{figure}[ht!]
                \centering
                \includegraphics[scale=0.35]{../images/8.2.2.a.png}
        \end{figure}
\end{proof}

\subsubsection{(b)}
\begin{proof}
        $R_2$ is not reflexive: \(3 \, \Ccancel{R_2} \,3\).
\end{proof}

\subsubsection{(c)}
\begin{proof}
        $R_2$ is not symmetric: \(2 \,R_2 \,3\) but \(3\, \Ccancel{R_2} \,2\).
\end{proof}

\subsubsection{(d)}
\begin{proof}
        $R_2$ is not transitive: \(0\, R_2\, 1\) and \(1 \,R_2 \,2\) but \(0 \,\Ccancel{R_2}\, 2\).
\end{proof}

\subsection{Exercise 3}
\(R_3 = \{(2, 3), (3, 2)\}\)

\subsubsection{(a)}

\begin{proof}
        \begin{figure}[ht!]
                \centering
                \includegraphics[scale=0.4]{../images/8.2.3.a.png}
        \end{figure}
\end{proof}

\subsubsection{(b)}

\begin{proof}
        $R_3$ is not reflexive: \(0 \, \Ccancel{R_3} \,0\).
\end{proof}

\subsubsection{(c)}

\begin{proof}
        $R_3$ is symmetric.
\end{proof}

\subsubsection{(d)}

\begin{proof}
        $R_3$ is not transitive: \(2\, R_3\, 3\) and \(3 \,R_3 \,2\) but \(2 \,\Ccancel{R_3}\, 2\).
\end{proof}

\subsection{Exercise 4}
\(R_4 = \{(1, 2), (2, 1), (1, 3), (3, 1)\}\)

\subsubsection{(a)}

\begin{proof}
        \begin{figure}[ht!]
                \centering
                \includegraphics[scale=0.4]{../images/8.2.4.a.png}
        \end{figure}
\end{proof}

\subsubsection{(b)}

\begin{proof}
        $R_4$ is not reflexive: \(0 \, \Ccancel{R_4} \,0\).
\end{proof}

\subsubsection{(c)}

\begin{proof}
        $R_4$ is symmetric.
\end{proof}

\subsubsection{(d)}

\begin{proof}
        $R_4$ is not transitive: \(2\, R_4\, 1\) and \(1 \,R_4 \,3\) but \(2 \,\Ccancel{R_4}\, 3\).
\end{proof}

\subsection{Exercise 5}
\(R_5 = \{(0, 0), (0, 1), (0, 2), (1, 2)\}\)

\subsubsection{(a)}

\begin{proof}
        \begin{figure}[ht!]
                \centering
                \includegraphics[scale=0.35]{../images/8.2.5.a.png}
        \end{figure}
\end{proof}

\subsubsection{(b)}

\begin{proof}
        $R_5$ is not reflexive: \(3 \, \Ccancel{R_5} \, 3\).
\end{proof}

\subsubsection{(c)}

\begin{proof}
        $R_5$ is not symmetric: \(1 \,R_5 \,2\) but \(2\, \Ccancel{R_5} \,1\).
\end{proof}

\subsubsection{(d)}

\begin{proof}
        $R_5$ is transitive.
\end{proof}

\subsection{Exercise 6}
\(R_6 = \{(0, 1), (0, 2)\}\)

\subsubsection{(a)}

\begin{proof}
        \begin{figure}[ht!]
                \centering
                \includegraphics[scale=0.35]{../images/8.2.6.a.png}
        \end{figure}
\end{proof}

\subsubsection{(b)}

\begin{proof}
        $R_6$ is not reflexive: \(3 \, \Ccancel{R_6} \, 3\).
\end{proof}

\subsubsection{(c)}

\begin{proof}
        $R_6$ is not symmetric: \(0 \,R_6 \,1\) but \(1\, \Ccancel{R_6} \,0\).
\end{proof}

\subsubsection{(d)}

\begin{proof}
        $R_6$ is transitive.
\end{proof}

\subsection{Exercise 7}
\(R_7 = \{(0, 3), (2, 3)\}\)

\subsubsection{(a)}

\begin{proof}
        \begin{figure}[ht!]
                \centering
                \includegraphics[scale=0.35]{../images/8.2.7.a.png}
        \end{figure}
\end{proof}

\subsubsection{(b)}

\begin{proof}
        $R_7$ is not reflexive: \(3 \, \Ccancel{R_7} \, 3\).
\end{proof}

\subsubsection{(c)}

\begin{proof}
        $R_7$ is not symmetric: \(0 \,R_7 \, 3\) but \(3\, \Ccancel{R_7} \,0\).
\end{proof}

\subsubsection{(d)}

\begin{proof}
        $R_7$ is transitive.
\end{proof}

\subsection{Exercise 8}
\(R_8 = \{(0, 0), (1, 1)\}\)

\subsubsection{(a)}

\begin{proof}
        \begin{figure}[ht!]
                \centering
                \includegraphics[scale=0.35]{../images/8.2.8.a.png}
        \end{figure}
\end{proof}

\subsubsection{(b)}

\begin{proof}
        $R_8$ is not reflexive: \(3 \, \Ccancel{R_8} \, 3\).
\end{proof}

\subsubsection{(c)}

\begin{proof}
        $R_8$ is symmetric.
\end{proof}

\subsubsection{(d)}

\begin{proof}
        $R_8$ is transitive.
\end{proof}

{\bf \cy In 9–33, determine whether the given relation is reflexive, symmetric, transitive, or none of these.
Justify your answers. }

\subsection{Exercise 9}
$R$ is the “greater than or equal to” relation on the set of real numbers: For every \(x, y \in \R, x \, R \, y \iff
x \geq y\).

\begin{proof}
        {\bf \(\bm{R}\) is reflexive:} \(R\) is reflexive iff for every real number \(x, x \, R \, x\). By definition of
        \(R\), this means that for every real number \(x, x \geq x\). In other words, for every real number \(x, x > x\) or
        \(x = x\), which is true.

                {\bf \(\bm{R}\) is not symmetric:} \(R\) is symmetric iff for all real numbers \(x\) and \(y\), if \(x \, R \, y\)
        then \(y \, R \, x\). By definition of \(R\), this means that for all real numbers \(x\) and \(y\), if \(x \geq y\)
        then \(y \geq x\). The following counterexample shows that this is false. \(x = 1\) and \(y = 0\). Then \(x \geq y\),
        but \(y \ngeq x\) because \(1 \geq 0\) and \(0 \ngeq 1\).

                {\bf \(\bm{R}\) is transitive:} \(R\) is transitive iff for all real numbers \(x, y\), and \(z\), if \(x \, R \, y\)
        and \(y \, R \, z\) then \(x \, R \, z\). By definition of \(R\), this means that for all real numbers \(x, y\), and
        \(z\), if \(x \geq y\) and \(y \geq z\) then \(x \geq z\). This is true by definition of \(\geq\) and the transitive
        property of order for the real numbers. (See Appendix A, T18.)
\end{proof}

\subsection{Exercise 10}
$C$ is the circle relation on the set of real numbers: For every \(x, y \in \R, x \, C \, y \iff x^2 + y^2 = 1\).

\begin{proof}
        {\bf \(\bm{C}\) is not reflexive:} Let \(x = 0\). Then \(0^2 + 0^2 = 0 \neq 1\), therefore \(0 \Ccancel{C} 0\).

                {\bf \(\bm{C}\) is symmetric:} Assume \(x \, C \, y\). Then \(x^2 + y^2 = 1\). So \(y^2 + x^2 = 1\). So \(y\, C \, x\).

                {\bf \(\bm{C}\) is not transitive:} Let \(x = 1, y = 0, z = 1\). Then \(x \, C \, y\) because \(1^2 + 0^2 = 1\), and
        \(y \, C \, z\) because \(0^2 + 1^2 = 1\). However \(x \, \Ccancel{C} \, z\) because \(1^2 + 1^2 = 2 \neq 1\).
\end{proof}

\subsection{Exercise 11}
$D$ is the relation defined on $\R$ as follows: For every \(x, y \in \R, x \, D \, y \iff xy \geq 0\).

\begin{proof}
        {\bf \(\bm{D}\) is reflexive:} For all real numbers \(x, x \cdot x = x^2 \geq 0\) so \(x \, D \, x\).

                {\bf \(\bm{D}\) is symmetric:} Assume \(x \, D \, y\). Then \(xy \geq 0\). So \(yx \geq 0\). So \(y \, D \, x\).

                {\bf \(\bm{D}\) is not transitive:} Let \(x = 1, y = 0, z = -1\). Then \(xy = 0 \geq 0\) so \(x \, D \, y\), and
        \(yz = 0 \geq 0\) so \(y \, D \, z\), but \(xz = -1 \ngeq 0\) so \(x \Ccancel{D} z\).
\end{proof}

\subsection{Exercise 12}
$E$ is the congruence modulo 4 relation on $\Z$: For every \(m, n \in \Z, m \, E \, n \iff 4 \mid (m - n)\).

\begin{proof}
        {\bf \(\bm{E}\) is reflexive:} For all \(m \in \Z, (m-m) = 0 = 4 \cdot 0\) so \(4 \mid (m-m)\) thus \(m\,E\,m\).

                {\bf \(\bm{E}\) is symmetric:} Assume \(m \, E \, n\). Then \(4\mid(m-n)\). So \(m-n = 4\cdot k\) for some integer $k$.
        So \(n-m = 4 \cdot (-k)\) where $-k$ is an integer. So \(4 \mid (n-m)\) and \(n \, E \, m\).

                {\bf \(\bm{E}\) is transitive:} Assume \(m \, E \, n\) and \(n \, E \, o\). Then \(4 \mid (m-n)\) and \(4\mid (n-o)\).
        So \(m-n = 4k\) and \(n-o = 4l\) for some integers \(k,l\). So \(m-o = (m-n) + (n-o) = 4k+4l = 4(k+l)\) where \(k+l\)
        is an integer. Thus \(4 \mid (m-o)\) and \(m \,E\,o\).
\end{proof}

\subsection{Exercise 13}
$F$ is the congruence modulo 5 relation on $\Z$: For every \(m, n \in \Z, m \, F \, n \iff 5 \mid (m - n)\).

\begin{proof}
        {\bf \(\bm{F}\) is reflexive:} The proof is the same as in exercise 12.

                {\bf \(\bm{F}\) is symmetric:} The proof is the same as in exercise 12.

                {\bf \(\bm{F}\) is transitive:} The proof is the same as in exercise 12.
\end{proof}

\subsection{Exercise 14}
$O$ is the relation defined on $\Z$ as follows: For every \(m, n \in \Z, m \, O \, n \iff m - n\) is odd.

\begin{proof}
        {\bf \(\bm{O}\) is not reflexive:} \(0 - 0 = 0\) is even, therefore \(0 \Ccancel{O} 0\).

                {\bf \(\bm{O}\) is symmetric:} Assume \(m \, O \, n\). So \(m-n\) is odd. So \(m-n = 2k+1\) for some integer \(k\). So
        \(n-m = -2k-1 = 2(-k-1)+1\) where \(-k-1\) is an integer. So \(n-m\) is odd and \(n \, O \, m\).

                {\bf \(\bm{O}\) is not transitive:} \(2 - 1 = 1\) is odd so \(2 \, O \, 1\), and \(1 - 0 = 1\) is odd so \(1 \,O\,0\),
        but \(2-0 = 2\) is even so \(2 \Ccancel{O} 0\).
\end{proof}

\subsection{Exercise 15}
$D$ is the “divides” relation on \(Z^+\): For all positive integers $m$ and \(n, m \, D \, n \iff m \mid n\).

\begin{proof}
        {\bf \(\bm{D}\) is reflexive:} For all \(m \in \Z^+\) \(m = m \cdot 1\) therefore \(m \mid m\), so \(m \, D \, m\).

                {\bf \(\bm{D}\) is not symmetric:} \(3 \, D \, 6\) because \(3 \mid 6\) because \(6 = 3 \cdot 2\), but \(6 \Ccancel{D}
        3\) because \(6 \nmid 3\) since \(3 / 6 = 1/2\) is not an integer.

                {\bf \(\bm{D}\) is transitive:} Assume \(m \, D \, n\) and \(n \, D \, o\). Then \(m \mid n\) and \(n \mid o\). So
        \(n = mk\) and \(o = nl\) for some integers \(k, l\). So \(o = nl = (mk)l = m(kl)\) where \(kl\) is an integer. So
        \(m \mid o\) and \(m \, D \, o\).
\end{proof}

\subsection{Exercise 16}
$A$ is the “absolute value” relation on $\R$: For all real numbers $x$ and $y$, \(x \, A \, y \iff |x| = |y|\).

\begin{proof}
        {\bf \(\bm{A}\) is reflexive:} For all real numbers \(x, |x| = |x|\) so \(x \, A \, x\).

                {\bf \(\bm{A}\) is symmetric:} Assume \(x \, A \, y\) so \(|x| = |y|\). Then \(|y| = |x|\) so \(y \, A \, x\).

                {\bf \(\bm{A}\) is transitive:} Assume \(x \, A \, y\) and \(y \, A \, z\), so \(|x| = |y|\) and \(|y| = |z|\). Then
        \(|x| = |y| = |z|\) so \(x \, A \, z\).
\end{proof}

\subsection{Exercise 17}
Recall that a prime number is an integer that is greater than 1 and has no positive integer divisors other than 1
and itself. (In particular, 1 is not prime.) A relation $P$ is defined on $\Z$ as follows: For every \(m, n \in \Z, m
\, P \, n \iff \exists\) a prime number $p$ such that \(p \mid m\) and \(p \mid n\).

\begin{proof}
        {\bf \(\bm{P}\) is not reflexive:} There is no prime number $p$ such that \(p \mid 1\) and \(p \mid 1\). Thus
        \(1 \Ccancel{P} 1\).

                {\bf \(\bm{P}\) is symmetric:} Assume \(m \, P \, n\). So there is a prime number $p$ such that \(p \mid m\) and
        \(p \mid n\). So \(p \mid n\) and \(p \mid m\), and thus \(n \, P \, m\).

                {\bf \(\bm{P}\) is not transitive:} Let \(m = 6, n = 15, o = 35\). Then the prime $p = 3$ divides both $m$ and $n$, so
        \(m \, P \, n\), and the prime $q = 5$ divides both $n$ and $o$, so \(n \, P \, o\), but there is no prime that divides
        both $m = 2 \cdot 3$ and $o = 5 \cdot 7$, so \(m \Ccancel{P} o\).
\end{proof}

\subsection{Exercise 18}
Define a relation $Q$ on $\R$ as follows: For all real numbers $x$ and $y$, \(x \, Q \, y \iff x - y\) is
rational.

\begin{proof}
        {\bf \(\bm{Q}\) is reflexive:} For all reals \(x \in \R, x-x = 0\) and 0 is rational, so \(x \, Q \, x\).

                {\bf \(\bm{Q}\) is symmetric:} Assume \(x \, Q \, y\). Then \(x - y\) is rational. Then \(y-x = -(x-y)\) is rational
        (being the negative of a rational). So \(y \, Q \, x\).

                {\bf \(\bm{Q}\) is transitive:} Assume \(x \, Q \, y\) and \(y \, Q \, z\). Then \(x - y\) and \(y - z\) are rational.
        So \(x - z = (x-y) + (y-z)\) is also rational (being the sum of two rationals). Thus \(x \, Q \, z\).
\end{proof}

\subsection{Exercise 19}
Define a relation $I$ on $\R$ as follows: For all real numbers $x$ and $y$, \(x \, I \, y \iff x - y\) is
irrational.

\begin{proof}
        {\bf \(\bm{I}\) is not reflexive:} For all reals \(x \in \R, x - x = 0\) and 0 is not irrational,
        so \(x \Ccancel{I} x\).

                {\bf \(\bm{I}\) is symmetric:} Assume \(x \, I \, y\). Then \(x - y\) is irrational. So \(y-x = -(x-y)\) is irrational
        (being the negative of an irrational). So \(y \, I \, x\).

                {\bf \(\bm{I}\) is not transitive:} Let \(x = \sqrt{2}, y = 0, z = \sqrt{2}\). Then \(x \, I \, y\) because \(x - y =
        \sqrt{2}\) is irrational. Also \(y \, I \, z\) because \(y - z = -\sqrt{2}\) is irrational. But \(x - z = 0\) is not
        irrational, thus \(x \Ccancel{I} z\).
\end{proof}

\subsection{Exercise 20}
Let \(X = \{a, b, c\}\) and \(\ps(X)\) be the power set of $X$ (the set of all subsets of $X$). A relation {\bf E} is
defined on \(\ps(X)\) as follows: For every \(A, B \in \ps(X), A \,\textbf{E}\, B \iff \) the number of elements
in $A$ equals the number of elements in $B$.

\begin{proof}
        {\bf E is reflexive:} For every \(A \in \ps(X)\), the number of elements in $A$ equals the number of elements in
        $A$. So \(A \,\textbf{E}\, A\).

                {\bf E is symmetric:} Assume \(A \,\textbf{E}\, B\). Then the number of elements in $A$ equals the number of elements
        in $B$. So, the number of elements in $B$ equals the number of elements in $A$. So \(B \,\textbf{E}\, A\).

                {\bf E is transitive:} Assume \(A \,\textbf{E}\, B\) and \(B \,\textbf{E}\, C\). Then the number of elements in $A$
        equals the number of elements in $B$, and the number of elements in $B$ equals the number of elements in $C$. So
        the number of elements in $A$ equals the number of elements
        in $C$. So \(A \,\textbf{E}\, C\).
\end{proof}

\subsection{Exercise 21}
Let \(X = \{a, b, c\}\) and \(\ps(X)\) be the power set of $X$. A relation {\bf L} is defined on \(\ps(X)\) as
follows: For every \(A, B \in \ps(X), A \,\textbf{L}\, B \iff \) the number of elements in $A$ is less than the
number of elements in $B$.

\begin{proof}
        {\bf L is not reflexive:} For all \(A \in \ps(X)\), the number of elements in $A$ is not less than the number of
        elements in $A$. So \(A \,\Ccancel{\textbf{L}}\, A\).

                {\bf L is not symmetric:} Let \(A = \es, B = \{a\}\). Then the number of elements in $A$ (which is 0) is less than the
        number of elements in $B$ (which is 1). So \(A \,\textbf{L}\, B\). But the number of elements in $B$ (which is 1) is
        not less than the number of elements in $A$ (which is 0). So \(B\,\Ccancel{\textbf{L}}\, A\).

                {\bf L is transitive:} Assume \(A \, \textbf{L} \, B\) and \(B \, \textbf{L} \, C\). Then the number of elements in
        $A$ is less than the number of elements in $B$, and the number of elements in $B$ is less than the number of
        elements in $C$. Then the number of elements in $A$ is less than the number of elements in $C$. So \(A \, \textbf{L} \, C\).
\end{proof}

\subsection{Exercise 22}
Let \(X = \{a, b, c\}\) and \(\ps(X)\) be the power set of $X$. A relation {\bf N} is defined on \(\ps(X)\) as
follows: For every \(A, B \in \ps(X), A \textbf{N} B \iff \) the number of elements in $A$ is not equal to the number
of elements in $B$.

\begin{proof}
        {\bf N is not reflexive:} Let \(A = \{a\}\) which has 1 element. Then the number of elements in $A$ is equal to the number of elements in $A$. So \(A \,\Ccancel{\textbf{N}}\, A\).

                {\bf N is symmetric:} Assume \(A \,\textbf{N}\, B\). Then the number of elements in $A$ is not equal to the number
        of elements in $B$. So the number of elements in $B$ is not equal to the number of elements in $A$, and
        \(B \,\textbf{N}\, A\).

                {\bf N is not transitive:} Let \(A = \{a\}, B = \es, C = \{c\}\). Then \(A \,\textbf{N}\, B\) because $A$ has 1
        element and $B$ has 0 elements, and $0 \neq 1$. Similarly \(B \,\textbf{N}\, C\). But \(A \,\Ccancel{\textbf{N}}\, C\)
        because both $A$ and $C$ have 1 element, and 1 = 1.
\end{proof}

\subsection{Exercise 23}
Let $X$ be a nonempty set and \(\ps(X)\) the power set of $X$. Define the “subset” relation {\bf S} on \(\ps(X)\) as
follows: For every \(A, B \in \ps(X), A \,\textbf{S}\, B \iff A \subseteq B\).

\begin{proof}
        {\bf S is reflexive:} For all \(A \in \ps(X), A \subseteq A\) therefore \(A \,\textbf{S}\, A\).

                {\bf S is not symmetric:} Let \(A = \{a\}, B = \{a, b\}\). Then \(A \subseteq B\), so \(A \,\textbf{S}\, B\). But
        \(B \nsubseteq A\) therefore \(B \,\Ccancel{\textbf{S}}\, A\).

                {\bf S is transitive:} Assume \(A \,\textbf{S}\, B\) and \(B \,\textbf{S}\, C\). So \(A \subseteq B\) and
        \(B \subseteq C\). Then by transitivity of subsets, \(A \subseteq C\), and \(A \,\textbf{S}\, C\).
\end{proof}

\subsection{Exercise 24}
Let \(X\) be a nonempty set and \(\ps(X)\) the power set of $X$. Define the “not equal to” relation {\bf U} on
\(\ps(X)\) as follows: For every \(A, B \in \ps(X), A \textbf{U} B \iff A \neq B\).

\begin{proof}
        {\bf U is not reflexive:} For every \(A \in \ps(X), A = A\) therefore \(A \, \Ccancel{\textbf{U}} \, A\).

                {\bf U is symmetric:} Assume \(A \, \textbf{U} \, B\). Then \(A \neq B\). So \(B \neq A\), and \(B \,\textbf{U} \, A\).

                {\bf U is not transitive:} Let \(X = \{x\}, A = \{x\}, B = \es, C = \{x\}\). Then \(A \, \textbf{U} \, B\) because
        \(A \neq B\), and \(B\,\textbf{U}\,C\) because \(B \neq C\), but \(A = C\) so \(A \, \Ccancel{\textbf{U}} \, C\).
\end{proof}

\subsection{Exercise 25}
Let $A$ be the set of all strings of $a$’s and $b$’s of length 4. Define a relation \(R\) on \(A\) as follows: For
every \(s, t \in A, s \, R \, t \iff s\) has the same first two characters as \(t\).

\begin{proof}
        {\bf \(\bm{R}\) is reflexive:} For every string \(s \in A\), $s$ has the same first two characters as $s$. Thus
        \(s \, R \, s\).

                {\bf \(\bm{R}\) is symmetric:} Assume \(s \, R \, t\). Then $s$ has the same first two characters as $t$. Then $t$ has
        the same first two characters as $s$, so \(t \, R \, s\).

                {\bf \(\bm{R}\) is transitive:} Assume \(s \, R \, t\) and \(t \, R \, r\). Then $s$ has the same first two characters
        as $t$, and $t$ has the same first two characters as $r$. So $s$ has the same first two characters as $r$, and
        \(s \, R \, r\).
\end{proof}

\subsection{Exercise 26}
Let \(A\) be the set of all strings of 0’s, 1’s, and 2’s that have length 4 and for which the sum of the characters
in the string is less than or equal to 2. Define a relation \(R\) on \(A\) as follows: For every \(s, t \in A, s \, R
\, t \iff\) the sum of the characters of \(s\) equals the sum of the characters of \(t\).

\begin{proof}
        {\bf \(\bm{R}\) is reflexive:} For every \(s, \in A\), the sum of the characters of \(s\) equals the sum of the
        characters of \(s\). So \(s \, R \, s\).

                {\bf \(\bm{R}\) is symmetric:} Assume \(s \, R \, t\). Then the sum of the characters of \(s\) equals the sum of the
        characters of \(t\). So the sum of the characters of \(t\) equals the sum of the characters of \(s\), and \(s\,R\,t\).

                {\bf \(\bm{R}\) is transitive:} Assume \(s \, R \, t\) and \(t \, R \, r\). Then the sum of the characters of \(s\)
        equals the sum of the characters of \(t\), and the sum of the characters of \(t\) equals the sum of the characters of
        \(r\). So the sum of the characters of \(s\) equals the sum of the characters of \(r\), and \(s\,R\,r\).
\end{proof}

\subsection{Exercise 27}
Let \(A\) be the set of all English statements. A relation {\bf I} is defined on \(A\) as follows: For every \(p, q
\in A, p \,\textbf{I}\, q \iff p \implies q\) is true.

\begin{proof}
        {\bf I is reflexive:} For every \(p \in A, p \implies p\) is true, therefore \(p \, \textbf{I} \, p\).

                {\bf I is not symmetric:} Let $p$ be ``$1 is greater than 2$'' and let $q$ be ``2 is greater than 1''. So $p$ is
        false and $q$ is true. Therefore \(p \implies q\) is true and \(q \implies p\) is false. So \(p \, \textbf{I} \, q\)
        but \(q \, \Ccancel{\textbf{I}} \, p\).

                {\bf I is transitive:} Assume \(p \, \textbf{I} \, q\) and \(q \, \textbf{I} \, r\). So \(p \implies q\) is true and
        \(q \implies r\) is true. By transitivity of implication, \(p \implies r\) is true, and \(p \, \textbf{I} \, r\).
\end{proof}

\subsection{Exercise 28}
Let \(A = \R \times \R\). A relation {\bf F} is defined on \(A\) as follows: For every \((x_1, y_1)\) and \((x_2, y_2)
\) in $A$, \((x_1, y_1) \textbf{F} (x_2, y_2) \iff x_1 = x_2\).

\begin{proof}
        {\bf F is reflexive:} For every \((x, y) \in A, x = x\), therefore \((x, y) \textbf{F} (x, y)\).

                {\bf F is symmetric:} Assume \((x_1, y_1) \textbf{F} (x_2, y_2)\). Then \(x_1 = x_2\). Then \(x_2 = x_1\). So
        \((x_2, y_2) \textbf{F} (x_1, y_1)\).

                {\bf F is transitive:} Assume \((x_1, y_1) \textbf{F} (x_2, y_2)\) and \((x_2, y_2) \textbf{F} (x_3, y_3)\). Then
        \(x_1 = x_2\) and \(x_2 = x_3\). Thus \(x_1 = x_3\) and so \((x_1, y_1) \textbf{F} (x_3, y_3)\).
\end{proof}

\subsection{Exercise 29}
Let \(A = \R \times \R\). A relation {\bf S} is defined on \(A\) as follows: For every \((x_1, y_1)\) and \((x_2, y_2)
\) in \(A, (x_1, y_1) \textbf{S} (x_2, y_2) \iff y_1 = y_2\).

\begin{proof}
        {\bf S is reflexive:} \((x, y) \textbf{S} (x, y)\) because \(y = y\).

                {\bf S is symmetric:} Assume \((x_1, y_1) \textbf{S} (x_2, y_2)\). Then \(y_1 = y_2\). So \(y_2 = y_1\).

        So \((x_2, y_2) \textbf{S} (x_1, y_1)\).

                {\bf S is transitive:} Assume \((x_1, y_1) \textbf{S} (x_2, y_2)\) and \((x_2, y_2) \textbf{S} (x_3, y_3)\). Then
        \(y_1 = y_2\) and \(y_2 = y_3\). So \(y_1 = y_3\). So \((x_1, y_1) \textbf{S} (x_3, y_3)\).
\end{proof}

\subsection{Exercise 30}
Let \(A\) be the “punctured plane”; that is, \(A\) is the set of all points in the Cartesian plane except the origin
(0, 0). A relation \(R\) is defined on \(A\) as follows: For every \(p_1\) and \(p_2\) in \(A, p_1 \, R \, p_2 \iff
p_1\) and \(p_2\) lie on the same half line emanating from the origin.

\begin{proof}
        {\bf \(\bm{R}\) is reflexive:} For all \(p \in A, p\) and \(p\) lie on the same half line emanating from the origin.
        So \(p \, R \, p\).

                {\bf \(\bm{R}\) is symmetric:} Assume \(p_1 \, R \, p_2\). Then \(p_1\) and \(p_2\) lie on the same half line
        emanating from the origin. Then \(p_2\) and \(p_1\) lie on the same half line emanating from the origin. So
        \(p_2 \, R \, p_1\).

                {\bf \(\bm{R}\) is transitive:} First notice that for any \(p \in A\) there is exactly one half line emanating from
        the origin on which $p$ lies.

        Assume \(p_1 \, R \, p_2\) and \(p_2 \, R \, p_3\). Then \(p_1\) and \(p_2\) lie on the same half line emanating from
        the origin, say $l_1$. And \(p_2\) and \(p_3\) lie on the same half line emanating from the origin, say $l_2$. Since
        $p_2$ lies on both $l_1$ and $l_2$, by the previous paragraph \(l_1 = l_2\). Then \(p_1\) and \(p_3\) lie on
        the same half line emanating from the origin. So \(p_1 \, R \, p_3\).
\end{proof}

\subsection{Exercise 31}
Let \(A\) be the set of people living in the world today. A relation \(R\) is defined on \(A\) as follows: For all
people \(p\) and \(q\) in \(A, p \, R \, q \iff p\) lives within 100 miles of \(q\).

\begin{proof}
        {\bf \(\bm{R}\) is reflexive:} For every person $p$, $p$ lives within 0 miles of $p$, so in particular $p$ lives
        within 100 miles of $p$. Therefore \(p \, R \, p\).

                {\bf \(\bm{R}\) is symmetric:} Assume \(p \, R \, q\). So $p$ lives within 100 miles of $q$. Then $q$ lives within
        100 miles of $p$. Thus \(q \, R \, p\).

                {\bf \(\bm{R}\) is not transitive:} As a counterexample, take $p$ to be an inhabitant of Chicago, Illinois, $q$ an
        inhabitant of Kankakee, Illinois, and $r$ an inhabitant of Champaign, Illinois. Then \(p \,R\, q\) because Chicago is
        less than 100 miles from Kankakee, and \(q \,R\, r\) because Kankakee is less than 100 miles from Champaign, but
        \(p \, \Ccancel{R}\, r\) because Chicago is not less than 100 miles from Champaign.
\end{proof}

\subsection{Exercise 32}
Let \(A\) be the set of all lines in the plane. A relation \(R\) is defined on \(A\) as follows: For every \(l_1\) and
\(l_2\) in \(A, l_1 \, R \, l_2 \iff l_1\) is parallel to \(l_2\). (Assume that a line is parallel to itself.)

\begin{proof}
        {\bf \(\bm{R}\) is reflexive:} For every line \(l \in A, l\) is parallel to itself, therefore \(l \, R \, l\).

                {\bf \(\bm{R}\) is symmetric:} Assume \(l_1 \, R \, l_2\). Then \(l_1\) is parallel to \(l_2\). Then \(l_2\) is
        parallel to \(l_1\), so \(l_2 \, R \, l_1\).

                {\bf \(\bm{R}\) is transitive:} Assume \(l_1 \, R \, l_2\) and \(l_2 \, R \, l_3\). Then \(l_1\) is parallel to $l_2$
        and \(l_2\) is parallel to $l_3$. By transitivity of parallelism \(l_1\) is parallel to \(l_3\) so \(l_1 \, R \, l_3\).
\end{proof}

\subsection{Exercise 33}
Let \(A\) be the set of all lines in the plane. A relation \(R\) is defined on \(A\) as follows: For every \(l_1\) and
\(l_2\) in \(A, l_1 \, R \, l_2 \iff l_1\) is perpendicular to \(l_2\).

\begin{proof}
        {\bf \(\bm{R}\) is not reflexive:} For evert line $l$ in $A$, $l$ is not perpendicular to itself ($l$ is parallel to
        itself). Therefore \(l \,\Ccancel{R}\, l\).

                {\bf \(\bm{R}\) is symmetric:} Assume \(l_1 \, R \, l_2\). Then \(l_1\) is perpendicular to \(l_2\). Then \(l_2\) is
        perpendicular to \(l_1\). So \(l_2 \, R \, l_1\).

                {\bf \(\bm{R}\) is not transitive:} Let $l_1$ be the line $y = 0$, let $l_2$ be the line $x = 0$ and $l_3$ be the
        line $y = 1$. Then $l_2$ is perpendicular to both $l_1$ and $l_3$ so \(l_1 \, R \, l_2\) and \(l_2 \, R \, l_3\). But
        $l_1$ is parallel to $l_3$ so \(l_1 \, \Ccancel{R} \,l_3\).
\end{proof}

{\bf \cy In $34-36$, assume that $R$ is a relation on a set $A$. Prove or disprove each statement.}

\subsection{Exercise 34}
If \(R\) is reflexive, then \(R^{-1}\) is reflexive.

\begin{proof}
        Suppose $R$ is any reflexive relation on a set $A$. {\it [We must show that \(R^{-1}\) is reflexive. To show this,
                                we must show that for every \(x\) in \(A, x \, R^{-1} x\).]} Given any element \(x\) in \(A\), since \(R\) is
        reflexive, \(x \,R\, x\), and by definition of relation, this means that \((x, x) \in R\). It follows, by definition
        of the inverse of a relation, that \((x, x) \in R^{-1}\), and so, by definition of relation, \(x \,R^{-1}\, x\) {\it
                        [as was to be shown].}
\end{proof}

\subsection{Exercise 35}
If \(R\) is symmetric, then \(R^{-1}\) is symmetric.

\begin{proof}
        Assume \(R\) is symmetric. {\it [We want to show \(R^{-1}\) is symmetric.]} Assume \(x \, R^{-1} \, y\). We need to
        show \(y \, R^{-1} \, x\). By definition of \(R^{-1}, y \, R \, x\). Since \(R\) is symmetric, \(x \, R \, y\). By
        definition of \(R^{-1}\) again, \(y \, R^{-1} \, x\).
\end{proof}

\subsection{Exercise 36}
If \(R\) is transitive, then \(R^{-1}\) is transitive.

\begin{proof}
        Assume \(R\) is transitive. {\it [We want to show \(R^{-1}\) is transitive.]} Assume \(x \, R^{-1} \, y\) and \(y \,
        R^{-1} \, z\). We need to show \(x \, R^{-1} \, z\). By definition of \(R^{-1}, y \, R \, x\) and \(z \, R \, y\).
        Since \(R\) is transitive, \(z \, R \, x\). By definition of \(R^{-1}\) again, \(x \, R^{-1} \, z\).
\end{proof}

{\bf \cy In $37-42$, assume that \(R\) and \(S\) are relations on a set \(A\). Prove or disprove each
statement.}

\subsection{Exercise 37}
If \(R\) and \(S\) are reflexive, is \(R \cap S\) reflexive? Why?

\begin{proof}
        Yes. Suppose $R$ and $S$ are reflexive. {\it [To show that \(R \cap S\) is reflexive, we must show that \(\forall x
                                \in A, (x, x) \in R \cap S\).]} So suppose \(x \in A\). Since $R$ is reflexive, \((x, x) \in R\), and since $S$ is
        reflexive, \((x, x) \in S\). Thus, by definition of intersection, \((x, x) \in R \cap S\) {\it [as was to be shown].}
\end{proof}

\subsection{Exercise 38}
If \(R\) and \(S\) are symmetric, is \(R \cap S\) symmetric? Why?

\begin{proof}
        Yes. Suppose $R$ and $S$ are symmetric. {\it [To show that \(R \cap S\) is symmetric, we must show that \(\forall x,y
                                \in A\), if \((x, y) \in R \cap S\) then \((y, x) \in R \cap S\).]} So suppose \(x,y \in A\) and \((x, y) \in R
        \cap S\). By definition of intersection \((x, y) \in R\) and \((x,y)\in S\). Since $R$ is symmetric, \((y,x)\in R\),
        and since $S$ is symmetric, \((y, x) \in S\). Thus, by definition of intersection, \((y, x) \in R \cap S\)
        {\it [as was to be shown].}
\end{proof}

\subsection{Exercise 39}
If \(R\) and \(S\) are transitive, is \(R \cap S\) transitive? Why?

\begin{proof}
        Yes. Suppose $R$ and $S$ are transitive. {\it [To show that \(R \cap S\) is transitive, we must show that \(\forall x,y
                                ,z \in A\), if \((x, y) \in R \cap S\) and \((y, z) \in R \cap S\) then \((x, z) \in R \cap S\).]} So suppose \(x,y,z
        \in A\) and \((x, y) \in R \cap S\) and \((y, z) \in R \cap S\). By definition of intersection \((x, y) \in R\) and
        \((x,y)\in S\) and \((y, z) \in R\) and \((y,z)\in S\). Since $R$ is transitive, \((x,z)\in R\), and since $S$ is
        transitive, \((x, z) \in S\). Thus, by definition of intersection, \((x, z) \in R \cap S\) {\it [as was to be shown].}
\end{proof}

\subsection{Exercise 40}
If \(R\) and \(S\) are reflexive, is \(R \cup S\) reflexive? Why?

\begin{proof}
        Yes. To prove this we must show that for all $x$ in $A$, \((x, x) \in R \cup S\). So suppose \(x\) is a particular
        but arbitrarily chosen element in \(A\). {\it [We must show that \((x, x) \in R \cup S\).]} Then \((x, x) \in R\)
        because $R$ is reflexive, and hence \((x, x) \in R \cup S\) by definition of union, {\it [as was to be shown]}.
\end{proof}

\subsection{Exercise 41}
If \(R\) and \(S\) are symmetric, is \(R \cup S\) symmetric? Why?

\begin{proof}
        Yes. To prove this we must show that for all $x$ and $y$ in $A$, if \((x, y) \in R \cup S\) then \((y, x) \in R \cup
        S\). So suppose \((x, y)\) is a particular but arbitrarily chosen element in \(R \cup S\). {\it [We must show that
                                \((y, x) \in R \cup S\).]} By definition of union, \((x, y) \in R\) or \((x, y) \in S\). In case \((x, y) \in R\), then
        \((y, x) \in R\) because $R$ is symmetric, and hence \((y, x) \in R \cup S\) by definition of union. In case \((x, y)
        \in S\) then \((y, x) \in S\) because $S$ is symmetric, and hence \((y, x) \in R \cup S\) by definition of union. Thus,
        in both cases, \((y, x) \in R \cup S\) {\it [as was to be shown]}.
\end{proof}

\subsection{Exercise 42}
If \(R\) and \(S\) are transitive, is \(R \cup S\) transitive? Why?

\begin{proof}
        No. Let \(A = \{a,b,c,d\}, R = \{(a,b),(b,c),(a,c)\}, S = \{(c,a),(a,d),(c,d)\}\). Then $R$ and $S$ are transitive but
        \(R \cup S\) is not: \((a,c) \in R \cup S\) and \((c,a) \in R \cup S\) but \((a,a) \notin R \cup S\).
\end{proof}

{\bf \cy In \(43-50\), the following definitions are used: a relation on a set \(A\) is defined to be irreflexive if,
and only if, for every \(x \in A, x \, \Ccancel[cyan]{R} \, x\); asymmetric if, and only if, for every \(x, y \in A\)
if \(x \, R \, y\) then \(y \Ccancel[cyan]{R} x\); intransitive if, and only if, for every \(x, y, z \in A\),
if \(x R y\) and \(y R z\) then \(x \Ccancel[cyan]{R} z\). For each of the relations in the referenced exercise,
determine whether the relation is irreflexive, asymmetric, intransitive, or none of these.}

\subsection{Exercise 43}
Exercise 1
\begin{proof}
        \(R_1\) is not irreflexive because \((0, 0) \in R_1\). \(R_1\) is not asymmetric because \((0, 1) \in R_1\) and
        \((1, 0) \in R_1\). \(R_1\) is not intransitive because \((0, 1) \in R_1\) and \((1, 0) \in R_1\) and \((0, 0) \in R_1\).
\end{proof}

\subsection{Exercise 44}
Exercise 2
\begin{proof}
        Recall \(R_2 = \{(0, 0), (0, 1), (1, 1), (1, 2), (2, 2), (2, 3)\}\).

        \(R_2\) is not irreflexive because \((0, 0) \in R_2\).

        \(R_2\) is not asymmetric because \((0, 0) \in R_2\) and \((0, 0) \in R_2\).

        \(R_2\) is not intransitive because \((0, 0) \in R_2\) and \((0,1) \in R_2\) and \((0,1) \in R_2\).
\end{proof}

\subsection{Exercise 45}
Exercise 3
\begin{proof}
        $R_3$ is irreflexive because no element of $A$ is related by $R_3$ to itself. $R_3$ is not asymmetric because
        \((2, 3) \in R_3\) and \((3, 2) \in R_3\). $R_3$ is intransitive. To see why, observe that $R_3$ consists only
        of \((2, 3)\) and \((3, 2)\). Now \((2, 3) \in R_3\) and \((3, 2) \in R_3\) but \((2, 2) \notin R_3\). Also
        \((3, 2) \in R_3\) and \((2, 3) \in R_3\) but \((3, 3) \notin R_3\).
\end{proof}

\subsection{Exercise 46}
Exercise 4
\begin{proof}
        Recall \(R_4 = \{(1, 2), (2, 1), (1, 3), (3, 1)\}\).

        \(R_4\) is irreflexive.

        \(R_4\) is not asymmetric because \((1, 2) \in R_4\) and \((2, 1) \in R_4\).

        \(R_4\) is intransitive.
\end{proof}

\subsection{Exercise 47}
Exercise 5
\begin{proof}
        Recall \(R_5 = \{(0, 0), (0, 1), (0, 2), (1, 2)\}\).

        \(R_5\) is not irreflexive because \((0,0) \in R_5\).

        \(R_5\) is not asymmetric because \((0, 0) \in R_5\) and \((0, 0) \in R_5\).

        \(R_5\) is not intransitive because \((0,1) \in R_5\) and \((1,2) \in R_5\) and \((0,2) \in R_5\).
\end{proof}

\subsection{Exercise 48}
Exercise 6
\begin{proof}
        Recall \(R_6 = \{(0, 1), (0, 2)\}\). \(R_6\) is irreflexive because no element of \(A\) is related by \(R_6\) to
        itself. \(R_6\) is asymmetric because \(R_6\) consists only of \((0, 1)\) and \((0, 2)\) and neither \((1, 0)\) nor
        \((2, 0)\) is in \(R_6\). \(R_6\) is intransitive.
\end{proof}

\subsection{Exercise 49}
Exercise 7
\begin{proof}
        Recall \(R_7 = \{(0, 3), (2, 3)\}\).

        \(R_7\) is irreflexive, asymmetric and intransitive.
\end{proof}

\subsection{Exercise 50}
Exercise 8
\begin{proof}
        Recall \(R_8 = \{(0, 0), (1, 1)\}\). \(R_8\) is not irreflexive because \((0,0) \in R_8\). \(R_8\) is not
        asymmetric because \((0,0) \in R_8\) and \((0,0) \in R_8\).
        \(R_8\) is intransitive.
\end{proof}

{\bf \cy In \(51-53, R, S\), and \(T\) are relations defined on \(A = \{0, 1, 2, 3\}\).}

\subsection{Exercise 51}
Let \(R = \{(0, 1), (0, 2), (1, 1), (1, 3), (2, 2), (3, 0)\}\). Find \(R^t\), the transitive closure of \(R\).

\begin{proof}
        \(R^t = R \cup \{(0, 0), (0, 3), (1, 0), (3, 1), (3, 2), (3, 3), (0, 2), (1, 2)\} = \)

        \(\{(0, 0), (0, 1), (0, 2), (0, 3), (1, 0), (1, 1), (1, 2), (1, 3), (2, 2), (3, 0), (3, 1), (3, 2), (3, 3)\}\).
\end{proof}

\subsection{Exercise 52}
Let \(S = \{(0, 0), (0, 3), (1, 0), (1, 2), (2, 0), (3, 2)\}\). Find \(S^t\), the transitive closure of \(S\).

\begin{proof}
        \(S^t = S \cup \{(0,2),(1,3),(2,2),(2,3),(3,3)\}\)

        \( = \{(0, 0), (0,2),(0, 3), (1, 0), (1, 2), (1,3),(2, 0),(2,2), (2,3),(3, 2),(3,3)\}\)
\end{proof}

\subsection{Exercise 53}
Let \(T = \{(0, 2), (1, 0), (2, 3), (3, 1)\}\). Find \(T^t\), the transitive closure of \(T\).

\begin{proof}
        \(T^t = T \cup \{(0,3),(0,1),(0,0),(1,2),(1,3),(1,1),(2,1),(2,0),(2,2),(3,0),(3,2),\)

        \((3,3)\} = \{(0,0),(0,1),(0, 2),(0,3), (1, 0),(1, 1),(1, 2),(1, 3), (2,0),(2,1),(2,2),(2, 3),\)

        \((3,0),(3,1),(3,2),(3, 1)\}\)
\end{proof}

\subsection{Exercise 54}
Write a computer algorithm to test whether a relation \(R\) defined on a finite set \(A\) is reflexive, where
\(A = \{a[1], a[2], \ldots, a[n]\}\).

\begin{tcolorbox}[colframe=cyan]
        {\bf \cy Algorithm: Test for Reflexivity}

        {\it [The input for this algorithm is a binary relation $R$ defined on a set $A$, that is represented as the one-
                        dimensional array \(a[1], a[2], \ldots, a[n].\) To test whether $R$ is reflexive, a variable called answer is
                        initially set equal to “yes,” and each element \(a[i]\) of $A$ is examined in turn to see whether it is related by $R$
                        to itself. If any element is not related to itself by $R$, then answer is set equal to “no,” the while loop is not
                        repeated, and processing terminates.]}
\end{tcolorbox}

\begin{tcolorbox}[colframe=cyan]
        {\bf Input:} $n$ {\it [a positive integer]}, \(a[1], a[2], \ldots, a[n]\) {\it [a one-dimensional array representing a set $A$]}, $R$ {\it [a subset of \(A \times A\)]}

        \begin{tabbing}
                {\bf Alg}\={\bf orithm Body:} \\
                \> \(i \coloneqq 1\), answer $\coloneqq$ ``yes''\\
                \> {\bf whi}\={\bf le} (answer = ``yes'' and \(i \leq n\)) \\
                \>          \> {\bf if} \((a[i], a[i]) \notin R\) {\bf then} answer \(\coloneqq\) ``no'' \\
                \>          \>  \(i \coloneqq i+1\) \\
                \> {\bf end while} \\
                {\bf Output:} answer {\it [a string]}
        \end{tabbing}
\end{tcolorbox}

\subsection{Exercise 55}
Write a computer algorithm to test whether a relation \(R\) defined on a finite set \(A\) is symmetric, where
\(A = \{a[1], a[2], \ldots, a[n]\}\).

\begin{tcolorbox}[colframe=cyan]
        {\bf \cy Algorithm: Test for Symmetry}

        {\bf Input:} $n$ {\it [a positive integer]}, \(a[1], a[2], \ldots, a[n]\) {\it [a one-dimensional array representing a set $A$]}, $R$ {\it [a subset of \(A \times A\)]}

        \begin{tabbing}
                {\bf Alg}\={\bf orithm Body:} \\
                \> \(i \coloneqq 1, j \coloneqq 1\), answer $\coloneqq$ ``yes''\\
                \> {\bf whi}\= {\bf le} (answer = ``yes'' and \(i \leq n\)) \\
                \>          \> {\bf whi}\={\bf le} (answer = ``yes'' and \(j \leq n\)) \\
                \>          \>           \> {\bf if} \((a[i], a[j]) \in R\) and \((a[j], a[i]) \notin R\) {\bf then}
                answer \(\coloneqq\) ``no'' \\
                \>          \>           \> \(j \coloneqq j+1\) \\
                \>          \> {\bf end while} \\
                \>          \>  \(i \coloneqq i+1\) \\
                \> {\bf end while} \\
                {\bf Output:} answer {\it [a string]}
        \end{tabbing}
\end{tcolorbox}

\subsection{Exercise 56}
Write a computer algorithm to test whether a relation \(R\) defined on a finite set \(A\) is transitive, where
\(A = \{a[1], a[2], \ldots, a[n]\}\).

\begin{tcolorbox}[colframe=cyan]
        {\bf \cy Algorithm: Test for Transitivity}

        {\bf Input:} $n$ {\it [a positive integer]}, \(a[1], a[2], \ldots, a[n]\) {\it [a one-dimensional array representing a set $A$]}, $R$ {\it [a subset of \(A \times A\)]}

                {\bf Algorithm Body:}

        \(i \coloneqq 1, j \coloneqq 1, k \coloneqq 1\), answer $\coloneqq$ ``yes''
\end{tcolorbox}

\begin{tcolorbox}[colframe=cyan]
        \begin{tabbing}
                {\bf whi}\= {\bf le} (answer = ``yes'' and \(i \leq n\)) \\
                \> {\bf whi}\={\bf le} (answer = ``yes'' and \(j \leq n\)) \\
                \>          \>{\bf whi}\={\bf le} (answer = ``yes'' and \(k \leq n\)) \\
                \>          \>         \> {\bf if} \((a[i], a[j]) \in R\) and \((a[j], a[k]) \in R\) and \((a[i], a[k]) \notin R\) \\
                \>          \>         \>  {\bf then} answer \(\coloneqq\) ``no'' \\
                \>           \>        \> \(k \coloneqq k+1\) \\
                \>           \> {\bf end while} \\
                \>           \> \(j \coloneqq j+1\) \\
                \> {\bf end while} \\
                \>  \(i \coloneqq i+1\) \\
                {\bf end while} \\
                {\bf Output:} answer {\it [a string]}
        \end{tabbing}
\end{tcolorbox}

\section{Exercise Set 8.3}

\subsection{Exercise 1}
Suppose that \(S = \{a, b, c, d, e\}\) and $R$ is a relation on $S$ such that \(a \,R\, b, b \,R\, c\), and
\(d \,R\, e\). List all of the following:
\[
        c\,R\,b, c\,R\,c, a\,R\,c, b\,R\,a, a\,R\,d, e\,R\,a, e\,R\,d, c\,R\,a
\]
that must be true if $R$ is:

\subsubsection{(a)}
reflexive (but not symmetric or transitive)
\begin{proof}
        \(c \,R\, c\)

\end{proof}

\subsubsection{(b)}
symmetric (but not reflexive or transitive)
\begin{proof}
        \(b \,R\, a, c \,R\, b, e \,R\, d\)
\end{proof}

\subsubsection{(c)}
transitive (but not reflexive or symmetric)
\begin{proof}
        \(a \,R\, c\)
\end{proof}

\subsubsection{(d)}
an equivalence relation.
\begin{proof}
        \(c \,R\, c, b \,R\, a, c \,R\, b, e \,R\, d, a \,R\, c, c \,R\, a\)
\end{proof}

\subsection{Exercise 2}
Each of the following partitions of \(\{0, 1, 2, 3, 4\}\) induces a relation \(R\) on \(\{0, 1, 2, 3, 4\}\). In each
case, find the ordered pairs in \(R\).

\subsubsection{(a)}
\(\{0, 2\}, \{1\}, \{3, 4\}\)
\begin{proof}
        \(R = \{(0, 0), (0, 2), (2, 0), (2, 2), (1, 1), (3, 3), (3, 4), (4, 3), (4, 4)\}\)
\end{proof}

\subsubsection{(b)}
\(\{0\}, \{1, 3, 4\}, \{2\}\)
\begin{proof}
        \(R = \{(0,0),(1,1),(3,3),(4,4),(1,3),(3,1),(1,4),(4,1),(3,4),(4,3),(2,2)\}\)
\end{proof}

\subsubsection{(c)}
\(\{0\}, \{1, 2, 3, 4\}\)

\begin{proof}
        \(R = \{(0,0),(1,1),(2,2),(3,3),(4,4),(1,2),(2,1),(1,3),(3,1),(1,4),(4,1),(2,3)\),

        \((3,2),(2,4),(4,2),(3,4),(4,3)\}\)
\end{proof}

{\bf \cy In each of $3-6$, the relation $R$ is an equivalence relation on $A$. As in example 8.3.5, first
find the specified equivalence classes. then state the number of distinct equivalence classes for $R$ and list them.}

\subsection{Exercise 3}
\(A = \{0, 1, 2, 3, 4\}, R = \{(0, 0), (0, 4), (1, 1), (1, 3), (2, 2), (3, 1), (3, 3), (4, 0), (4, 4)\}\),
equivalence classes: [0], [1], [2], [3]

\begin{proof}
        \([0] = \{0, 4\}, [1] = \{1, 3\}, [2] = \{2\}, [3] = \{1, 3\}\). So there are three distinct equivalence classes:
        \([0] = \{0, 4\} = [4], [1] = \{1, 3\} = [3], [2] = \{2\}\)
\end{proof}

\subsection{Exercise 4}
\(A = \{a, b, c, d\}, R = \{(a, a), (b, b), (b, d), (c, c), (d, b), (d, d)\}\), classes: \([a], [b], [c], [d]\)

\begin{proof}
        \([a] = \{a\}, [b] = \{b, d\}, [c] = \{c\}, [d] = \{b, d\}\). So there are four distinct equivalence classes:
        \([a] = \{a\}, [b] = \{b, d\} = [d], [c] = \{c\}\).
\end{proof}

\subsection{Exercise 5}
\(A = \{1, 2, 3, 4, \ldots, 20\}\). \(R\) is defined on \(A\) as follows: for all \(x, y \in A, x \,R\, y \iff 4
\mid (x - y)\). Equivalence classes: [1], [2], [3], [4], [5]

\begin{proof}
        \([1] = \{1, 5, 9, 13, 17\}, [2] = \{2, 6, 10, 14, 18\}, [3] = \{3, 7, 11, 15, 19\}, [4] = \{4, 8, 12,\)

        \(16, 20\}, [5] = \{5, 9, 13, 17, 1\} = [1]\), four distinct equivalence classes: [1], [2], [3], [4]
\end{proof}

\subsection{Exercise 6}
\(A = \{-4, -3, -2, -1, 0, 1, 2, 3, 4, 5\}\). \(R\) is defined on \(A\) as follows: for all \(x, y \in A, x \,R\,
y \iff 3 \mid (x - y)\). Equivalence classes: [0], [1], [2], [3]

\begin{proof}
        \([0] = \{-3, 0, 3\}, [1] = \{-2, 1, 4\}, [2] = \{-4, -1, 2, 5\}, [3] = \{-3, 0, 3\}\)

        There are 3 distinct equivalence classes: [0], [1], [2]
\end{proof}

{\bf \cy In each of $7-14$, the relation $R$ is an equivalence relation on the set $A$. Find the distinct
equivalence classes of $R$.}

\subsection{Exercise 7}
\(A = \{(1, 3), (2, 4), (-4, -8), (3, 9), (1, 5), (3, 6)\}\). \(R\) is defined on \(A\) as follows: For every
\((a, b), (c, d) \in A, (a, b) \, R \, (c, d) \iff ad = bc\).

\begin{proof}
        \(\{(1, 3), (3, 9)\}, \{(2, 4), (24, 28), (3, 6)\}, \{(1, 5)\}\)
\end{proof}

\subsection{Exercise 8}
\(X = \{a, b, c\}\) and \(A = \ps(X)\). \(R\) is defined on \(A\) as follows: For all sets \(u, v\) in \(\ps(X)\),
\(u\, R\, v \iff N(u) = N(v)\). (That is, the number of elements in $u$ equals the number of elements in $v$.)

\begin{proof}
        \(\{\es\}, \{\{a\}, \{b\}, \{c\}\}, \{\{a, b\}, \{a, c\}, \{b, c\}\}, \{\{a, b, c\}\}\)
\end{proof}

\subsection{Exercise 9}
\(X = \{-1, 0, 1\}\) and \(A = \ps(X)\). \(R\) is defined on \(\ps(X)\) as follows: For all sets $s$ and $t$ in
\(\ps(X), s \, R\, t \iff\) the sum of the elements in $s$
equals the sum of the elements in $t$.

\begin{proof}
        \(\{\es, \{0\}, \{-1, 1\}, \{-1, 0, 1\}\}, \{\{-1\}, \{-1, 0\}\}, \{\{1\}, \{0, 1\}\}\)
\end{proof}

\subsection{Exercise 10}
\(A = \{-5, -4, -3, -2, -1, 0, 1, 2, 3, 4, 5\}\). \(R\) is defined on \(A\) as follows: For all \(m, n \in \Z,
m \,R\, n \iff 3 \mid (m^2 - n^2)\).

\begin{proof}
        \(\{-5, -4, -2, -1, 1, 2, 4, 5\}, \{-3, 0, 3\}\)
\end{proof}

\subsection{Exercise 11}
\(A = \{-4, -3, -2, -1, 0, 1, 2, 3, 4\}\). \(R\) is defined
on \(A\) as follows: For every \((m, n) \in A, m \,R\, n
\iff 4 \mid (m^2 - n^2)\).

\begin{proof}
        \(\{-4, -2, 0, 2, 4\}, \{-3, -1, 1, 3\}\)
\end{proof}

\subsection{Exercise 12}
\(A = \{-4, -3, -2, -1, 0, 1, 2, 3, 4\}\). \(R\) is defined on \(A\) as follows: For all \((m, n) \in A, m \,R\, n \iff
5 \mid (m^2 - n^2)\).

\begin{proof}
        \(\{-4, -1, 1, 4\}, \{-3, -2, 2, 3\}, \{0\}\)
\end{proof}

\subsection{Exercise 13}
\(A\) is the set of all strings of length 4 in \(a\)’s and \(b\)’s. \(R\) is defined on \(A\) as follows: For all
strings \(s\) and \(t\) in \(A, s \,R\, t \iff s\) has the same first two characters as \(t\).

\begin{proof}
        \(\{aaaa, aaab, aaba, aabb\}\), \(\{abaa, abab, abba, abbb\}\), \(\{baaa, baab, baba, babb\}\),

        \(\{bbaa, bbab, bbba, bbbb\}\)
\end{proof}

\subsection{Exercise 14}
\(A\) is the set of all strings of 0’s, 1’s, and 2’s that have length 4 and for which the sum of the characters in
the string is less than or equal to 2. \(R\) is defined on \(A\) as follows: For every \(s, t \in A, s \,R\, t \iff\)
the sum of the characters of \(s\) equals the sum of the characters of \(t\).

\begin{proof}
        \(\{0000\}, \{0001, 0010, 0100, 1000\}, \{0011, 0101, 1001, 0110, 1010, 1100, 0002, 0020,\)

        \(0200, 2000\}\)
\end{proof}

\subsection{Exercise 15}
Determine which of the following congruence relations are true and which are false.

\subsubsection{(a)}
\(17 \equiv 2 \pmod 5\)
\begin{proof}
        True. \(17 - 2 = 15 = 5 \cdot 3\)
\end{proof}

\subsubsection{(b)}
\(4 \equiv -5 \pmod 7\)
\begin{proof}
        False. \(4 - (-5) = 9\) is not divisible by 7.
\end{proof}

\subsubsection{(c)}
\(-2 \equiv -8 \pmod 3\)
\begin{proof}
        True. \(-2 - (-8) = 6 = 3 \cdot 2\)
\end{proof}

\subsubsection{(d)}
\(-6 \equiv -2 \pmod 2\)
\begin{proof}
        True. \(-6 - (-2) = -4 = 2 \cdot (-2)\)
\end{proof}

\subsection{Exercise 16}
\subsubsection{(a)}
Let \(R\) be the relation of congruence modulo 3. Which of the following equivalence classes are equal?
\([7], [-4], [-6], [17], [4], [27], [19]\)

\begin{proof}
        \([7] = [4] = [19], [-4] = [17], [-6] = [27]\)
\end{proof}

\subsubsection{(b)}
Let \(R\) be the relation of congruence modulo 7. Which of the following equivalence classes are equal?
\([35], [3], [-7], [12], [0], [-2], [17]\)

\begin{proof}
        \([35] = [0] = [-7], [3] = [17], [12] = [-2]\)
\end{proof}

\subsection{Exercise 17}
\subsubsection{(a)}
Prove that for all integers $m$ and $n$, \(m \equiv n \pmod 3\) iff \(m \mod 3 = n \mod 3\).

\begin{proof}
        \(\bm{(\implies)}\) Suppose $m, n$ are integers such that \(m \equiv n \pmod 3\). {\it [We want to show that \(m \mod
                                3 = n \mod 3\)].} By definition of congruence, \(3 \mid (m-n)\), and so, by definition of divisibility, \(m - n = 3a\)
        for some integer $a$. Let \(r = m \mod 3\). Then \(m = 3b + r\) for some integer $b$. Since \(m - n = 3a\), it follows
        that \(m - n = (3b + r) - n = 3a\), or, equivalently, \(n = 3(b - a) + r\). Now \(b - a\) is an integer and \(0 \leq r
        < 3\). So, by definition of mod, \(n \mod 3 = r\), which equals \(m \mod 3\).

        \(\bm{(\impliedby)}\) Suppose $m, n$ are integers such that \(m \mod 3 = n \mod 3\). {\it [We want to show that \(m
                                \equiv n \pmod 3\)].} Let \(r = m \mod 3 = n \mod 3\). Then, by definition of mod, \(m = 3p + r\) and \(n = 3q +
        r\) for some integers $p$ and $q$. By substitution, \(m - n = (3p + r) - (3q + r) = 3(p - q)\). Since \(p - q\) is an
        integer, it follows that \(3 \mid (m - n)\), and so, by definition of congruence, \(m \equiv n \pmod 3\).
\end{proof}

\subsubsection{(b)}
Prove for all integers $d>0$ and $m,n$, \(m \equiv n \pmod d\) iff \(m \mod d = n \mod d\).

\begin{proof}
        Assume $m,n,d$ are integers with $d > 0$.

        \(\bm{(\implies)}\)
        1. Assume \(m \equiv n \pmod d\).

        2. By 1 and definition of congruence, \(d \mid (m-n)\).

        3. By 2 and definition of divisibility, \(m-n = da\) for some integer $a$.

        4. Let \(r = m \mod d\). Then by definition of mod, \(m = db + r\) for some integer $b$ and \(0 \leq r < d\).

        5. By 3 and 4, \(m-n = (db+r) - n = da\), so \(n = (db+r) - da = d(b-a)+r\) where $b-a$ is an integer and \(0 \leq r < d\).

        6. By 5 and definition of mod, \(r = n \mod d\). Therefore \(n \mod d = r = m \mod d\).

        \(\bm{(\impliedby)}\)
        1. Assume \(m \mod d = n \mod d\). Let \(r = m \mod d = n \mod d\).

        2. By 1 and definition of mod, \(m = da+r\) and \(n = db+r\) for some integers $a,b$.

        3. By 2, \(m-n = da+r - (db+r) = d(a-b)\) where $a-b$ is an integer. Thus \(d \mid (m-n)\).

        4. By 3 and definition of congruence, \(m \equiv n \pmod d\).
\end{proof}

\subsection{Exercise 18}

\subsubsection{(a)}
Give an example of two sets that are distinct but not disjoint.

\begin{proof}
        One possible answer: Let \(A = \{1, 2\}\) and \(B = \{2, 3\}\). Then \(A \neq B\), so \(A\) and \(B\) are distinct.
        But \(A\) and \(B\) are not disjoint since \(2 \in A \cap B\).
\end{proof}

\subsubsection{(b)}
Find sets \(A_1\) and \(A_2\) and elements \(x, y\), and \(z\) such that \(x\) and \(y\) are in \(A_1\) and \(y\) and
\(z\) are in \(A_2\) but \(x\) and \(z\) are not both in either of the sets \(A_1\) or \(A_2\).

\begin{proof}
        Let \(A_1 = \{x, y\}, A_2 = \{y, z\}\).
\end{proof}

{\bf \cy In $19-31$, (1) prove that the relation is an equivalence relation, and (2) describe the distinct
equivalence classes of each relation.}

\subsection{Exercise 19}
\(A\) is the set of all students at your college.

\subsubsection{(a)}
\(R\) is the relation defined on \(A\) as follows: For every \(x\) and \(y\) in \(A\), \(x \,R\, y \iff x\) has
the same major (or double major) as \(y\). (Assume “undeclared” is a major.)

\begin{proof}
        (1) \(R\) is reflexive because it is true that for each student \(x\) at a college, \(x\) has the same major (or
        double major) as \(x\).

        \(R\) is symmetric because it is true that for all students \(x\) and \(y\) at a college, if \(x\) has the same major
        (or double major) as \(y\), then \(y\) has the same major (or double major) as \(x\).

        \(R\) is transitive because it is true that for all students \(x, y\), and \(z\) at a college, if \(x\) has the
        same major (or double major) as \(y\) and \(y\) has the same major (or double major) as \(z\), then \(x\) has the
        same major (or double major) as \(z\).

        \(R\) is an equivalence relation because it is reflexive, symmetric, and transitive.

        (2) There is one equivalence class for each major and double major at the college. Each class consists of all
        students with that major (or double major).
\end{proof}

\subsubsection{(b)}
\(S\) is the relation defined on \(A\) as follows: For every \(x, y \in A, x \,S\, y \iff x\) is the same age as \(y\).

\begin{proof}
        (1) \(S\) is reflexive because for each student \(x\) at a college, \(x\) is the same age as \(x\).

        \(S\) is symmetric because it is true that for all students \(x\) and \(y\) at a college, if \(x\) is the same age as
        \(y\), then \(y\) is the same age as \(x\).

        \(S\) is transitive because it is true that for all students \(x, y\), and \(z\) at a college, if \(x\) is the
        same age as \(y\) and \(y\) is the same age as \(z\), then \(x\) is the same age as \(z\).

        \(S\) is an equivalence relation because it is reflexive, symmetric, and transitive.

        (2) There is one equivalence class for each age at the college. Each class consists of all students with that age.
\end{proof}

\subsection{Exercise 20}
\(E\) is the relation defined on \(\Z\) as follows: For every \(m, n \in \Z, m \,E\, n \iff 4 \mid (m - n)\).

\begin{proof}
        (1) The solution to exercise 12 in Section 8.2 proved that \(E\) is reflexive, symmetric, and transitive.
        Thus \(E\) is an equivalence relation.

        (2) Observe that for any integer \(a\), the equivalence class of \(a\) is

        \begin{tabular}{rcll}
                \([a]\) & = & \(\{x \in \Z \, | \, x \,E\, a\}\)                            & {\cy by definition of equivalence class} \\
                        & = & \(\{x \in \Z \,|\, x - a \text{ is divisible by } 4\}\)       & {\cy by definition of \(E\)}             \\
                        & = & \(\{x \in \Z \,|\, x - a = 4k \text{ for some integer } k\}\) & {\cy by definition of divisibility}      \\
                        & = & \(\{x \in \Z \,|\, x = 4k + a \text{ for some integer } k\}\) & {\cy by algebra.}
        \end{tabular}

        Now when any integer \(a\) is divided by 4, the only possible remainders are 0, 1, 2, and 3 and no integer has
        two distinct remainders when it is divided by 4. Thus every integer is contained in exactly one of the following four
        equivalence classes:

        \(\{x \in \Z \,|\, x = 4k \text{ for some integer } k\}\),
        \(\{x \in \Z \,|\, x = 4k + 1 \text{ for some integer } k\}\),

        \(\{x \in \Z \,|\, x = 4k + 2 \text{ for some integer } k\}\),
        \(\{x \in \Z \,|\, x = 4k + 3 \text{ for some integer } k\}\)
\end{proof}

\subsection{Exercise 21}
\(R\) is the relation defined on \(\Z\) as follows: For every \(m, n \in \Z, m \,R\, n \iff 7m - 5n\) is even.

\begin{proof}
        (1) \(R\) is reflexive because for all \(m \in \Z, 7m - 5m = 2m\) is even, therefore \(m \, R \, m\).

        \(R\) is symmetric: assume \(m \,R\, n\). Then \(7m - 5n\) is even. So \(7m-5n=2k\) for some integer $k$. Then
        \(7n-5m = (-5n+12n) + (7m-12m) = (12n-12m) + (7m-5n) = 2(6n-6m) + 2k = 2(6n-6m+k)\) where \(6n-6m+k\) is an
        integer. Thus \(7n-5m\) is even and \(n \,R\, m\).

        \(R\) is transitive: assume \(m \,R\, n\) and \(n \,R\, o\). Then \(7m-5n=2k\) and \(7n-5o=2l\) for some integers
        \(k, l\). So \(7m-5o = (7m-5n+5n) + (-7n+7n-5o) = (7m-5n) + (5n-7n) + (7n-5o) = 2k - 2n + 2l = 2(k-n+l)\), where
        \(k-n+l\) is an integer. So \(7m-5o\) is even and \(m \,R\, o\).

        (2) Assume \(m \,R\, n\). So \(7m-5n\) is even. By properties of even and odd integers, either \(7m\) and
        \(5n\) are both even, or they are both odd. Since 7 and 5 are odd, and odd times odd is odd, this means that either
        $m$ and $n$ are both even, or they are both odd. Thus $R$ has two equivalence classes: the set of all even integers
        and the set of all odd integers.
\end{proof}

\subsection{Exercise 22}
Let \(A\) be the set of all statement forms in three variables \(p, q\), and \(r\). {\bf R} is the relation
defined on \(A\) as follows: For all \(P\) and \(Q\) in \(A\), \(P \,{\bf R}\, Q \iff P\) and \(Q\) have the same
truth table.

\begin{proof}
        (1) {\bf R} is reflexive because for all \(P \in A\), \(P\) and \(P\) have the same truth table, so \(P\,{\bf R}\, P\).

                {\bf R} is symmetric: assume \(P \,{\bf R}\, Q\). Then \(P\) and \(Q\) have the same truth table. Then \(Q\) and
        \(P\) have the same truth table, so \(Q \,{\bf R}\, P\).

                {\bf R} is transitive: assume \(P \,{\bf R}\, Q\) and \(Q \,{\bf R}\, S\). Then \(P\) and \(Q\), and \(Q\) and \(S\)
        have the same truth tables. So \(P\) and \(S\) have the same truth table, and \(P \,{\bf R}\, S\).

        (2) There is an equivalence class corresponding to every possible truth table in 3 variables $p,q,r$. There are 8
        lines in every truth table, and each line has 2 options (true or false), so there are \(2^8\) equivalence classes.
\end{proof}

\subsection{Exercise 23}
Let \(P\) be a set of parts shipped to a company from various suppliers. \(S\) is the relation defined on \(P\)
as follows: For every \(x, y \in P, x \,S\, y \iff x\) has the same part number and is shipped from the same supplier
as \(y\).

\begin{proof}
        (1) \(S\) is reflexive because for all \(x \in P\), \(x\) has the same part number and is shipped from the same
        supplier as \(x\), so \(x \, S \, x\).

        \(S\) is symmetric: assume \(x \,S\, y\). Then \(x\) has the same part number and is shipped from the same supplier
        as \(y\). So \(y\) has the same part number and is shipped from the same supplier as \(x\). Thus \(y \,S\, x\).

        \(S\) is transitive: assume \(x \,S\, y\) and \(y \,S\, z\). So \(x\) has the same part number and is shipped from
        the same supplier as \(y\) and \(y\) has the same part number and is shipped from the same supplier as \(z\), so
        \(x\) has the same part number and is shipped from the same supplier as \(z\). Therefore \(x \,S\, z\).

        (2) For each distinct part number shipped from each distinct supplier, there is a distinct equivalence class
        corresponding to that part number.
\end{proof}

\subsection{Exercise 24}
Let \(A\) be the set of identifiers in a computer program. It is common for identifiers to be used for only a short
part of the execution time of a program and not to be used again to execute other parts of the program. In such cases,
arranging for identifiers to share memory locations makes efficient use of a computer’s memory capacity. Define a
relation \(R\) on \(A\) as follows: For all identifiers \(x\) and \(y\), \(x \,R\, y \iff\) the values of \(x\) and
\(y\) are stored in the same memory location during execution of the program.

\begin{proof}
        (1) \(R\) is reflexive because for all identifiers \(x\), the values of \(x\) and \(x\) are stored in the same memory
        location during execution of the program.

        \(R\) is symmetric: assume \(x \,R\, y\). Then the values of \(x\) and \(y\) are stored in the same memory
        location during execution of the program. So the values of \(y\) and \(x\) are stored in the same memory location during execution of the program. So \(y \,R\, x\).

        \(R\) is transitive: assume \(x \,R\, y\) and \(y \,R\, z\). So the values of \(x\) and \(y\) are stored in the
        same memory location during execution of the program, and the values of \(y\) and \(z\) are stored in the same memory
        location during execution of the program. Then the values of \(x\) and \(z\) are stored in the same memory location
        during execution of the program. So \(x \,R\, z\).

        (2) There is a distinct equivalence class corresponding to each distinct memory location during execution of the
        program.
\end{proof}

\subsection{Exercise 25}
\(A\) is the “absolute value” relation defined on \(\R\) as follows: For every \(x, y \in \R, x \,A\, y \iff |x| = |y|\).

\begin{proof}
        (1) \(A\) is reflexive because for all \(x \in \R, |x| = |x|\), so \(x \, A \, x\).

        \(A\) is symmetric: assume \(x \,A\, y\). Then \(|x| = |y|\). So \(|y| = |x|\), and \(y \,A\, x\).

        \(A\) is transitive: assume \(x \,A\, y\) and \(y \,A\, z\). Then \(|x| = |y|\) and \(|y| = |z|\), so \(|x| = |z|\)
        and therefore \(x \,A\, z\).

        (2) There is a distinct equivalence class for each nonnegative real number. Each class is a set of the form
        \(\{x, -x\}\) where $x$ is a nonnegative real number.
\end{proof}

\subsection{Exercise 26}
$D$ is the relation defined on \(\Z\) as follows: For every \(m, n \in \Z, m \,D\, n \iff 3 \mid (m^2 - n^2)\).

\begin{proof}
        (1) \(D\) is reflexive because for all \(m \in \Z, m^2 - m^2 = 0 = 3 \cdot 0\), so \(3 \mid (m^2-m^2)\) and thus
        \(m \,D\, m\).

        \(D\) is symmetric: assume \(m \,D\, n\). Then \(3 \mid (m^2 - n^2)\). Then \(m^2-n^2 = 3r\) for some integer $r$.
        Then \(n^2-m^2 = 3\cdot(-r)\) where $-r$ is an integer. So \(3 \mid (n^2-m^2)\) and thus \(n \,D\, m\).

        \(D\) is transitive: assume \(m \,D\, n\) and \(n \,D\, o\). So \(3 \mid (m^2-n^2)\) and \(3 \mid (n^2-o^2)\). So
        \(m^2-n^2 = 3r\) and \(n^2-o^2 = 3s\) for some integers $r,s$. Then \(m^2-o^2 = (m^2-n^2)+(n^2-o^2) = 3(r+s)\)
        where $r+s$ is an integer. Therefore \(3 \mid (m^2-n^2)\) and \(m \,D\, o\).

        (2) There are two distinct equivalence classes: \([0] = \{\ldots, -6, -3, 0, 3, 6, \ldots\}\) and
        \([1] = \{\ldots, -5, -4, -2, -1, 1, 2, 4, 5, \ldots\}\).
\end{proof}

\subsection{Exercise 27}
\(R\) is the relation defined on \(\Z\) as follows: For every \((m, n) \in \Z, m \,R\, n \iff 4 \mid (m^2 - n^2)\).

\begin{proof}
        (1) $R$ is an equivalence relation, the proofs are exactly the same as above in Exercise 26 (replace 3 with 4).

        (2) There are 2 distinct equivalence classes: [0] = the set of all even integers, [1] = the set of all odd integers.
        This is because, if we want \(4 \mid (m-n)(m+n)\), it is sufficient that both \(m-n\) and \(m+n\) are even; this is
        the case when either both $m,n$ are odd or both $m,n$ are even.
\end{proof}

\subsection{Exercise 28}
$I$ is the relation defined on $\R$ as follows: For every \(x, y \in \R, m \,I\, n \iff x - y\) is an integer.

\begin{proof}
        (1) $I$ is reflexive because the difference between each real number and itself is 0, which is an integer.

        $I$ is symmetric because for all real numbers $x$ and $y$, if \(x - y\) is an integer, then \(y - x = (-1)(x - y)\),
        which is also an integer.

        $I$ is transitive because for all real numbers \(x, y\), and \(z\), if \(x - y\) is an integer and \(y - z\) is an
        integer, then \(x - z = (x - y) + (y - z)\) is the sum of two integers and thus is an integer.

        $I$ is an equivalence relation because it is reflexive, symmetric, and transitive.

        (2) There is one class for each real number $x$ with \(0 \leq x < 1\). The distinct classes are all sets of the form
        \(\{y \in \R \,|\, y = n + x, \text{ for some integer } n\}\), where $x$ is a real number such that \(0 \leq x < 1\).
\end{proof}

\subsection{Exercise 29}
Define $P$ on the set \(\R \times \R\) of ordered pairs of real numbers as follows: For every \((w, x), (y, z) \in \R
\times \R, (w, x) \,P\, (y, z) \iff w = y\).

\begin{proof}
        (1) $P$ is reflexive because each ordered pair of real numbers has the same first element as itself.

        $P$ is symmetric for the following reason: Suppose \((w, x)\) and \((y, z)\) are ordered pairs of real numbers such
        that \((w, x) \,P\, (y, z)\). Then, by definition of $P$, \(w = y\). Now by the symmetric property of equality, this
        implies that \(y = w\), and so, by definition of $P$, \((y, z) \,P\, (w, x)\).

        $P$ is transitive for the following reason: Suppose \((u, v), (w, x)\), and \((y, z)\) are ordered pairs of real
        numbers such that \((u, v) \,P\, (w, x)\) and \((w, x) \,P\, (y, z)\). Then, by definition of $P$, \(u = w\) and
        \(w = y\). It follows from the transitive property of equality that \(u = y\). Hence, by definition of $P$, \((u, v) \,P\, (y, z)\).

        $P$ is an equivalence relation because it is reflexive, symmetric, and transitive.

        (2) There is one equivalence class for each real number. The distinct equivalence classes are all sets of ordered
        pairs \(\{(x, y) \in \R \times \R \,|\, x = a\}\), for each real number $a$. Equivalently, the equivalence classes
        consist of all vertical lines in the Cartesian plane.
\end{proof}

\subsection{Exercise 30}
Define $Q$ on the set \(\R \times \R\) as follows: For every \((w, x), (y, z) \in \R \times \R, (w, x) \,Q \,
(y, z) \iff x = z\).

\begin{proof}
        (1) $Q$ is reflexive because for all \((w, x) \in \R \times \R, x = x\) so \((w, x) \,Q\, (w, x)\).

        $Q$ is symmetric: assume \((w, x) \,Q\, (y,z)\). Then \(x = z\). So \(z = x\) and thus \((y,z) \,Q\, (w, x)\).

        $Q$ is transitive: assume \((w, x) \,Q\, (y,z)\) and \((y,z) \,Q\, (s,t)\). Then \(x = z\) and \(z = t\). So
        \(x = t\) and thus \((w,x) \,Q\, (s,t)\).

        (2) $Q$ has a distinct equivalence class, for each real number $a$, of the form \(\{(w,x) \in \R \times \R \,|\,
        x = a\}\).
\end{proof}

\subsection{Exercise 31}
Let $P$ be the set of all points in the Cartesian plane except the origin. $R$ is the relation defined on $P$ as
follows: For every \(p_1\) and \(p_2\) in \(P, p_1 \,R\, p_2 \iff p_1\) and \(p_2\) lie on the same half-line
emanating from the origin.

\begin{proof}
        (1) $R$ is reflexive: for every \(p \in P\), $p$ and $p$ lie on the same half-line emanating from the origin (namely
        the half-line that connects $p$ to the origin). Thus \(p \,R\, p\).

        $R$ is symmetric: assume \(p\,R\,q\). Then $p$ and $q$ lie on the same half-line $l$ emanating from the origin. Then
        $q$ and $p$ lie on the same half-line $l$ emanating from the origin. Thus \(q \,R\,p\).

        $R$ is transitive: assume \(p \,R\,q\) and \(q \,R\,r\). Then $p$ and $q$ lie on the same half-line $l_1$ emanating
        from the origin and $q$ and $r$ lie on the same half-line $l_2$ emanating from the origin. Then it must be that
        \(l_1 = l_2\) since $q$ lies on both half-lines. Thus $p$ and $r$ lie on the same half-line emanating from the
        origin, and \(p\,R\,r\).

        (2) Each equivalence class is a half-line $l$ emanating from the origin, containing all the points $p$ that lie on $l$.
\end{proof}

\subsection{Exercise 32}
Let $A$ be the set of all straight lines in the Cartesian plane. Define a relation \(\parallel\) on $A$ as follows:
For every $l_1$ and $l_2$ in $A$, \(l_1 \parallel l_2 \iff l_1\) is parallel to $l_2$. Then \(\parallel\) is an
equivalence relation on $A$. Describe the equivalence classes of this relation.

\begin{proof}
        There is one equivalence class for each real number $t$ such that \(0 \leq t < \pi\). One line in each class goes
        through the origin, and that line makes an angle of $t$ with the positive horizontal axis.

        \begin{figure}[ht!]
                \centering
                \includegraphics[scale=0.5]{../images/8.3.32.png}
        \end{figure}

        Alternatively, there is one equivalence class for every
        possible slope: all real numbers plus “undefined.”
\end{proof}

\subsection{Exercise 33}
Let $A$ be the set of points in the rectangle with $x$ and $y$ coordinates between 0 and 1. That is,
\[A = \{(x, y) \in \R \times \R \,|\, 0 \leq x \leq 1 \text{ and } 0 \leq y \leq 1\}.
\]
Define a relation $R$ on $A$ as follows: For all \((x_1, y_1)\) and \((x_2, y_2)\) in $A$,

\begin{center}
        \begin{tabular}{ccccccc}
                \((x_1, y_1) \,R\, (x_2, y_2) \iff\) & \((x_1, y_1)\) & =   & \((x_2, y_2);\) & or                        \\
                                                     & \(x_1=0\)      & and & \(x_2=1\)       & and & \(y_1 = y_2\); & or \\
                                                     & \(x_1=1\)      & and & \(x_2=0\)       & and & \(y_1 = y_2\); & or \\
                                                     & \(y_1=0\)      & and & \(y_2=1\)       & and & \(x_1 = x_2\); & or \\
                                                     & \(y_1=1\)      & and & \(y_2=0\)       & and & \(x_1 = x_2\).
        \end{tabular}
\end{center}

In other words, all points along the top edge of the rectangle are related to the points along the bottom edge
directly beneath them, and all points directly opposite each other along the left and right edges are related to
each other. The points in the interior of the rectangle are not related to anything other than themselves. Then $R$ is
an equivalence relation on $A$. Imagine gluing together all the points that are in the same equivalence class. Describe
the resulting figure.

\begin{proof}
        Gluing the top and bottom edges of the rectangle results in a horizontal cylinder. Then, if we also glue the left and
        right circular ends of this cylinder, we get a doughnut shaped figure:

        \begin{figure}[ht!]
                \centering
                \includegraphics[scale=0.2]{../images/8.3.33.png}
        \end{figure}
\end{proof}

\subsection{Exercise 34}
The documentation for the computer language Java recommends that when an “equals method” is defined for an object, it
be an equivalence relation. That is, if $R$ is defined as follows: \(x \,R\, y \iff x.equals(y)\) for all objects in
the class, then $R$ should be an equivalence relation. Suppose that in trying to optimize some of the mathematics
of a graphics application, a programmer creates an object called a point, consisting of two coordinates in the plane.
The programmer defines an equals method as follows: If $p$ and $q$ are any points, then \(p.equals(q)\) iff the
distance from $p$ to $q$ is less than or equal to $c$ where $c$ is a small positive number that depends on the
resolution of the computer display. Is the programmer’s equals method an equivalence relation? Justify your answer.

\begin{proof}
        No. If points \(p, q\), and $r$ all lie on a straight line with $q$ in the middle, and if $p$ is $c$ units from $q$
        and $q$ is $c$ units from $r$, then $p$ is more than $c$ units from $r$.
\end{proof}

\subsection{Exercise 35}
Find an additional representative circuit for the input/output table of Example 8.3.9.

\begin{proof}
        Recall that the output table is:

        \begin{center}
                \arrayrulecolor{cyan}
                \begin{tabular}{|cc|c|}
                        \hline
                        \multicolumn{2}{|c|}{\cy Input} & {\cy Output}              \\
                        \hline
                        \(\bm{P}\)                      & \(\bm{Q}\)   & \(\bm{R}\) \\
                        \hline
                        1                               & 1            & 0          \\
                        \hline
                        1                               & 0            & 0          \\
                        \hline
                        0                               & 1            & 0          \\
                        \hline
                        0                               & 0            & 1          \\
                        \hline
                \end{tabular}
                \arrayrulecolor{black} % change it back!
        \end{center}

        And two representative circuits for this table were given:

        \begin{figure}[ht!]
                \centering
                \includegraphics[scale=0.5]{../images/8.3.35.png}
        \end{figure}

        Here is another:

        \begin{figure}[ht!]
                \centering
                \includegraphics[scale=0.5]{../images/8.3.35.2.png}
        \end{figure}
\end{proof}

{\bf \cy Let $R$ be an equivalence relation on a set $A$. Prove each of the statements in $36-41$ directly from the
definitions of equivalence relation and equivalence class without using the results of Lemma 8.3.2, Lemma 8.3.3, or
Theorem 8.3.4.}

\subsection{Exercise 36}
For every $a$ in $A$, \(a \in [a]\).

\begin{proof}
        Suppose $R$ is an equivalence relation on a set $A$ and \(a \in A\). Because $R$ is an equivalence relation, $R$ is
        reflexive, and because $R$ is reflexive, each element of $A$ is related to itself by $R$. In particular \(a\,R\,a\).
        Hence, by definition of equivalence class, \(a \in [a]\).
\end{proof}

\subsection{Exercise 37}
For every $a$ and $b$ in $A$, if \(b \in [a]\) then \(a \,R\, b\).

\begin{proof}
        Yes, by definition of \([a], b \in [a] \iff b \,R\,a\). So \(b \,R\, a\). By symmetry, \(a \,R\, b\).
\end{proof}

\subsection{Exercise 38}
For every \(a, b\), and \(c\) in \(A\), if \(b \,R\, c\) and \(c \in [a]\) then \(b \in [a]\).

\begin{proof}
        Suppose $R$ is an equivalence relation on a set $A$ and \(a, b\), and $c$ are elements of $A$ with \(b \,R\, c\) and
        \(c \in [a]\). Since \(c \in [a]\), then \(c \,R\, a\) by definition of equivalence class. Now \(R\) is transitive
        because \(R\) is an equivalence relation. Thus, since \(b \,R\, c\) and \(c \,R\, a\), then \(b \,R\, a\). It follows
        that \(b \in [a]\) by definition of equivalence class.
\end{proof}

\subsection{Exercise 39}
For every $a$ and $b$ in $A$, if \([a] = [b]\) then \(a \,R\, b\).

\begin{proof}
        Assume \([a] = [b]\). By Exercise 36 \(b \in [b]\), and since \([a] = [b]\) we have \(b \in [a]\). So by Exercise
        37 \(a \,R\, b\).
\end{proof}

\subsection{Exercise 40}
For every \(a, b\), and \(x\) in \(A\), if \(a \,R\, b\) and \(x \in [a]\) then \(x \in [b]\).

\begin{proof}
        Suppose \(a, b\), and \(x\) are in \(A, a \,R\, b\), and \(x \in [a]\). By definition of equivalence class,
        \(x \,R\, a\). So \(x \,R\, a\) and \(a \,R\, b\), and thus, by transitivity, \(x \,R\, b\). Hence \(x \in [b]\).
\end{proof}

\subsection{Exercise 41}
For every $a$ and $b$ in $A$, if \(a \in [b]\) then \([a] = [b]\).

\begin{proof}
        Assume \(x \in [a]\). Then \(x \,R\, a\) by definition of \([a]\). Since \(a \in [b]\), by Exercise 37 \(b \,R\, a\).
        By symmetry, \(a \,R\, b\). Then by transitivity \(x \,R\, b\). So \(x \in [b]\) by definition of \([b]\). Thus
        \([a] \subseteq [b]\).

        Assume \(x \in [b]\). Then \(x \,R\, b\) by definition of \([b]\). Since \(a \in [b]\), by Exercise 37 \(b \,R\, a\).
        Then by transitivity \(x \,R\, a\). So \(x \in [a]\) by definition of \([a]\). Thus \([b] \subseteq [a]\).

        So by definition of set equality \([a] = [b]\).
\end{proof}

\subsection{Exercise 42}
Let $R$ be the relation defined in Example 8.3.12: \((a,b) \,R\, (c,d) \iff ad = bc\).

\subsubsection{(a)}
Prove that $R$ is reflexive.

\begin{proof}
        For all \(a \in \Z, b \in \Z - \{0\}, ab = ba\), therefore \((a,b) \,R\, (a,b)\). So $R$ is reflexive.
\end{proof}

\subsubsection{(b)}
Prove that $R$ is symmetric.

\begin{proof}
        Assume \((a,b) \,R\, (c,d)\). Then \(ad = bc\). So \(bc = ad\) and thus \((c,d) \,R\, (a,b)\). So $R$ is symmetric.
\end{proof}

\subsubsection{(c)}
List four distinct elements in [(1, 3)].

\begin{proof}
        One possible answer: \((2, 6), (-2, -6), (3, 9), (-3, -9)\).
\end{proof}

\subsubsection{(d)}
List four distinct elements in [(2, 5)].

\begin{proof}
        One possible answer: \((4, 10), (6, 15), (8, 20), (10, 25)\).
\end{proof}

\subsection{Exercise 43}
In Example 8.3.12, define operations of addition (+) and multiplication (\(\cdot\)) as follows: For every \((a, b),
(c, d) \in A, [(a, b)] + [(c, d)] = [(ad + bc, bd)]\) and \([(a, b)] \cdot [(c, d)] = [(ac, bd)]\).

\subsubsection{(a)}
Prove that this addition is well defined. That is, show that if \([(a, b)] = [(a', b')]\) and \([(c, d)] = [(c',
d')]\), then \([(ad + bc, bd)] = [(a'd' + b'c', b'd')]\).

\begin{proof}
        Suppose that \((a, b), (a', b'), (c, d)\), and \((c', d')\) are any elements of \(A\) such that \([(a, b)]=[(a', b')]\)
        and \([(c, d)] = [(c', d')]\). By definition of \(R, ab' = ba'\) (*) and \(cd' = dc'\) (**). We must show that
        \([(a, b)] + [(c, d)] = [(a', b')] + [(c', d')]\). By definition of the addition on \(A\), this equation is true
        if, and only if, \([(ad + bc, bd)] = [(a'd' + b'c', b'd')]\). And, by definition of the relation, this equation is
        true if, and only if, \((ad + bc)b'd' = bd(a'd' + b'c')\). After multiplying out, this becomes \(adb'd' + bcb'd' =
        bda'd' + bdb'c'\), and regrouping, turns it into \((ab')(dd') + (cd')(bb') = (ba')(dd') + (dc')(bb')\).
        Substituting the values from (*) and (**) shows that this last equation is true.
\end{proof}

\subsubsection{(b)}
Prove that this multiplication is well defined. That is, show that if \([(a, b)] = [(a', b')]\) and \([(c, d)] =
[(c', d')]\), then \([(ac, bd)] = [(a'c', b'd')]\).

\begin{proof}
        1. Assume \([(a, b)] = [(a', b')]\) and \([(c, d)] = [(c', d')]\).

        2. By 1 and the definition of equivalence classes of $R$, \((a,b) \,R\, (a',b')\) and \((c,d) \,R\, (c',d')\).

        3. By 2 and definition of $R$, \(ab' = ba'\) and \(cd' = dc'\).

        4. By 3, multiplying the left hand sides together and the right hand sides together, \(ab'cd' = ba'dc'\).

        5. By 4 and commutativity, reorganizing, \((ac)(b'd') = (bd)(a'c')\).

        6. By 5 and definition of $R$, \((ad, bc) \,R\, (a'c', b'd')\).

        7. By 6 and definition of equivalence classes of $R$, \([(ad, bc)] = [(a'c', b'd')]\).
\end{proof}

\subsubsection{(c)}
Show that \([(0, 1)]\) is an identity element for addition. That is, show that for any \((a, b) \in A, [(a, b)] + [(0,
1)] = [(0, 1)] + [(a, b)] = [(a, b)]\).

\begin{proof}
        Suppose that \((a, b)\) is any element of \(A\). We must show that \([(a, b)] + [(0, 1)] = [(a, b)]\). By definition
        of the addition on \(A\), this equation is true if, and only if, \([(a \cdot 1 + b \cdot 0, b \cdot 1)]=[(a, b)]\).
        And this last equation is true because \(a \cdot 1 + b \cdot 0 = a\) and \(b \cdot 1 = b\).
\end{proof}

\subsubsection{(d)}
Find an identity element for multiplication. That is, find \((i, j)\) in \(A\) so that for every \((a, b)\) in \(A\),
\([(a, b)] \cdot [(i, j)] = [(i, j)] \cdot [(a, b)] = [(a, b)]\).

\begin{proof}
        The multiplicative identity is \((1,1)\). Indeed, by definition of multiplication on \(A\), \([(a, b)] \cdot
        [(1, 1)] = [(a \cdot 1, b \cdot 1)] = [(a,b)]\) and \([(1, 1)] \cdot [(a, b)] = [(1 \cdot a, 1 \cdot b)] = [(a,b)]\).
\end{proof}

\subsubsection{(e)}
For any \((a, b) \in A\), show that \([(-a, b)]\) is an inverse for \([(a, b)]\) for addition. That is, show that
\([(-a, b)] + [(a, b)] = [(a, b)] + [(-a, b)] = [(0, 1)]\).

\begin{proof}
        Suppose that \((a, b)\) is any element of \(A\). We must show that \([(a, b)] + [(-a, b)] = [(-a, b)] + [(a, b)] =
        [(0, 1)]\). By definition of the addition on \(A\), this equation is true if, and only if, \([(ab + b(-a), bb)] =
        [(0, 1)]\), or, equivalently, \([(0, bb)] = [(0, 1)]\). By definition of the relation, this last equation is true if,
        and only if, \(0 \cdot 1 = bb \cdot 0\), which is true.
\end{proof}

\subsubsection{(f)}
Given any \((a, b) \in A\) with \(a \neq 0\), find an inverse for \([(a, b)]\) for multiplication. That is, find
\((c, d)\) in \(A\) so that \([(a, b)] \cdot [(c, d)] = [(c, d)] \cdot [(a, b)] = [(i, j)]\), where \([(i, j)]\) is
the identity element you found in part (d).

\begin{proof}
        Given \([(a,b)]\) we want to find \([(c,d)]\) such that \([(a,b)] \cdot [(c,d)] = [(ac,bd)] = [(1,1)]\).

        So by the definition of equivalence classes of $R$ on $A$, \((ac,bd)\) is related to \((1,1)\) by $R$, in other words
        \(ac \cdot 1 = bd \cdot 1\), or \(ac = bd\). Then let \((c,d) = (b,a)\). So \(ac = ab = ba = bd\), therefore
        \((ac,bd) \,R\, (1,1)\) and thus \([(ac,bd)] = [(1,1)]\), in other words \([(a,b)] \cdot [(c,d)] = [(1,1)]\).

        Similarly we can prove that \([(c,d)] \cdot [(a,b)] = [(1,1)]\).
\end{proof}

\subsection{Exercise 44}
Let \(A = Z^+ \times Z^+\). Define a relation \(R\) on \(A\) as follows: For every \((a, b)\) and \((c, d)\) in
\(A, (a, b) \,R\, (c, d) \iff a + d = c + b\).

\subsubsection{(a)}
Prove that $R$ is reflexive.

\begin{proof}
        Let \((a, b)\) be any element of \(Z^+ \times Z^+\). We must show that \((a, b) \,R\, (a, b)\). By definition of
        $R$, this relationship holds if, and only if, \(a + b = b + a\). But this equation is true by the commutative law of
        addition for real numbers. Hence \(R\) is reflexive.
\end{proof}

\subsubsection{(b)}
Prove that $R$ is symmetric.

\begin{proof}
        Assume \((a, b) \,R\, (c, d)\). Then by definition of $R$, \(a + d = c + b\). So \(c + b = a + d\). Thus
        \((c, d) \,R\, (a, b)\) by definition of $R$. So $R$ is symmetric.
\end{proof}

\subsubsection{(c)}
Prove that $R$ is transitive.

\begin{proof}
        Assume \((a, b) \,R\, (c, d)\) and \((c, d)\,R\,(e, f)\). By definition of $R$, \(a + d=c + b\) and \(c + f=e + d\).
        Adding the two equations we get \(a+d+c+f = c+b+e+d\). Canceling \(c+d\) on both sides we get \(a+f = e+b\), thus
        \((a,b) \,R\, (e,f)\) so $R$ is transitive.
\end{proof}

\subsubsection{(d)}
List five elements in [(1, 1)].

\begin{proof}
        One possible answer: (1, 1), (2, 2), (3, 3), (4, 4), (5, 5)
\end{proof}

\subsubsection{(e)}
List five elements in [(3, 1)].

\begin{proof}
        One possible answer: (4, 2), (5, 3), (6, 4), (7, 5), (8, 6)
\end{proof}

\subsubsection{(f)}
List five elements in [(1, 2)].

\begin{proof}
        One possible answer: (1, 2), (2, 3), (3, 4), (4, 5), (5, 6)
\end{proof}

\subsubsection{(g)}
Describe the distinct equivalence classes of $R$.

\begin{proof}
        Observe that for any positive integers $a$ and $b$, the equivalence class of \((a, b)\) consists of all ordered
        pairs in \(Z^+ \times Z^+\) for which the difference between the first and second coordinates equals \(a - b\).
        Thus there is one equivalence class for each integer: positive, negative, and zero. Each positive integer \(n\)
        corresponds to the class of \((n + 1, 1)\); each negative integer \(-n\) corresponds to the class of \((1, n + 1)\);
        and zero corresponds to the class \((1, 1)\).
\end{proof}

\subsection{Exercise 45}
The following argument claims to prove that the requirement that an equivalence relation be reflexive is redundant. In
other words, it claims to show that if a relation is symmetric and transitive, then it is reflexive. Find the
mistake in the argument.

“Proof: Let $R$ be a relation on a set $A$ and suppose $R$ is symmetric and transitive. For any two elements $x$ and
$y$ in $A$, if \(x \,R\, y\) then \(y \,R\, x\) since $R$ is symmetric. Thus it follows by transitivity that
\(x \,R\, x\), and hence $R$ is reflexive.”

\begin{proof}
        The conclusion \(x \,R\, x\) only follows under the assumption that \(x \,R\, y\), which has not been
        discharged from the proof. (See next exercise.)
\end{proof}

\subsection{Exercise 46}
Let $R$ be a relation on a set $A$ and suppose $R$ is symmetric and transitive. Prove the following: If for every
$x$ in $A$ there is a $y$ in $A$ such that \(x \,R\, y\), then $R$ is an equivalence relation.

\begin{proof}
        Let $R$ be a relation on a set $A$ and suppose $R$ is symmetric and transitive. Assume $x$ is any element in $A$.
        By the assumption, there exists $y$ in $A$ such that \(x \,R\, y\). Then \(y \,R\, x\) since $R$ is symmetric. Thus
        it follows by transitivity that \(x \,R\, x\), and hence $R$ is reflexive. Hence $R$ is an equivalence relation.
\end{proof}

\subsection{Exercise 47}
Refer to the quote at the beginning of this section to answer the following questions.

\subsubsection{(a)}
What is the name of the Knight’s song called?

\begin{proof}
        ... The name of the song is called ‘Haddocks’ Eyes.’
\end{proof}

\subsubsection{(b)}
What is the name of the Knight’s song?

\begin{proof}
        The name really is ‘The Aged Aged Man.’
\end{proof}

\subsubsection{(c)}
What is the Knight’s song called?

\begin{proof}
        “Ways and Means”
\end{proof}

\subsubsection{(d)}
What is the Knight’s song?

\begin{proof}
        The song really is ‘A-sitting on a Gate’
\end{proof}

\subsubsection{(e)}
What is your (full, legal) name?

\begin{proof}
        Spam, Egg
\end{proof}

\subsubsection{(f)}
What are you called?

\begin{proof}
        Spam
\end{proof}

\subsubsection{(g)}
What are you? (Do not answer this on paper; just think about it.)

\begin{proof}
        {\it ???}
\end{proof}

\section{Exercise Set 8.4}

\subsection{Exercise 1}
\subsubsection{(a)}
Use the Caesar cipher to encrypt the message WHERE SHALL WE MEET.

\begin{proof}
        ZKHUH VKDOO ZH PHHW
\end{proof}

\subsubsection{(b)}
Use the Caesar cipher to decrypt the message LQ WKH FDIHWHULD.

\begin{proof}
        IN THE CAFETERIA
\end{proof}

\subsection{Exercise 2}
\subsubsection{(a)}
Use the Caesar cipher to encrypt the message AN APPLE A DAY.

\begin{proof}
        DQ DSSOH D GDB
\end{proof}

\subsubsection{(b)}
Use the Caesar cipher to decrypt the message NHHSV WKH GRFWRU DZDB.

\begin{proof}
        KEEPS THE DOCTOR AWAY
\end{proof}

\subsection{Exercise 3}
Let \(a = 25, b = 19, n = 3\).

\subsubsection{(a)}
Verify that \(3 \mid (25 - 19)\).

\begin{proof}
        The relation \(3 \mid (25 - 19)\) is true because \(25 - 19 = 6\) and \(3 \mid 6\) (since \(6 = 3 \cdot 2\)).
\end{proof}

\subsubsection{(b)}
Explain why \(25 \equiv 19 \pmod 3\).

\begin{proof}
        By definition of congruence modulo \(n\), to show that \(25 \equiv 19 \pmod 3\), one must show that \(3 \mid (25-19)\).
        This was verified in part (a).
\end{proof}

\subsubsection{(c)}
What value of \(k\) has the property that \(25 = 19 + 3k\)?

\begin{proof}
        To show that \(25 = 19 + 3k\) for some integer \(k\), one solves the equation for \(k\) and checks that the result is
        an integer. In this case, \(k = (25 - 19)/3 = 2\), which is an integer. Thus \(25 = 19 + 2 \cdot 3\).
\end{proof}

\subsubsection{(d)}
What is the (nonnegative) remainder obtained when 25 is divided by 3? When 19 is divided by 3?

\begin{proof}
        When 25 is divided by 3, the remainder is 1 because \(25 = 3 \cdot 8 + 1\). When 19 is divided by 3, the remainder is
        also 1 because \(19 = 3 \cdot 6 + 1\). Thus 25 and 19 have the same remainder when divided by 3.
\end{proof}

\subsubsection{(e)}
Explain why \(25 \mod 3 = 19 \mod 3\).

\begin{proof}
        By definition, \(25 \mod 3\) is the remainder obtained when 25 is divided by 3, and \(19 \mod 3\) is the remainder
        obtained when 19 is divided by 3. In part (d) these two numbers were shown to be equal.
\end{proof}

\subsection{Exercise 4}
Let \(a = 68, b = 33, n = 7\).

\subsubsection{(a)}
Verify that \(7 \mid (68 - 33)\).

\begin{proof}
        The relation \(7 \mid (68 - 33)\) is true because \(68 - 33 = 35\) and \(7 \mid 35\) (since \(35 = 7 \cdot 5\)).
\end{proof}

\subsubsection{(b)}
Explain why \(68 \equiv 33 \pmod 7\).

\begin{proof}
        By definition of congruence modulo \(n\), to show that \(68 \equiv 33 \pmod 7\), one must show that \(7 \mid (68-33)\).
        This was verified in part (a).
\end{proof}

\subsubsection{(c)}
What value of $k$ has the property that \(68 = 33 + 7k\)?

\begin{proof}
        To show that \(68 = 33 + 7k\) for some integer \(k\), one solves the equation for \(k\) and checks that the result is
        an integer. In this case, \(k = (68 - 33)/7 = 5\), which is an integer. Thus \(68 = 33 + 7 \cdot 5\).
\end{proof}

\subsubsection{(d)}
What is the (nonnegative) remainder obtained when 68 is divided by 7? When 33 is divided by 7?

\begin{proof}
        When 68 is divided by 7, the remainder is 5 because \(68 = 7 \cdot 9 + 5\). When 33 is divided by 7, the remainder is
        also 5 because \(33 = 7 \cdot 4 + 5\). Thus 68 and 33 have the same remainder when divided by 7.
\end{proof}

\subsubsection{(e)}
Explain why \(68 \mod 7 = 33 \mod 7\).

\begin{proof}
        By definition, \(68 \mod 7\) is the remainder obtained when 68 is divided by 7, and \(33 \mod 7\) is the remainder
        obtained when 33 is divided by 7. In part (d) these two numbers were shown to be equal.
\end{proof}

\subsection{Exercise 5}
Prove the transitivity of modular congruence. That is, prove that for all integers \(a, b, c\), and \(n\) with
\(n > 1\), if \(a \equiv b \pmod n\) and \(b \equiv c \pmod n\) then \(a \equiv c \pmod n\).

\begin{proof}
        1. Assume \(a \equiv b \pmod n\) and \(b \equiv c \pmod n\).

        2. By 1 and definition of modular congruence, \(n \mid (a-b)\) and \(n \mid (b-c)\).

        3. By 2 and definition of divides, \(a-b = en\) and \(b-c = fn\) for some integers \(e,f\).

        4. By 3, adding the equations we get \(a-c = (a-b)+(b-c) = en + fn = (e+f)n\) where \(e+f\) is an integer.

        5. By 4 and definition of divides, \(n \mid (a-c)\).

        6. By 5 and definition of modular congruence, \(a \equiv c \pmod n\).
\end{proof}

\subsection{Exercise 6}
Prove that the distinct equivalence classes of the relation of congruence modulo \(n\) are the sets \([0], [1], [2],
\ldots, [n - 1]\), where for each \(a = 0, 1, 2, \ldots, n - 1, [a] = \{m \in \Z \,|\, m \equiv a \pmod n\}\).

\begin{proof}
        Assume \(a \in \Z\). By the quotient-remainder theorem, \(a = nq + r\) for some integers \(q,r\) with \(0 \leq r < n\).
        So \(a \equiv r \pmod n\) and by definition of \([r], a \in [r]\). Since \(0 \leq r < n\), $a$ belongs to one of the
        sets \([0], [1], [2], \ldots, [n - 1]\).

        Now we need to show the sets \([0], [1], [2], \ldots, [n - 1]\) are distinct equivalence classes. Assume \(0\leq a<n\)
        and \(0 \leq b < n\). We need to show that if \([a] = [b]\) then \(a = b\).

        Assume \([a] = [b]\). Since \(a \equiv a \pmod n\), by definition of \([a], a \in [a]\). Since \([a] = [b], a \in [b]\). Then by definition of \([b]\), \(a \equiv b \pmod n\). So \(n \mid (a-b)\). So \(a-b = kn\) for some integer $k$. Notice that since \(0\leq a < n\) and \(0 \leq b < n\), we have \(-n < a-b < n\). Therefore $k=0$. So $a=b$.

        We have shown that every integer belongs to one of the classes \([0], [1], [2], \ldots, [n-1]\), and we have show
        that these classes are all distinct.
\end{proof}

\subsection{Exercise 7}
Verify the following statements.

\subsubsection{(a)}
\(128 \equiv 2 \pmod 7\) and \(61 \equiv 5 \pmod 7\)

\begin{proof}
        \(128 \equiv 2 \pmod 7\) because \(128 - 2 = 126 = 7 \cdot 18\),
        \(61 \equiv 5 \pmod 7\) because \(61 - 5 = 56 = 7 \cdot 8\)
\end{proof}

\subsubsection{(b)}
\((128 + 61) \equiv (2 + 5) \pmod 7\)

\begin{proof}
        \(128 + 61 \equiv (2 + 5) \pmod 7\) because \(128 + 61 = 189, 2 + 5 = 7\), and \(189 - 7 = 182 = 7 \cdot 26\).
\end{proof}

\subsubsection{(c)}
\((128 - 61) \equiv (2 - 5) \pmod 7\)

\begin{proof}
        \(128 - 61 \equiv (2 - 5) \pmod 7\) because \(128 - 61 = 67, 2 - 5 = -3\), and \(67 - (-3) = 70 = 7 \cdot 10\).
\end{proof}

\subsubsection{(d)}
\((128 \cdot 61) \equiv (2 \cdot 5) \pmod 7\)

\begin{proof}
        \(128 \cdot 61 \equiv (2 \cdot 5) \pmod 7\) because \(128 \cdot 61 = 7808, 2 \cdot 5 = 10\), and \(7808 - (10) = 7798
        = 7 \cdot 1114\).
\end{proof}

\subsubsection{(e)}
\(128^2 \equiv 2^2 \pmod 7\)

\begin{proof}
        \(128^2 \equiv 2^2 \pmod 7\) because \(128^2 = 16384, 2^2 = 4\), and \(16384 - 4 = 16380 = 7 \cdot 2340\).
\end{proof}

\subsection{Exercise 8}
Verify the following statements.

\subsubsection{(a)}
\(45 \equiv 3 \pmod 6\) and \(104 \equiv 2 \pmod 6\)

\begin{proof}
        \(45 \equiv 3 \pmod 6\) because \(45 - 3 = 42 = 6 \cdot 7\)

        \(104 \equiv 2 \pmod 6\) because \(104 - 2 = 102 = 6 \cdot 17\)
\end{proof}

\subsubsection{(b)}
\((45 + 104) \equiv (3 + 2) \pmod 6\)

\begin{proof}
        \(45 + 104 \equiv (3 + 2) \pmod 6\) because \(45 + 104 = 149, 3 + 2 = 5\), and \(149 - 5 = 144 = 6 \cdot 24\).
\end{proof}

\subsubsection{(c)}
\((45 - 104) \equiv (3 - 2) \pmod 6\)

\begin{proof}
        \(45 - 104 \equiv (3 - 2) \pmod 6\) because \(45 - 104 = -59, 3 - 2 = 1\), and \(-59 - (1) = -60 = 6 \cdot (-10)\).
\end{proof}

\subsubsection{(d)}
\((45 \cdot 104) \equiv (3 \cdot 2) \pmod 6\)

\begin{proof}
        \(45 \cdot 104 \equiv (3 \cdot 2) \pmod 6\) because \(45 \cdot 104 = 4680, 3 \cdot 2 = 6\), and \(4680 - (6) = 4674
        = 6 \cdot 779\).
\end{proof}

\subsubsection{(e)}
\(45^2 \equiv 3^2 \pmod 6\)

\begin{proof}
        \(45^2 \equiv 3^2 \pmod 6\) because \(45^2 = 2025, 3^2 = 9\), and \(2025 - 9 = 2016 = 6 \cdot 336\).
\end{proof}

{\bf \cy In $9-11$, prove each of the given statements, assuming that \(a, b, c, d\), and \(n\) are integers with
\(n > 1\) and that \(a \equiv c \pmod n\) and \(b \equiv d \pmod n\).}

\subsection{Exercise 9}
\((a + b) \equiv (c + d) \pmod n\)

\subsubsection{(a)}

\begin{proof}
        Suppose \(a, b, c, d\), and \(n\) are integers with \(n > 1, a \equiv c \pmod n\), and \(b \equiv d \pmod n\). By
        Theorem 8.4.1, \(a - c = nr\) and \(b - d = ns\) for some integers \(r\) and \(s\). Then \((a + b) - (c + d) =
        (a - c) + (b - d) = nr + ns = n(r + s)\). Now \(r + s\) is an integer, and so, by Theorem 8.4.1,
        \(a + b \equiv (c + d) \pmod n\).
\end{proof}

\subsubsection{(b)}
\((a - b) \equiv (c - d) \pmod n\)

\begin{proof}
        Suppose \(a, b, c, d\), and \(n\) are integers with \(n > 1, a \equiv c \pmod n\), and \(b \equiv d \pmod n\). By
        Theorem 8.4.1, \(a - c = nr\) and \(b - d = ns\) for some integers \(r\) and \(s\). Then \((a - b) - (c - d) =
        (a - c) - (b - d) = nr - ns = n(r - s)\). Now \(r - s\) is an integer, and so, by Theorem 8.4.1,
        \(a - b \equiv (c - d) \pmod n\).
\end{proof}

\subsection{Exercise 10}
\(a^2 \equiv c^2 \pmod n\)

\begin{proof}
        Suppose \(a, c\), and \(n\) are integers with \(n > 1\) and \(a \equiv c \pmod n\). By Theorem 8.4.1, \(a - c = nr\)
        for some integers \(r\). Then \(a^2-c^2 = (a-c)(a+c) = nr(a+c)\). Now \(r(a+c)\) is an integer, and so, by Theorem
        8.4.1, \(a^2 \equiv c^2 \pmod n\).
\end{proof}

\subsection{Exercise 11}
\(a^m \equiv c^m \pmod n\) for every integer \(m \geq 1\) (Use mathematical induction on \(m\).)

\begin{proof}
        {\it Show that \(a \equiv c \pmod n\)}: This holds by assumption.

                {\it Show that for every integer \(m \geq 1\) if \(a^m \equiv c^m \pmod n\) then \(a^{m+1} \equiv c^{m+1} \pmod
                        n\):} Assume \(m \geq 1\) and \(a^m \equiv c^m \pmod n\). We also know that \(a \equiv c \pmod n\) by assumption, so
        \(a-c = ns\) for some integer $s$. Then

        \begin{center}
                \begin{tabular}{rcl}
                        \(a^{m+1} - c^{m+1}\) & = & \((a-c)(a^{m} + a^{m-1}c + \ldots + ac^{m-1} + c^{m})\) \\
                                              & = & \(ns(a^{m} + a^{m-1}c + \ldots + ac^{m-1} + c^{m})\)    \\
                \end{tabular}
        \end{center}

        therefore \(n \mid (a^{m+1}-c^{m+1})\), so by definition of congruence, \(a^{m+1} \equiv c^{m+1} \pmod n\).
\end{proof}

\subsection{Exercise 12}
\subsubsection{(a)}
Prove that for every integer \(n \geq 0, 10^n \equiv 1 \pmod 9\).

\begin{proof}
        Let the property \(P(n)\) be the congruence \(10^n \equiv 1 \pmod 9\).

                {\bf Show that \(P(0)\) is true:} When \(n = 0\), the left-hand side of the congruence is \(10^0 = 1\) and the right-
        hand side is also 1.

                {\bf Show that for every integer \(k \geq 0\), if \(P(k)\) is true, then \(P(k + 1)\) is true:} Let \(k\) be any
        integer with \(k \geq 0\), and suppose \(P(k)\) is true. That is, suppose \(10^k \equiv 1 \pmod 9\). (*) {\it [This
                                is the inductive hypothesis.]} By Theorem 8.4.1, \(10 \equiv 1 \pmod 9\) (**) because \(10 - 1 = 9 = 9 \cdot 1\).
        And by Theorem 8.4.3, we can multiply the left- and right-hand sides of (*) and (**) to obtain \(10^k \cdot 10 \equiv
        1 \cdot 1 \pmod 9\), or, equivalently, \(10^{k+1} \equiv 1 \pmod 9\). Hence \(P(k + 1)\) is true.

        Alternative Proof: Note that \(10 \equiv 1 \pmod 9\) because \(10 - 1 = 9\) and \(9 \mid 9\). Thus by Theorem
        8.4.3(4), \(10^n \equiv 1^n \equiv 1 \pmod 9\).
\end{proof}

\subsubsection{(b)}
Use part (a) to prove that a positive integer is divisible by 9 if, and only if, the sum of its digits is divisible by 9.

\begin{proof}
        Assume \(n\) is a positive integer. We can write \(n\) in base 10 expansion:
        \[
                n = a_k \cdot 10^k + a_{k-1} \cdot 10^{k-1} + \cdots + a_1 \cdot 10 + a_0 \cdot 10^0
        \]
        where \(a_k, \ldots, a_0\) are the decimal digits of $n$.

                {\it [We want to show $n$ is divisible by 9 if and only if \(a_k + a_{k-1} + \cdots +a_1+ a_0\) is divisible by 9.]}

        By part (a), \(10^k \equiv 10^{k-1} \equiv \cdots \equiv 10^1 \equiv 10^0 \equiv 1 \pmod 9\). Therefore
        \[
                n \equiv a_k \cdot 10^k + a_{k-1} \cdot 10^{k-1} + \cdots + a_1 \cdot 10 + a_0 \cdot 10^0 \equiv a_k \cdot 1 + \cdots +
                a_1 \cdot 1 + a_0 \cdot 1 \pmod 9
        \]
        If \(n\) is divisible by 9, \(n \equiv 0 \pmod 9\), so by the above congruence, \(a_k +\cdots+a_0 \equiv 0 \pmod 9\),
        so \(a_k + \cdots + a_0\) is divisible by 9. Similarly if \(a_k + \cdots + a_0\) is divisible by 9 then $n$ is
        divisible by 9 by the same reasoning.
\end{proof}

\subsection{Exercise 13}
\subsubsection{(a)}
Prove that for every integer \(n \geq 1, 10^n \equiv (-1)^n \pmod{11}\).

\begin{proof}
        Let the property \(P(n)\) be the congruence \(10^n \equiv (-1)^n \pmod{11}\).

                {\bf Show that \(P(0)\) is true:} When \(n = 0\), the left-hand side of the congruence is \(10^0 = 1\) and the right-
        hand side is also \((-1)^n = 1\).

                {\bf Show that for every integer \(k \geq 0\), if \(P(k)\) is true, then \(P(k + 1)\) is true:} Let \(k\) be any
        integer with \(k \geq 0\), and suppose \(P(k)\) is true. That is, suppose \(10^k \equiv (-1)^k \pmod{11}\). (*) {\it
                        [This is the inductive hypothesis.]} By Theorem 8.4.1, \(10 \equiv -1 \pmod{11}\) (**) because \(10 - (-1) = 11 = 11
        \cdot 1\). And by Theorem 8.4.3, we can multiply the left- and right-hand sides of (*) and (**) to obtain \(10^k \cdot
        10 \equiv (-1)^k \cdot (-1) \pmod{11}\), or, equivalently, \(10^{k+1} \equiv (-1)^{k+1} \pmod{11}\). Hence \(P(k + 1)\) is true.
\end{proof}

\subsubsection{(b)}
Use part (a) to prove that a positive integer is divisible by 11 if, and only if, the alternating sum of its digits is
divisible by 11. (For instance, the alternating sum of the digits of 82,379 is \(8 - 2 + 3 - 7 + 9 = 11\) and
\(82,379 = 11 \cdot 7489\).)

\begin{proof}
        Assume \(n\) is a positive integer. We can write \(n\) in base 10 expansion:
        \[
                n = a_k \cdot 10^k + a_{k-1} \cdot 10^{k-1} + \cdots + a_1 \cdot 10 + a_0 \cdot 10^0
        \]
        where \(a_k, \ldots, a_0\) are the decimal digits of $n$.

                {\it [We want to show $n$ is divisible by 11 if and only if \(a_k \cdot (-1)^k + a_{k-1} \cdot (-1)^{k-1} + \cdots +
                                a_1 \cdot (-1)^1 + a_0 \cdot (-1)^0\) is divisible by 11.]}

        By part (a), \(10^k \equiv (-1)^{k} \pmod {11}\). Therefore
        \[
                n \equiv a_k \cdot (-1)^k + a_{k-1} \cdot (-1)^{k-1} + \cdots + a_1 \cdot (-1)^1 + a_0 \cdot (-1)^0 \pmod{11}
        \]
        If \(n\) is divisible by 11, \(n \equiv 0 \pmod{11}\), so by the above congruence, \(a_k \cdot (-1)^k + \cdots + a_0
        \cdot (-1)^0 \equiv 0 \pmod{11}\), so \(a_k \cdot (-1)^k + \cdots + a_0 \cdot (-1)^0\) is divisible by 11. Similarly
        if \(a_k + \cdots + a_0\) is divisible by 11 then $n$ is
        divisible by 11 by the same reasoning.
\end{proof}

\subsection{Exercise 14}
Use the technique of Example 8.4.4 to find \(14^2 \mod 55, 14^4 \mod 55, 14^8 \mod 55\), and \(14^{16} \mod 55\).

\begin{proof}
        \begin{tabular}{lclclcl}
                \(14^1 \mod 55\)    & = & \(14\)                                                           \\
                \(14^2 \mod 55\)    & = & \(196 \mod 55\)              & = & \(31\)                        \\
                \(14^4 \mod 55\)    & = & \((14^2 \mod 55)^2 \mod 55\) & = & \(31^2 \mod 55\) & = & \(26\) \\
                \(14^8 \mod 55\)    & = & \((14^4 \mod 55)^2 \mod 55\) & = & \(26^2 \mod 55\) & = & \(16\) \\
                \(14^{16} \mod 55\) & = & \((14^8 \mod 55)^2 \mod 55\) & = & \(16^2 \mod 55\) & = & \(36\)
        \end{tabular}
\end{proof}

\subsection{Exercise 15}
Use the result of exercise 14 and the technique of Example 8.4.5 to find \(14^{27} \mod 55\).

\begin{proof}
        \(14^{27} \mod 55 = 14^{16+8+2+1} \mod 55\)

        \begin{tabular}{rcl}
                 & = & \((14^{16} \mod 55)(14^8 \mod 55)(14^2 \mod 55)(14 \mod 55) \mod 55\) \\
                 & = & \(36 \cdot 16 \cdot 31 \cdot 14 \mod 55 = 249985 \mod 55 = 9\)
        \end{tabular}
\end{proof}

{\bf \cy In $16-18$, use the techniques of example 8.4.4 and example 8.4.5 to find the given numbers.}

\subsection{Exercise 16}
\(675^{307} \mod 713\)

\begin{proof}
        Note that \(307 = 256 + 32 + 16 + 2 + 1\).

        \begin{tabular}{lclclcl}
                \(675^1 \mod 713\)     & = & \(675\)                                                     \\
                \(675^2 \mod 713\)     & = & \(455625 \mod 713\) & = & \(18\)                            \\
                \(675^4 \mod 713\)     & = & \(18^2 \mod 713\)   & = & \(324\)                           \\
                \(675^8 \mod 713\)     & = & \(324^2 \mod 713\)  & = & \(104976 \mod 713\) & = & \(165\) \\
                \(675^{16} \mod 713\)  & = & \(165^2 \mod 713\)  & = & \(27225 \mod 713\)  & = & \(131\) \\
                \(675^{32} \mod 713\)  & = & \(131^2 \mod 713\)  & = & \(17161 \mod 713\)  & = & \(49\)  \\
                \(675^{64} \mod 713\)  & = & \(49^2 \mod 713\)   & = & \(2401 \mod 713\)   & = & \(262\) \\
                \(675^{128} \mod 713\) & = & \(262^2 \mod 713\)  & = & \(68644 \mod 713\)  & = & \(196\) \\
                \(675^{256} \mod 713\) & = & \(196^2 \mod 713\)  & = & \(38416 \mod 713\)  & = & \(627\)
        \end{tabular}

        So \(675^{307} \mod 713\) = \(675^{256 + 32 + 16 + 2 + 1} \mod 713\) = \((675^{256} \cdot 675^{32} \cdot 675^{16}
        \cdot 675^{2} \cdot 675^{1}) \mod 713\) = \((627 \cdot 49 \cdot 131 \cdot 18 \cdot 675) \mod 713\) = \(48900262950
        \mod 713 = 3\)
\end{proof}

\subsection{Exercise 17}
\(89^{307} \mod 713\)

\begin{proof}
        Note that \(307 = 256 + 32 + 16 + 2 + 1\).

        \begin{tabular}{lclclcl}
                \(89^1 \mod 713\)     & = & \(89\)                                                     \\
                \(89^2 \mod 713\)     & = & \(7921 \mod 713\)  & = & \(78\)                            \\
                \(89^4 \mod 713\)     & = & \(78^2 \mod 713\)  & = & \(380\)                           \\
                \(89^8 \mod 713\)     & = & \(380^2 \mod 713\) & = & \(144400 \mod 713\) & = & \(374\) \\
                \(89^{16} \mod 713\)  & = & \(374^2 \mod 713\) & = & \(139876 \mod 713\) & = & \(128\) \\
                \(89^{32} \mod 713\)  & = & \(128^2 \mod 713\) & = & \(16384 \mod 713\)  & = & \(698\) \\
                \(89^{64} \mod 713\)  & = & \(698^2 \mod 713\) & = & \(487204 \mod 713\) & = & \(225\) \\
                \(89^{128} \mod 713\) & = & \(225^2 \mod 713\) & = & \(50625 \mod 713\)  & = & \(2\)   \\
                \(89^{256} \mod 713\) & = & \(2^2 \mod 713\)   & = & \(4\)
        \end{tabular}

        So \(89^{307} \mod 713\) = \(89^{256 + 32 + 16 + 2 + 1} \mod 713\) = \((89^{256} \cdot 89^{32} \cdot 89^{16}
        \cdot 89^{2} \cdot 89^{1}) \mod 713\) = \((4 \cdot 698 \cdot 128 \cdot 78 \cdot 89) \mod 713\) = \(2480904192 \mod
        713 = 15\)
\end{proof}

\subsection{Exercise 18}
\(48^{307} \mod 713\)

\begin{proof}
        Note that \(307 = 256 + 32 + 16 + 2 + 1\).

        \begin{tabular}{lclclcl}
                \(48^1 \mod 713\)     & = & \(48\)                                                     \\
                \(48^2 \mod 713\)     & = & \(2304 \mod 713\)  & = & \(165\)                           \\
                \(48^4 \mod 713\)     & = & \(165^2 \mod 713\) & = & \(131\)                           \\
                \(48^8 \mod 713\)     & = & \(131^2 \mod 713\) & = & \(17161 \mod 713\)  & = & \(49\)  \\
                \(48^{16} \mod 713\)  & = & \(49^2 \mod 713\)  & = & \(2401 \mod 713\)   & = & \(262\) \\
                \(48^{32} \mod 713\)  & = & \(262^2 \mod 713\) & = & \(68644 \mod 713\)  & = & \(196\) \\
                \(48^{64} \mod 713\)  & = & \(196^2 \mod 713\) & = & \(38416 \mod 713\)  & = & \(627\) \\
                \(48^{128} \mod 713\) & = & \(627^2 \mod 713\) & = & \(393129 \mod 713\) & = & \(266\) \\
                \(48^{256} \mod 713\) & = & \(266^2 \mod 713\) & = & \(70756 \mod 713\)  & = & \(169\)
        \end{tabular}

        So \(48^{307} \mod 713\) = \(48^{256 + 32 + 16 + 2 + 1} \mod 713\) = \((48^{256} \cdot 48^{32} \cdot 48^{16}
        \cdot 48^{2} \cdot 48^{1}) \mod 713\) = \((169 \cdot 196 \cdot 262 \cdot 165 \cdot 48) \mod 713\) = \(68733624960
        \mod 713 = 12\)
\end{proof}

{\bf \cy In $19-24$, use the RSA cipher from examples 8.4.9 and 8.4.10. In $19-21$, translate the message into its
numeric equivalent and encrypt it. In $22-24$, decrypt the ciphertext and translate the result into letters of the
alphabet to discover the message.}

\subsection{Exercise 19}
HELLO

\begin{proof}
        The letters in HELLO translate numerically into 08, 05, 12, 12, and 15. By Example 8.4.9, the H is encrypted as 17. To
        encrypt E, we compute \(5^3 \mod 55 = 15\). To encrypt L, we compute \(12^3 \mod 55 = 23\). And to encrypt O, we
        compute \(15^3 \mod 55 = 20\). Thus the ciphertext is 17 15 23 23 20. (In practice, individual letters of the alphabet
        are grouped together in blocks during encryption so that deciphering cannot be accomplished through knowledge of
        frequency patterns of letters or words.)
\end{proof}

\subsection{Exercise 20}
WELCOME

\begin{proof}
        The letters in WELCOME translate numerically into 23, 05, 12, 03, 15, 13 and 05. To encrypt W, we compute \(23^3 \mod
        55 = 12\). To encrypt E, we compute \(5^3 \mod 55 = 15\). To encrypt L, we compute \(12^3 \mod 55 = 23\). To encrypt
        C, we compute \(3^3 \mod 55 = 27\). To encrypt O, we compute \(15^3 \mod 55 = 20\). And to encrypt M, we compute
        \(13^3 \mod 55 = 52\). Thus the ciphertext is 12 15 23 27 20 52 15.
\end{proof}

\subsection{Exercise 21}
EXCELLENT

\begin{proof}
        The letters in EXCELLENT translate numerically into 05, 24,  03, 05, 12, 12, 05, 14 and 20. To encrypt E, we compute
        \(5^3 \mod 55 = 15\). To encrypt X, we compute \(24^3 \mod 55 = 19\). To encrypt C, we compute \(3^3 \mod 55 = 27\).
        To encrypt L, we compute \(12^3 \mod 55 = 23\). To encrypt N, we compute \(14^3 \mod 55 = 49\). To encrypt T, we
        compute \(20^3 \mod 55 = 25\). Thus the ciphertext is 15 19 27 15 23 23 15 49 25.
\end{proof}

\subsection{Exercise 22}
13 20 20 09

\begin{proof}
        By Example 8.4.10, the decryption key is 27. Thus the residues modulo 55 for \(13^{27}, 20^{27}\), and \(9^{27}\)
        must be found and then translated into letters of the alphabet. Because \(27 = 16 + 8 + 2 + 1\), we first perform
        the following computations:

        \begin{tabular}{lll}
                \(13^{1} \equiv 13 \pmod{55}\)             & \(20^{1} \equiv 20 \pmod{55}\)              & \(9^{1} \equiv 9 \pmod{55}\)              \\
                \(13^{2} \equiv 4 \pmod{55}\)              & \(20^{2} \equiv 15 \pmod{55}\)              & \(9^{2} \equiv 26 \pmod{55}\)             \\
                \(13^{4} \equiv 4^2 \equiv 16 \pmod{55}\)  & \(20^{4} \equiv 15^2 \equiv 5 \pmod{55}\)   & \(9^{4} \equiv 26^2 \equiv 16
                \pmod{55}\)                                                                                                                          \\
                \(13^{8} \equiv 16^2 \equiv 36 \pmod{55}\) & \(20^{8} \equiv 5^2 \equiv 25 \pmod{55}\)   & \(9^{8} \equiv 16^2 \equiv 36 \pmod{55}\) \\
                \(13^{16} \equiv 36^2 \equiv 31\pmod{55}\) & \(20^{16} \equiv 25^2 \equiv 20 \pmod{55}\) & \(9^{16} \equiv 36^2
                \equiv 31 \pmod{55}\)
        \end{tabular}

        Then we compute

        \begin{center}
                \begin{tabular}{lclcl}
                        \(13^{27} \mod 55\) & = & \((31 \cdot 36 \cdot 4 \cdot 13) \mod 55\)  & = & 7  \\
                        \(20^{27} \mod 55\) & = & \((20 \cdot 25 \cdot 15 \cdot 20) \mod 55\) & = & 15 \\
                        \(9^{27} \mod 55\)  & = & \((31 \cdot 36 \cdot 26 \cdot 9) \mod 55\)  & = & 4
                \end{tabular}
        \end{center}

        Finally, because 7, 15, and 4 translate into letters as G, O, and D, we see that the message is GOOD.
\end{proof}

\subsection{Exercise 23}
08 05 15
\begin{proof}
        By Example 8.4.10, the decryption key is 27. Thus the residues modulo 55 for \(8^{27}, 5^{27}\), and \(15^{27}\)
        must be found and then translated into letters of the alphabet. Because \(27 = 16 + 8 + 2 + 1\), we first perform
        the following computations:

        \begin{tabular}{lll}
                \(8^{1} \equiv 8 \pmod{55}\)               & \(5^{1} \equiv 5 \pmod{55}\)              & \(15^{1} \equiv 15 \pmod{55}\)             \\
                \(8^{2} \equiv 9 \pmod{55}\)               & \(5^{2} \equiv 25 \pmod{55}\)             & \(15^{2} \equiv 5 \pmod{55}\)              \\
                \(8^{4} \equiv 9^2 \equiv 26 \pmod{55}\)   & \(5^{4} \equiv 25^2 \equiv 20 \pmod{55}\) & \(15^{4} \equiv 5^2 \equiv 25
                \pmod{55}\)                                                                                                                         \\
                \(8^{8} \equiv 26^2 \equiv 16 \pmod{55}\)  & \(5^{8} \equiv 20^2 \equiv 15 \pmod{55}\) & \(15^{8} \equiv 25^2 \equiv 20 \pmod{55}\) \\
                \(8^{16} \equiv 16^2 \equiv 36 \pmod{55}\) & \(5^{16} \equiv 15^2 \equiv 5 \pmod{55}\) & \(15^{16} \equiv 20^2
                \equiv 15 \pmod{55}\)
        \end{tabular}

        Then we compute

        \begin{center}
                \begin{tabular}{lclcl}
                        \(8^{27} \mod 55\)  & = & \((36 \cdot 16 \cdot 9 \cdot 8) \mod 55\)  & = & 2  \\
                        \(5^{27} \mod 55\)  & = & \((5 \cdot 15 \cdot 25 \cdot 5) \mod 55\)  & = & 25 \\
                        \(15^{27} \mod 55\) & = & \((15 \cdot 20 \cdot 5 \cdot 15) \mod 55\) & = & 5
                \end{tabular}
        \end{center}

        Finally, because 2, 25, and 5 translate into letters as B, Y, and E, we see that the message is BYE.
\end{proof}

\subsection{Exercise 24}
51 14 49 15
\begin{proof}
        By Example 8.4.10, the decryption key is 27. Thus the residues modulo 55 for \(51^{27},14^{27}\), and \(49^{27}\)
        must be found and then translated into letters of the alphabet (\(15^{27}\) translates to E). Because \(27 = 16 +
        8 + 2 + 1\), we first perform the following computations:

        \begin{tabular}{lll}
                \(51^{1} \equiv 51 \pmod{55}\)              & \(14^{1} \equiv 14 \pmod{55}\)              & \(49^{1} \equiv 49 \pmod{55}\)             \\
                \(51^{2} \equiv 16 \pmod{55}\)              & \(14^{2} \equiv 31 \pmod{55}\)              & \(49^{2} \equiv 36 \pmod{55}\)             \\
                \(51^{4} \equiv 16^2 \equiv 36 \pmod{55}\)  & \(14^{4} \equiv 31^2 \equiv 26 \pmod{55}\)  & \(49^{4} \equiv 36^2
                \equiv 31 \pmod{55}\)                                                                                                                  \\
                \(51^{8} \equiv 36^2 \equiv 31 \pmod{55}\)  & \(14^{8} \equiv 26^2 \equiv 16 \pmod{55}\)  & \(49^{8} \equiv 31^2 \equiv 26 \pmod{55}\) \\
                \(51^{16} \equiv 31^2 \equiv 26 \pmod{55}\) & \(14^{16} \equiv 16^2 \equiv 36 \pmod{55}\) & \(49^{16} \equiv 26^2
                \equiv 16 \pmod{55}\)
        \end{tabular}

        Then we compute

        \begin{center}
                \begin{tabular}{lclcl}
                        \(51^{27} \mod 55\) & = & \((26 \cdot 31 \cdot 16 \cdot 51) \mod 55\) & = & 6  \\
                        \(14^{27} \mod 55\) & = & \((36 \cdot 16 \cdot 31 \cdot 14) \mod 55\) & = & 9  \\
                        \(49^{27} \mod 55\) & = & \((16 \cdot 26 \cdot 36 \cdot 49) \mod 55\) & = & 14
                \end{tabular}
        \end{center}

        Finally, because 6, 9, 24 and 15 translate into letters as F, I, N, and E, we see that the message is FINE.
\end{proof}

\subsection{Exercise 25}
Use Theorem 5.2.2 to prove that if $a$ and $n$ are integers greater than 1 and \(a^n - 1\) is prime, then \(a = 2\) and
$n$ is prime.

\begin{proof}
        By Theorem 5.2.2, using $a$ in place of $r$ and \(n - 1\)
        in place of $n$, we have \(1 + a + \cdots + a^{n-1} =
        \frac{a^n-1}{a-1}\). Multiplying both sides by \(a - 1\) gives \(a^n - 1 = (a-1)(1 + a + a^2 + \cdots + a^{n-1})\).
        So \(a-1 \mid (a^n - 1)\). Since \(a^n-1\) is prime, this forces \(a-1 = 1\) so \(a = 2\).

        So \(2^n-1\) is prime. Argue by contradiction and assume $n$ is not prime. So \(n = ab\) for some integers
        \(1 < a,b < n\). Then by Theorem 5.2.2,
        \[
                2^{n} - 1 = 2^{ab} - 1 = (2^a)^b - 1^b = (2^a - 1)((2^a)^{b-1} + (2^a)^{b-2} + \cdots + 1)
        \]
        where both \(2^a-1 > 1\) and \((2^a)^{b-1} + (2^a)^{b-2} + \cdots + 1 > 1\), so \(2^n - 1\) is not prime, a
        contradiction. So $n$ is prime.
\end{proof}

{\bf \cy In 26 and 27, use the extended euclidean algorithm to find the greatest common divisor of the given numbers
and express it as a linear combination of the two numbers.}

\subsection{Exercise 26}
6664 and 765

\begin{proof}
        {\cy Step 1:} \(6664 = 765 \cdot 8 + 544\), and so \(544 = 6664 - 765 \cdot 8\)

        {\cy Step 2:} \(765 = 544 \cdot 1 + 221\), and so \(221 = 765 - 544\)

        {\cy Step 3:} \(544 = 221 \cdot 2 + 102\), and so \(102 = 544 - 221 \cdot 2\)

        {\cy Step 4:} \(221 = 102 \cdot 2 + 17\), and so \(17 = 221 - 102 \cdot 2\)

        {\cy Step 5:} \(102 = 17 \cdot 6 + 0\)

        Thus gcd\((6664, 765) = 17\) (which is the remainder obtained just before the final division). Substitute back
        through steps $4-1$ to express 17 as a linear combination of 6664 and 765:

        \begin{tabular}{rclcl}
                \(17\) & = & \(221 - 102 \cdot 2\)                           & = & \(\)                                \\
                \(\)   & = & \(221 - (554 - 221 \cdot 2)\)                   & = & \(221 \cdot 5 - 544 \cdot 2\)       \\
                \(\)   & = & \((765 - 544) \cdot 5 - 544 \cdot 2\)           & = & \(765 \cdot 5 - 544 \cdot 7\)       \\
                \(\)   & = & \(765 \cdot 5 - (6664 - 765 \cdot 8) \cdot 7 \) & = & \((-7) \cdot 6664 + 61 \cdot 765\).
        \end{tabular}

        (When you have finished this final step, it is wise to verify that you have not made a mistake by checking that
        the final expression really does equal the greatest common divisor.)
\end{proof}

\subsection{Exercise 27}
4158 and 1568

\begin{proof}
        {\cy Step 1:} \(4158 = 1568 \cdot 2 + 1022\), and so \(1022 = 4158 - 1568 \cdot 2\)

        {\cy Step 2:} \(1568 = 1022 \cdot 1 + 546\), and so \(546 = 1568 - 1022\)

        {\cy Step 3:} \(1022 = 546 \cdot 1 + 476\), and so \(476 = 1022 - 546\)

        {\cy Step 4:} \(546 = 476 \cdot 1 + 70\), and so \(70 = 546 - 476\)

        {\cy Step 5:} \(476 = 70 \cdot 6 + 56\), and so \(56 = 476 - 70 \cdot 6\)

        {\cy Step 6:} \(70 = 56 \cdot 1 + 14\), and so \(14 = 70 - 56\)

        {\cy Step 7:} \(56 = 14 \cdot 4 + 0\).

        Thus gcd\((4158, 1568) = 14\) (which is the remainder obtained just before the final division). Substitute back
        through steps $7-1$ to express 14 as a linear combination of 4158 and 1568:

        \begin{tabular}{rclcl}
                \(14\) & = & \(70 - 56\)                                        & = & \(\)                              \\
                \(\)   & = & \(70 - (476 - 70 \cdot 6)\)                        & = & \(70 \cdot 7 - 476\)              \\
                \(\)   & = & \((546 - 476) \cdot 7 - 476\)                      & = & \(546 \cdot 7 - 476 \cdot 8\)     \\
                \(\)   & = & \(546 \cdot 7 - (1022 - 546) \cdot 8\)             & = & \(546 \cdot 15 - 1022 \cdot 8\)   \\
                \(\)   & = & \((1568 - 1022) \cdot 15 - 1022 \cdot 8\)          & = & \(1568 \cdot 15 - 1022 \cdot 23\) \\
                \(\)   & = & \(1568 \cdot 15 - (4158 - 1568 \cdot 2) \cdot 23\) & = & \(1568 \cdot 61 - 4158 \cdot 23\)
        \end{tabular}
\end{proof}

{\bf \cy Exercises 28 and 29 refer to the following formal version of the extended euclidean algorithm.}

\begin{tcolorbox}[colframe=cyan]
        {\bf \cy Algorithm 8.4.1 Extended Euclidean Algorithm}

        {\bf Input:} $A, B$ {\it [integers with \(A > B > 0\)]}

        \begin{tabbing}
                {\bf Algorithm Body:} \\
                \(a \coloneqq A, b \coloneqq B, s \coloneqq 1, t \coloneqq 0, u \coloneqq 0, v \coloneqq 1\) \\
                {\it [pre-condition: \(a = sA + tB\) and \(b = uA + vB\)]} \\
                {\bf wh}\={\bf ile} (\(b \neq 0\)) {\it [loop invariant: \(a = sA + tB, b = uA + vB, gcd(a,b) = gcd(A,B)\)]} \\
                \> \(r \coloneqq a \mod b, q \coloneqq a \text{ div } b, a \coloneqq b, b \coloneqq r\) \\
                \> \(u_{new} \coloneqq s-uq, v_{new} \coloneqq t-vq\) \\
                \> \(s \coloneqq u, t \coloneqq v, u \coloneqq u_{new}, v \coloneqq v_{new}\) \\
                {\bf end while} \\
                gcd \(\coloneqq a\), {\it [post-condition: \(gcd(A,B) = a = sA + tB\)]} \\
                {\bf Output:} gcd {\it [a positive integer]}, \(s,t\) {\it [integers]}
        \end{tabbing}
\end{tcolorbox}

{\bf \cy In 28 and 29, for the given values of $A$ and $B$, make a table showing the value of \(s, t\), and \(sA + tB\)
before the start of the while loop and after each iteration of the loop.}

\subsection{Exercise 28}
\begin{proof}
        \begin{center}
                \arrayrulecolor{cyan}
                \begin{tabular}{|l|c|c|c|c|c|}
                        \hline
                        \(\bm{a}\)       & 330 & 156  & 18   & 12    & 6     \\
                        \hline
                        \(\bm{b}\)       & 156 & 18   & 12   & 6     & 0     \\
                        \hline
                        \(\bm{r}\)       &     & 18   & 12   & 6     & 0     \\
                        \hline
                        \(\bm{q}\)       &     & 2    & 8    & 1     & 2     \\
                        \hline
                        \(\bm{s}\)       & 1   & 0    & 1    & $-8$  & 9     \\
                        \hline
                        \(\bm{t}\)       & 0   & 1    & $-2$ & 17    & $-19$ \\
                        \hline
                        \(\bm{u}\)       & 0   & 1    & $-8$ & 9     & $-26$ \\
                        \hline
                        \(\bm{v}\)       & 1   & $-2$ & 17   & $-19$ & 55    \\
                        \hline
                        \(\bm{u_{new}}\) &     & 1    & $-8$ & 9     & $-26$ \\
                        \hline
                        \(\bm{v_{new}}\) &     & $-2$ & 17   & $-19$ & 55    \\
                        \hline
                        \(\bm{sA+tB}\)   & 330 & 156  & 18   & 12    & 6     \\
                        \hline
                \end{tabular}
                \arrayrulecolor{black} % change it back!
        \end{center}
\end{proof}

\subsection{Exercise 29}
\begin{proof}
        \begin{center}
                \arrayrulecolor{cyan}
                \begin{tabular}{|l|c|c|c|c|c|c|}
                        \hline
                        \(\bm{a}\)       & 284 & 168  & 116  & 52   & 12    & 4     \\
                        \hline
                        \(\bm{b}\)       & 168 & 116  & 52   & 12   & 4     & 0     \\
                        \hline
                        \(\bm{r}\)       &     & 116  & 52   & 12   & 4     & 0     \\
                        \hline
                        \(\bm{q}\)       &     & 1    & 1    & 2    & 4     & 3     \\
                        \hline
                        \(\bm{s}\)       & 1   & 0    & 1    & $-1$ & 3     & $-13$ \\
                        \hline
                        \(\bm{t}\)       & 0   & 1    & $-1$ & 2    & $-5$  & 22    \\
                        \hline
                        \(\bm{u}\)       & 0   & 1    & $-1$ & 3    & $-13$ & 42    \\
                        \hline
                        \(\bm{v}\)       & 1   & $-1$ & 2    & $-5$ & 22    & $-71$ \\
                        \hline
                        \(\bm{u_{new}}\) &     & 1    & $-1$ & 3    & $-13$ & 42    \\
                        \hline
                        \(\bm{v_{new}}\) &     & $-1$ & 2    & $-5$ & 22    & $-71$ \\
                        \hline
                        \(\bm{sA+tB}\)   & 284 & 168  & 116  & 52   & 12    & 4     \\
                        \hline
                \end{tabular}
                \arrayrulecolor{black} % change it back!
        \end{center}
\end{proof}

\subsection{Exercise 30}
Finish the proof of Theorem 8.4.5 by proving that if \(a, b\), and $c$ are as in the proof, then \(c \mid b\).

\begin{proof}
        By the quotient-remainder theorem \(b = cp + r\) for some integers \(p,r\) with \(0 \leq r < c\). Then \(r = b - cp\)
        and substituting \(c = as+bt\), we get \(r = b - cp = b-(as+bt)p = b - asp - bpt = b(1-pt) - asp\), so $r$ is a
        linear combination of $a$ and $b$. If \(r > 0\) then $r$ would be in $S$, so $r$ would be a smaller element of $S$
        than $c$, a contradiction. Hence $r = 0$ and \(b = cp\) which implies \(c \mid b\).
\end{proof}

\subsection{Exercise 31}
\subsubsection{(a)}
Find an inverse for 210 modulo 13.

\begin{proof}
        {\cy Step 1:} \(210 = 13 \cdot 16 + 2\), and so \(2 = 210 - 16 \cdot 13\)

        {\cy Step 2:} \(13 = 2 \cdot 6 + 1\), and so \(1 = 13 - 2 \cdot 6\)

        {\cy Step 3:} \(6 = 1 \cdot 6 + 0\), and so \(gcd(210, 13) = 1\)

        Substitute back through steps 2 and 1:
        \(1 = 13 - 2 \cdot 6 = 13 - (210 - 16 \cdot 13) \cdot 6 = (-6) \cdot 210 + 97 \cdot 13\)

        Thus \(210 \cdot (-6) \equiv 1 \pmod{13}\), and so \(-6\) is an inverse for 210 modulo 13.
\end{proof}

\subsubsection{(b)}
Find a positive inverse for 210 modulo 13.

\begin{proof}
        Compute \(13 - 6 = 7\). Note that \(7 \equiv -6 \pmod{13}\) because \(7 - (-6) = 13 = 13 \cdot 1\). Thus, by Theorem
        8.4.3(3), \(210 \cdot 7 \equiv 210 \cdot (-6) \pmod{13}\). By part (a), \(-6\) is an inverse for 210 modulo 13, and so
        \(210 \cdot (-6) \equiv 1 \pmod{13}\). It follows, by the symmetric and transitive properties of congruence, that
        \(210 \cdot 7 \equiv 1 \pmod{13}\), and so 7 is a positive inverse for 210 modulo 13.
\end{proof}

\subsubsection{(c)}
Find a positive solution for the congruence \(210x \equiv 8 \pmod{13}\).

\begin{proof}
        This problem can be solved using either the result of part (a) or that of part (b). By part (b) \(210 \cdot 7 \equiv 1
        \pmod{13}\). Multiply both sides by 8 and apply Theorem 8.4.3(3) to obtain \(210\cdot 56 \equiv 8 \pmod{13}\). Thus
        a positive solution for \(210x \equiv 8 \pmod{13}\) is \(x = 56\). Note that the least positive residue corresponding
        to this solution is also a solution. By Theorem 8.4.1, \(56 \equiv 4 \pmod{13}\) because \(56 = 13 \cdot 4 + 4\), and
        so, by Theorem 8.4.3(3), \(210 \cdot 56 \equiv 210 \cdot 4 \equiv 9 \pmod{13}\). This shows that 4 is also a solution
        for the congruence, and because \(0 \leq 4 < 13\), 4 is the least positive solution for the congruence.
\end{proof}

\subsection{Exercise 32}
\subsubsection{(a)}
Find an inverse for 41 modulo 660.

\begin{proof}
        {\cy Step 1:} \(660 = 41 \cdot 16 + 4\), and so \(4 = 660 - 41 \cdot 16\)

        {\cy Step 2:} \(41 = 4 \cdot 10 + 1\), and so \(1 = 41 - 4 \cdot 10\)

        {\cy Step 3:} \(4 = 1 \cdot 4 + 0\), and so \(gcd(660, 41) = 1\)

        Substitute back through steps 2 and 1:
        \(1 = 41 - 4 \cdot 10 = 41 - (660 - 41 \cdot 16) \cdot 10 = (-10) \cdot 660 + 161 \cdot 41\)

        Thus \(41 \cdot 161 \equiv 1 \pmod{660}\), and so \(161\) is an inverse for 41 modulo 660.
\end{proof}

\subsubsection{(b)}
Find the least positive solution for the following congruence: \(41x \equiv 125 \pmod{660}\).

\begin{proof}
        By part (a) \(41 \cdot 161 \equiv 1 \pmod{660}\). Multiply both sides by 125 and apply Theorem 8.4.3(3) to obtain
        \(41 \cdot 20125 \equiv 125 \pmod{660}\). Thus a positive solution for \(41x \equiv 125 \pmod{660}\) is
        \(x = 20125\). Note that the least positive residue corresponding to this solution is also a solution. By
        Theorem 8.4.1, \(20125 \equiv 325 \pmod{660}\) because \(20125 = 660 \cdot 30 + 325\), and so, by Theorem
        8.4.3(3), \(41 \cdot 20125 \equiv 41 \cdot 325 \equiv 125 \pmod{660}\). This shows that 325 is also a solution for
        the congruence, and because \(0 \leq 325 < 660\), 325 is the least positive solution for the congruence.
\end{proof}

\subsection{Exercise 33}
Use Theorem 8.4.5 to prove that for all integers \(a, b\), and $c$, if \(gcd(a, b) = 1\) and \(a \mid c\) and \(b \mid
c\), then \(ab \mid c\).

\begin{proof}
        Since \(gcd(a,b) = 1\), by Theorem 8.4.5 \(1 = as+bt\) for some integers \(s,t\).

        Since \(a \mid c, au = c\) for some integer $u$. Since \(b \mid c, bv = c\) for some integer $v$.

        Since \(1 = as+bt\), multiplying by $c$ we get \(c = cas + cbt = (bv)as + (au)bt = ab(sv + tu)\).

        Notice that \(sv+tu\) is an integer, thus \(ab \mid c\).
\end{proof}

\subsection{Exercise 34}
Give a counterexample to show that the statement of exercise 33 is false if the hypothesis that \(gcd(a, b)=1\)
is removed.

\begin{proof}
        Let \(a = 4, b = 6, c = 12\). Then \(gcd(a,b) = gcd(4,6) = 2 \neq 1\). And \(a \mid c\) because \(12 = 4 \cdot 3\) and
        \(b \mid c\) because \(12 = 6 \cdot 2\). But \(ab \nmid c\) because \(4 \cdot 6 = 24 \nmid 12\).
\end{proof}

\subsection{Exercise 35}
Corollary 8.4.7 guarantees the existence of an inverse modulo $n$ for an integer $a$ when $a$ and $n$ are
relatively prime. Use Euclid’s lemma to prove that the inverse is unique modulo $n$. In other words, show that if
$s$ and $t$ are any two integers whose product with $a$ is congruent to 1 modulo $n$, then $s$ and $t$ are congruent
to each other modulo $n$.

\begin{proof}
        Let $a$ be any integer and let $n$ be any positive integer, and suppose $s$ and $t$ are any inverses for a modulo $n$. Thus \(as \equiv 1 \pmod n\) and \(at \equiv 1 \pmod n\). Note that \(ast = (as) \cdot t = (at) \cdot s\). By Theorem
        8.4.3(3), \((as) \cdot t \equiv t \pmod n\) and \((at) \cdot s \equiv s \pmod n\). Thus, by symmetry and
        transitivity of congruence modulo $n$, \(s \equiv t \pmod n\). Because $s$ and $t$ were chosen arbitrarily, we
        conclude that any two inverses for a are congruent modulo $n$.
\end{proof}

{\bf \cy In 36, 37, 39, and 40, use the RSA cipher with public key \(n = 713 = 23 \cdot 31\) and \(e = 43\). In 36
and 37, encode the messages into their numeric equivalents and encrypt them. In 39 and 40, decrypt the given
ciphertext and find the original messages.}

\subsection{Exercise 36}
HELP
\begin{proof}
        The numeric equivalents of H, E, L, and P are 08, 05, 12, and 16. To encrypt these letters, the following quantities
        must be computed: \(8^{43} \mod 713, 5^{43} \mod 713, 12^{43} \mod 713\), and \(16^{43} \mod 713\). We use the
        fact that 43 = 32 + 8 + 2 + 1.

        \begin{tabular}{lclcll}
                \(8^1\)    & = & \(8\)     & = & \(\)    & \(\mod 713\) \\
                \(8^2\)    & = & \(64\)    & = & \(\)    & \(\mod 713\) \\
                \(8^4\)    & = & \(64^2\)  & = & \(531\) & \(\mod 713\) \\
                \(8^8\)    & = & \(531^2\) & = & \(326\) & \(\mod 713\) \\
                \(8^{16}\) & = & \(326^2\) & = & \(39\)  & \(\mod 713\) \\
                \(8^{32}\) & = & \(39^2\)  & = & \(95\)  & \(\mod 713\)
        \end{tabular}

        Thus the ciphertext is \(8^{43} \mod 713 = (95 \cdot 326 \cdot 64 \cdot 8) \mod 713 = 233\).

        \begin{tabular}{lclcll}
                \(5^1\)    & = & \(5\)     & = & \(\)    & \(\mod 713\) \\
                \(5^2\)    & = & \(25\)    & = & \(\)    & \(\mod 713\) \\
                \(5^4\)    & = & \(625\)   & = & \(\)    & \(\mod 713\) \\
                \(5^8\)    & = & \(625^2\) & = & \(614\) & \(\mod 713\) \\
                \(5^{16}\) & = & \(614^2\) & = & \(532\) & \(\mod 713\) \\
                \(5^{32}\) & = & \(532^2\) & = & \(676\) & \(\mod 713\)
        \end{tabular}

        Thus the ciphertext is \(5^{43} \mod 713 = (676 \cdot 614 \cdot 25 \cdot 5) \mod 713 = 129\).

        \begin{tabular}{lclcll}
                \(12^1\)    & = & \(12\)    & = & \(\)    & \(\mod 713\) \\
                \(12^2\)    & = & \(144\)   & = & \(\)    & \(\mod 713\) \\
                \(12^4\)    & = & \(144^2\) & = & \(59\)  & \(\mod 713\) \\
                \(12^8\)    & = & \(59^2\)  & = & \(629\) & \(\mod 713\) \\
                \(12^{16}\) & = & \(629^2\) & = & \(639\) & \(\mod 713\) \\
                \(12^{32}\) & = & \(639^2\) & = & \(485\) & \(\mod 713\)
        \end{tabular}

        Thus the ciphertext is \(12^{43} \mod 713 = (485 \cdot 629 \cdot 144 \cdot 12) \mod 713 = 48\).

        \begin{tabular}{lclcll}
                \(16^1\)    & = & \(16\)    & = & \(\)    & \(\mod 713\) \\
                \(16^2\)    & = & \(256\)   & = & \(\)    & \(\mod 713\) \\
                \(16^4\)    & = & \(256^2\) & = & \(653\) & \(\mod 713\) \\
                \(16^8\)    & = & \(653^2\) & = & \(35\)  & \(\mod 713\) \\
                \(16^{16}\) & = & \(35^2\)  & = & \(512\) & \(\mod 713\) \\
                \(16^{32}\) & = & \(512^2\) & = & \(473\) & \(\mod 713\)
        \end{tabular}

        Thus the ciphertext is \(12^{43} \mod 713 = (473 \cdot 35 \cdot 256 \cdot 16) \mod 713 = 128\).

        Therefore, the encrypted message is 233 129 048 128. (Again, note that in practice, individual letters of the
        alphabet are grouped together in blocks during encryption so that deciphering cannot be accomplished through
        knowledge of frequency patterns of letters or words. We kept them separate so that the numbers in the computations
        would be smaller and easier to work with.)
\end{proof}

\subsection{Exercise 37}
COME
\begin{proof}
        The numeric equivalents of C, O, M, and E are 03, 15, 13, and 05. The letter E was encrypted in Exercise 36 as 129.
        To encrypt the other letters, the following quantities must be computed: \(3^{43} \mod 713, 15^{43} \mod 713\), and
        \(13^{43} \mod 713\). We use the fact that 43 = 32 + 8 + 2 + 1.

        \begin{tabular}{lclcll}
                \(3^1\)    & = & \(3\)     & = & \(\)    & \(\mod 713\) \\
                \(3^2\)    & = & \(9\)     & = & \(\)    & \(\mod 713\) \\
                \(3^4\)    & = & \(9^2\)   & = & \(81\)  & \(\mod 713\) \\
                \(3^8\)    & = & \(81^2\)  & = & \(144\) & \(\mod 713\) \\
                \(3^{16}\) & = & \(144^2\) & = & \(59\)  & \(\mod 713\) \\
                \(3^{32}\) & = & \(59^2\)  & = & \(629\) & \(\mod 713\)
        \end{tabular}

        Thus the ciphertext is \(3^{43} \mod 713 = (629 \cdot 144 \cdot 9 \cdot 3) \mod 713 = 675\).

        \begin{tabular}{lclcll}
                \(15^1\)    & = & \(15\)    & = & \(\)    & \(\mod 713\) \\
                \(15^2\)    & = & \(225\)   & = & \(\)    & \(\mod 713\) \\
                \(15^4\)    & = & \(225^2\) & = & \(2\)   & \(\mod 713\) \\
                \(15^8\)    & = & \(2^2\)   & = & \(4\)   & \(\mod 713\) \\
                \(15^{16}\) & = & \(4^2\)   & = & \(16\)  & \(\mod 713\) \\
                \(15^{32}\) & = & \(16^2\)  & = & \(256\) & \(\mod 713\)
        \end{tabular}

        Thus the ciphertext is \(15^{43} \mod 713 = (256 \cdot 4 \cdot 225 \cdot 15) \mod 713 = 89\).

        \begin{tabular}{lclcll}
                \(13^1\) & = & \(13\)    & = & \(\)   & \(\mod 713\) \\
                \(13^2\) & = & \(169\)   & = & \(\)   & \(\mod 713\) \\
                \(13^4\) & = & \(169^2\) & = & \(41\) & \(\mod 713\) \\
        \end{tabular}

        \begin{tabular}{lclcll}
                \(13^8\)    & = & \(41^2\)  & = & \(255\) & \(\mod 713\) \\
                \(13^{16}\) & = & \(255^2\) & = & \(142\) & \(\mod 713\) \\
                \(13^{32}\) & = & \(142^2\) & = & \(200\) & \(\mod 713\)
        \end{tabular}

        Thus the ciphertext is \(13^{43} \mod 713 = (200 \cdot 255 \cdot 169 \cdot 13) \mod 713 = 476\).

        Therefore, the encrypted message is 675 089 476 129.
\end{proof}

\subsection{Exercise 38}
Find the least positive inverse for 43 modulo 660.

\begin{proof}
        {\cy Step 1:} \(660 = 43 \cdot 15 + 15\), so \(15 = 660 - 43 \cdot 15\)

        {\cy Step 2:} \(43 = 15 \cdot 2 + 13\), so \(13 = 43 - 15 \cdot 2\)

        {\cy Step 3:} \(15 = 13 \cdot 1 + 2\), so \(2 = 15 - 13 \cdot 1\)

        {\cy Step 4:} \(13 = 2 \cdot 6 + 1\), so \(1 = 13 - 2 \cdot 6\)

        {\cy Step 5:} \(2 = 1 \cdot 2 + 0\), so \(gcd(660, 43) = 1\).

        Stepping back through steps 4-1 we get: \(1 = 13 - 2 \cdot 6 = 13 - (15 - 13) \cdot 6 = 7 \cdot 13 - 6 \cdot 15\)

        \(= 7 \cdot (43 - 15 \cdot 2) - 6 \cdot 15 = 7 \cdot 43 - 20 \cdot 15\)

        \(= 7 \cdot 43 - 20 \cdot (660 - 43 \cdot 15) = 307 \cdot 43 - 20 \cdot 660\)

        So \(307 \cdot 43 \equiv 1 \pmod{660}\) and since \(0 \leq 307 < 660\), 307 is the least positive inverse of 43.
\end{proof}

\subsection{Exercise 39}
675 089 089 048

\begin{proof}
        By Exercise 37, 675 decrypts to C and 089 decrypts to O. By Exercise 36, 048 decrypts to L. So the plaintext is COOL.
\end{proof}

\subsection{Exercise 40}
028 018 675 129

\begin{proof}
        Similarly, 675 = C, 129 = E. We must decrypt 028 and 018. By exercise 38, the decryption key, $d$, is 307. So we must
        compute \(28^{307} \mod 713\) and \(18^{307} \mod 713\). We use the fact 307 = 256 + 32 + 16 + 2 + 1.

        \begin{tabular}{lclcll}
                \(28^1\)     & = & \(28\)    & = & \(\)    & \(\mod 713\) \\
                \(28^2\)     & = & \(784\)   & = & \(71\)  & \(\mod 713\) \\
                \(28^4\)     & = & \(71^2\)  & = & \(50\)  & \(\mod 713\) \\
                \(28^8\)     & = & \(50^2\)  & = & \(361\) & \(\mod 713\) \\
                \(28^{16}\)  & = & \(361^2\) & = & \(555\) & \(\mod 713\) \\
                \(28^{32}\)  & = & \(555^2\) & = & \(9\)   & \(\mod 713\) \\
                \(28^{64}\)  & = & \(9^2\)   & = & \(81\)  & \(\mod 713\) \\
                \(28^{128}\) & = & \(81^2\)  & = & \(144\) & \(\mod 713\) \\
                \(28^{256}\) & = & \(144^2\) & = & \(59\)  & \(\mod 713\) \\
        \end{tabular}

        Thus the ciphertext is \(28^{307} \mod 713 = (59 \cdot 9 \cdot 555 \cdot 71 \cdot 28) \mod 713 = 14\), which is N.

        \begin{tabular}{lclcll}
                \(18^1\)     & = & \(18\)    & = & \(\)    & \(\mod 713\) \\
                \(18^2\)     & = & \(324\)   & = & \(\)    & \(\mod 713\) \\
                \(18^4\)     & = & \(324^2\) & = & \(165\) & \(\mod 713\) \\
                \(18^8\)     & = & \(165^2\) & = & \(131\) & \(\mod 713\) \\
                \(18^{16}\)  & = & \(131^2\) & = & \(49\)  & \(\mod 713\) \\
                \(18^{32}\)  & = & \(49^2\)  & = & \(262\) & \(\mod 713\) \\
                \(18^{64}\)  & = & \(262^2\) & = & \(196\) & \(\mod 713\) \\
                \(18^{128}\) & = & \(196^2\) & = & \(627\) & \(\mod 713\) \\
                \(18^{256}\) & = & \(627^2\) & = & \(266\) & \(\mod 713\) \\
        \end{tabular}

        Thus the ciphertext is \(18^{307} \mod 713 = (266 \cdot 262 \cdot 49 \cdot 324 \cdot 18) \mod 713 = 9\), which is I.

        So the plaintext is NICE.
\end{proof}

\subsection{Exercise 41}
\subsubsection{(a)}
Use mathematical induction and Euclid’s lemma to prove that for every positive integer $s$, if $p$ and \(q_1, q_2,
\ldots, q_s\) are prime numbers and \(p \mid q_1q_2 \cdots q_s\), then \(p=q_i\) for some $i$ with \(1\leq i \leq s\).

\begin{proof}
        Let $P(s)$ be the statement: ``if $p$ and \(q_1, q_2, \ldots, q_s\) are prime numbers and \(p \mid q_1q_2 \cdots
        q_s\), then \(p=q_i\) for some $i$ with \(1\leq i \leq s\).''

        {\bf Show \(P(1)\) is true:} Assume \(p \mid q_1\). Since $p$ and $q_1$ are both prime, \(p = q_1\) as needed.

                {\bf Show that for any integer \(s \geq 1\) if \(P(s)\) is true then \(P(s+1)\) is true:} Assume \(P(s)\), and assume
        $p$ and \(q_1, q_2, \ldots, q_{s+1}\) are prime numbers and \(p \mid q_1q_2 \cdots q_s q_{s+1}\).

        Let \(a = q_1q_2 \cdots q_s\). Then \(p \mid a q_{s+1}\), and either \(p = q_{s+1}\), or by Euclid’s lemma
        \(p \mid q_1q_2 \cdots q_s\), in which case \(p = q_i\) for some \(1 \leq i \leq s\) by the inductive hypothesis.
\end{proof}

\subsubsection{(b)}
The uniqueness part of the unique factorization theorem for the integers says that given any integer $n$, if \(n = p_1
p_2 \cdots p_r = q_1q_2 \cdots q_s\) for some positive integers $r$ and $s$ and prime numbers \(p1 \leq p2 \leq
\cdots \leq p_r\) and \(q1 \leq q_2 \leq \cdots \leq q_s\), then \(r = s\) and \(p_i = q_i\) for every integer $i$ with
\(1 \leq i \leq r\).

Use the result of part (a) to fill in the details of the following sketch of a proof: Suppose that $n$ is an integer
with two different prime factorizations: \(n = p_1 p_2 \cdots p_t = q_1 q_2 \cdots q_u\). All the prime factors
that appear on both sides can be canceled (as many times as they appear on both sides) to arrive at the situation where
\(p_1 p_2 \cdots p_r = q_1 q_2 \cdots q_s, p_1 \leq p_2 \leq \cdots \leq p_r, q_1 \leq q_2 \leq \cdots \leq q_s\),
and \(p_i \neq q_j\) for any integers $i$ and $j$. Then use part (a) to deduce a contradiction, and conclude that the
prime factorization of $n$ is unique except, possibly, for the order in which the prime factors are written.

\begin{proof}
        Assume \(p_1 p_2 \cdots p_r = q_1 q_2 \cdots q_s, p_1 \leq p_2 \leq \cdots \leq p_r, q_1 \leq q_2 \leq \cdots \leq
        q_s\), and \(p_i \neq q_j\) for any integers $i$ and $j$.

        Notice \(p_1 \mid q_1 q_2 \cdots q_s\). Then by part (a), \(p_1 = q_i\) for some \(1 \leq i \leq s\). This contradicts
        the fact that \(p_1 \neq q_j\) for any integer and $j$.

                {\it Conclusion:} Our supposition was false, so $n$ does not have two different prime factorizations. Thus the
        prime factorization of $n$ is unique except, possibly, for the order in which the prime factors are written.
\end{proof}

\subsection{Exercise 42}
According to Fermat’s little theorem, if $p$ is a prime number and $a$ and $p$ are relatively prime, then \(a^{p-1}
\equiv 1 \pmod p\). Verify that this theorem gives correct results for the following:

\subsubsection{(a)}
\(a = 15\) and \(p = 7\)

\begin{proof}
        \(a^{p-1} = 15^6 = 11390625 \equiv 1 \pmod 7\) because \(11390625 - 1 = 7 \cdot 1627232\).
\end{proof}

\subsubsection{(b)}
\(a = 8\) and \(p = 11\)

\begin{proof}
        \(a^{p-1} = 8^{10} = 1073741824 \equiv 1 \pmod 11\) because \(1073741824 - 1 = 11 \cdot 97612893\).
\end{proof}

\subsection{Exercise 43}
Fermat’s little theorem can be used to show that a number is not prime by finding a number a relatively prime to $p$
with the property that \(a^{p-1} \not\equiv 1 \pmod p\). However, it cannot be used to show that a number is prime.
Find an example to illustrate this fact. That is, find integers $a$ and $p$ such that $a$ and $p$ are relatively
prime and \(a^{p-1} \equiv 1 \pmod p\) but $p$ is not prime.

\begin{proof}
        Let \(a = 5, p = 4\). Then $a$ and $p$ are relatively prime, and \(a^{p-1} = 5^{4-1}=5^3 = 125 \equiv 1 \pmod 4\)
        because \(125 - 1 = 124 = 4 \cdot 31\), but $p$ is not prime because \(4 = 2 \cdot 2\).
\end{proof}

\section{Exercise Set 8.5}

\subsection{Exercise 1}
Each of the following is a relation on \(\{0, 1, 2, 3\}\). Draw directed graphs for each relation, and indicate which
relations are antisymmetric.

\subsubsection{(a)}
\(R_1 = \{(0, 0), (0, 2), (1, 0), (1, 3), (2, 2), (3, 0), (3, 1)\}\)

\begin{proof}
        \begin{figure}[ht!]
                \centering
                \includegraphics[scale=0.5]{../images/8.5.1.a.png}
        \end{figure}
\end{proof}

\subsubsection{(b)}
\(R_2 = \{(0, 1), (0, 2), (1, 1), (1, 2), (1, 3), (2, 2), (3, 2)\}\)

\begin{proof}
        \begin{figure}[ht!]
                \centering
                \includegraphics[scale=0.5]{../images/8.5.1.b.png}
        \end{figure}
\end{proof}

\subsubsection{(c)}
\(R_3 = \{(0, 0), (0, 3), (1, 0), (1, 3), (2, 2), (3, 3),(3, 2)\}\)

\begin{proof}
        \(R_3\) is antisymmetric: there are no cases where \(a \,R\, b\) and \(b \,R\, a\) and \(a \neq b\).
        \begin{figure}[ht!]
                \centering
                \includegraphics[scale=0.5]{../images/8.5.1.c.png}
        \end{figure}
\end{proof}

\subsubsection{(d)}
\(R_4 = \{(0, 0), (1, 0), (1, 2), (1, 3), (2, 0), (2, 1),(3, 2), (3, 0)\}\)

\begin{proof}
        \(R_4\) is not antisymmetric: \(1 \,R\, 2\) and \(2 \,R\, 1\) and \(1 \neq 2\).
        \begin{figure}[ht!]
                \centering
                \includegraphics[scale=0.5]{../images/8.5.1.d.png}
        \end{figure}
\end{proof}

\subsection{Exercise 2}
Let $P$ be the set of all people in the world and define a relation $R$ on $P$ as follows: For all people $x$ and $y$,
\(x \,R\, y \iff x\) is no older than $y$. Is $R$ antisymmetric? Prove or give a counterexample.

\begin{proof}
        $R$ is not antisymmetric. Let $x$ and $y$ be any two distinct people of the same age. Then \(x \,R\, y\) and
        \(y \,R\, x\) but \(x \neq y\).
\end{proof}

\subsection{Exercise 3}
Let $S$ be the set of all strings of $a$’s and $b$’s. Define a relation $R$ on $S$ as follows: For every \(s, t
\in S, s \,R\, t \iff L(s) \leq L(t)\), where \(L(x)\) denotes the length of a string $x$. Is $R$ antisymmetric?
Prove or give a counterexample.

\begin{proof}
        $R$ is not antisymmetric. Let \(s = a, t = b\). Then \(L(s) = L(t) = 1\) so both \(L(s) \leq L(t)\) and \(L(t) \leq
        L(s)\), so \(s \,R\, t\) and \(t \,R\,s\) but \(s \neq t\).
\end{proof}

\subsection{Exercise 4}
Let $R$ be the “less than” relation on $\R$, the set of all real numbers: For every \(x, y \in \R, x \,R\, y \iff x <
y\). Is $R$ antisymmetric? Prove or give a counterexample.

\begin{proof}
        Yes. If \(x \neq y\) are two real numbers and \(x < y\) then \(y \nless x\).
\end{proof}

\subsection{Exercise 5}
Let $\R$ be the set of all real numbers and define a relation $R$ on \(\R \times \R\) as follows: For every
\((a, b)\) and \((c, d)\) in \(\R \times R, (a, b) \,R\, (c, d) \iff\) either \(a < c\) or both \(a = c\) and
\(b \leq d\). Is $R$ a partial order relation? Prove or give a counterexample.

\begin{proof}
        $R$ is a partial order relation.

                {\bf $R$ is reflexive:} Suppose \((a, b) \in \R \times \R\). Then \((a, b) \,R\, (a, b)\) because \(a = a\) and
        \(b \leq b\).

                {\bf $R$ is antisymmetric:} Suppose \((a, b)\) and \((c, d)\) are ordered pairs of real numbers such that \((a, b) \,R
        \, (c, d)\) and \((c, d) \,R\, (a, b)\). Then either \(a < c\) or both \(a = c\) and \(b \leq d\) and either \(c < a\)
        or both \(c = a\) and \(d \leq b\). Thus \(a \leq c\) and \(c \leq a\) and so \(a = c\). Consequently, \(b \leq d\)
        and \(d \leq b\) and so \(b = d\). Hence \((a, b) = (c, d)\).

                {\bf $R$ is transitive:} Suppose \((a, b), (c, d)\), and \((e, f)\) are ordered pairs of real numbers such that
        \((a, b) \,R\, (c, d)\) and \((c, d) \,R\, (e, f)\). Then either \(a < c\) or both \(a = c\) and \(b \leq d\) and
        either \(c < e\) or both \(c = e\) and \(d \leq f\).

        It follows that one of the following cases must occur.

                {\bf Case 1 (\(a < c\) and \(c < e\)):} Then by transitivity of \(<\), \(a < e\), and so \((a, b) \,R\,
        (e, f)\) by definition of \(R\).

                {\bf Case 2 (\(a < c\) and \(c = e\)):} Then by substitution, \(a < e\), and so \((a, b) \,R\, (e, f)\) by
        definition of \(R\).

                {\bf Case 3 (\(a = c\) and \(c < e\)):} Then by substitution, \(a < e\), and so \((a, b) \,R\, (e, f)\) by
        definition of \(R\).

                {\bf Case 4 (\(a = c\) and \(c = e\)):} Then by definition of \(R\), \(b \leq d\) and \(d \leq f\), and so by
        transitivity of \(\leq\), \(b \leq f\). Hence \(a = e\) and \(b \leq f\), and so \((a, b) \,R\, (e, f)\) by definition
        of \(R\).

        In each case, \((a, b) \,R\, (e, f)\). Therefore, \(R\) is transitive. Since \(R\) is reflexive, antisymmetric, and
        transitive, \(R\) is a partial order relation.
\end{proof}

\subsection{Exercise 6}
Let $P$ be the set of all people who have ever lived and define a relation $R$ on $P$ as follows: For every \(r, s
\in P, r \,R\, s \iff r\) is an ancestor of \(s\) or \(r = s\). Is $R$ a partial order relation? Prove or give a
counterexample.

\begin{proof}
        $R$ is a partial order relation.

                {\bf $R$ is reflexive:} For all \(r \in P, r = r\) therefore \(r \,R\, r\).

                {\bf $R$ is antisymmetric:} Assume \(r \,R\, s\) and \(s \,R\, r\). {\it [We want to show \(r = s\).]}

        Since \(r \,R\, s\), either \(r=s\) or $r$ is an ancestor of $s$. If \(r=s\) then there is nothing to show. So assume
        $r$ is an ancestor of $s$.

        Since \(s \,R\, r\), either \(s=r\) or $s$ is an ancestor of $r$. If $s=r$ then there is nothing to show. So assume
        $s$ is an ancestor of $r$.

        So both $r$ is an ancestor of $s$ and $s$ is an ancestor of $r$. This is not possible unless \(r=s\).

                {\bf $R$ is transitive:} Assume \(r \,R\, s\) and \(s \,R\, t\). {\it [We want to show \(r \,R\, t\).]} Either $r=s$ or
        $r$ is an ancestor of $s$.

                {\bf Case 1 (\(r=s\)):} Either $s=t$ or $s$ is an ancestor of $t$. If $s=t$ then $r=s=t$ so \(r \,R\, t\). If $s$ is
        an ancestor of $t$, then since $r=s$, $r$ is an ancestor of $t$, thus \(r \,R\, t\) again.

                {\bf Case 2 ($r$ is an ancestor of $s$):} Either $s=t$ or $s$ is an ancestor of $t$. If $s=t$ then since $r$ is an
        ancestor of $s$, $r$ is an ancestor of $t$, thus \(r \,R\, t\). If $s$ is an ancestor of $t$, then by transitivity of
        ancestry, $r$ is an ancestor of $t$, so \(r \,R\, t\) again.
\end{proof}

\subsection{Exercise 7}
Define a relation $R$ on $\Z$, the set of all integers as follows: For every \(m, n \in \Z, m \,R\, n \iff\) every
prime factor of $m$ is a prime factor of $n$. Is $R$ a partial order relation? Prove or give a counterexample.

\begin{proof}
        $R$ is not antisymmetric: let \(m = 2, n = 4\). Then \(m \,R\, n\) and \(n \,R\, m\) but \(m \neq n\).
\end{proof}

\subsection{Exercise 8}
Define a relation $R$ on $\Z$, the set of all integers as follows: For every \(m, n \in \Z, m \,R\, n \iff m + n\)
is even. Is $R$ a partial order relation? Prove or give a counterexample.

\begin{proof}
        $R$ is not a partial order relation because $R$ is not antisymmetric.

        \underline{Counterexample:} \(1 \,R\, 3\) (because \(1 + 3\) is even) and \(3 \,R\, 1\) (because \(3 + 1\) is even)
        but \(1 \neq 3\).
\end{proof}

\subsection{Exercise 9}
Define a relation $R$ on $\R$, the set of all real numbers as follows: For every \(x,y\in\R,x\,R\,y\iff x^2\leq y^2\).
Is $R$ a partial order relation? Prove or give a counterexample.

\begin{proof}
        $R$ is not antisymmetric: let \(x = -2, y = 2\). Then \(x^2 = 4 = y^2\) so both \(x \,R\, y\) and \(y \,R\, x\) but
        \(x \neq y\).
\end{proof}

\subsection{Exercise 10}
Suppose $R$ and $S$ are antisymmetric relations on a set $A$. Must \(R \cup S\) also be antisymmetric? Explain.

\begin{proof}
        No. \underline{Counterexample:} Define relations $R$ and $S$ on the set \(\{1, 2\}\) as follows: \(R = \{(1, 2)\}\)
        and \(S = \{(2, 1)\}\). Then both $R$ and $S$ are antisymmetric, but \(R \cup S = \{(1, 2), (2, 1)\}\) is not antisymmetric because \((1, 2) \in R \cup S\) and \((2, 1)
        \in R \cup S\) but \(1 \neq 2\).
\end{proof}

\subsection{Exercise 11}
Let \(A = \{a, b\}\), and suppose \(A\) has the partial order relation \(R = \{(a, a),(a, b), (b, b)\}\).
Let \(S\) be the set of all strings in \(a\)’s and \(b\)’s and let be the corresponding lexicographic order on \(S\).
Indicate which of the following statements are true, and for each true statement cite as a reason part (1), (2), or
(3) of the definition of lexicographic order given in Theorem 8.5.1.

\subsubsection{(a)}
\(aab \preceq aaba\)

\begin{proof}
        True, by (1).
\end{proof}

\subsubsection{(b)}
\(bbab \preceq bba\)

\begin{proof}
        False. \(bba \preceq bbab\) by (1).
\end{proof}

\subsubsection{(c)}
\(\lambda \preceq aba\)

\begin{proof}
        True, by (3).
\end{proof}

\subsubsection{(d)}
\(aba \preceq abb\)

\begin{proof}
        True, by (2).
\end{proof}

\subsubsection{(e)}
\(bbab \preceq bbaa\)

\begin{proof}
        False. \(bbaa \preceq bbab\) by (2).
\end{proof}

\subsubsection{(f)}
\(ababa \preceq ababaa\)

\begin{proof}
        True, by (1).
\end{proof}

\subsubsection{(g)}
\(bbaba \preceq bbabb\)

\begin{proof}
        True, by (2).
\end{proof}

\subsection{Exercise 12}
Prove Theorem 8.5.1.

        {\it Note:} The errata for the book has a correction for this theorem:

{\bf 550 - Theorem 8.5.1} In \#2 (line 7), change each $m$ to a $k$. (5 changes)

But the errata does not mention what $k$ is. The problem with (2) in the Theorem is this: it requires $s$ and $t$ to
be the same {\it only except for the LAST character of $s$.} This does not work, since it is possible for $t$ to
be much shorter than $s$ and still be bigger: \(aaaaaaaaa \preceq z\) for example.

So here $k$ should refer to any length up to $m$, so that, if the first $k-1$ characters of $s$ and $t$ are the same,
and they differ at {\it some point} where $t$ has a ``bigger'' character, then $t$ is bigger: \(aaaaaaaaa
\preceq z\) for example. Here $k = 1$ because $s$ and $t$ have their first $1-1 = 0$ characters the same.

Notice that $k$ has to be less than or equal to the lengths of both $s$ and $t$.

So we can rephrase (2) as follows:

If there exists an integer $k$ such that \(1 \leq k \leq min(m,n)\), the first $k-1$ characters of $s$ and $t$ are
the same, \(s_k \,R\, t_k\) and \(s_k \neq t_k\), then \(s \preceq t\).

This makes the proofs below {\it a lot more} complicated. But it has to be done! Following these proofs can be very
difficult; when you have trouble, always write down some examples like the ones above.

\begin{proof}
        Let $m,n,o$ be the lengths of $s,t,u$ and let \(s_i, t_i, u_i\) be the $i$th characters of $s$, $t$ and $u$,
        respectively, for each \(1 \leq i \leq max(m, n, o)\). (Note that $m,n$ and $o$ could be 0, in which case \(s_i\), \(t_i\) and \(u_i\) are not defined.)

        {\bf \(\bm{\preceq}\) is reflexive:} For any string $s$, \(m \leq m\), and the first $m$ characters of $s$ and $s$
        are the same. Therefore by (1), \(s \preceq s\).

                {\bf \(\bm{\preceq}\) is antisymmetric:} Assume \(s \preceq t\) and \(t \preceq s\). {\it [We want to show $s=t$.]}

                {\bf Case 1: \(\bm{s \preceq t}\) because of (1).} Then \(m \leq n\) where both $m$ and $n$ are positive, and the first
        $m$ characters of $s$ and $t$ are the same.

        Now consider the fact \(t \preceq s\).

        This cannot be because of (3) since $n$ is positive and therefore \(t \neq \lambda\).

        It cannot be because of (2) either, because there is no integer $k$ such that \(1 \leq k \leq min(m,n)\) and \(s_k \neq t_k\) since the first $m$ characters of $s$ and $t$ are the same and the length of $s$ is $m$.

        So it must be because of (1). Then \(n \leq m\), so \(m = n\). Since the first $m$ characters of $s$ and $t$ are the
        same, and they both have length $m$, \(s = t\).

                {\bf Case 2: \(\bm{s \preceq t}\) because of (2).} So there is an integer $k$ such that \(1 \leq k \leq min(m,n)\), the
        first $k-1$ characters of $s$ and $t$ are the same, \(s_k \neq t_k\) and \(s_k \,R\, t_k\).

        Now consider the fact \(t \preceq s\).

        This cannot be because of (3) since $n$ is positive and therefore \(t \neq \lambda\).

        It cannot be because of (1) either, since \(s_k \neq t_k\).

        So it must be because of (2). So there is an integer $j$ such that \(1 \leq j \leq min(n,m)\), the first $j-1$
        characters of $t$ and $s$ are the same, \(t_j \neq s_j\), and \(t_j \,R\, s_j\).

        Since the first $k-1$ characters of $s$ and $t$ are the same, the first $j-1$ characters of $s$ and $t$ are the
        same, the $j$th characters are different, and the $k$th characters are different, all of this forces $j=k$. So
        \(s_k \,R\, t_k\) and \(t_k \,R\, s_k\). Since $R$ is antisymmetric, \(s_k = t_k\), a contradiction.
        So this case is not possible.

                {\bf Case 3: \(\bm{s \preceq t}\) because of (3).} Then \(s = \lambda\). So \(t \preceq \lambda\). Since (1) and (2)
        require positive string lengths for both strings, this must be because of (3) as well. Thus \(t=\lambda\) and \(s=t\).

                {\bf \(\bm{\preceq}\) is transitive:} Assume \(s\preceq t\) and \(t\preceq u\). {\it[We want to show \(s\preceq u\).]}

                {\bf Case 1: \(\bm{s \preceq t}\) because of (1).} Then \(m \leq n\) where both $m$ and $n$ are positive, and the first
        $m$ characters of $s$ and $t$ are the same.

        Now consider the fact \(t \preceq u\).

        This cannot be because of (3) since $n$ is positive and therefore \(t \neq \lambda\).

        If it is because of (1) then \(n \leq o\) and the first $n$ characters of $t$ and $u$ are the same. Then the first $m$
        characters of $s$ and $u$ are the same, since \(m \leq n \leq o\), so \(s \preceq u\) by (1).

        If it is because of (2) then there is an integer $k$ such that \(1 \leq k \leq min(n,o)\), the first $k-1$ characters
        of $t$ and $u$ are the same, \(t_k \neq u_k\) and \(t_k \,R\, u_k\).

        If \(m \leq k-1\) then the first $m$ characters of $s$ and $u$ are the same, so \(s \preceq u\) by (1).

        If \(k \leq m\) then the first $k-1$ characters of $s$ and $u$ are the same, \(s_k = t_k \neq u_k\) and \(s_k = t_k \,R\, u_k\), so \(s \preceq u\) by (2).

                {\bf Case 2: \(\bm{s \preceq t}\) because of (2).} So there is an integer $k$ such that \(1 \leq k \leq min(m,n)\), the
        first $k-1$ characters of $s$ and $t$ are the same, \(s_k \neq t_k\) and \(s_k \,R\, t_k\).

        Now consider the fact \(t \preceq u\).

        This cannot be because of (3) since $n$ is positive and therefore \(t \neq \lambda\).

        If it is because of (1), then \(n \leq o\) and the first $n$ characters of $t$ and $u$ are the same. Then the first
        $k-1$ characters of $s$ and $u$ are the same, \(s_k \neq t_k = u_k\) and \(s_k \,R\, t_k = u_k\). Therefore
        \(s \preceq u\) by (2).

        If it is because of (2), then there is an integer $j$ such that \(1 \leq j \leq min(n, o)\), the first $j-1$
        characters of $t$ and $u$ are the same, \(t_j \neq u_j\) and \(t_j \,R\, u_j\).

        Let \(i = min(j,k)\). Then the first $i-1$ characters of $s$ and $u$ are the same, \(s_i \neq u_i\) and \(s_i \,R\,
        u_i\). Therefore \(s \preceq u\) by (2).

                {\bf Case 3: \(\bm{s \preceq t}\) because of (3).} Then \(s = \lambda\). So \(s \preceq u\) by (3).
\end{proof}

\subsection{Exercise 13}
Let \(A = \{a, b\}\). Describe all partial order relations on \(A\).

\begin{proof}
        \(R_1 = \{(a, a), (b, b)\}, R_2 = \{(a, a), (b, b), (a, b)\}, R_3 = \{(a, a), (b, b), (b, a)\}\)
\end{proof}

\subsection{Exercise 14}
Let \(A = \{a, b, c\}\).

\subsubsection{(a)}
Describe all partial order relations on \(A\) for which \(a\) is a maximal element.

\begin{proof}
        \(R_1 = \{(a, a), (b, b), (c, c)\}\), \(R_2 = \{(a, a), (b, b), (c, c), (b, a)\}\)

        \(R_3 = \{(a, a), (b, b), (c, c), (c, a)\}\), \(R_4 = \{(a, a), (b, b), (c, c), (b, a), (c, a)\}\)

        \(R_5 = \{(a, a), (b, b), (c, c), (c, b), (c, a)\}\), \(R_6 = \{(a, a), (b, b), (c, c), (b, c), (b, a)\}\)

        \(R_7 = \{(a, a), (b, b), (c, c), (c, b), (b, a), (c, a)\}\), \(R_8 = \{(a, a), (b, b), (c, c), (b, c), (b, a), (c, a)\}\)

        \(R_9 = \{(a, a), (b, b), (c, c), (b, c)\}\), \(R_{10} = \{(a, a), (b, b), (c, c), (c, b)\}\)
\end{proof}

\subsubsection{(b)}
Describe all partial order relations on \(A\) for which \(a\) is a minimal element.

\begin{proof}
        \(R_1 = \{(a, a), (b, b), (c, c)\}\), \(R_2 = \{(a, a), (b, b), (c, c), (a, b)\}\)

        \(R_3 = \{(a, a), (b, b), (c, c), (a, c)\}\), \(R_4 = \{(a, a), (b, b), (c, c), (a, b), (a, c)\}\)

        \(R_5 = \{(a, a), (b, b), (c, c), (b, c), (a, c)\}\), \(R_6 = \{(a, a), (b, b), (c, c), (c, b), (a, b)\}\)

        \(R_7 = \{(a, a), (b, b), (c, c), (c, b), (a, b), (a, c)\}\), \(R_8 = \{(a, a), (b, b), (c, c), (b, c), (a, b), (a, c)\}\)

        \(R_9 = \{(a, a), (b, b), (c, c), (b, c)\}\), \(R_{10} = \{(a, a), (b, b), (c, c), (c, b)\}\)

\end{proof}

\subsection{Exercise 15}
Suppose a relation $R$ on a set $A$ is reflexive, symmetric, transitive, and antisymmetric. What can you
conclude about $R$? Prove your answer.

        {\it Hint:} $R$ is the identity relation on $A$: \(x \,R\, x\) for each \(x \in A\), and \(x\,\Ccancel{R}\,y\) if
\(x \neq y\).

\begin{proof}
        Since $R$ is reflexive, symmetric, transitive, it's an equivalence relation. Consider any element \(a \in A\) and
        consider its equivalence class \([a]\). If \(b \in [a]\) then \(a \,R\, b\). By symmetry \(b \,R\, a\). By
        antisymmetry, \(a = b\). So \([a] = \{a\}\). Every element of $A$ is its own equivalence class. Moreover we also
        proved that \(a \,R\, b\) if and only if \(a = b\). In other words, if \(a \neq b\) then \(a \,\Ccancel{R}\, b\).
        So $R$ is the identity relation on $A$.
\end{proof}

\subsection{Exercise 16}
Consider the “divides” relation on each of the following sets A. Draw the Hasse diagram for each relation.

\subsubsection{(a)}
\(A = \{1, 2, 4, 5, 10, 15, 20\}\)

\begin{proof}
        \begin{figure}[ht!]
                \centering
                \includegraphics[scale=0.3]{../images/8.5.16.a.png}
        \end{figure}
\end{proof}

\subsubsection{(b)}
\(A = \{2, 3, 4, 6, 8, 9, 12, 18\}\)

\begin{proof}
        \begin{figure}[ht!]
                \centering
                \includegraphics[scale=0.5]{../images/8.5.16.b.png}
        \end{figure}
\end{proof}

\subsection{Exercise 17}
Consider the “subset” relation on \(\ps(S)\) for each of the following sets $S$.
Draw the Hasse diagram for each relation.

\subsubsection{(a)}
\(S = \{0, 1\}\)

\begin{proof}
        \begin{figure}[ht!]
                \centering
                \includegraphics[scale=0.4]{../images/8.5.17.a.png}
        \end{figure}
\end{proof}

\subsubsection{(b)}
\(S = \{0, 1, 2\}\)

\begin{proof}
        \begin{figure}[ht!]
                \centering
                \includegraphics[scale=0.4]{../images/8.5.17.b.png}
        \end{figure}
\end{proof}

\subsection{Exercise 18}
Let \(S = \{0, 1\}\) and consider the partial order relation $R$ defined on \(S \times S\) as follows: For all
ordered pairs \((a, b)\) and \((c, d)\) in \(S \times S, (a, b) \,R\, (c, d) \iff\) either \(a <c\) or both \(a= c\)
and \(b \leq d\), where \(<\) denotes the usual “less than” and \(\leq\) denotes the usual “less than or equal to”
relation for real numbers. Draw the Hasse diagram for $R$.

\begin{proof}
        \begin{figure}[ht!]
                \centering
                \includegraphics[scale=0.4]{../images/8.5.18.png}
        \end{figure}
\end{proof}

\subsection{Exercise 19}
Let \(S = \{0, 1\}\) and consider the partial order relation $R$ defined on \(S \times S\) as follows: For all
ordered pairs \((a, b)\) and \((c, d)\) in \(S \times S, (a, b) \,R\, (c, d) \iff\) \(a \leq c\) and \(b \leq d\),
where \(\leq\) denotes the usual “less than or equal to” relation for real numbers. Draw the Hasse diagram for $R$.

\begin{proof}
        \begin{figure}[ht!]
                \centering
                \includegraphics[scale=0.4]{../images/8.5.19.png}
        \end{figure}
\end{proof}

\subsection{Exercise 20}
Let \(S = \{0, 1\}\) and consider the partial order relation $R$ defined on \(S \times S \times S\) as follows:
For all ordered triples \((a, b, c)\) and \((d, e, f)\) in \(S \times S \times S, (a, b, c) \,R\, (d, e, f) \iff\)
\(a \leq d, b \leq e\) and \(c \leq f\), where \(\leq\) denotes the usual “less than or equal to” relation for real
numbers. Draw the Hasse diagram for $R$.

\begin{proof}
        \begin{figure}[ht!]
                \centering
                \includegraphics[scale=0.4]{../images/8.5.20.png}
        \end{figure}
\end{proof}

\subsection{Exercise 21}
Consider the “divides” relation defined on the set \(A = \{1, 2, 2^2, 2^3, \ldots, 2^n\}\), where $n$ is a
nonnegative integer.

\subsubsection{(a)}
Prove that this relation is a total order relation on $A$.

\begin{proof}
        {\it [We must show that for all $a$ and $b$ in $A$, \(a \mid b\) or \(b \mid a\).]} Let $a$ and $b$ be particular
        but arbitrarily chosen elements of $A$. By definition of $A$, there are nonnegative integers $r$ and $s$ such that
        \(a = 2^r\) and \(b = 2^s\). Now either \(r \leq s\) or \(s < r\). If \(r \leq s\), then \(b = 2^s = 2^r \cdot 2^{s-r}
        = a \cdot 2^{s-r}\), where \(s - r \geq 0\). It follows, by definition of divisibility, that \(a \mid b\). By a similar
        argument, if \(s < r\), then \(b \mid a\). Hence either \(a \mid b\) or \(b \mid a\), {\it [as was to be shown].}
\end{proof}

\subsubsection{(b)}
Draw the Hasse diagram for this relation for \(n = 4\).

\begin{proof}
        \begin{figure}[ht!]
                \centering
                \includegraphics[scale=0.4]{../images/8.5.21.b.png}
        \end{figure}
\end{proof}

{\bf \cy In $22-29$, find all greatest, least, maximal, and minimal elements for the relations in each of the
referenced exercises.}

\subsection{Exercise 22}
Exercise 16(a)
\begin{proof}
        greatest element: none; least element: 1; maximal elements: 15, 20; minimal element: 1
\end{proof}

\subsection{Exercise 23}
Exercise 16(b)
\begin{proof}
        greatest element: none; least element: none; maximal elements: 8, 12, 18; minimal elements: 2, 3
\end{proof}

\subsection{Exercise 24}
Exercise 17(a)
\begin{proof}
        greatest element: \(\{0, 1\}\); least element: \(\es\); maximal element: \(\{0, 1\}\); minimal element: \(\es\)
\end{proof}

\subsection{Exercise 25}
Exercise 17(b)
\begin{proof}
        greatest element: \(\{0,1,2\}\); least element: \(\es\); maximal elements: \(\{0,1,2\}\); minimal elements: \(\es\)
\end{proof}

\subsection{Exercise 26}
Exercise 18
\begin{proof}
        greatest element: (1, 1); least element: (0, 0); maximal element: (1, 1); minimal element: (0, 0)
\end{proof}

\subsection{Exercise 27}
Exercise 19
\begin{proof}
        greatest element: (1,1); least element: (0,0); maximal elements: (1,1); minimal elements: (0,0)
\end{proof}

\subsection{Exercise 28}
Exercise 20
\begin{proof}
        greatest element: (1,1,1); least element: (0,0,0); maximal elements: (1,1,1); minimal elements: (0,0,0)
\end{proof}

\subsection{Exercise 29}
Exercise 21
\begin{proof}
        greatest element: \(2^n\); least element: 1; maximal elements: \(2^n\); minimal elements: 1
\end{proof}

\subsection{Exercise 30}
Each of the following sets is partially ordered with respect to the “less than or equal to” relation, \(\leq\),
for real numbers. In each case, determine whether the set has a greatest or least element.

\subsubsection{(a)}
\(\R\)
\begin{proof}
        No greatest element, no least element

\end{proof}

\subsubsection{(b)}
\(\{x \in \R \,|\, 0 \leq x \leq 1\}\)
\begin{proof}
        Least element is 0, greatest element is 1
\end{proof}

\subsubsection{(c)}
\(\{x \in \R \,|\, 0 < x < 1\}\)
\begin{proof}
        No greatest element, no least element
\end{proof}

\subsubsection{(d)}
\(\{x \in \Z \,|\, 0 < x < 10\}\)
\begin{proof}
        Least element is 1, greatest element is 9
\end{proof}

\subsection{Exercise 31}
Let \(A = \{a, b, c, d\}\), and let \(R\) be the relation \(R = \{(a, a), (b, b), (c, c), (d, d), (c, a), (a, d)\),

\((c, d), (b, c), (b, d), (b, a)\}\). Is \(R\) a total order on \(A\)? Justify your answer.

\begin{proof}
        $R$ is a total order relation because it is reflexive, antisymmetric, and transitive (so it is a partial order)
        and because \([b, c, a, d]\) is a chain that contains every element of \(A: b \,R\, c, c \,R\, a\), and \(a \,R\, d\).
\end{proof}

\subsection{Exercise 32}
Let \(A = \{a, b, c, d\}\), and let \(R\) be the relation \(R = \{(a, a), (b, b), (c, c), (d, d), (c, b), (a, d)\),

\((b, a), (b, d), (c, d), (c, a)\}\). Is \(R\) a total order on \(A\)? Justify your answer.

\begin{proof}
        $R$ is a total order relation because it is reflexive, antisymmetric, and transitive (so it is a partial order)
        and because \([c, b, a, d]\) is a chain that contains every element of \(A: c \,R\, b, b \,R\, a\), and \(a \,R\, d\).
\end{proof}

\subsection{Exercise 33}
Consider the set \(A = \{12, 24, 48, 3, 9\}\) ordered by the “divides” relation. Is $A$ totally ordered with respect
to the relation? Justify your answer.

\begin{proof}
        The relation is reflexive, antisymmetric and transitive, but it's not a total order: there is no chain that includes
        all 5 elements of $A$, since 9 does not divide any of the larger elements 12, 24, 48.
\end{proof}

\subsection{Exercise 34}
Suppose that $R$ is a partial order relation on a set $A$ and that $B$ is a subset of $A$. The restriction of $R$ to
$B$ is defined as follows:
\[
        R' = \{(x,y)\,|\,x\in B, y \in B, \text{ and } (x, y) \in R\}.
\]
In other words, two elements of $B$ are related by $R'$ if, and only if, they are related by $R$. Prove that the
restriction of $R$ to $B$ is a partial order relation on $B$. (In less formal language, this says that a subset of a
partially ordered set is partially ordered.)

\begin{proof}
        {\bf $R'$ is reflexive:} For all \(b \in B\), since \(B \subseteq A\), we have \(b \in A\). Since $R$ is reflexive,
        \(b \,R\, b\). Thus by definition of $R'$, \(b \,R'\, b\).

                {\bf $R'$ is antisymmetric:} Assume \(b_1, b_2 \in B\) and assume \(b_1 \,R'\, b_2\) and \(b_2 \,R'\, b_1\). By
        definition of $R'$, \(b_1 \,R\, b_2\) and \(b_2 \,R\, b_1\). Since $R$ is antisymmetric, \(b_1 = b_2\).

                {\bf $R'$ is transitive:} Assume \(b_1, b_2, b_3 \in B\) and assume \(b_1 \,R'\, b_2\) and \(b_2 \,R'\, b_3\). By
        definition of $R'$, \(b_1 \,R\, b_2\) and \(b_2 \,R\, b_3\). Since $R$ is transitive, \(b_1 \,R\, b_3\). Since
        both \(b_1\) and \(b_2\) are in $B$, by definition of $R'$, \(b_1 \,R'\, b_3\).
\end{proof}

\subsection{Exercise 35}
The set \(\ps(\{w, x, y, z\})\) is partially ordered with respect to the “subset” relation \(\subseteq\). Find a
chain of length 4 in \(\ps(\{w, x, y, z\})\).

\begin{proof}
        One possible solution: \(\es \subseteq \{w\} \subseteq \{w, x\} \subseteq \{w, x, y\} \subseteq \{w, x, y, z\}\)
\end{proof}

\subsection{Exercise 36}
The set \(A = \{2, 4, 3, 6, 12, 18, 24\}\) is partially ordered with respect to the “divides” relation. Find a
chain of length 3 in \(A\).

\begin{proof}
        One possible solution: \(2 \mid 6 \mid 12 \mid 24\)
\end{proof}

\subsection{Exercise 37}
Find a chain of length 2 for the relation defined in exercise 19.

\begin{proof}
        \((0,0) \,R\, (1,0) \,R\, (1,1)\)
\end{proof}

\subsection{Exercise 38}
Prove that a partially ordered set is totally ordered if, and only if, it is a chain.

\begin{proof}
        Suppose $A$ is a partially ordered set with respect to a relation \(\preceq\). By definition of total order, $A$ is
        totally ordered if, and only if, any two elements of $A$ are comparable. By definition of chain, this is true if,
        and only if, $A$ is a chain.
\end{proof}

\subsection{Exercise 39}
Suppose that $A$ is a totally ordered set. Use mathematical induction to prove that for any integer \(n \geq 1\), every
subset of $A$ with $n$ elements has both a least element and a greatest element.

\begin{proof}
        Let $A$ be a set that is totally ordered with respect to a relation \(\preceq\), and let the property \(P(n)\) be the
        sentence “Every subset of $A$ with $n$ elements has both a least element and a greatest element.”

        {\bf Show that \(P(1)\) is true:} If \(A = \es\), then \(P(1)\) is true by default. So assume that \(A\) has at
        least one element, and suppose \(S = \{a_1\}\) is a subset of \(A\) with one element. Because \(\preceq\) is
        reflexive, \(a_1 \preceq a_1\). So, by definition of least element and greatest element, \(a_1\) is both a least
        element and a greatest element of \(S\), and thus the property is true for \(n = 1\).

                {\bf Show that for every integer \(k \geq 1\), if \(P(k)\) is true, then \(P(k + 1)\) is true:} Let $k$ be any integer
        with \(k \geq 1\), and suppose that any subset of $A$ with $k$ elements has both a least element and a greatest
        element. {\it [Inductive hypothesis.]}

        We must show that any subset of $A$ with \(k + 1\) elements has both a least element and a greatest element. If $A$ has
        fewer than \(k + 1\) elements, then the statement is true by default. So assume that $A$ has at least \(k + 1\)
        elements and that \(S = \{a_1, a_2, \ldots, a_{k+1}\}\) is a subset of $A$ with \(k + 1\) elements. By inductive
        hypothesis, \(S - \{a_{k+1}\}\) has both a least element $s$ and a greatest element $t$. Now because $A$ is totally
        ordered, \(a_{k+1}\) and $s$ are comparable. If \(a_{k+1} \preceq s\), then, by transitivity of \(\preceq, a_{k+1}\)
        is the least element of $S$; otherwise, $s$ remains the least element of $S$. And if \(t \preceq a_{k+1}\), then,
        by transitivity of \(\preceq, a_{k+1}\) is the greatest element of $S$; otherwise, $t$ remains the greatest element
        of $S$. Thus $S$ has both a greatest element and a least element, {\it [as was to be shown].}
\end{proof}

\subsection{Exercise 40}
Prove that a nonempty, finite, partially ordered set has

\subsubsection{(a)}
at least one minimal element,

\begin{proof}
        Suppose not. Suppose $A$ is a nonempty, finite set that is partially ordered with respect to a relation \(\preceq\),
        and suppose no element of $A$ is minimal. Construct a sequence of elements \(x_1, x_2, x_3, \ldots\) of $A$ as
        follows:

        1. Pick any element of $A$ and call it \(x_1\).

        2. For each \(i = 2, 3, 4, \ldots\), pick \(x_i\) to be an element of $A$ for which \(x_i \preceq x_{i-1}\) and \(x_i
        \neq x_{i-1}\). (Such an element must exist because otherwise \(x_{i-1}\) would be minimal, and we are
        supposing that no element of $A$ is minimal.)

        Now \(x_i \neq x_j\) for any \(i \neq j\):

        Because otherwise, if \(x_i = x_j\), where \(i < j\), then on the one hand, \(x_j \preceq x_{j-1} \preceq \cdots
        \preceq x_{i+1} \preceq x_i\) and so \(x_j \preceq x_{i+1}\). On the other hand, since \(x_i = x_j\) and \(x_{i+1}
        \preceq x_i\), then \(x_{i+1} \preceq x_j\). Hence by antisymmetry, \(x_j = x_{i+1}\), and so \(x_i = x_{i+1}\)
        because \(x_i = x_j\). But this contradicts the definition of the sequence \(x_1, x_2 , x_3, \ldots\).

        Thus \(x_1, x_2, x_3, \ldots\) is an infinite sequence of distinct elements, and consequently

        \(\{x_1, x_2, x_3, \ldots\}\) is an infinite subset of the finite set $A$, which is impossible. Hence the supposition
        is false and we conclude that any partially ordered subset of a finite set has a minimal element.
\end{proof}

\subsubsection{(b)}
at least one maximal element.

\begin{proof}
        The proof is almost identical to part (a), except replace ``minimal'' with ``maximal'' and replace \(x_i \preceq
        x_{i-1}\) with \(x_{i-1} \preceq x_i\). Also replace \(x_j \preceq x_{j-1} \preceq \cdots \preceq x_{i+1}\preceq x_i\)
        with \(x_i \preceq x_{i+1} \preceq \cdots \preceq x_{j-1} \preceq x_j\), and \(x_j \preceq x_{i+1}\) with \(x_{i+1}
        \preceq x_j\), and so on.
\end{proof}

\subsection{Exercise 41}
Prove that a finite, partially ordered set has

\subsubsection{(a)}
at most one greatest element,

\begin{proof}
        Suppose not. Suppose $A$ is a finite set that is partially ordered with respect to a relation \(\preceq\), and suppose
        two distinct elements \(a, b\) of $A$ are both greatest, with \(a \neq b\). Then by definition of greatest, \(a
        \preceq b\) and \(b \preceq a\). By antisymmetry, \(a = b\), a contradiction.
\end{proof}

\subsubsection{(b)}
at most one least element.

\begin{proof}
        The proof is almost identical. Replace ``greatest'' with ``least''.
\end{proof}

\subsection{Exercise 42}
Draw a Hasse diagram for a partially ordered set that has two maximal elements and two minimal elements and is such
that each element is comparable to exactly two other elements.

\begin{proof}
        \begin{figure}[ht!]
                \centering
                \includegraphics[scale=0.5]{../images/8.5.42.png}
        \end{figure}
\end{proof}

\subsection{Exercise 43}
Draw a Hasse diagram for a partially ordered set that has three maximal elements and three minimal elements and is
such that each element is either greater than or less than exactly two other elements.

\begin{proof}
        \begin{figure}[ht!]
                \centering
                \includegraphics[scale=0.5]{../images/8.5.43.png}
        \end{figure}
\end{proof}

\subsection{Exercise 44}
Use the algorithm given in the text to find a topological sorting for the relation of exercise 16(a) that is different from the “less than or equal to” relation \(\leq\).

\begin{proof}
        One such total order is 1, 5, 2, 15, 10, 4, 20.
\end{proof}

\subsection{Exercise 45}
Use the algorithm given in the text to find a topological sorting for the relation of exercise 16(b) that is different from the “less than or equal to” relation \(\leq\).

\begin{proof}
        One such total order is 3, 2, 9, 6, 18, 4, 12, 8.
\end{proof}

\subsection{Exercise 46}
Use the algorithm given in the text to find a topological sorting for the relation of exercise 19.

\begin{proof}
        One such total order is (0, 0), (1, 0), (0, 1), (1, 1).
\end{proof}

\subsection{Exercise 47}
Use the algorithm given in the text to find a topological sorting for the relation of exercise 20.

\begin{proof}
        One such total order is (0,0,0), (0,0,1), (0,1,0), (1,0,0), (1,1,0), (1,0,1), (0,1,1), (1,1,1).
\end{proof}

\subsection{Exercise 48}
Use the algorithm given in the text to find a topological sorting for the “subset” relation on \(\ps(\{a, b, c, d\})\).

\begin{proof}
        One such total order is \(\es, \{a\}, \{b\}, \{c\}, \{d\}, \{a,b\}, \{a,c\}, \{a,d\}, \{b,c\}, \{b,d\}\),

        \(\{c,d\}, \{a,b,c\}, \{a,b,d\}, \{a,c,d\}, \{b,c,d\}, \{a,b,c,d\}\)
\end{proof}

\subsection{Exercise 49}
\begin{figure}[ht!]
        \centering
        \includegraphics[scale=0.5]{../images/8.5.1.png}
\end{figure}

Refer to the prerequisite structure shown in Figure 8.5.1.

\subsubsection{(a)}
Find a list of six noncomparable courses that is different from the list given in the text.

\begin{proof}
        The list given in the text was: 350, 360, 345, 301, 230, 200.

        Here is another: 350, 390, 345, 301, 230, 200.
\end{proof}

\subsubsection{(b)}
Find two topological sortings that are different from the one given in the text.

\begin{proof}
        The one given in the text was: 140, 150, 141, 155, 200, 225, 230, 300, 250, 301, 340, 345, 350, 360, 390.

        Here is another: 150, 140, 141, 155, 200, 225, 230, 300, 250, 301, 340, 345, 350, 360, 390.

        Here is another: 140, 150, 141, 155, 200, 225, 230, 250, 350, 300, 301, 340, 345, 360, 390.
\end{proof}

\subsection{Exercise 50}
A set $S$ of jobs can be ordered by writing \(x \preceq y\) to mean that either \(x = y\) or \(x\) must be done before
\(y\), for all \(x\) and \(y\) in \(S\). The following is a Hasse diagram for this relation for a particular set \(S\)
of jobs.

\begin{figure}[ht!]
        \centering
        \includegraphics[scale=0.5]{../images/8.5.50.png}
\end{figure}

\subsubsection{(a)}
If one person is to perform all the jobs, one after another, find an order in which the jobs can be done.

\begin{proof}
        One possible answer: 1, 6, 10, 9, 5, 7, 2, 4, 8, 3
\end{proof}

\subsubsection{(b)}
Suppose enough people are available to perform any number of jobs simultaneously.

(i) If each job requires one day to perform, what is the least number of days needed to perform all ten jobs?

(ii) What is the maximum number of jobs that can be performed at the same time?

\begin{proof}
        (i) Jobs 1, 9, 2 can be performed on day 1. Jobs 10, 6, 8 can be performed on day 2. Jobs 5, 7 can be performed on
        day 3. Job 4 can be performed on day 4. Finally job 3 can be performed on day 5, which is the least number of days.

        (ii) At most 3 jobs can be performed at the same time.
\end{proof}

\subsection{Exercise 51}
Suppose the tasks described in Example 8.5.12 require the following performance times:

\begin{center}
        \arrayrulecolor{cyan}
        \begin{tabular}{|c|c|c|}
                \hline
                {\bf \cy Task} & {\bf \cy Immediately Preceding Tasks} & {\bf \cy Time Needed to Perform Task} \\
                \hline
                1              &                                       & 9 hours                               \\
                \hline
                2              & 1                                     & 7 hours                               \\
                \hline
                3              & 1                                     & 4 hours                               \\
                \hline
                4              & 2                                     & 5 hours                               \\
                \hline
                5              & 2, 3                                  & 7 hours                               \\
                \hline
                6              & 4                                     & 3 hours                               \\
                \hline
                7              & 2, 3                                  & 2 hours                               \\
                \hline
                8              & 4, 5                                  & 4 hours                               \\
                \hline
                9              & 6, 7, 8                               & 6 hours                               \\
                \hline
        \end{tabular}
        \arrayrulecolor{black} % change it back!
\end{center}

\subsubsection{(a)}
What is the minimum time required to assemble a car?

\begin{proof}
        This is the Hasse diagram:

        \begin{figure}[ht!]
                \centering
                \includegraphics[scale=0.4]{../images/8.5.51.png}
        \end{figure}

        If we fill the diagram we get the following minimum times:

        {\bf Task 1}: 9 hours,

        {\bf Task 2} (depends on 1): 9+7 = 16 hours, {\bf Task 3} (depends on 1): 9+4 = 13 hours,

        {\bf Task 4} (depends on 2): 16+5 = 21 hours, {\bf Task 5} (depends on 2 and 3): max(16, 13) + 7 = 23 hours,
        {\bf Task 7} (depends on 2 and 3): max(16, 13) + 2 = 18 hours,

        {\bf Task 6} (depends on 4): 23+3 = 26 hours, {\bf Task 8} (depends on 4 and 5): max(21, 23) + 4 = 27 hours,

        {\bf Task 9} (depends on 6, 7 and 8): max(26, 18, 27) + 6 = 33 hours.
\end{proof}

\subsubsection{(b)}
Find a critical path for the assembly process.

\begin{proof}
        Critical path: 1, 2, 5, 8, 9 because 9+7+7+4+6 is exactly 33 hours.
\end{proof}

\end{document}
