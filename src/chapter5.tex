\documentclass[14pt]{extarticle}

\usepackage[table]{xcolor}
\usepackage{amsmath,mathtools,amsfonts,amsthm,amssymb,hyperref,wasysym,pifont}
\usepackage{parskip,geometry,latexsym,bookmark,mathtools,float,cancel,tcolorbox}

\newtheorem{defn}{Definition}
\newtheorem{thm}{Theorem}
\newtheorem{claim}{Claim}
\newtheorem{lemma}{Lemma}

\newcommand{\dps}{\displaystyle}
\newcommand{\fbl}{\underline{\hspace{1cm}}\,\,}
\newcommand{\R}{\mathbb{R}}
\newcommand{\Z}{\mathbb{Z}}
\newcommand{\from}{\leftarrow}
\newcommand{\true}{{\bf t}}
\newcommand{\false}{{\bf c}}
\newcommand{\bic}{\leftrightarrow}
\newcommand{\base}[1]{{\color{cyan}#1}}
\newcommand{\floor}[1]{{\left\lfloor#1\right\rfloor}}
\newcommand{\ceil}[1]{{\lceil#1\rceil}}
\newcommand{\da}{\downarrow}
\newcommand{\fa}{\forall}
\newcommand{\te}{\exists}
\newcommand{\cy}{\color{cyan}}
\newcommand\Ccancel[2][black]{\renewcommand\CancelColor{\color{#1}}\cancel{#2}}
\newcommand\Cbcancel[2][black]{\renewcommand\CancelColor{\color{#1}}\bcancel{#2}}

\hypersetup{colorlinks,allcolors=blue,linktoc=all}
\geometry{a4paper}
\geometry{margin=0.42in}

\title{Chapter 5 Solutions, Susanna Epp Discrete Math 5th Edition}

\author{https://github.com/spamegg1}

\begin{document}
\maketitle
\tableofcontents

\section{Exercise Set 5.1}

{\bf \cy Write the first four terms of the sequences defined by the formulas in $1-6$.}

\subsection{Exercise 1}
$\dps a_k  = \frac{k}{10 + k}$, for every integer $k \geq 1$.

\begin{proof}
$\dps\frac{1}{11}, \frac{2}{12}, \frac{3}{13}, \frac{4}{14}$
\end{proof}

\subsection{Exercise 2}
$\dps b_j  = \frac{5-j}{5+j}$, for every integer $j \geq 1$.

\begin{proof}
$\dps\frac{4}{6}, \frac{3}{7}, \frac{2}{8}, \frac{1}{9}$
\end{proof}

\subsection{Exercise 3}
$\dps c_i  = \frac{(-1)^i}{3^i}$, for every integer $i \geq 0$.

\begin{proof}
$\dps 1, -\frac{1}{3}, \frac{1}{9}, -\frac{1}{27}$
\end{proof}

\subsection{Exercise 4}
$\dps d_m  = 1 + \left(\frac{1}{2}\right)^m$, for every integer $m \geq 0$.

\begin{proof}
$\dps 2, \frac{3}{2}, \frac{5}{4}, \frac{9}{8}$
\end{proof}

\subsection{Exercise 5}
$\dps e_n  = \floor{\frac{n}{2}}\cdot 2$, for every integer $n \geq 0$.

\begin{proof}
$0, 0, 2, 2$
\end{proof}

\subsection{Exercise 6}
$\dps f_n  = \floor{\frac{n}{4}}\cdot 4$, for every integer $n \geq 1$.

\begin{proof}
0, 0, 0, 4
\end{proof}

\subsection{Exercise 7}
Let $a_k = 2k + 1$ and $b_k = (k - 1)^3 + k + 2$ for every integer $k \geq 0$. Show that the first three terms of these sequences are identical but that their fourth terms differ.

\begin{proof}
$a_0 = 2(0) + 1 = 1, a_1 = 2(1) + 1 = 3, a_2 = 2(2) + 1 = 5, a_3 = 2(3) + 1 = 7$.

$b_0 = (0-1)^3 + 0 + 2 = 1, b_1 = (1-1)^3 + 1 + 2 = 3, b_2 = (2-1)^3 + 2 + 2 = 5, b_3 = (3-1)^3 + 3 + 2 = 13.$
\end{proof}

{\bf\cy Compute the first fifteen terms of each of the sequences in 8 and 9, and describe the general behavior of these sequences in words. (a definition of logarithm is given in Section 7.1.)}

\subsection{Exercise 8}
$g_n = \floor{\log_2 n}$ for every integer $n \geq 1$.

\begin{proof}
$g_1 = \floor{\log_2 1} = 0, g_2 = \floor{\log_2 2} = 1, g_3 = \floor{\log_2 3} = 1, g_4 = \floor{\log_2 4} = 2$,

$g_5 = \floor{\log_2 5} = 2, g_6 = \floor{\log_2 6} = 2, g_7 = \floor{\log_2 7} = 2, g_8 = \floor{\log_2 8} = 3$, 

$g_9 = \floor{\log_2 9} = 3, g_{10} = \floor{\log_2 10} = 3, g_{11} = \floor{\log_2 11} = 3, g_{12} = \floor{\log_2 12} = 3$, 

$g_{13} = \floor{\log_2 13} = 3, g_{14} = \floor{\log_2 14} = 3, g_{15} = \floor{\log_2 15} = 3$.

When $n$ is an integral power of 2, $g_n$ is the exponent of that power. For instance, $8 = 2^3$ and $g_8 = 3$. More generally, if $n = 2k$, where $k$ is an integer, then $g_n = k$. All terms of the sequence from $g_{2^k}$ up to, but not including, $g_{2^{k+1}}$ have the same value, namely $k$. For instance, all terms of the sequence from $g_8$ through $g_{15}$ have the value 3.
\end{proof}

\subsection{Exercise 9}
$h_n = n\floor{\log_2 n}$ for every integer $n \geq 1$.

\begin{proof}
$h_1 = 1\floor{\log_2 1} = 0, h_2 = 2\floor{\log_2 2} = 2, h_3 = 3\floor{\log_2 3} = 3, h_4 = 4\floor{\log_2 4} = 8$,

$h_5 = 5\floor{\log_2 5} = 10, h_6 = 6\floor{\log_2 6} = 12, h_7 = 7\floor{\log_2 7} = 14, h_8 = 8\floor{\log_2 8} = 24$, 

$h_9 = 9\floor{\log_2 9} = 27, h_{10} = 10\floor{\log_2 10} = 30, h_{11} = 11\floor{\log_2 11} = 33$, 

$h_{12} = 12\floor{\log_2 12} = 36, h_{13} = 13\floor{\log_2 13} = 39, h_{14} = 14\floor{\log_2 14} = 42,$ 

$h_{15} = 15\floor{\log_2 15} = 45$.
\end{proof}

{\bf\cy Find explicit formulas for sequences of the form $a_1, a_2, a_3, \ldots$ with the initial terms given in $10-16$.}

{\bf\cy Exercises $10-16$ have more than one correct answer.}

\subsection{Exercise 10}
$-1, 1, -1, 1, -1, 1$

\begin{proof}
$a_n = (-1)^n$, where $n$ is an integer and $n \geq 1$
\end{proof}

\subsection{Exercise 11}
$0, 1, -2, 3, -4, 5$

\begin{proof}
$a_n = (n-1)(-1)^n$, where $n$ is an integer and $n \geq 1$
\end{proof}

\subsection{Exercise 12}
$\dps \frac{1}{4}, \frac{2}{9}, \frac{3}{16}, \frac{4}{25}, \frac{5}{36}, \frac{6}{49}$

\begin{proof}
$\dps a_n = \frac{n}{(n+1)^2}$, where $n$ is an integer and $n \geq 1$
\end{proof}

\subsection{Exercise 13}
$\dps 1 - \frac{1}{2}, \frac{1}{2} - \frac{1}{3}, \frac{1}{3} - \frac{1}{4}, \frac{1}{4} - \frac{1}{5}, \frac{1}{5} - \frac{1}{6}, \frac{1}{6} - \frac{1}{7}$

\begin{proof}
$\dps a_n = \frac{1}{n} - \frac{1}{n+1}$, where $n$ is an integer and $n \geq 1$
\end{proof}

\subsection{Exercise 14}
$\dps \frac{1}{3}, \frac{4}{9}, \frac{9}{27}, \frac{16}{81}, \frac{25}{243}, \frac{36}{729}$

\begin{proof}
$\dps a_n = \frac{n^2}{3^n}$, where $n$ is an integer and $n \geq 1$
\end{proof}

\subsection{Exercise 15}
$\dps 0, -\frac{1}{2}, \frac{2}{3}, -\frac{3}{4}, \frac{4}{5}, -\frac{5}{6}, \frac{6}{7}$

\begin{proof}
$\dps a_n = \frac{n-1}{n}\cdot(-1)^{n-1}$, where $n$ is an integer and $n \geq 1$
\end{proof}

\subsection{Exercise 16}
3, 6, 12, 24, 48, 96

\begin{proof}
$\dps a_n = 3\cdot 2^{n-1}$, where $n$ is an integer and $n \geq 1$
\end{proof}

\subsection{Exercise 17}
Consider the sequence defined by $\dps a_n = \frac{2n + (-1)^n - 1}{4}$ for every integer $n \geq 0$. Find an alternative explicit formula for an that uses the floor notation.

\begin{proof}
$a_0 = 0, a_1 = 0, a_2 = 1, a_3 = 1, a_4 = 2, a_5 = 2$. It seems to be following the pattern: $\dps a_n = \floor{\frac{n}{2}}$. Let's try to prove this. When $n$ is even, $n = 2k$ for some integer $k$, so we have
\[
a_n = a_{2k} = \frac{2(2k) + (-1)^{2k} - 1}{4} = \frac{4k + 1 - 1}{4} = \frac{4k}{4} = k = \frac{n}{2} = \floor{\frac{n}{2}}
\]
When $n$ is odd, $n = 2k+1$ for some integer $k$, so we have
\[
a_n = a_{2k+1} = \frac{2(2k+1) + (-1)^{2k+1} - 1}{4} = \frac{4k + 2 - 1 - 1}{4} = \frac{4k}{4} = k = \frac{n-1}{2} = \floor{\frac{n}{2}}
\]
So $\dps a_n = \floor{\frac{n}{2}}$ for all $n \geq 0$.
\end{proof}

\subsection{Exercise 18}
Let $a_0 = 2, a_1 = 3, a_2 = -2, a_3 = 1, a_4 = 0, a_5 = -1$, and $a_6 = -2$. Compute each of the summations and products below.

\subsubsection{(a)}
$\dps\sum_{i=0}^{6}a_i$

\begin{proof}
$2 + 3 + (-2) + 1 + 0 + (-1) + (-2) = 1$

\end{proof}

\subsubsection{(b)}
$\dps\sum_{i=0}^{0}a_i$

\begin{proof}
$a_0 = 2$
\end{proof}

\subsubsection{(c)}
$\dps\sum_{j=1}^{3}a_{2j}$

\begin{proof}
$a_2 + a_4 + a_6 = -2 + 0 + (-2) = -4$
\end{proof}

\subsubsection{(d)}
$\dps\prod_{k=0}^{6}a_k$

\begin{proof}
$2 \cdot 3 \cdot (-2) \cdot 1 \cdot 0 \cdot (-1) \cdot (-2) = 0$
\end{proof}

\subsubsection{(e)}
$\dps\prod_{k=2}^{2}a_k$

\begin{proof}

\end{proof}

{\bf\cy Compute the summations and products in $19-28$.}

\subsection{Exercise 19}
$\dps\sum_{k=1}^{5}(k+1)$

\begin{proof}
$2+3+4+5+6=20$
\end{proof}

\subsection{Exercise 20}
$\dps\prod_{k=2}^{4}{k}^2$

\begin{proof}
$2^2\cdot 3^2\cdot 4^2 = 576$
\end{proof}

\subsection{Exercise 21}
$\dps\sum_{k=1}^{3}(k^2+1)$

\begin{proof}
$(1^2+1) + (2^2+1) + (3^2+1) = 2+5+10 = 17$
\end{proof}

\subsection{Exercise 22}
$\dps\prod_{j=0}^{4}{(-1)}^j$

\begin{proof}
$(-1)^0\cdot (-1)^1\cdot (-1)^2\cdot (-1)^3\cdot(-1)^4 = 1$
\end{proof}

\subsection{Exercise 23}
$\dps\sum_{i=1}^{1}i(i+1)$

\begin{proof}
1(1+1) = 2
\end{proof}

\subsection{Exercise 24}
$\dps\sum_{j=0}^{0}(j+1)\cdot2^j$

\begin{proof}
$(0+1)\cdot 2^0 = 1$
\end{proof}

\subsection{Exercise 25}
$\dps\prod_{k=2}^{2}\left(1 - \frac{1}{k}\right)$

\begin{proof}
$(1 - 1/2) = 1/2$
\end{proof}

\subsection{Exercise 26}
$\dps\sum_{k=-1}^{1}(k^2+3)$

\begin{proof}
$((-1)^2+3) + (0^2+3) + (1^2+3) = 11$
\end{proof}

\subsection{Exercise 27}
$\dps\sum_{n=1}^{6}\left(\frac{1}{n} - \frac{1}{n+1}\right)$

\begin{proof}
$\dps \left(\frac{1}{1} - \Ccancel[cyan]{\frac{1}{2}}\right) + \left(\Ccancel[cyan]{\frac{1}{2}} - \Ccancel[red]{\frac{1}{3}}\right) + \left(\Ccancel[red]{\frac{1}{3}} - \Ccancel[green]{\frac{1}{4}}\right) + \left(\Ccancel[green]{\frac{1}{4}} - \Ccancel[magenta]{\frac{1}{5}}\right) + \left(\Ccancel[magenta]{\frac{1}{5}} - \Ccancel[black]{\frac{1}{6}}\right) + \left(\Ccancel[black]{\frac{1}{6}} - \frac{1}{7}\right)$

$\dps = 1 - \frac{1}{7} = \frac{6}{7}$
\end{proof}

\subsection{Exercise 28}
$\dps\prod_{i=2}^{5}\frac{i(i+2)}{(i-1)\cdot(i+1)}$

\begin{proof}
$\dps \frac{2(2+2)}{(2-1)(2+1)}\cdot\frac{3(3+2)}{(3-1)(3+1)}\cdot\frac{4(4+2)}{(4-1)(4+1)}\cdot\frac{5(5+2)}{(5-1)(5+1)}$

$\dps = \frac{\Ccancel[cyan]{8}}{3}\cdot\frac{\Ccancel[red]{15}}{\Ccancel[cyan]{8}}\cdot\frac{\Ccancel[green]{24}}{\Ccancel[red]{15}}\cdot\frac{35}{\Ccancel[green]{24}}\cdot = \frac{35}{3}$
\end{proof}

{\bf\cy Write the summations in $29-32$ in expanded form.}

\subsection{Exercise 29}
$\dps\sum_{i=1}^{n}(-2)^i$

\begin{proof}
$(-2)^1 + (-2)^2 + (-2)^3 + \cdots + (-2)^n = -2 + 2^2 - 2^3 + \cdots + (-1)^n2^n$
\end{proof}

\subsection{Exercise 30}
$\dps\sum_{j=1}^{n}j(j+1)$

\begin{proof}
$1\cdot2 + 2\cdot3 + 3\cdot4 + \cdots + n(n+1)$
\end{proof}

\subsection{Exercise 31}
$\dps\sum_{k=0}^{n+1}\frac{1}{k!}$

\begin{proof}
$\dps\sum_{k=0}^{n+1}\frac{1}{k!} = \frac{1}{0!} + \frac{1}{1!} + \frac{1}{2!} + \cdots + \frac{1}{(n+1)!}$
\end{proof}

\subsection{Exercise 32}
$\dps\sum_{i=1}^{k+1}i(i!)$

\begin{proof}
$1(1!) + 2(2!) + 3(3!) + \cdots + (k+1)(k+1)!$
\end{proof}

{\bf\cy Evaluate the summations and products in $33-36$ for the indicated values of the variable.}

\subsection{Exercise 33}
$\dps \frac{1}{1^2} + \frac{1}{2^2} + \frac{1}{3^2} + \cdots + \frac{1}{n^2}$; $n = 1$

\begin{proof}
$\dps\frac{1}{1^2} = 1$
\end{proof}

\subsection{Exercise 34}
$\dps 1(1!) + 2(2!) + 3(3!) + \cdots + m(m!)$; $m = 2$

\begin{proof}
$\dps 1(1!) + 2(2!) = 1 + 4 = 5$
\end{proof}

\subsection{Exercise 35}
$\dps \left(\frac{1}{1+1}\right)\left(\frac{2}{2+1}\right)\left(\frac{3}{3+1}\right) \cdots \left(\frac{k}{k+1}\right)$; $k = 3$

\begin{proof}
$\dps \left(\frac{1}{1+1}\right)\left(\frac{2}{2+1}\right)\left(\frac{3}{3+1}\right) = \frac{1}{2}\frac{2}{3}\frac{3}{4} = \frac{1}{4}$
\end{proof}

\subsection{Exercise 36}
$\dps \left(\frac{1\cdot2}{3\cdot4}\right)\left(\frac{2\cdot3}{4\cdot5}\right)\left(\frac{3\cdot4}{5\cdot6}\right)\cdots\left(\frac{m\cdot(m+1)}{(m+2)\cdot(m+3)}\right); m = 1$

\begin{proof}
$\dps \frac{1\cdot2}{3\cdot4} = \frac{3}{8}$
\end{proof}

{\bf\cy Write each of $37-39$ as a single summation.}

\subsection{Exercise 37}
$\dps\sum_{i=1}^{k}i^3 + (k+1)^3$

\begin{proof}
$\dps\sum_{i=1}^{k+1}i^3$
\end{proof}

\subsection{Exercise 38}
$\dps\sum_{k=1}^{m}\frac{k}{k+1} + \frac{m+1}{m+2}$

\begin{proof}
$\dps\sum_{k=1}^{m+1}\frac{k}{k+1}$
\end{proof}

\subsection{Exercise 39}
$\dps\sum_{m=0}^{n}(m+1)2^n + (n+2)2^{n+1}$

\begin{proof}
$\dps\sum_{m=0}^{n+1}(m+1)2^n$
\end{proof}

{\bf\cy Rewrite $40-42$ by separating off the final term.}

\subsection{Exercise 40}
$\dps\sum_{i=1}^{k+1}i(i!)$

\begin{proof}
$\dps\sum_{i=1}^{k}i(i!) + (k+1)(k+1)!$
\end{proof}

\subsection{Exercise 41}
$\dps\sum_{k=1}^{m+1}k^2$

\begin{proof}
$\dps\sum_{k=1}^{m}k^2 + (m+1)^2$
\end{proof}

\subsection{Exercise 42}
$\dps\sum_{m=1}^{n+1}m(m+1)$

\begin{proof}
$\dps\sum_{m=1}^{n}m(m+1) + (n+1)(n+2)$
\end{proof}

{\bf\cy Write each of $43-52$ using summation or product
notation.}

{\bf\cy Exercises $43-52$ have more than one correct answer.}

\subsection{Exercise 43}
$1^2 - 2^2 + 3^2 - 4^2 + 5^2 - 6^2 + 7^2$

\begin{proof}
$\dps \sum_{k=1}^{7}(-1)^{k+1}k^2$ or $\dps \sum_{k=0}^{6}(-1)^{k}(k+1)^2$
\end{proof}

\subsection{Exercise 44}
$(1^3 - 1) - (2^3 - 1) + (3^3 - 1) - (4^3 - 1) + (5^3 - 1)$

\begin{proof}
$\dps\sum_{k=1}^{5}(k^3-1)$
\end{proof}

\subsection{Exercise 45}
$(2^2 - 1)\cdot(3^2 - 1)\cdot(4^2 - 1)$

\begin{proof}
$\dps\prod_{k=2}^{4}(k^2-1)$
\end{proof}

\subsection{Exercise 46}
$\dps\frac{2}{3\cdot4} - \frac{3}{4\cdot5} + \frac{4}{5\cdot6} - \frac{5}{6\cdot7} + \frac{6}{7\cdot8}$

\begin{proof}
$\dps\sum_{j=2}^{6}\frac{(-1)^j j}{(j+1)(j+2)}$
\end{proof}

\subsection{Exercise 47}
$1 - r + r^2 - r^3 + r^4 - r^5$

\begin{proof}
$\dps\sum_{i = 0}^{5}(-1)^i r^i$
\end{proof}

\subsection{Exercise 48}
$(1 - t)\cdot(1 - t^2)\cdot(1 - t^3)\cdot(1 - t^4)$

\begin{proof}
$\dps\prod_{k=1}^{4}(1-t^k)$
\end{proof}

\subsection{Exercise 49}
$1^3 + 2^3 + 3^3 + \cdots + n^3$

\begin{proof}
$\dps\sum_{k=1}^{n}k^3$
\end{proof}

\subsection{Exercise 50}
$\dps \frac{1}{2!} + \frac{2}{3!} + \frac{3}{4!} + \cdots \frac{n}{(n+1)!}$

\begin{proof}
$\dps\sum_{k=1}^{n}\frac{k}{(k+1)!}$
\end{proof}

\subsection{Exercise 51}
$n + (n - 1) + (n - 2) + \cdots + 1$

\begin{proof}
$\dps\sum_{i=0}^{n-1}(n-i)$
\end{proof}

\subsection{Exercise 52}
$\dps n + \frac{n-1}{2!} + \frac{n-2}{3!} + \frac{n-3}{4!} + \cdots + \frac{1}{n!}$

\begin{proof}
$\dps\sum_{i=0}^{n-1}\frac{n-i}{(i+1)!}$
\end{proof}

{\bf\cy Transform each of 53 and 54 by making the change of
variable $i = k + 1$.}

\subsection{Exercise 53}
$\dps\sum_{k=0}^{5}k(k-1)$

\begin{proof}
When $k = 0$, we have $i = 0+1 = 1$ and when $k = 5$ we have $i = 5+1 = 6$. Solving for $k$ we get $k = i-1$. So
\[
\sum_{k=0}^{5}k(k-1) = \sum_{i=1}^{6}(i-1)(i-2)
\]
\end{proof}

\subsection{Exercise 54}
$\dps\prod_{k=1}^{n}\frac{k}{k^2+4}$

\begin{proof}
When $k = 1$, we have $i = 1+1 = 2$ and when $k = n$ we have $i = n+1$. Solving for $k$ we get $k = i-1$. So
\[
\prod_{k=1}^{n}\frac{k}{k^2+4} = \prod_{i=2}^{n+1}\frac{i-1}{(i-1)^2+4}
\]
\end{proof}

{\bf\cy Transform each of $55-58$ by making the change of variable $j = i - 1$.}

\subsection{Exercise 55}
$\dps\sum_{i=1}^{n+1}\frac{(i-1)^2}{i\cdot n}$

\begin{proof}
When $i = 1$, we have $j = 1-1 = 0$ and when $i = n+1$ we have $j = n+1-1 = n$. Solving for $i$ we get $i = j+1$. So
\[
\sum_{i=1}^{n+1}\frac{(i-1)^2}{i\cdot n} = \sum_{j=0}^{n}\frac{(j+1-1)^2}{(j+1)\cdot n} = \sum_{j=0}^{n}\frac{j^2}{(j+1)\cdot n}
\]
\end{proof}

\subsection{Exercise 56}
$\dps\sum_{i=3}^{n}\frac{i}{i+n-1}$

\begin{proof}
When $i = 3$, we have $j = 3-1 = 2$ and when $i = n$ we have $j = n-1$. Solving for $i$ we get $i = j+1$. So
\[
\sum_{i=3}^{n}\frac{i}{i+n-1} = \sum_{j=2}^{n-1}\frac{j+1}{j+1+n-1} = \sum_{j=2}^{n-1}\frac{j+1}{j+n}
\]
\end{proof}

\subsection{Exercise 57}
$\dps\sum_{i=1}^{n-1}\frac{i}{(n-i)^2}$

\begin{proof}
When $i = 1$, we have $j = 1-1 = 0$ and when $i = n-1$ we have $j = n-1-1 = n-2$. Solving for $i$ we get $i = j+1$. So
\[
\sum_{i=1}^{n-1}\frac{i}{(n-i)^2} = \sum_{j=0}^{n-2}\frac{j+1}{(n-(j+1))^2}
\]
\end{proof}

\subsection{Exercise 58}
$\dps\prod_{i=n}^{2n}\frac{n-i+1}{n+i}$

\begin{proof}
When $i = n$, we have $j = n-1$ and when $i = 2n$ we have $j = 2n-1$. Solving for $i$ we get $i = j+1$. So
\[
\prod_{i=n}^{2n}\frac{n-i+1}{n+i} = \prod_{j=n-1}^{2n-1}\frac{n-(j+1)+1}{n+j+1} = \prod_{j=n-1}^{2n-1}\frac{n-j}{n+j+1}
\]
\end{proof}

{\bf\cy Write each of $59-61$ as a single summation or product.}

\subsection{Exercise 59}
$\dps 3\sum_{k=1}^{n}(2k-3) + \sum_{k=1}^{n}(4-5k)$

\begin{proof}
$\dps\sum_{k=1}^{n}[3(2k-3) + (4-5k)] = \sum_{k=1}^{n}[6k-9 + 4-5k] = \sum_{k=1}^{n}[k-5]$
\end{proof}

\subsection{Exercise 60}
$\dps 2\sum_{k=1}^{n}(3k^2+4) + 5\sum_{k=1}^{n}(2k^2-1)$

\begin{proof}
$\dps \sum_{k=1}^{n}[2(3k^2+4) + 5(2k^2-1)] = \sum_{k=1}^{n}[6k^2+8 + 10k^2-5] = \sum_{k=1}^{n}[16k^2+3]$
\end{proof}

\subsection{Exercise 61}
$\dps\prod_{k=1}^{n}\frac{k}{k+1} \prod_{k=1}^{n}\frac{k+1}{k+2}$

\begin{proof}
$\dps\prod_{k=1}^{n}\frac{k}{k+1} \prod_{k=1}^{n}\frac{k+1}{k+2} = \prod_{k=1}^{n}\frac{k}{\Ccancel[cyan]{k+1}}\frac{\Ccancel[cyan]{k+1}}{k+2} = \prod_{k=1}^{n}\frac{k}{k+2}$
\end{proof}

{\bf\cy Compute each of $62-76$. Assume the values of the variables are restricted so that the expressions are defined.}

\subsection{Exercise 62}
$\dps\frac{4!}{3!}$

\begin{proof}
$\dps \frac{4\cdot\Ccancel[cyan]{3\cdot2\cdot1}}{\Ccancel[cyan]{3\cdot2\cdot1}} = 4$
\end{proof}

\subsection{Exercise 63}
$\dps\frac{6!}{8!}$

\begin{proof}
$\dps \frac{\Ccancel[cyan]{6\cdot5\cdot4\cdot3\cdot2\cdot1}}{8\cdot7\cdot\Ccancel[cyan]{6\cdot5\cdot4\cdot3\cdot2\cdot1}} = \frac{1}{56}$
\end{proof}

\subsection{Exercise 64}
$\dps\frac{4!}{0!}$

\begin{proof}
$\dps\frac{4!}{0!} = \frac{24}{1} = 24$
\end{proof}

\subsection{Exercise 65}
$\dps\frac{n!}{(n-1)!}$

\begin{proof}
$\dps\frac{n\cdot\Ccancel[cyan]{(n-1)\cdots2\cdot1}}{\Ccancel[cyan]{(n-1)\cdots2\cdot1}} = n$
\end{proof}

\subsection{Exercise 66}
$\dps\frac{(n-1)!}{(n+1)!}$

\begin{proof}
$\dps\frac{\Ccancel[cyan]{(n-1)\cdots2\cdot1}}{(n+1)\cdot n\cdot\Ccancel[cyan]{(n-1)\cdots2\cdot1}} = \frac{1}{(n+1)n}$
\end{proof}

\subsection{Exercise 67}
$\dps\frac{n!}{(n-2)!}$

\begin{proof}
$\dps\frac{n\cdot(n-1)\cdot\Ccancel[cyan]{(n-2)\cdots2\cdot1}}{\Ccancel[cyan]{(n-2)\cdots2\cdot1}} = n(n-1)$
\end{proof}

\subsection{Exercise 68}
$\dps\frac{((n+1)!)^2}{(n!)^2}$

\begin{proof}
$ = \dps\left(\frac{(n+1)!}{n!}\right)^2 = \left(\frac{(n+1)\Ccancel[cyan]{n(n-1)\cdots2\cdot1}}{\Ccancel[cyan]{n(n-1)\cdots2\cdot1}}\right)^2 = (n+1)^2$
\end{proof}

\subsection{Exercise 69}
$\dps\frac{n!}{(n-k)!}$

\begin{proof}
$\dps\frac{n\cdot(n-1)\cdots(n-k+1)\cdot\Ccancel[cyan]{(n-k)(n-k-1)\cdots2\cdot1}}{\Ccancel[cyan]{(n-k)(n-k-1)\cdots2\cdot1}} = n(n-1)\cdots(n-k+1)$
\end{proof}

\subsection{Exercise 70}
$\dps\frac{n!}{(n-k+1)!}$

\begin{proof}
$\dps\frac{n\cdot(n-1)\cdots(n-k+2)\cdot\Ccancel[cyan]{(n-k+1)(n-k)\cdots2\cdot1}}{\Ccancel[cyan]{(n-k+1)(n-k)\cdots2\cdot1}} = n(n-1)\cdots(n-k+2)$
\end{proof}

\subsection{Exercise 71}
$\dps\binom{5}{3}$

\begin{proof}
$\dps\binom{5}{3} = \frac{5!}{3!(5-3)!} = \frac{5!}{3! \cdot 2!} = \frac{5 \cdot 4 \cdot \Ccancel[cyan]{3 \cdot 2 \cdot 1}}{(\Ccancel[cyan]{3 \cdot 2 \cdot 1}) \cdot (2 \cdot 1)} = 10$

\end{proof}

\subsection{Exercise 72}
$\dps\binom{7}{4}$

\begin{proof}
$\dps\binom{7}{4} = \frac{7!}{4!(7-4)!} = \frac{7!}{4! \cdot 3!} = \frac{7 \cdot \Ccancel[red]{6} \cdot 5 \cdot \Ccancel[cyan]{4 \cdot 3 \cdot 2 \cdot 1}}{(\Ccancel[cyan]{4 \cdot 3 \cdot 2 \cdot 1}) \cdot (\Ccancel[red]{3 \cdot 2} \cdot 1)} = 35$
\end{proof}

\subsection{Exercise 73}
$\dps\binom{3}{0}$

\begin{proof}
1
\end{proof}

\subsection{Exercise 74}
$\dps\binom{5}{5}$

\begin{proof}
1
\end{proof}

\subsection{Exercise 75}
$\dps\binom{n}{n-1}$

\begin{proof}
$\dps\binom{n}{n-1} = \frac{n!}{(n-1)!(n-(n-1))!} = \frac{n!}{(n-1)! \cdot 1!} = \frac{n \cdot \Ccancel[cyan]{(n-1) \cdots 2 \cdot 1}}{\Ccancel[cyan]{(n-1) \cdots 2 \cdot 1}} = n$
\end{proof}

\subsection{Exercise 76}
$\dps\binom{n+1}{n-1}$

\begin{proof}
$\dps\binom{n+1}{n-1} = \frac{(n+1)!}{(n-1)!(n+1-(n-1))!} = \frac{(n+1)!}{(n-1)! \cdot 2!}$ 

$\dps= \frac{(n+1) \cdot n \cdot \Ccancel[cyan]{(n-1) \cdots 2 \cdot 1}}{\Ccancel[cyan]{(n-1) \cdots 2 \cdot 1}\cdot 2} = \frac{(n+1)n}{2}$
\end{proof}

\subsection{Exercise 77}

\subsubsection{(a)}
Prove that $n! + 2$ is divisible by 2, for every integer $n \geq 2$.

\begin{proof}
Let $n$ be an integer such that $n \geq 2$. By definition of factorial,
\[
n! =
\left\{
\begin{array}{lr}
2 \cdot 1 & \text{\cy if $n = 2$} \\
3 \cdot 2 \cdot 1 & \text{\cy if $n = 3$} \\
n \cdot (n-1) \cdots 2 \cdot 1 & \text{\cy if $n > 3$} \\
\end{array}
\right.
\]
In each case, $n!$ has a factor of 2, and so $n! = 2k$ for some integer $k$. Then $n!+2 = 2k+2 = 2(k+1)$. Since $k+1$ is an integer, $n!+2$ is divisible by 2.
\end{proof}

\subsubsection{(b)}
Prove that $n! + k$ is divisible by $k$, for every integer $n \geq 2$ and $k = 2, 3, \ldots, n$.

\begin{proof}
For every $k = 2, 3, \ldots , n$, from the definition of $n!$ in part (a), we can see that $n!$ has a factor of $k$, so $n! = ka$ for some integer $a$. Then $n!+k = ka + k = k(a+1)$ where $a+1$ is an integer. Therefore $n!+k$ is divisible by $k$ for every $k = 2, 3, \ldots , n$.
\end{proof}

\subsubsection{(c)}
Given any integer $m \geq 2$, is it possible to find a sequence of $m - 1$ consecutive positive integers none of which is prime? Explain your answer.

\begin{proof}
Yes. By part (b), $m! + k$ is divisible by $k$, for all $k = 2, 3, \ldots, m$. So $m!+2, m!+3, \ldots, m!+m$ are $m-1$ consecutive integers none of which is prime.
\end{proof}

\subsection{Exercise 78}
Prove that for all nonnegative integers $n$ and $r$ with
$r+1 \leq n$, $\dps\binom{n}{r+1} = \frac{n-r}{r+1}\binom{n}{r}$.

\begin{proof}
Suppose $n$ and $r$ are nonnegative integers with $r + 1 \leq n$. The right-hand side of the equation to be shown is
\[
\begin{array}{rcl}
\dps\frac{n-r}{r+1} \cdot \binom{n}{r} & = & \dps\frac{n-r}{r+1} \cdot \frac{n!}{r!(n-r)!} \\
& = & \dps\frac{\Ccancel[cyan]{n-r}}{r+1} \cdot \frac{n!}{r!\Ccancel[cyan]{(n-r)}(n-r-1)!} \\
& = & \dps \frac{n!}{(r+1)!(n-r-1)!} \\
& = & \dps \frac{n!}{(r+1)!(n-(r+1))!} \\
& = & \dps \binom{n}{r+1}
\end{array}
\]
which is the left-hand side of the equality to be shown.
\end{proof}

\subsection{Exercise 79}
Prove that if $p$ is a prime number and $r$ is an integer
with $0 < r < p$, then $\dps\binom{p}{r}$ is divisible by $p$.

\begin{proof}
We know that
\[
\binom{p}{r} = \frac{p!}{r!(p-r)!} = \frac{p \cdot (p-1) \cdots 2 \cdot 1}{[r \cdot (r-1) \cdots 2 \cdot 1][(p-r) \cdot (p-r-1) \cdots 2 \cdot 1]}
\]
is an integer. Notice that all the factors in the denominator are less than $p$. So, since $p$ is prime, $p$ is not divisible by any of the factors in the denominator. This means that every factor in the denominator is canceled out by the factors of $(p-1) \cdots 2 \cdot 1$. Thus
\[
M = \frac{(p-1) \cdots 2 \cdot 1}{[r \cdot (r-1) \cdots 2 \cdot 1][(p-r) \cdot (p-r-1) \cdots 2 \cdot 1]}
\]
is also an integer (otherwise $p \cdot M$ would not be an integer, since $p$ cannot cancel out anything in the denominator). Therefore $\dps\binom{p}{r} = p\cdot M$ where $M$ is an integer, so it is divisible by $p$.
\end{proof}

\subsection{Exercise 80}
Suppose $a[1], a[2], a[3], \ldots, a[m]$ is a one-dimensional array and consider the following algorithm segment:

\begin{tabbing}
\hspace{7cm}
\= $sum \coloneqq 0$ \\
\> {\bf for} \= ($k \coloneqq 1$ {\bf to} $m$) \\
\>           \> $sum \coloneqq sum +  a[k]$ \\
\> {\bf next} $k$
\end{tabbing}

Fill in the blanks below so that each algorithm segment performs the same job as the one shown in the exercise statement.

\subsubsection{(a)}
\begin{tabbing}
$sum \coloneqq 0$ \\
{\bf for} \= ($i \coloneqq 0$ {\bf to} \fbl) \\
          \> $sum \coloneqq$ \fbl \\
{\bf next} $i$
\end{tabbing}

\begin{proof}
$m - 1, sum + a[i + 1]$
\end{proof}

\subsubsection{(b)}
\begin{tabbing}
$sum \coloneqq 0$ \\
{\bf for} \= ($j \coloneqq 2$ {\bf to} \fbl) \\
          \> $sum \coloneqq$ \fbl \\
{\bf next} $j$
\end{tabbing}

\begin{proof}
$m + 1, sum + a[j - 1]$
\end{proof}

{\bf\cy Use repeated division by 2 to convert (by hand) the integers in $81-83$ from base 10 to base 2.}

\subsection{Exercise 81}
90
\begin{proof}
\begin{center}
\begin{tabular}{rcll}
90 / 2 & = & 45, & remainder = 0 \\
45 / 2 & = & 22, & remainder = 1 \\
22 / 2 & = & 11, & remainder = 0 \\
11 / 2 & = & 5, & remainder = 1 \\
5 / 2 & = & 2, & remainder = 1 \\
2 / 2 & = & 1, & remainder = 0 \\
1 / 2 & = & 0, & remainder = 1
\end{tabular}
\end{center}
So $90_{10} = 1011010_2$.
\end{proof}

\subsection{Exercise 82}
98
\begin{proof}
\begin{center}
\begin{tabular}{rcll}
98 / 2 & = & 49, & remainder = 0 \\
49 / 2 & = & 24, & remainder = 1 \\
24 / 2 & = & 12, & remainder = 0 \\
12 / 2 & = & 6, & remainder = 0 \\
6 / 2 & = & 3, & remainder = 0 \\
3 / 2 & = & 1, & remainder = 1 \\
1 / 2 & = & 0, & remainder = 1
\end{tabular}
\end{center}
So $98_{10} = 1100010_2$.
\end{proof}

\subsection{Exercise 83}
205
\begin{proof}
\begin{center}
\begin{tabular}{rcll}
205 / 2 & = & 102, & remainder = 1 \\
102 / 2 & = & 51, & remainder = 0 \\
51 / 2 & = & 25, & remainder = 1 \\
25 / 2 & = & 12, & remainder = 1 \\
12 / 2 & = & 6, & remainder = 0 \\
6 / 2 & = & 3, & remainder = 0 \\
3 / 2 & = & 1, & remainder = 1 \\
1 / 2 & = & 0, & remainder = 1
\end{tabular}
\end{center}
So $205_{10} = 11001101_2$.
\end{proof}

{\bf\cy Make a trace table to trace the action of algorithm 5.1.1 on the input in $84-86$.}

\subsection{Exercise 84}
23
\begin{proof}
\begin{center}
\arrayrulecolor{cyan}
\begin{tabular}{|c|c|c|c|c|c|c|}
\hline
$a$&23&&&&& \\
\hline
$i$&0&1&2&3&4&5 \\
\hline
$q$&23&11&5&2&1&0 \\
\hline
$r[0]$&&1&&&& \\
\hline
$r[1]$&&&1&&& \\
\hline
$r[2]$&&&&1&& \\
\hline
$r[3]$&&&&&0& \\
\hline
$r[4]$&&&&&&1 \\
\hline
\end{tabular}
\arrayrulecolor{black} % change it back!
\end{center}
\end{proof}

\subsection{Exercise 85}
28
\begin{proof}
\begin{center}
\arrayrulecolor{cyan}
\begin{tabular}{|c|c|c|c|c|c|c|}
\hline
$a$&28&&&&& \\
\hline
$i$&0&1&2&3&4&5 \\
\hline
$q$&28&14&7&3&1&0 \\
\hline
$r[0]$&&0&&&& \\
\hline
$r[1]$&&&0&&& \\
\hline
$r[2]$&&&&1&& \\
\hline
$r[3]$&&&&&1& \\
\hline
$r[4]$&&&&&&1 \\
\hline
\end{tabular}
\arrayrulecolor{black} % change it back!
\end{center}
\end{proof}

\subsection{Exercise 86}
44
\begin{proof}
\begin{center}
\arrayrulecolor{cyan}
\begin{tabular}{|c|c|c|c|c|c|c|c|}
\hline
$a$&44&&&&&& \\
\hline
$i$&0&1&2&3&4&5&6 \\
\hline
$q$&44&22&11&5&2&1&0 \\
\hline
$r[0]$&&0&&&&& \\
\hline
$r[1]$&&&0&&&& \\
\hline
$r[2]$&&&&1&&& \\
\hline
$r[3]$&&&&&1&& \\
\hline
$r[4]$&&&&&&0& \\
\hline
$r[5]$&&&&&&&1 \\
\hline
\end{tabular}
\arrayrulecolor{black} % change it back!
\end{center}
\end{proof}

\subsection{Exercise 87}
Write an informal description of an algorithm (using repeated division by 16) to convert a nonnegative integer from decimal notation to hexadecimal notation (base 16).

\begin{proof}
Suppose $a$ is a nonnegative integer. Divide a by 16 using the quotient-remainder theorem to obtain a quotient $q[0]$ and a remainder $r[0]$. If the quotient is nonzero, divide by 16 again to obtain a quotient $q[1]$ and a remainder $r[1]$. Continue this process until a quotient of 0 is obtained. At each stage, the remainder must be less than the divisor, which is 16. Thus each remainder is always among $0, 1, 2, \ldots, 15$. Read the divisions from the bottom up.
\end{proof}

{\bf\cy Use the algorithm you developed for exercise 87 to convert the integers in $88-90$ to hexadecimal notation.}

\subsection{Exercise 88}
287
\begin{proof}
\begin{center}
\begin{tabular}{rcll}
287 / 16 & = & 17, & remainder = 15 = F \\
17 / 16 & = & 1, & remainder = 1 \\
1 / 16 & = & 0, & remainder = 1
\end{tabular}
\end{center}
So $287_{10} = 11F_{16}$.
\end{proof}

\subsection{Exercise 89}
693
\begin{proof}
\begin{center}
\begin{tabular}{rcll}
693 / 16 & = & 43, & remainder = 5 \\
43 / 16 & = & 2, & remainder = 11 = B \\
2 / 16 & = & 0, & remainder = 2
\end{tabular}
\end{center}
So $693_{10} = 2B5_{16}$.
\end{proof}

\subsection{Exercise 90}
2301
\begin{proof}
\begin{center}
\begin{tabular}{rcll}
2301 / 16 & = & 143, & remainder = 13 = D \\
143 / 16 & = & 8, & remainder = 15 = F \\
8 / 16 & = & 0, & remainder = 8
\end{tabular}
\end{center}
So $2301_{10} = 8FD_{16}$.
\end{proof}

\subsection{Exercise 91}
Write a formal version of the algorithm you developed for exercise 87.\\
{\it Proof:}
\begin{tcolorbox}[colframe=cyan]
{\bf \cy Decimal to Hexadecimal Conversion Using Repeated Division by 16} 

{\bf \cy Input:} $a$ {\it[a nonnegative integer]}

\begin{tabbing}
{\bf\cy Alg}\={\bf\cy orithm Body:} \\
         \> $q \coloneqq a, i \coloneqq 0$ \\
         \> {\bf wh}\={\bf ile} ($i = 0$ or $q \neq 0$) \\
         \>          \> $r[i] \coloneqq q \mod 16$ \\
         \>          \> $q \coloneqq q \text{ div } 16$ \\
         \>          \> {\it [$r[i]$ and $q$ can be obtained by calling the division algorithm.]} \\
         \> {\bf end while} \\
         \> {\it [After execution of this step, the values $r[0], r[1], \ldots, r[i-1]$ are all 0's and 1's,} \\ 
         \> {\it and $a = (r[i-1] r[i-2] \ldots r[1] r[0])_{16}$].} \\
{\bf\cy Output:} $r[0], r[1], \ldots, r[i-1] $ {\it [a sequence of integers]}
\end{tabbing}
\end{tcolorbox}

\section{Exercise Set 5.2}

\subsection{Exercise 1}
Use the technique illustrated at the beginning of this section to show that the statements in (a) and (b) are true.

\subsubsection{(a)}
If $\dps \left(1 - \frac{1}{2}\right)\left(1 - \frac{1}{3}\right)\left(1 - \frac{1}{4}\right)\left(1 - \frac{1}{5}\right) = \frac{1}{5}$ then 

$\dps \left(1 - \frac{1}{2}\right)\left(1 - \frac{1}{3}\right)\left(1 - \frac{1}{4}\right)\left(1 - \frac{1}{5}\right)\left(1 - \frac{1}{6}\right) = \frac{1}{6}$.

\begin{proof}
The statement in part (a) is true because if 
\[
\dps \left(1 - \frac{1}{2}\right)\left(1 - \frac{1}{3}\right)\left(1 - \frac{1}{4}\right)\left(1 - \frac{1}{5}\right) = \frac{1}{5}
\]
then
\[\dps \left(1 - \frac{1}{2}\right)\left(1 - \frac{1}{3}\right)\left(1 - \frac{1}{4}\right)\left(1 - \frac{1}{5}\right)\left(1 - \frac{1}{6}\right) = \frac{1}{5} \cdot \frac{5}{6} = \frac{1}{6}.
\]
\end{proof}

\subsubsection{(b)}
If $\dps \left(1 - \frac{1}{2}\right)\left(1 - \frac{1}{3}\right)\left(1 - \frac{1}{4}\right)\left(1 - \frac{1}{5}\right)\left(1 - \frac{1}{6}\right) = \frac{1}{6}$ then 

$\dps \left(1 - \frac{1}{2}\right)\left(1 - \frac{1}{3}\right)\left(1 - \frac{1}{4}\right)\left(1 - \frac{1}{5}\right)\left(1 - \frac{1}{6}\right)\left(1 - \frac{1}{7}\right) = \frac{1}{7}$.

\begin{proof}
The statement in part (a) is true because if 
\[
\dps \left(1 - \frac{1}{2}\right)\left(1 - \frac{1}{3}\right)\left(1 - \frac{1}{4}\right)\left(1 - \frac{1}{5}\right)\left(1 - \frac{1}{6}\right) = \frac{1}{6}
\]
then
\[\dps \left(1 - \frac{1}{2}\right)\left(1 - \frac{1}{3}\right)\left(1 - \frac{1}{4}\right)\left(1 - \frac{1}{5}\right)\left(1 - \frac{1}{6}\right)\left(1 - \frac{1}{7}\right) = \frac{1}{6} \cdot \frac{6}{7} = \frac{1}{7}.
\]
\end{proof}

\subsection{Exercise 2}
For each positive integer $n$, let $P(n)$ be the formula
\[
1 + 3 + 5 + \cdots + (2n - 1) = n^2.
\]
\subsubsection{(a)}
Write $P(1)$. Is $P(1)$ true?

\begin{proof}
$P(1)$ is the equation $1 = 1^2$, which is true.
\end{proof}

\subsubsection{(b)}
Write $P(k)$.

\begin{proof}
$P(k)$ is the equation $1 + 3 + 5 + \cdots + (2k - 1) = k^2$.
\end{proof}

\subsubsection{(c)}
Write $P(k + 1)$.

\begin{proof}
$P(k + 1)$ is the equation $1 + 3 + 5 + \cdots + (2(k + 1) - 1) = (k + 1)^2$.
\end{proof}

\subsubsection{(d)}
In a proof by mathematical induction that the formula holds for every integer $n \geq 1$, what must be shown in the inductive step?

\begin{proof}
In the inductive step, show that if $k$ is any integer for which $k \geq 1$ and $1 + 3 + 5 + \cdots + (2k - 1) = k^2$ is true, then $1 + 3 + 5 + \cdots + (2(k + 1) - 1) = (k + 1)^2$ is also true.
\end{proof}

\subsection{Exercise 3}
For each positive integer $n$, let $P(n)$ be the formula
\[
1^2 + 2^2 + \cdots + n^2 = \frac{n(n+1)(2n+1)}{6}
\]
\subsubsection{(a)}
Write $P(1)$. Is $P(1)$ true?

\begin{proof}
$P(1)$ is ``$1^2 = \frac{1(1+1)(2\cdot 1 + 1)}{6}$.'' $P(1)$ is true because the left-hand side equals $1^2 = 1$ and the right-hand side equals $\frac{1(1+1)(2 + 1)}{6} = \frac{2 \cdot 3}{6} = 1$ also.
\end{proof}

\subsubsection{(b)}
Write $P(k)$.

\begin{proof}
$P(k)$ is ``$\dps k^2 = \frac{k(k+1)(2k + 1)}{6}$.''
\end{proof}

\subsubsection{(c)}
Write $P(k + 1)$.

\begin{proof}
$P(k+1)$ is ``$\dps (k+1)^2 = \frac{(k+1)(k+2)(2k + 3)}{6}$.''
\end{proof}

\subsubsection{(d)}
In a proof by mathematical induction that the formula holds for every integer $n \geq 1$, what must be shown in the inductive step?

\begin{proof}
In the inductive step, show that if $k$ is any integer for which $k \geq 1$ and $\dps k^2 = \frac{k(k+1)(2k + 1)}{6}$ is true, then $\dps (k+1)^2 = \frac{(k+1)(k+2)(2k + 3)}{6}$ is also true.
\end{proof}

\subsection{Exercise 4}
For each integer $n$ with $n \geq 2$, let $P(n)$ be the formula
\[
\sum_{i=1}^{n-1}i(i+1) = \frac{n(n-1)(n+1)}{3}
\]
\subsubsection{(a)}
Write $P(2)$. Is $P(2)$ true?

\begin{proof}
$P(2)$ is ``$\dps \sum_{i=1}^{1}i(i+1) = \frac{2(2-1)(2+1)}{3}$'' It's true because the left-hand side is $1(1+1) = 2$ and the right-hand side is $\dps \frac{2(1)(3)}{3} = 2$ also.
\end{proof}

\subsubsection{(b)}
Write $P(k)$.

\begin{proof}
$P(k)$ is ``$\dps \sum_{i=1}^{k-1}i(i+1) = \frac{k(k-1)(k+1)}{3}$.''
\end{proof}

\subsubsection{(c)}
Write $P(k + 1)$.

\begin{proof}
$P(k+1)$ is ``$\dps \sum_{i=1}^{k}i(i+1) = \frac{(k+1)k(k+2)}{3}$.''
\end{proof}

\subsubsection{(d)}
In a proof by mathematical induction that the formula holds for every integer $n \geq 2$, what must be shown in the inductive step?

\begin{proof}
In the inductive step, show that if $k$ is any integer for which $k \geq 2$ and \\ $\dps \sum_{i=1}^{k-1}i(i+1) = \frac{k(k-1)(k+1)}{3}$ is true, then $\dps \sum_{i=1}^{k}i(i+1) = \frac{(k+1)k(k+2)}{3}$ is also true.
\end{proof}

\subsection{Exercise 5}
Fill in the missing pieces in the following proof that
\[
1 + 3 + 5 + \cdots + (2n - 1) = n^2
\]
for every integer $n \geq 1$.

{\bf Proof:} Let the property $P(n)$ be the equation 
\[
1 + 3 + 5 + \cdots + (2n - 1) = n^2. \,\,\,{\cy \from P(n)}
\]
{\bf Show that $P(1)$ is true:} To establish $P(1)$, we must show that when 1 is substituted in place of $n$, the left-hand side equals the right-hand side. But when $n = 1$, the left-hand side is the sum of all the odd integers from 1 to $2\cdot 1 - 1$, which is the sum of the odd integers from 1 to 1 and is just 1. The right-hand side is {\cy (a) \fbl}, which also equals 1. So $P(1)$ is true. 

{\bf Show that for every integer $k \geq 1$, if $P(k)$ is true then $P(k+1)$ is true:} Let $k$ be any integer with $k \geq 1$. 

{\it [Suppose $P(k)$ is true. That is:]} Suppose 
\[
1 + 3 + 5 + \cdots + (2k - 1) = {\cy (b) \fbl \from P(k)}
\]
{\it [This is the inductive hypothesis.]} 

{\it [We must show that $P(k + 1)$ is true. That is:]} We must show that 
\[
{\cy (c) \fbl} = {\cy (d) \fbl \from P(k + 1)}
\]
Now the left-hand side of $P(k + 1)$ is $1 + 3 + 5 + \cdots + (2(k + 1) - 1)$ 
\[
\begin{array}{lll}
= & 1 + 3 + 5 + \cdots + (2k + 1) & \text{\cy by algebra} \\
= & [1 + 3 + 5 + \cdots + (2k - 1)] + (2k+1) & \text{\cy because (e) \fbl} \\
= & k^2 + (2k+1) & \text{\cy by (f) \fbl} \\
= & (k + 1)^2 & \text{\cy by algebra,}
\end{array}
\]
which is the right-hand side of $P(k+1)$ {\it [as was to be shown.]}

{\it [Since we have proved the basis step and the inductive step, we conclude that the given statement is true.]}

{\it Note:} This proof was annotated to help make its logical flow more obvious. In standard mathematical writing, such annotation is omitted.

\begin{proof}
a. $1^2$; b. $k^2$; c. $1 + 3 + 5 + \cdots + [2(k + 1) - 1]$; d. $(k + 1)^2$; e. the next-to-last term is $2k-1$ because the odd integer just before $2k + 1$ is $2k - 1$; f. inductive hypothesis
\end{proof}

{\bf\cy Prove each statement in $6-9$ using mathematical induction. Do not derive them from theorem 5.2.1 or Theorem 5.2.2.}

\subsection{Exercise 6}
For every integer $n \geq 1$,
\[
2 + 4 + 6 + \cdots + 2n = n^2 + n.
\]
\begin{proof}
For the given statement, the property $P(n)$ is the equation
\[
2 + 4 + 6 + \cdots + 2n = n^2 + n. \,\,\, {\cy \from P(n)}
\]
{\bf Show that $P(1)$ is true:} To prove $P(1)$, we must show that when 1 is substituted into the equation in place of $n$, the left-hand side equals the right-hand side. But when 1 is substituted for $n$, the left-hand side is the sum of all the even integers from 2 to $2 \geq 1$, which is just 2, and the right-hand side is $1^2 + 1$, which also equals 2. Thus $P(1)$ is true.

{\bf Show that for every integer $k \geq 1$, if $P(k)$ is true then $P(k + 1)$ is true:}

Let $k$ be any integer with $k \geq 1$, and suppose $P(k)$ is true. That is, suppose
\[
2 + 4 + 6 + \cdots + 2k = k^2 + k. \,\,\, {\cy \from P(k) \text{: inductive hypothesis}}
\]
We must show that $P(k + 1)$ is true. That is, we must show that
\[
2 + 4 + 6 + \cdots + 2(k + 1) = (k + 1)^2 + (k + 1).
\]
Because $(k + 1)^2 + (k + 1) = k^2 + 2k + 1 + k + 1 =
k^2 + 3k + 2$, this is equivalent to showing that
\[
2 + 4 + 6 + \cdots + 2(k + 1) = k^2 + 3k + 2. \,\, \from P(k + 1)
\]
Now the left-hand side of $P(k + 1)$ is $2 + 4 + 6 + \cdots + 2(k + 1)$
\[
\begin{array}{lll}
= & 2 + 4 + 6 + \cdots + 2k + 2(k + 1) & \text{\cy make next-to-last term explicit} \\
= & (k^2+k) + 2(k+1) & \text{\cy by inductive hypothesis} \\
= & k^2 + 3k + 2 & \text{\cy by algebra,} 
\end{array}
\]
and this is the right-hand side of $P(k + 1)$. Hence $P(k + 1)$ is true. {\it [Since both the basis step and the inductive step have been proved, $P(n)$ is true for every integer $n \geq 1$.]}
\end{proof}

\subsection{Exercise 7}
For every integer $n \geq 1$,
\[
1 + 6 + 11 + 16 + \cdots + (5n-4) = \frac{n(5n-3)}{2}.
\]
\begin{proof}
For the given statement, the property $P(n)$ is the equation
\[
1 + 6 + 11 + 16 + \cdots + (5n-4) = \frac{n(5n-3)}{2}. \,\,\, {\cy \from P(n)}
\]
{\bf Show that $P(1)$ is true:} To prove $P(1)$, we must show that when 1 is substituted into the equation in place of $n$, the left-hand side equals the right-hand side. But when 1 is substituted for $n$, the left-hand side is the sum from 1 to $1 \geq 1$, which is just 1, and the right-hand side is $\frac{1(5-3)}{2}$, which also equals 1. Thus $P(1)$ is true.

{\bf Show that for every integer $k \geq 1$, if $P(k)$ is true then $P(k + 1)$ is true:}

Let $k$ be any integer with $k \geq 1$, and suppose $P(k)$ is true. That is, suppose
\[
1 + 6 + 11 + 16 + \cdots + (5k-4) = \frac{k(5k-3)}{2}. \,\,\, {\cy \from P(k) \text{: inductive hypothesis}}
\]
We must show that $P(k + 1)$ is true. That is, we must show that
\[
1 + 6 + 11 + 16 + \cdots + (5(k+1)-4) = \frac{(k+1)(5(k+1)-3)}{2}.
\]
Because $5(k + 1) - 4 = 5k+1$ and $5(k + 1) - 3 = 5k+2$, this is equivalent to showing that
\[
1 + 6 + 11 + 16 + \cdots + (5k+1) = \frac{(k+1)(5k+2)}{2}. \,\, {\cy \from P(k + 1)}
\]
Now the left-hand side of $P(k + 1)$ is $1 + 6 + 11 + 16 + \cdots + (5k+1)$

\[
\begin{array}{lll}
= & \dps 1 + 6 + 11 + 16 + \cdots + (5k-4) + (5k+1)  & \text{\cy make next-to-last term explicit} \\
= & \dps \frac{k(5k-3)}{2} + (5k+1) & \text{\cy by inductive hypothesis} \\
= & \dps \frac{5k^2-3k}{2} + \frac{10k+2}{2} & \text{\cy by algebra} \\
= & \dps \frac{5k^2-3k + 10k + 2}{2} & \text{\cy by algebra} \\
= & \dps \frac{5k^2 + 7k + 2}{2} & \text{\cy by algebra} \\
= & \dps \frac{(k+1)(5k+2)}{2} & \text{\cy by factoring,}
\end{array}
\]
and this is the right-hand side of $P(k + 1)$. Hence $P(k + 1)$ is true. {\it [Since both the basis step and the inductive step have been proved, $P(n)$ is true for every integer $n \geq 1$.]}
\end{proof}

\subsection{Exercise 8}
For every integer $n \geq 0$,
\[
1 + 2 + 2^2 + \cdots + 2^n = 2^{n+1} - 1.
\]
\begin{proof}
For the given statement, the property $P(n)$ is the equation
\[
1 + 2 + 2^2 + \cdots + 2^n = 2^{n+1} - 1. \,\,\, {\cy \from P(n)}
\]
{\bf Show that $P(0)$ is true:} The left-hand side of $P(0)$ is 1, and the right-hand side is $2^{0+1} - 1 = 2 - 1 = 1$ also. Thus $P(0)$ is true.

{\bf Show that for every integer $k \geq 0$, if $P(k)$ is true then $P(k + 1)$ is true:}

Let $k$ be any integer with $k \geq 0$, and suppose $P(k)$ is true. That is, suppose
\[
1 + 2 + 2^2 + \cdots + 2^k = 2^{k+1} - 1. \,\,\, {\cy \from P(k) \text{: inductive hypothesis}}
\]
We must show that $P(k + 1)$ is true. That is, we must show that
\[
1 + 2 + 2^2 + \cdots + 2^{k+1} = 2^{(k+1)+1} - 1,
\]
or, equivalently,
\[
1 + 2 + 2^2 + \cdots + 2^{k+1} = 2^{k+2} - 1. \,\, {\cy \from P(k + 1)}
\]
Now the left-hand side of $P(k + 1)$ is $1 + 2 + 2^2 + \cdots + 2^{k+1}$
\[
\begin{array}{lll}
= & 1 + 2 + 2^2 + \cdots + 2^k + 2^{k+1} & \text{\cy make next-to-last term explicit} \\
= & (2^{k+1} - 1) + 2^{k+1} & \text{\cy by inductive hypothesis} \\
= & 2 \cdot 2^{k+1} - 1& \text{\cy by combining like terms} \\
= & 2^{k+2} - 1& \text{\cy by the laws of exponents} 
\end{array}
\]
and this is the right-hand side of $P(k + 1)$. Hence $P(k + 1)$ is true. {\it [Since both the basis step and the inductive step have been proved, $P(n)$ is true for every integer $n \geq 0$.]}
\end{proof}

\subsection{Exercise 9}
For every integer $n \geq 3$,
\[
4^3 + 4^4 + 4^5 + \cdots + 4^n = \frac{4(4^n-16)}{3}.
\]
\begin{proof}
For the given statement, the property $P(n)$ is the equation
\[
4^3 + 4^4 + 4^5 + \cdots + 4^n = \frac{4(4^n-16)}{3}. \,\,\, {\cy \from P(n)}
\]
{\bf Show that $P(3)$ is true:} The left-hand side of $P(3)$ is $4^3 = 64$, and the right-hand side is $\frac{4(4^3-16)}{3} = 4 \cdot 48 / 3 = 64$ also. Thus $P(3)$ is true.

{\bf Show that for every integer $k \geq 3$, if $P(k)$ is true then $P(k + 1)$ is true:}

Let $k$ be any integer with $k \geq 3$, and suppose $P(k)$ is true. That is, suppose
\[
4^3 + 4^4 + 4^5 + \cdots + 4^k = \frac{4(4^k-16)}{3}. \,\,\, {\cy \from P(k) \text{: inductive hypothesis}}
\]
We must show that $P(k + 1)$ is true. That is, we must show that
\[
4^3 + 4^4 + 4^5 + \cdots + 4^{k+1} = \frac{4(4^{k+1}-16)}{3}. \,\, {\cy \from P(k+1)}
\]
Now the left-hand side of $P(k + 1)$ is $4^3 + 4^4 + 4^5 + \cdots + 4^{k+1}$
\[
\begin{array}{lll}
= & 4^3 + 4^4 + 4^5 + \cdots + 4^k + 4^{k+1} & \text{\cy make next-to-last term explicit} \\
= & \dps \frac{4(4^k-16)}{3} + 4^{k+1} & \text{\cy by inductive hypothesis} \\
= & \dps \frac{4(4^k-16)}{3} + 4 \cdot 4^k & \text{\cy by the laws of exponents} \\
= & \dps 4\left(\frac{4^k-16}{3} + 4^k \right) & \text{\cy by factoring} \\
= & \dps 4\left(\frac{4^k-16}{3} + \frac{3 \cdot 4^k}{3}\right) & \text{\cy by algebra} \\
= & \dps 4\left(\frac{4^k-16 + 3 \cdot 4^k}{3}\right) & \text{\cy by algebra} \\
= & \dps 4\left(\frac{4 \cdot 4^k-16}{3}\right) & \text{\cy by combining like terms} \\
= & \dps 4\left(\frac{4^{k+1}-16}{3}\right) & \text{\cy by the laws of exponents} 
\end{array}
\]
and this is the right-hand side of $P(k + 1)$. Hence $P(k + 1)$ is true. {\it [Since both the basis step and the inductive step have been proved, $P(n)$ is true for every integer $n \geq 3$.]}
\end{proof}

{\bf\cy Prove each of the statements in $10-18$ by mathematical induction.}

\subsection{Exercise 10}
$\dps 1^2 + 2^2 + \cdots + n^2 = \frac{n(n+1)(2n+1)}{6}$, for every integer $n \geq 1$.

\begin{proof}
For the given statement, the property $P(n)$ is the equation
\[
1^2 + 2^2 + \cdots + n^2 = \frac{n(n+1)(2n+1)}{6}. \,\,\, {\cy \from P(n)}
\]
{\bf Show that $P(1)$ is true:} The left-hand side of $P(1)$ is $1^2 = 1$, and the right-hand side is $\dps \frac{1(1+1)(2+1)}{6} = \frac{2 \cdot 3}{6} = 1$ also. Thus $P(1)$ is true.

{\bf Show that for every integer $k \geq 1$, if $P(k)$ is true then $P(k + 1)$ is true:}

Let $k$ be any integer with $k \geq 1$, and suppose $P(k)$ is true. That is, suppose
\[
1^2 + 2^2 + \cdots + k^2 = \frac{k(k+1)(2k+1)}{6}. \,\,\, {\cy \from P(k) \text{: inductive hypothesis}}
\]
We must show that $P(k + 1)$ is true. That is, we must show that
\[
1^2 + 2^2 + \cdots + (k+1)^2 = \frac{(k+1)((k+1)+1)(2(k+1)+1)}{6},
\]
or, equivalently,
\[
1^2 + 2^2 + \cdots + (k+1)^2 = \frac{(k+1)(k+2)(2k+3)}{6}. \,\,{\cy \from P(k+1)}
\]
Now the left-hand side of $P(k + 1)$ is $1^2 + 2^2 + \cdots + (k+1)^2$
\[
\begin{array}{lll}
= & 1^2 + 2^2 + \cdots + k^2 + (k+1)^2 & \text{\cy make next-to-last term explicit} \\
= & \dps \frac{k(k+1)(2k+1)}{6} + (k+1)^2 & \text{\cy by inductive hypothesis} \\
= & \dps \frac{k(k+1)(2k+1)}{6} + \frac{6(k+1)^2}{6} & \text{\cy because $\frac{6}{6} = 1$} \\
= & \dps \frac{k(k+1)(2k+1) + 6(k+1)^2}{6} & \text{\cy by adding fractions} \\
= & \dps \frac{(k+1)[k(2k+1) + 6(k+1)]}{6} & \text{\cy by factoring out $(k+1)$} \\
= & \dps \frac{(k+1)[2k^2+k + 6k+6]}{6} & \text{\cy by multiplying out} \\
= & \dps \frac{(k+1)[2k^2+7k+6]}{6} & \text{\cy by combining like terms} \\
= & \dps \frac{(k+1)(k+2)(2k+3)}{6} & \text{\cy by factoring} \\
\end{array}
\]
and this is the right-hand side of $P(k + 1)$. Hence $P(k + 1)$ is true. {\it [Since both the basis step and the inductive step have been proved, $P(n)$ is true for every integer $n \geq 1$.]}
\end{proof}

\subsection{Exercise 11}
$\dps 1^3 + 2^3 + \cdots + n^3 = \left[\frac{n(n+1)}{2}\right]^2$, for every integer $n \geq 1$.

\begin{proof}
For the given statement, the property $P(n)$ is the equation
\[
\dps 1^3 + 2^3 + \cdots + n^3 = \left[\frac{n(n+1)}{2}\right]^2. \,\,\, {\cy \from P(n)}
\]
{\bf Show that $P(1)$ is true:} The left-hand side of $P(1)$ is $1^3 = 1$, and the right-hand side is $\dps \left[ \frac{1(1+1)}{2}\right]^2 = 1^2 = 1$ also. Thus $P(1)$ is true.

{\bf Show that for every integer $k \geq 1$, if $P(k)$ is true then $P(k + 1)$ is true:}

Let $k$ be any integer with $k \geq 1$, and suppose $P(k)$ is true. That is, suppose
\[
\dps 1^3 + 2^3 + \cdots + k^3 = \left[\frac{k(k+1)}{2}\right]^2. \,\,\, {\cy \from P(k) \text{: inductive hypothesis}}
\]
We must show that $P(k + 1)$ is true. That is, we must show that
\[
\dps 1^3 + 2^3 + \cdots + (k+1)^3 = \left[\frac{(k+1)(k+2)}{2}\right]^2. \,\,\,{\cy \from P(k+1)}
\]
Now the left-hand side of $P(k + 1)$ is $1^3 + 2^3 + \cdots + (k+1)^3$
\[
\begin{array}{lll}
= & 1^3 + 2^3 + \cdots + k^3+ (k+1)^3 & \text{\cy make next-to-last term explicit} \\
= & \dps \left[\frac{k(k+1)}{2}\right]^2 + (k+1)^3 & \text{\cy by inductive hypothesis} \\
= & \dps \frac{k^2(k+1)^2}{4} + (k+1)(k+1)^2 & \text{\cy by algebra} \\
= & \dps \frac{k^2(k+1)^2}{4} + \frac{4(k+1)(k+1)^2}{4} & \text{\cy because $\frac{4}{4} = 1$} \\
= & \dps \frac{k^2(k+1)^2 + 4(k+1)(k+1)^2}{4} & \text{\cy by adding fractions} \\
= & \dps \frac{(k+1^2)[k^2 + 4(k+1)]}{4} & \text{\cy by factoring out $(k+1)^2$}  \\
= & \dps \frac{(k+1)^2[k^2+4k+4]}{4} & \text{\cy by multiplying out} \\
= & \dps \frac{(k+1)^2(k+2)^2}{4} & \text{\cy by factoring} \\
= & \dps \left[\frac{(k+1)(k+2)}{2}\right]^2 & \text{\cy by factoring}
\end{array}
\]
and this is the right-hand side of $P(k + 1)$. Hence $P(k + 1)$ is true. {\it [Since both the basis step and the inductive step have been proved, $P(n)$ is true for every integer $n \geq 1$.]}
\end{proof}

\subsection{Exercise 12}
$\dps \frac{1}{1 \cdot 2} + \frac{1}{2 \cdot 3} + \cdots + \frac{1}{n(n+1)} = \frac{n}{n+1}$, for every integer $n \geq 1$.

\begin{proof}
For the given statement, the property $P(n)$ is the equation
\[
\dps \frac{1}{1 \cdot 2} + \frac{1}{2 \cdot 3} + \cdots + \frac{1}{n(n+1)} = \frac{n}{n+1}. \,\,\, {\cy \from P(n)}
\]
{\bf Show that $P(1)$ is true:} The left-hand side of $P(1)$ is $\frac{1}{1\cdot (1+1)} = 1/2$, and the right-hand side is $\frac{1}{1+1} = 1/2$ also. Thus $P(1)$ is true.

{\bf Show that for every integer $k \geq 1$, if $P(k)$ is true then $P(k + 1)$ is true:}

Let $k$ be any integer with $k \geq 1$, and suppose $P(k)$ is true. That is, suppose
\[
\dps \frac{1}{1 \cdot 2} + \frac{1}{2 \cdot 3} + \cdots + \frac{1}{k(k+1)} = \frac{k}{k+1}. \,\,\, {\cy \from P(k) \text{: inductive hypothesis}}
\]
We must show that $P(k + 1)$ is true. That is, we must show that
\[
\dps \frac{1}{1 \cdot 2} + \frac{1}{2 \cdot 3} + \cdots + \frac{1}{(k+1)((k+1)+1)} = \frac{k+1}{(k+1)+1}. \,\,\,{\cy \from P(k+1)}
\]
Now the left-hand side of $P(k + 1)$ is $\dps \frac{1}{1 \cdot 2} + \frac{1}{2 \cdot 3} + \cdots + \frac{1}{(k+1)(k+2)}$
\[
\begin{array}{lll}
= & \dps \frac{1}{1 \cdot 2} + \frac{1}{2 \cdot 3} + \cdots + \frac{1}{k(k+1)} + \frac{1}{(k+1)(k+2)} & \text{\cy make next-to-last term explicit} \\
= & \dps \frac{k}{k+1} + \frac{1}{(k+1)(k+2)} & \text{\cy by inductive hypothesis} \\
= & \dps \frac{k(k+2)}{(k+1)(k+2)} + \frac{1}{(k+1)(k+2)} & \text{\cy because $\frac{k+2}{k+2} = 1$} \\
= & \dps \frac{k^2+2k}{(k+1)(k+2)} + \frac{1}{(k+1)(k+2)} & \text{\cy by algebra} \\
= & \dps \frac{k^2+2k+1}{(k+1)(k+2)} & \text{\cy by algebra} \\
= & \dps \frac{(k+1)^2}{(k+1)(k+2)} & \text{\cy by algebra} \\
= & \dps \frac{k+1}{k+2} & \text{\cy by canceling $k+1$}
\end{array}
\]
and this is the right-hand side of $P(k + 1)$. Hence $P(k + 1)$ is true. {\it [Since both the basis step and the inductive step have been proved, $P(n)$ is true for every integer $n \geq 1$.]}
\end{proof}

\subsection{Exercise 13}
$\dps \sum_{i=1}^{n-1}i(i+1) = \frac{n(n-1)(n+1)}{3}$, for every integer $n \geq 2$.

\begin{proof}
For the given statement, the property $P(n)$ is the equation
\[
\dps \sum_{i=1}^{n-1}i(i+1) = \frac{n(n-1)(n+1)}{3}. \,\,\, {\cy \from P(n)}
\]
{\bf Show that $P(2)$ is true:} The left-hand side of $P(1)$ is $\dps \sum_{i=1}^{2-1}i(i+1) = 1(1+1) = 2$, and the right-hand side is $\dps \frac{2(2-1)(2+1)}{3} = \frac{6}{3} = 2$ also. Thus $P(2)$ is true.

{\bf Show that for every integer $k \geq 2$, if $P(k)$ is true then $P(k + 1)$ is true:}

Let $k$ be any integer with $k \geq 2$, and suppose $P(k)$ is true. That is, suppose
\[
\dps \sum_{i=1}^{k-1}i(i+1) = \frac{k(k-1)(k+1)}{3}. \,\,\, {\cy \from P(k) \text{: inductive hypothesis}}
\]
We must show that $P(k + 1)$ is true. That is, we must show that
\[
\dps \sum_{i=1}^{k+1-1}i(i+1) = \frac{(k+1)(k+1-1)(k+1+1)}{3},
\]
or, equivalently,
\[
\dps \sum_{i=1}^{k}i(i+1) = \frac{(k+1)k(k+2)}{3}. \,\,\,{\cy \from P(k+1)}
\]
Now the left-hand side of $P(k + 1)$ is $\dps \sum_{i=1}^{k}i(i+1)$
\[
\begin{array}{lll}
= & \dps \sum_{i=1}^{k-1}i(i+1) + k(k+1) & \text{\cy make next-to-last term explicit} \\
= & \dps \frac{k(k-1)(k+1)}{3} + k(k+1) & \text{\cy by inductive hypothesis} \\
= & \dps \frac{k(k-1)(k+1)}{3} + \frac{3k(k+1)}{3} & \text{\cy because $\frac{3}{3} = 1$} \\
= & \dps \frac{k(k-1)(k+1) + 3k(k+1)}{3} & \text{\cy by adding fractions} \\
= & \dps \frac{k(k+1)[(k-1) + 3]}{3} & \text{\cy by factoring out $k(k+1)$} \\
= & \dps \frac{k(k+1)(k+2)}{3} & \text{\cy by algebra}
\end{array}
\]
and this is the right-hand side of $P(k + 1)$. Hence $P(k + 1)$ is true. {\it [Since both the basis step and the inductive step have been proved, $P(n)$ is true for every integer $n \geq 2$.]}
\end{proof}

\subsection{Exercise 14}
$\dps \sum_{i=1}^{n+1}i \cdot 2^i = n \cdot 2^{n+2} + 2$, for every integer $n \geq 0$.

\begin{proof}
For the given statement, the property $P(n)$ is the equation
\[
\dps \sum_{i=1}^{n+1}i \cdot 2^i = n \cdot 2^{n+2} + 2. \,\,\, {\cy \from P(n)}
\]
{\bf Show that $P(0)$ is true:} The left-hand side of $P(0)$ is $\dps \sum_{i=1}^{0+1}i \cdot 2^i = 1 \cdot 2^i = 2$, and the right-hand side is $0 \cdot 2^{0+2} + 2 = 0 + 2 = 2$ also. Thus $P(0)$ is true.

{\bf Show that for every integer $k \geq 0$, if $P(k)$ is true then $P(k + 1)$ is true:}

Let $k$ be any integer with $k \geq 0$, and suppose $P(k)$ is true. That is, suppose
\[
\dps \sum_{i=1}^{k+1}i \cdot 2^i = k \cdot 2^{k+2} + 2. \,\,\, {\cy \from P(k) \text{: inductive hypothesis}}
\]
We must show that $P(k + 1)$ is true. That is, we must show that
\[
\dps \sum_{i=1}^{(k+1)+1}i \cdot 2^i = (k+1) \cdot 2^{(k+1) +2} + 2,
\]
or, equivalently,
\[
\dps \sum_{i=1}^{k+2}i \cdot 2^i = (k+1) \cdot 2^{k+3} + 2. \,\,\,{\cy \from P(k+1)}.
\]
Now the left-hand side of $P(k + 1)$ is $\sum_{i=1}^{k+2}i \cdot 2^i$
\[
\begin{array}{lll}
= & \dps \sum_{i=1}^{k+1}i \cdot 2^i + (k+2) \cdot 2^{k+2} & \text{\cy make next-to-last term explicit} \\
= & k \cdot 2^{k+2} + 2 + (k+2) \cdot 2^{k+2} & \text{\cy by inductive hypothesis} \\
= & (2k+2) \cdot 2^{k+2} + 2 & \text{\cy by combining like terms} \\
= & 2(k+1) \cdot 2^{k+2} + 2 & \text{\cy by factoring out 2} \\
= & (k+1) \cdot 2^{k+3} + 2 & \text{\cy by laws of exponents} 
\end{array}
\]
and this is the right-hand side of $P(k + 1)$. Hence $P(k + 1)$ is true. {\it [Since both the basis step and the inductive step have been proved, $P(n)$ is true for every integer $n \geq 0$.]}
\end{proof}

\subsection{Exercise 15}
$\dps \sum_{i=1}^{n}i(i!) = (n+1)! - 1$, for every integer $n \geq 1$.

\begin{proof}
For the given statement, the property $P(n)$ is the equation
\[
\dps \sum_{i=1}^{n}i(i!) = (n+1)! - 1. \,\,\, {\cy \from P(n)}
\]
{\bf Show that $P(1)$ is true:} The left-hand side of $P(1)$ is $\dps \sum_{i=1}^{1}i(i!) = 1(1!) = 1$, and the right-hand side is $(1+1)! - 1 = 2 - 1 = 1$ also. Thus $P(1)$ is true.

{\bf Show that for every integer $k \geq 1$, if $P(k)$ is true then $P(k + 1)$ is true:}

Let $k$ be any integer with $k \geq 1$, and suppose $P(k)$ is true. That is, suppose
\[
\dps \sum_{i=1}^{k}i(i!) = (k+1)! - 1. \,\,\, {\cy \from P(k) \text{: inductive hypothesis}}
\]
We must show that $P(k + 1)$ is true. That is, we must show that
\[
\dps \sum_{i=1}^{k+1}i(i!) = (k+2)! - 1. \,\,\,{\cy \from P(k+1)}
\]
Now the left-hand side of $P(k + 1)$ is $\sum_{i=1}^{k+1}i(i!)$
\[
\begin{array}{lll}
= & \dps \sum_{i=1}^{k}i(i!) + (k+1)(k+1)! & \text{\cy make next-to-last term explicit} \\
= & (k+1)! - 1 + (k+1)(k+1)! & \text{\cy by inductive hypothesis} \\
= & (k+1)![1 + (k+1)] - 1 & \text{\cy by factoring out $(k+1)!$} \\
= & (k+1)!(k+2) - 1 & \text{\cy by algebra} \\
= & (k+2)! - 1 & \text{\cy by definition of !}
\end{array}
\]
and this is the right-hand side of $P(k + 1)$. Hence $P(k + 1)$ is true. {\it [Since both the basis step and the inductive step have been proved, $P(n)$ is true for every integer $n \geq 1$.]}
\end{proof}

\subsection{Exercise 16}
$\dps \left(1 - \frac{1}{2^2}\right)\left(1 - \frac{1}{3^2}\right) \cdots \left(1 - \frac{1}{n^2}\right) = \frac{n+1}{2n}$, for every integer $n \geq 2$.

\begin{proof}
For the given statement, the property $P(n)$ is the equation
\[
\dps \left(1 - \frac{1}{2^2}\right)\left(1 - \frac{1}{3^2}\right) \cdots \left(1 - \frac{1}{n^2}\right) = \frac{n+1}{2n}. \,\,\, {\cy \from P(n)}
\]
{\bf Show that $P(2)$ is true:} The left-hand side of $P(2)$ is $\dps 1 - \frac{1}{2^2} = 1 - \frac{1}{4} = \frac{3}{4}$, and the right-hand side is $\dps \frac{2+1}{2 \cdot 2} = \frac{3}{4}$ also. Thus $P(2)$ is true.

{\bf Show that for every integer $k \geq 2$, if $P(k)$ is true then $P(k + 1)$ is true:}

Let $k$ be any integer with $k \geq 2$, and suppose $P(k)$ is true. That is, suppose
\[
\dps \left(1 - \frac{1}{2^2}\right)\left(1 - \frac{1}{3^2}\right) \cdots \left(1 - \frac{1}{k^2}\right) = \frac{k+1}{2k}. \,\,\, {\cy \from P(k) \text{: inductive hypothesis}}
\]
We must show that $P(k + 1)$ is true. That is, we must show that
\[
\dps \left(1 - \frac{1}{2^2}\right)\left(1 - \frac{1}{3^2}\right) \cdots \left(1 - \frac{1}{(k+1)^2}\right) = \frac{(k+1)+1}{2(k+1)}. \,\,\,{\cy \from P(k+1)}
\]
Now the left-hand side of $P(k + 1)$ is $\dps \left(1 - \frac{1}{2^2}\right)\left(1 - \frac{1}{3^2}\right) \ldots \left(1 - \frac{1}{(k+1)^2}\right)$
\[
\begin{array}{lll}
= & \dps \left(1 - \frac{1}{2^2}\right)\left(1 - \frac{1}{3^2}\right) \cdots \left(1 - \frac{1}{k^2}\right) \left(1 - \frac{1}{(k+1)^2}\right) & \text{\cy show next-to-last term} \\
= & \dps \frac{k+1}{2k} \cdot \left(1 - \frac{1}{(k+1)^2}\right) & \text{\cy by inductive hypothesis} \\
= & \dps \frac{k+1}{2k} \cdot \left(\frac{(k+1)^2}{(k+1)^2} - \frac{1}{(k+1)^2}\right) & \text{\cy because $\frac{(k+1)^2}{(k+1)^2} = 1$} \\
= & \dps \frac{k+1}{2k} \cdot \frac{(k+1)^2 - 1}{(k+1)^2} & \text{\cy by adding fractions} \\
= & \dps \frac{k+1}{2k} \cdot \frac{k^2+2k+1 - 1}{(k+1)^2} & \text{\cy by algebra} \\
= & \dps \frac{k+1}{2k} \cdot \frac{k^2+2k}{(k+1)^2} & \text{\cy by algebra} \\
= & \dps \frac{k+1}{2k} \cdot \frac{k(k+2)}{(k+1)^2} & \text{\cy by factoring out $k$} \\
= & \dps \frac{k+1}{2} \cdot \frac{k+2}{(k+1)^2} & \text{\cy by canceling out $k$} \\
= & \dps \frac{1}{2} \cdot \frac{k+2}{k+1} & \text{\cy by canceling out $k+1$} \\
= & \dps \frac{k+2}{2(k+1)} & \text{\cy by multiplying fractions} 
\end{array}
\]
and this is the right-hand side of $P(k + 1)$. Hence $P(k + 1)$ is true. {\it [Since both the basis step and the inductive step have been proved, $P(n)$ is true for every integer $n \geq 2$.]}
\end{proof}

\subsection{Exercise 17}
$\dps \prod_{i=0}^{n}\left(\frac{1}{2i+1} \cdot \frac{1}{2i+2}\right) = \frac{1}{(2n+2)!}$, for every integer $n \geq 0$.

\begin{proof}
For the given statement, the property $P(n)$ is the equation
\[
\dps \prod_{i=0}^{n}\left(\frac{1}{2i+1} \cdot \frac{1}{2i+2}\right) = \frac{1}{(2n+2)!}. \,\,\, {\cy \from P(n)}
\]
{\bf Show that $P(0)$ is true:} The left-hand side of $P(0)$ is $\dps \prod_{i=0}^{0}\left(\frac{1}{2i+1} \cdot \frac{1}{2i+2}\right) = \frac{1}{2\cdot0+1} \cdot \frac{1}{2\cdot0+2} = \frac{1}{1}\cdot \frac{1}{2} = \frac{1}{2}$, and the right-hand side is $\dps \frac{1}{(2\cdot 0 +2)!} = \frac{1}{2}$ also. Thus $P(0)$ is true.

{\bf Show that for every integer $k \geq 0$, if $P(k)$ is true then $P(k + 1)$ is true:}

Let $k$ be any integer with $k \geq 0$, and suppose $P(k)$ is true. That is, suppose
\[
\dps \prod_{i=0}^{k}\left(\frac{1}{2i+1} \cdot \frac{1}{2i+2}\right) = \frac{1}{(2k+2)!}. \,\,\, {\cy \from P(k) \text{: inductive hypothesis}}
\]
We must show that $P(k + 1)$ is true. That is, we must show that
\[
\dps \prod_{i=0}^{k+1}\left(\frac{1}{2i+1} \cdot \frac{1}{2i+2}\right) = \frac{1}{(2(k+1)+2)!}. \,\,\,{\cy \from P(k+1)}
\]
Now the left-hand side of $P(k + 1)$ is $\dps \prod_{i=0}^{k+1}\left(\frac{1}{2i+1} \cdot \frac{1}{2i+2}\right)$
\[
\begin{array}{lll}
= & \dps \prod_{i=0}^{k}\left(\frac{1}{2i+1} \frac{1}{2i+2}\right) \left(\frac{1}{2(k+1)+1} \frac{1}{2(k+1)+2}\right) & \text{\cy make next-to-last term explicit} \\
= & \dps \frac{1}{(2k+2)!} \cdot \left(\frac{1}{2(k+1)+1} \cdot \frac{1}{2(k+1)+2}\right) & \text{\cy by inductive hypothesis} \\
= & \dps \frac{1}{(2k+2)!} \cdot \frac{1}{2k+3} \cdot \frac{1}{2k+4} & \text{\cy by algebra} \\
= & \dps \frac{1}{(2k+2)!\cdot(2k+3)\cdot(2k+4)} & \text{\cy by multiplying fractions} \\
= & \dps \frac{1}{(2k+4)!} & \text{\cy by definition of factorial}
\end{array}
\]
and this is the right-hand side of $P(k + 1)$. Hence $P(k + 1)$ is true. {\it [Since both the basis step and the inductive step have been proved, $P(n)$ is true for every integer $n \geq 0$.]}
\end{proof}

\subsection{Exercise 18}
$\dps \prod_{i=2}^{n}\left(1 - \frac{1}{i}\right) = \frac{1}{n}$, for every integer $n \geq 2$.

{\it Hint:} See the discussion at the beginning of this section.

\begin{proof}
For the given statement, the property $P(n)$ is the equation
\[
\dps \prod_{i=2}^{n}\left(1 - \frac{1}{i}\right) = \frac{1}{n}. \,\,\, {\cy \from P(n)}
\]
{\bf Show that $P(2)$ is true:} The left-hand side of $P(2)$ is $\dps \prod_{i=2}^{2}\left(1 - \frac{1}{i}\right) = 1 - \frac{1}{2} = \frac{1}{2}$, and the right-hand side is $\dps \frac{1}{2}$ also. Thus $P(2)$ is true.

{\bf Show that for every integer $k \geq 2$, if $P(k)$ is true then $P(k + 1)$ is true:}

Let $k$ be any integer with $k \geq 2$, and suppose $P(k)$ is true. That is, suppose
\[
\dps \prod_{i=2}^{k}\left(1 - \frac{1}{i}\right) = \frac{1}{k}. \,\,\, {\cy \from P(k) \text{: inductive hypothesis}}
\]
We must show that $P(k + 1)$ is true. That is, we must show that
\[
\dps \prod_{i=2}^{k+1}\left(1 - \frac{1}{i}\right) = \frac{1}{k+1}. \,\,\,{\cy \from P(k+1)}
\]
Now the left-hand side of $P(k + 1)$ is $\dps \prod_{i=2}^{k+1}\left(1 - \frac{1}{i}\right)$
\[
\begin{array}{lll}
= & \dps \prod_{i=2}^{k}\left(1 - \frac{1}{i}\right) \cdot \left(1 - \frac{1}{k+1}\right) & \text{\cy make next-to-last term explicit} \\
= & \dps \frac{1}{k} \cdot \left(1 - \frac{1}{k+1}\right) & \text{\cy by inductive hypothesis} \\
= & \dps \frac{1}{k} \cdot \left(\frac{k+1}{k+1} - \frac{1}{k+1}\right) & \text{\cy because $\frac{k+1}{k+1} = 1$} \\
= & \dps \frac{1}{k} \cdot \frac{k}{k+1} & \text{\cy by adding fractions} \\
= & \dps \frac{1}{k+1} & \text{\cy by canceling $k$}
\end{array}
\]
and this is the right-hand side of $P(k + 1)$. Hence $P(k + 1)$ is true. {\it [Since both the basis step and the inductive step have been proved, $P(n)$ is true for every integer $n \geq 2$.]}
\end{proof}

\subsection{Exercise 19}
(For students who have studied calculus) Use mathematical induction, the product rule from calculus, and the facts that $\dps \frac{d(x)}{dx} = 1$ and that $x^{k+1} = x\cdot x^k$ to prove that for every integer $\dps n \geq 1, \frac{d(x^n)}{dx} = n x^{n-1}$.

\begin{proof}
For the given statement, the property $P(n)$ is the equation
\[
\dps \frac{d(x^n)}{dx} = n x^{n-1}. \,\,\, {\cy \from P(n)}
\]
{\bf Show that $P(1)$ is true:} The left-hand side of $P(1)$ is $\dps \frac{d(x^1)}{dx} = \frac{d(x)}{dx} = 1$, and the right-hand side is $1 \cdot x^{1-1} = 1$ also. Thus $P(1)$ is true.

{\bf Show that for every integer $k \geq 1$, if $P(k)$ is true then $P(k + 1)$ is true:}

Let $k$ be any integer with $k \geq 1$, and suppose $P(k)$ is true. That is, suppose
\[
\dps \frac{d(x^k)}{dx} = k x^{k-1}. \,\,\, {\cy \from P(k) \text{: inductive hypothesis}}
\]
We must show that $P(k + 1)$ is true. That is, we must show that
\[
\dps \frac{d(x^{k+1})}{dx} = (k+1) x^{k+1-1}. \,\,\,{\cy \from P(k+1)}
\]
Now the left-hand side of $P(k+1)$ is $\dps \frac{d(x^{k+1})}{dx}$
\[
\begin{array}{lll}
= & \dps \frac{d(x\cdot x^k)}{dx} & \text{\cy because $x^{k+1} = x\cdot x^k$} \\
= & \dps \frac{d(x)}{dx}\cdot x^k + x\cdot\frac{d(x^k)}{dx} & \text{\cy by the product rule} \\
= & \dps 1 \cdot x^k + x\cdot\frac{d(x^k)}{dx} & \text{\cy because $\frac{d(x)}{dx} = 1$} \\
= & x^k + x\cdot(k x^{k-1}) & \text{\cy by inductive hypothesis} \\
= & x^k + kx^k & \text{\cy by algebra} \\
= & (k+1)x^k & \text{\cy by combining like terms}
\end{array}
\]
and this is the right-hand side of $P(k + 1)$. Hence $P(k + 1)$ is true. {\it [Since both the basis step and the inductive step have been proved, $P(n)$ is true for every integer $n \geq 1$.]}
\end{proof}

{\bf \cy Use the formula for the sum of the first $n$ integers and/or the formula for the sum of a geometric sequence to evaluate the sums in $20-29$ or to write them in closed form.}

\subsection{Exercise 20}
$4 + 8 + 12 + 16 + \cdots + 200$

\begin{proof}
$\dps 4 + 8 + 12 + 16 + \cdots + 200 = 4(1+2+\cdots+50) = 4 \cdot \frac{50 \cdot 51}{2} = 5100$
\end{proof}

\subsection{Exercise 21}
$5 + 10 + 15 + 20 + \cdots + 300$

\begin{proof}
$\dps 5 + 10 + 15 + 20 + \cdots + 300 = 5(1+2+\cdots+60) = 5 \cdot \frac{60 \cdot 61}{2} = 9150$
\end{proof}

\subsection{Exercise 22}
\subsubsection{(a)}
$3 + 4 + 5 + 6 + \cdots + 1000$

\begin{proof}
$\dps 3 + 4 + 5 + 6 + \cdots + 1000 = (1+2+\cdots+1000) - (1+2) = \frac{1000 \cdot 1001}{2} - 3 = 500500-3 = 500497$
\end{proof}

\subsubsection{(b)}
$3 + 4 + 5 + 6 + \cdots + m$

\begin{proof}
$\dps 3 + 4 + 5 + 6 + \cdots + m = (1+2+\cdots+m) - (1+2) = \frac{m(m+1)}{2} - 3$
\end{proof}

\subsection{Exercise 23}
\subsubsection{(a)}
$7 + 8 + 9 + 10 + \cdots + 600$

\begin{proof}
$\dps 7 + 8 + 9 + 10 + \cdots + 600 = 1 + 2 + \cdots + 600 - (1+2+3+4+5+6) = \frac{600 \cdot 601}{2} - \frac{6 \cdot 7}{2} = 18300 - 21 = 18279$
\end{proof}

\subsubsection{(b)}
$7 + 8 + 9 + 10 + \cdots + k$

\begin{proof}
$\dps 7 + 8 + 9 + 10 + \cdots + k = 1 + 2 + \cdots + k - (1+2+3+4+5+6) = \frac{k(k+1)}{2} - 21$
\end{proof}

\subsection{Exercise 24}
$1 + 2 + 3 + \cdots + (k - 1)$, where $k$ is any integer with $k \geq 2$.

\begin{proof}
$\dps 1 + 2 + 3 + \cdots + (k - 1) = \frac{(k-1)(k-1+1)}{2} = \frac{(k-1)k}{2}$
\end{proof}

\subsection{Exercise 25}
$1 + 2 + 2^2 + \cdots + 2^{25}$

\subsubsection{(a)}

\begin{proof}
$\dps 1 + 2 + 2^2 + \cdots + 2^{25} = \frac{2^{26} - 1}{2-1} = 2^{26} - 1$
\end{proof}

\subsubsection{(b)}
$2 + 2^2 + 2^3 + \cdots + 2^{26}$

\begin{proof}
$2 + 2^2 + 2^3 + \cdots + 2^{26} = 2(1 + 2 + 2^2 + \cdots + 2^{25}) = 2(2^{26} - 1) = 2^{27}-2$
\end{proof}

\subsubsection{(c)}
$2 + 2^2 + 2^3 + \cdots + 2^n$

\begin{proof}
$2 + 2^2 + 2^3 + \cdots + 2^{n} = 2(1 + 2 + 2^2 + \cdots + 2^{n-1}) = 2(2^{n} - 1) = 2^{n+1}-2$
\end{proof}

\subsection{Exercise 26}
$3 + 3^2 + 3^3 + \cdots + 3^n$, where $n$ is any integer with $n \geq 1$.

\begin{proof}
$\dps 3 + 3^2 + 3^3 + \cdots + 3^n = 1 + 3 + 3^2 + 3^3 + \cdots + 3^n - 1 = \frac{3^{n+1}-1}{3-1} - 1 = \frac{3^{n+1}-1}{2} - 1$
\end{proof}

\subsection{Exercise 27}
$5^3 + 5^4 + 5^5 + \cdots + 5^k$, where $k$ is any integer with $k \geq 3$.

\begin{proof}
$5^3 + 5^4 + 5^5 + \cdots + 5^k = 1 + 5 + 5^2 + 5^3 + 5^4 + 5^5 + \cdots + 5^k - (1 + 5 + 5^2) = (5^{k+1} - 1) - 31 = 5^{k+1} - 32$
\end{proof}

\subsection{Exercise 28}
$\dps 1 + \frac{1}{2} + \frac{1}{2^2} + \cdots + \frac{1}{2^n}$, where $n$ is any positive integer.

\begin{proof}
$\dps 1 + \frac{1}{2} + \frac{1}{2^2} + \cdots + \frac{1}{2^n} = \frac{\left(\frac{1}{2}\right)^{n+1} - 1}{\frac{1}{2} - 1} = -2\left[\left(\frac{1}{2}\right)^{n+1} - 1\right] = 2 - \frac{1}{2^n}$
\end{proof}

\subsection{Exercise 29}
$1 - 2 + 2^2 - 2^3 + \cdots + (-1)^n 2^n$, where $n$ is any positive integer.

\begin{proof}
$\dps 1 - 2 + 2^2 - 2^3 + \cdots + (-1)^n 2^n = \frac{(-2)^{n+1} - 1}{-2 - 1} = \frac{(-2)^{n+1} - 1}{-3} = \frac{1 - (-2)^{n+1}}{3}$
\end{proof}

\subsection{Exercise 30}
Observe that $\,\,\,\dps \frac{1}{1 \cdot 3} = \dps \frac{1}{3}, \,\,\, \frac{1}{1 \cdot 3} + \frac{1}{3 \cdot 5} = \dps \frac{2}{5}, \,\,\, \frac{1}{1 \cdot 3} + \frac{1}{3 \cdot 5} + \frac{1}{5 \cdot 7} = \dps \frac{3}{7}$, 

$\dps \frac{1}{1 \cdot 3} + \frac{1}{3 \cdot 5} + \frac{1}{5 \cdot 7} + \frac{1}{7 \cdot 9} = \dps \frac{4}{9}$

Guess a general formula and prove it by induction.

\begin{proof}
{\it General formula:} For every integer $n \geq 1$,
\[
\dps \frac{1}{1 \cdot 3} + \frac{1}{3 \cdot 5} + \cdots + \frac{1}{(2n-1)(2n+1)} = \frac{n}{2n+1}
\]
\underline{Proof by mathematical induction:}
For the given statement, the property $P(n)$ is the equation
\[
\dps \frac{1}{1 \cdot 3} + \frac{1}{3 \cdot 5} + \cdots + \frac{1}{(2n-1)(2n+1)} = \frac{n}{2n+1}. \,\,\, {\cy \from P(n)}
\]
{\bf Show that $P(1)$ is true:} The left-hand side of $P(1)$ is $\dps \frac{1}{1 \cdot 3} = \frac{1}{3}$, and the right-hand side is $\dps \frac{1}{2 \cdot 1 + 1} = \frac{1}{3}$ also. Thus $P(1)$ is true.

{\bf Show that for every integer $k \geq 1$, if $P(k)$ is true then $P(k + 1)$ is true:}

Let $k$ be any integer with $k \geq 1$, and suppose $P(k)$ is true. That is, suppose
\[
\dps \frac{1}{1 \cdot 3} + \frac{1}{3 \cdot 5} + \cdots + \frac{1}{(2k-1)(2k+1)} = \frac{k}{2k+1}. \,\,\, {\cy \from P(k) \text{: inductive hypothesis}}
\]
We must show that $P(k + 1)$ is true. That is, we must show that
\[
\dps \frac{1}{1 \cdot 3} + \frac{1}{3 \cdot 5} + \cdots + \frac{1}{(2(k+1)-1)(2(k+1)+1)} = \frac{k+1}{2(k+1)+1}, 
\]
or, equivalently,
\[
\dps \frac{1}{1 \cdot 3} + \frac{1}{3 \cdot 5} + \cdots + \frac{1}{(2k+1)(2k+3)} = \frac{k+1}{2k+3}. \,\,\,{\cy \from P(k+1)}
\]
Now the left-hand side of $P(k + 1)$ is $\dps \frac{1}{1 \cdot 3} + \frac{1}{3 \cdot 5} + \cdots + \frac{1}{(2k+1) (2k+3)}$

\[
\begin{array}{lll}
= & \dps \frac{1}{1 \cdot 3} + \frac{1}{3 \cdot 5} + \cdots + \frac{1}{(2k-1)(2k+1)} + \frac{1}{(2k+1)(2k+3)} & \text{\cy show next-to-last term} \\
= & \dps \frac{k}{2k+1} + \frac{1}{(2k+1)(2k+3)} & \text{\cy by inductive hypothesis} \\
= & \dps \frac{k(2k+3)}{(2k+1)(2k+3)} + \frac{1}{(2k+1)(2k+3)} & \text{\cy because $\frac{2k+3}{2k+3} = 1$} \\
= & \dps \frac{2k^2+3k+1}{(2k+1)(2k+3)} & \text{\cy by adding fractions} \\
= & \dps \frac{(2k+1)(k+1)}{(2k+1)(2k+3)} & \text{\cy by factoring} \\
= & \dps \frac{k+1}{2k+3} & \text{\cy by canceling $2k+1$}
\end{array}
\]
and this is the right-hand side of $P(k + 1)$. Hence $P(k + 1)$ is true. {\it [Since both the basis step and the inductive step have been proved, $P(n)$ is true for every integer $n \geq 1$.]}
\end{proof}

\subsection{Exercise 31}
Compute values of the product
\[
\left(1 + \frac{1}{1}\right)\left(1 + \frac{1}{2}\right)\left(1 + \frac{1}{3}\right) \cdots \left(1 + \frac{1}{n}\right)
\]
for small values of $n$ in order to conjecture a general formula for the product. Prove your conjecture by mathematical induction.

\begin{proof}
$\dps \left(1 + \frac{1}{1}\right) = 2, \left(1 + \frac{1}{1}\right)\left(1 + \frac{1}{2}\right) = \dps 3,
\left(1 + \frac{1}{1}\right)\left(1 + \frac{1}{2}\right)\left(1 + \frac{1}{3}\right) = \dps 4$.

{\it General formula:} For every integer $n \geq 1$,
\[
\dps \left(1 + \frac{1}{1}\right)\left(1 + \frac{1}{2}\right)\left(1 + \frac{1}{3}\right) \cdots \left(1 + \frac{1}{n}\right) = n+1.
\]
\underline{Proof by mathematical induction:}
For the given statement, the property $P(n)$ is the equation
\[
\dps \left(1 + \frac{1}{1}\right)\left(1 + \frac{1}{2}\right)\left(1 + \frac{1}{3}\right) \cdots \left(1 + \frac{1}{n}\right) = n+1. \,\,\, {\cy \from P(n)}
\]
{\bf Show that $P(1)$ is true:} The left-hand side of $P(1)$ is $1 + \frac{1}{1} = 2$, and the right-hand side is $1 + 1 = 2$ also. Thus $P(1)$ is true.

{\bf Show that for every integer $k \geq 1$, if $P(k)$ is true then $P(k + 1)$ is true:}

Let $k$ be any integer with $k \geq 1$, and suppose $P(k)$ is true. That is, suppose
\[
\dps \left(1 + \frac{1}{1}\right)\left(1 + \frac{1}{2}\right)\left(1 + \frac{1}{3}\right) \cdots \left(1 + \frac{1}{k}\right) = k+1. \,\,\, {\cy \from P(k) \text{: inductive hypothesis}}
\]
We must show that $P(k + 1)$ is true. That is, we must show that
\[
\dps \left(1 + \frac{1}{1}\right)\left(1 + \frac{1}{2}\right)\left(1 + \frac{1}{3}\right) \cdots \left(1 + \frac{1}{k+1}\right) = k+2. \,\,\,{\cy \from P(k+1)}
\]
Now the left-hand side of $P(k + 1)$ is $\left(1 + \frac{1}{1}\right)\left(1 + \frac{1}{2}\right)\left(1 + \frac{1}{3}\right) \cdots \left(1 + \frac{1}{k+1}\right)$
\[
\begin{array}{lll}
= & \dps \left(1 + \frac{1}{1}\right)\left(1 + \frac{1}{2}\right)\left(1 + \frac{1}{3}\right) \cdots \left(1 + \frac{1}{k}\right) \left(1 + \frac{1}{k+1}\right) & \text{\cy show next-to-last term} \\
= & \dps (k+1) \cdot \left(1 + \frac{1}{k+1}\right) & \text{\cy by inductive hypothesis} \\
= & \dps (k+1) \cdot \left(\frac{k+1}{k+1} + \frac{1}{k+1}\right) & \text{\cy because $\frac{k+1}{k+1} = 1$} \\
= & \dps (k+1) \cdot \frac{k+2}{k+1} & \text{\cy by adding fractions} \\
= & k+2 & \text{\cy by canceling $k+1$}
\end{array}
\]
and this is the right-hand side of $P(k + 1)$. Hence $P(k + 1)$ is true. {\it [Since both the basis step and the inductive step have been proved, $P(n)$ is true for every integer $n \geq 1$.]}
\end{proof}

\subsection{Exercise 32}
Observe that
\[
\begin{array}{rcl}
1 & = & 1 \\
1 - 4 & = & -(1+2) \\
1 - 4 + 9 & = & 1+2+3 \\
1 - 4 + 9 - 16 & = & -(1+2+3+4) \\
1 - 4 + 9 - 16 + 25 & = & 1+2+3+4+5
\end{array}
\]
Guess a general formula and prove it by mathematical induction.

\begin{proof}
{\it General formula:} For every integer $n \geq 1$,
\[
\dps \sum_{i=1}^n (-1)^{i+1}i^2 = (-1)^{n+1}\sum_{j=1}^{n}j = (-1)^{n+1}\frac{n(n+1)}{2}.
\]
\underline{Proof by mathematical induction:}
For the given statement, the property $P(n)$ is the equation
\[
\dps \sum_{i=1}^n (-1)^{i+1}i^2 = (-1)^{n+1}\frac{n(n+1)}{2}. \,\,\, {\cy \from P(n)}
\]
{\bf Show that $P(1)$ is true:} The left-hand side of $P(1)$ is $\dps \sum_{i=1}^1 (-1)^{i+1}i^2 = (-1)^2 1^2 = 1$, and the right-hand side is $\dps (-1)^{1+1}\frac{1(1+1)}{2} = 1$ also. Thus $P(1)$ is true.

{\bf Show that for every integer $k \geq 1$, if $P(k)$ is true then $P(k + 1)$ is true:}

Let $k$ be any integer with $k \geq 1$, and suppose $P(k)$ is true. That is, suppose
\[
\dps \sum_{i=1}^k (-1)^{i+1}i^2 = (-1)^{k+1}\frac{k(k+1)}{2}. \,\,\, {\cy \from P(k) \text{: inductive hypothesis}}
\]
We must show that $P(k + 1)$ is true. That is, we must show that
\[
\dps \sum_{i=1}^{k+1} (-1)^{i+1}i^2 = (-1)^{k+2}\frac{(k+1)(k+2)}{2}. \,\,\,{\cy \from P(k+1)}
\]
Now the left-hand side of $P(k + 1)$ is $\dps \sum_{i=1}^{k+1} (-1)^{i+1}i^2$
\[
\begin{array}{lll}
= & \dps \sum_{i=1}^{k} (-1)^{i+1}i^2 + (-1)^{k+2}(k+1)^2 & \text{\cy make next-to-last term explicit} \\
= & \dps (-1)^{k+1}\frac{k(k+1)}{2} + (-1)^{k+2}(k+1)^2 & \text{\cy by inductive hypothesis} \\
= & \dps (-1)^{k+1}\frac{k(k+1)}{2} + (-1)^{k+1}(-1)(k+1)^2 & \text{\cy by factoring a power of $-1$} \\
= & \dps (-1)^{k+1}\left[\frac{k(k+1)}{2} - (k+1)^2\right] & \text{\cy by factoring out $(-1)^{k+1}$} \\
\end{array}
\]
\[
\begin{array}{lll}
\vspace{0.2cm}
= & \dps (-1)^{k+1}\left[\frac{k(k+1)}{2} -\frac{2(k+1)^2}{2}\right] & \text{\cy because $\frac{2}{2} = 1$} \\
\vspace{0.2cm}
= & \dps (-1)^{k+1}\frac{k(k+1)-2(k+1)^2}{2}  & \text{\cy by adding fractions} \\
\vspace{0.2cm}
= & \dps (-1)^{k+1}\frac{k^2 + k - 2(k^2 + 2k + 1)}{2} & \text{\cy by algebra} \\
\vspace{0.2cm}
= & \dps (-1)^{k+1}\frac{k^2 + k - 2k^2 - 4k -2}{2} & \text{\cy by algebra} \\
\vspace{0.2cm}
= & \dps (-1)^{k+1}\frac{-k^2 - 3k - 2}{2} & \text{\cy by algebra} \\
\vspace{0.2cm}
= & \dps (-1)^{k+2}\frac{k^2 + 3k + 2}{2} & \text{\cy by factoring out $(-1)$} \\
\vspace{0.2cm}
= & \dps (-1)^{k+2}\frac{(k+2)(k+1)}{2} & \text{\cy by factoring}
\end{array}
\]
and this is the right-hand side of $P(k + 1)$. Hence $P(k + 1)$ is true. {\it [Since both the basis step and the inductive step have been proved, $P(n)$ is true for every integer $n \geq 1$.]}
\end{proof}

\subsection{Exercise 33}
Find a formula in $n, a, m$, and $d$ for the sum $(a + md) + (a + (m + 1)d) + (a + (m + 2)d) + \cdots + (a + (m + n)d)$, where $m$ and $n$ are integers, $n \geq 0$, and $a$ and $d$ are real numbers. Justify your answer.

\begin{proof}
$(a + md) + (a + (m + 1)d) + (a + (m + 2)d) + \cdots + (a + (m + n)d)$
\[
\begin{array}{lll}
= & \dps \sum_{i = 1}^{n} (a+(m+i)d) & \text{\cy by summation notation} \\
= & \dps \sum_{i = 1}^{n} (a+md+id) & \text{\cy by multiplying} \\
= & \dps \sum_{i = 1}^{n} (a+md) + \sum_{i = 1}^{n} id & \text{\cy by splitting the sum} \\
= & \dps (a+md)\sum_{i = 1}^{n} 1 + d\sum_{i = 1}^{n} i & \text{\cy by moving out constants} \\
= & \dps (a+md)n + d \sum_{i = 1}^{n} i & \text{\cy because $\sum_{i=1}^{n}1 = n$} \\
= & \dps (a+md)n + d \cdot \frac{n(n+1)}{2} & \text{\cy because $\sum_{i=1}^{n} i = \frac{n(n+1)}{2}$}
\end{array}
\]
\end{proof}

\subsection{Exercise 34}
Find a formula in $a, r, m$, and $n$ for the sum $ar^m + ar^{m+1} + ar^{m+2} + \cdots + ar^{m+n}$ where $m$ and $n$ are integers, $n \geq 0$, and $a$ and $r$ are real numbers. Justify your answer.

\begin{proof}
$\dps ar^m + ar^{m+1} + ar^{m+2} + \cdots + ar^{m + n} = ar^m(1 + r + r^2 + \cdots + r^n) = ar^m \cdot \frac{r^{n + 1} - 1}{r - 1}$
\end{proof}

\subsection{Exercise 35}
You have two parents, four grandparents, eight great-grandparents, and so forth.

\subsubsection{(a)}
If all your ancestors were distinct, what would be the total number of your ancestors for the past 40 generations (counting your parents’ generation as number one)? (Hint: Use the formula for the sum of a geometric sequence.)

\begin{proof}
$\dps 2^1 + 2^2 + \cdots + 2^{40} = 2(1 + 2^1 + \cdots 2^{39}) = 2 \cdot \frac{2^{40} - 1}{2 - 1} = 2^{41} - 2$
\end{proof}

\subsubsection{(b)}
Assuming that each generation represents 25 years, how long is 40 generations?

\begin{proof}
$25 \cdot 40 = 1000$
\end{proof}

\subsubsection{(c)}
The total number of people who have ever lived is approximately 10 billion, which equals $10^{10}$ people. Compare this fact with the answer to part (a). What can you deduce?

\begin{proof}
$2^{41} - 2 = 2,199,023,255,550 > 10,000,000,000 = 10^{10}$ so many of my ancestors are not distinct.
\end{proof}

{\bf \cy Find the mistakes in the proof fragments in $36-38$.}

\subsection{Exercise 36}
{\bf Theorem:} For any integer $\dps n \geq 1, \,\,\, 1^2 + 2^2 + \cdots + n^2 = \frac{n(n+1)(2n+1)}{6}.$

{\bf ``Proof (by mathematical induction):} Certainly the theorem is true for $n = 1$ because $1^2 = 1$ and $\dps \frac{1(1+1)(2+1)}{6} = 1$. So the basis step is true. For the inductive step, suppose that $k$ is any integer with $\dps k \geq 1, k^2 = \frac{k(k+1)(2k+1)}{6}$. We must show that $\dps (k+1)^2 = \frac{(k+1)((k+1)+1)(2(k+1)+1)}{6}$.''

\begin{proof}
In the inductive step, both the inductive hypothesis and what is to be shown are wrong. The inductive hypothesis should be: 

Suppose that for some integer $k \geq 1$, 

\[
\dps 1^2 + 2^2 + \cdots + k^2 = \frac{k(k+1)(2k+1)}{6}.
\]

And what is to be shown should be:
\[
\dps 1^2 + 2^2 + \cdots + (k + 1)^2 = \frac{(k + 1)((k + 1) + 1)(2(k+1)+1)}{6}.
\]
\end{proof}

\subsection{Exercise 37}
{\bf Theorem:} For any integer $n \geq 0$,
\[
1 + 2 + 2^2 + \cdots + 2^n = 2^{n+1} - 1.
\]
{\bf ``Proof (by mathematical induction):} Let the property $P(n)$ be
\[
1 + 2 + 2^2 + \cdots + 2^n = 2^{n+1} - 1.
\]
{\bf Show that $P(0)$ is true:} \\
The left-hand side of $P(0)$ is $1 + 2 + 2^2 + \cdots + 2^0 = 1$ and the right-hand side is $2^{0+1} - 1 = 2 - 1 = 1$ also. So $P(0)$ is true.''

{\it Hint:} See the Caution note in Section 5.1, page 262.

\begin{proof}
The left-hand side of $P(0)$ is wrong; it should be simply $1$ instead of $1 + 2 + 2^2 + \cdots + 2^0$.
\end{proof}

\subsection{Exercise 38}
{\bf Theorem:} For any integer $n \geq 1$,
\[
\sum_{i=1}^{n}i(i!) = (n+1)! - 1.
\]
{\bf ``Proof (by mathematical induction):} Let the property $P(n)$ be
\[
1 + 2 + 2^2 + \cdots + 2^n = 2^{n+1} - 1.
\]
{\bf Show that $P(1)$ is true:} When $n = 1$,
\[
\sum_{i=1}^{1}i(i!) = (1+1)! - 1.
\]
So $1(1!) = 2! - 1$, and $1 = 1$. Thus $P(1)$ is true.

{\it Hint:} See the subsection Proving an Equality on page 284 in Section 5.2.

\begin{proof}
For $P(1)$ this proof is assuming what is to be shown. The equality \\ $\dps \sum_{i=1}^{1}i(i!) = (1+1)! - 1$ is $P(1)$ which is what we need to show, but the proof assumes that this is true, then does operations on both sides to reach a true conclusion $1 = 1$. This proves that if $P(1)$ is true, then $1 = 1$ is true, but that does not prove $P(1)$ is true.
\end{proof}

\subsection{Exercise 39}
Use Theorem 5.2.1 to prove that if $m$ and $n$ are any positive integers and $m$ is odd, then $\dps \sum_{k=0}^{m-1}(n+k)$ is divisible by $m$. Does the conclusion hold if $m$ is even? Justify your answer.

\begin{proof}
$\dps \sum_{k=0}^{m-1}(n+k) = \sum_{k=0}^{m-1}n + \sum_{k=0}^{m-1}k = n\sum_{k=0}^{m-1}1 + (0 + \sum_{k=1}^{m-1}k) = nm + \frac{(m-1)(m-1+1)}{2}$

$\dps = nm + \frac{(m-1)m}{2} = m(n + (m-1)/2)$ which is divisible by $m$ if and only if $n + (m-1)/2$ is an integer, if and only if $m-1$ is even, if and only if $m$ is odd. So the statement is true when $m$ is odd and false when $m$ is even.
\end{proof}

\subsection{Exercise 40}
Use Theorem 5.2.1 and the result of Exercise 10 to prove that if $p$ is any prime number with $p \geq 5$, then the sum of the squares of any $p$ consecutive integers is divisible by $p$.

\begin{proof}
Assume $p$ is any prime number with $p \geq 5$. Assume $n$ is any integer and consider the $p$ consecutive integers $n, n+1, \ldots, n+p-1$. {\it [We want to show $n^2 + (n+1)^2 + \cdots + (n+p-1)^2$ is divisible by $p$.]} Then

$n^2 + (n+1)^2 + (n+2)^2 + \cdots + (n+p-1)^2$
\[
\begin{array}{lll}
= & n^2 + (n^2 + 2n + 1) + (n^2 + 2n \cdot 2 + 2^2) + \cdots + (n^2 + 2n(p-1) + (p-1)^2) \\
= & pn^2 + 2n(1 + 2 + \cdots + (p-1)) + (1^2 + 2^2 + \cdots + (p-1)^2) \\
= & \dps pn^2 + 2n \cdot \frac{(p-1)(p-1+1)}{2} + \frac{(p-1)(p-1+1)(2(p-1)+1)}{6} \\
= & \dps pn^2 + n(p-1)p + \frac{(p-1)p(2p-1)}{6}
\end{array}
\]
We know that the sum of squares formula gives us an integer, so $\dps \frac{(p-1)p(2p-1)}{6}$ is an integer. Since $p \geq 5$ is a prime, it is not divisible by 6, therefore $\dps\frac{(p-1)(2p-1)}{6}$ is an integer. So
\[
\begin{array}{lll}
= & \dps pn^2 + n(p-1)p + p\frac{(p-1)(2p-1)}{6} \\
= & \dps p\left[n^2 + n(p-1) + \frac{(p-1)(2p-1)}{6}\right] 
\end{array}
\]
is divisible by $p$.
\end{proof}

\section{Exercise Set 5.3}

\subsection{Exercise 1}
Use mathematical induction (and the proof of proposition 5.3.1 as a model) to show that any amount of money of at least 14¢ can be made up using 3¢ and 8¢ coins.

\begin{proof}

\end{proof}

\subsection{Exercise 2}
Use mathematical induction to show that any postage of at least 12¢ can be obtained using 3¢ and 7¢ stamps.

\begin{proof}

\end{proof}

\subsection{Exercise 3}
Stamps are sold in packages containing either 5 stamps or 8 stamps.

\subsubsection{(a)}
Show that a person can obtain 5, 8, 10, 13, 15, 16, 20, 21, 24, or 25 stamps by buying a collection of 5-stamp packages and 8-stamp packages.

\begin{proof}

\end{proof}

\subsubsection{(b)}
Use mathematical induction to show that any quantity of at least 28 stamps can be obtained by buying a collection of 5-stamp packages and 8-stamp packages.

\begin{proof}

\end{proof}

\subsection{Exercise 4}
For each positive integer $n$, let $P(n)$ be the sentence
that describes the following divisibility property: $5^n - 1$ is divisible by 4.

\subsubsection{(a)}
Write $P(0)$. Is $P(0)$ true?

\begin{proof}

\end{proof}

\subsubsection{(b)}
Write $P(k)$.

\begin{proof}

\end{proof}

\subsubsection{(c)}
Write $P(k + 1)$.

\begin{proof}

\end{proof}

\subsubsection{(d)}
In a proof by mathematical induction that this divisibility property holds for every integer $n \geq 0$, what must be shown in the inductive step?

\begin{proof}

\end{proof}

\subsection{Exercise 5}
For each positive integer $n$, let $P(n)$ be the inequality
$2^n < (n + 1)!$.

\subsubsection{(a)}
Write $P(2)$. Is $P(2)$ true?

\begin{proof}

\end{proof}

\subsubsection{(b)}
Write $P(k)$.

\begin{proof}

\end{proof}

\subsubsection{(c)}
Write $P(k + 1)$.

\begin{proof}

\end{proof}

\subsubsection{(d)}
In a proof by mathematical induction that this inequality holds for every integer $n \geq 2$, what must be shown in the inductive step?

\begin{proof}

\end{proof}

\subsection{Exercise 6}
For each positive integer $n$, let $P(n)$ be the sentence:

Any checkerboard with dimensions $2 \times 3n$ can be completely covered with L-shaped trominoes.

\subsubsection{(a)}
Write $P(1)$. Is $P(1)$ true?

\begin{proof}

\end{proof}

\subsubsection{(b)}
Write $P(k)$.

\begin{proof}

\end{proof}

\subsubsection{(c)}
Write $P(k + 1)$.

\begin{proof}

\end{proof}

\subsubsection{(d)}
In a proof by mathematical induction that $P(n)$ is true for each integer $n \geq 1$, what must be shown in the inductive step?

\begin{proof}

\end{proof}

\subsection{Exercise 7}
For each positive integer $n$, let $P(n)$ be the sentence:

In any round-robin tournament involving $n$ teams, the teams can be labeled $T_1, T_2, T_3, \ldots$, $T_n$, so that $T_i$ beats $T_{i + 1}$ for every $i = 1, 2, \ldots, n$.

\subsubsection{(a)}
Write $P(2)$. Is $P(2)$ true?

\begin{proof}

\end{proof}

\subsubsection{(b)}
Write $P(k)$.

\begin{proof}

\end{proof}

\subsubsection{(c)}
Write $P(k + 1)$.

\begin{proof}

\end{proof}

\subsubsection{(d)}
In a proof by mathematical induction that $P(n)$ is true for each integer $n \geq 2$, what must be shown in the inductive step?

\begin{proof}

\end{proof}

{\bf \cy Prove each statement in $8-23$ by mathematical induction.}

\subsection{Exercise 8}
$5^n - 1$ is divisible by 4, for every integer $n \geq 0$.

\begin{proof}

\end{proof}

\subsection{Exercise 9}
$7^n - 1$ is divisible by 6, for each integer $n \geq 0$.

\begin{proof}

\end{proof}

\subsection{Exercise 10}
$n^3 - 7n + 3$ is divisible by 3, for each integer $n \geq 0$.

\begin{proof}

\end{proof}

\subsection{Exercise 11}
$3^{2n} - 1$ is divisible by 8, for each integer $n \geq 0$.

\begin{proof}

\end{proof}

\subsection{Exercise 12}
For any integer $n \geq 0$, $7^n - 2^n$ is divisible by 5.

\begin{proof}

\end{proof}

\subsection{Exercise 13}
For any integer $n \geq 0, x^n - y^n$ is divisible by $x - y$, where $x$ and $y$ are any integers with $x \neq y$.

\begin{proof}

\end{proof}

\subsection{Exercise 14}
$n^3 - n$ is divisible by 6, for each integer $n \geq 0$.

\begin{proof}

\end{proof}

\subsection{Exercise 15}
$n(n^2 + 5)$ is divisible by 6, for each integer $n \geq 0$.

\begin{proof}

\end{proof}

\subsection{Exercise 16}
$2^n < (n + 1)!$, for every integer $n \geq 2$.

\begin{proof}

\end{proof}

\subsection{Exercise 17}
$1 + 3n \leq 4^n$, for every integer $n \geq 0$.

\begin{proof}

\end{proof}

\subsection{Exercise 18}
$5^n + 9 < 6^n$, for each integer $n \geq 2$.

\begin{proof}

\end{proof}

\subsection{Exercise 19}
$n^2 < 2^n$, for every integer $n \geq 5$.

\begin{proof}

\end{proof}

\subsection{Exercise 20}
$2^n < (n + 2)!$, for each integer $n \geq 0$.

\begin{proof}

\end{proof}

\subsection{Exercise 21}
$\dps \sqrt{n} < \frac{1}{\sqrt{1}} + \frac{1}{\sqrt{2}} + \cdots + \frac{1}{\sqrt{n}}$, for every integer $n \geq 2$.

\begin{proof}

\end{proof}

\subsection{Exercise 22}
$1 + nx \leq (1 + x)^n$, for every real number $x > -1$
and every integer $n \geq 2$.

\begin{proof}

\end{proof}

\subsection{Exercise 23}

\subsubsection{(a)}
$n^3 > 2n + 1$, for each integer $n \geq 2$.

\begin{proof}

\end{proof}

\subsubsection{(b)}
$n! > n^2$, for each integer $n \geq 4$.

\begin{proof}

\end{proof}

\subsection{Exercise 24}
A sequence $a_1, a_2, a_3, \ldots$ is defined by letting $a_1 = 3$ and $a_k = 7a_{k-1}$ for each integer $k \geq 2$. Show that $a_n = 3\cdot 7^n - 1$ for every integer $n \geq 1$.

\begin{proof}

\end{proof}

\subsection{Exercise 25}
A sequence $b_0, b_1, b_2, \ldots$ is defined by letting
$b_0 = 5$ and $b_k = 4 + b_{k-1}$ for each integer $k \geq 1$. Show that $b_n > 4n$ for every integer $n \geq 0$.

\begin{proof}

\end{proof}

\subsection{Exercise 26}
A sequence $c_0, c_1, c_2, \ldots$ is defined by letting
$c_0 = 3$ and $c_k = (c_{k-1})^2$ for every integer $k \geq 1$. Show that $c_n = 3^{2n}$ for each integer $n \geq 0$.

\begin{proof}

\end{proof}

\subsection{Exercise 27}
A sequence $d_1, d_2, d_3, \ldots$ is defined by letting $d1 = 2$ and $d_k = \frac{d_{k-1}}{k}$ for each integer $k \geq 2$. Show that for every integer $n \geq 1, d_n = \frac{2}{n!}$. 

\begin{proof}

\end{proof}

\subsection{Exercise 28}
Prove that for every integer $n \geq 1$,
\[
\frac{1}{3} = \frac{1 + 3 + 5 + \cdots + (2n-1)}{(2n+1) + (2n+3) + \cdots + (2n + (2n-1))}
\]
\begin{proof}

\end{proof}

{\bf \cy Exercises 29 and 30 use the definition of string and string length from page 13 in Section 1.4. Recursive definitions for these terms are given in Section 5.9.}

\subsection{Exercise 29}
A set $L$ consists of strings obtained by juxtaposing one or more of $abb, bab$, and $bba$. Use mathematical induction to prove that for every integer $n \geq 1$, if a string $s$ in $L$ has length $3n$, then $s$ contains an even number of $b$’s.

\begin{proof}

\end{proof}

\subsection{Exercise 30}
A set $S$ consists of strings obtained by juxtaposing one or more copies of 1110 and 0111. Use mathematical induction to prove that for every integer $n \geq 1$, if a string $s$ in $S$ has length $4n$, then the number of 1’s in $s$ is a multiple of 3.

\begin{proof}

\end{proof}

\subsection{Exercise 31}
Use mathematical induction to give an alternative proof for the statement proved in Example 4.9.9: For any positive integer $n$, a complete graph on $n$ vertices has $\frac{n(n - 1)}{2}$ edges. {\it Hint:} Let $P(n)$ be the sentence, ``the number of edges in a complete graph on $n$ vertices is $\frac{n(n - 1)}{2}$.''

\begin{proof}

\end{proof}

\subsection{Exercise 32}
Some $5 \times 5$ checkerboards with one square removed can be completely covered by L-shaped trominoes, whereas other $5 \times 5$ checkerboards cannot. Find examples of both kinds of checkerboards. Justify your answers.

\begin{proof}

\end{proof}

\subsection{Exercise 33}
Consider a $4 \times 6$ checkerboard. Draw a covering of the board by L-shaped trominoes.

\begin{proof}

\end{proof}

\subsection{Exercise 34}
\subsubsection{(a)}
Use mathematical induction to prove that for each integer $n \geq 1$, any checkerboard with dimensions $2 \times 3n$ can be completely covered with L-shaped trominoes.

\begin{proof}

\end{proof}

\subsubsection{(b)}
Let $n$ be any integer greater than or equal to 1. Use the result of part (a) to prove by mathematical induction that for every integer $m$, any checkerboard with dimensions $2m \times 3n$ can be completely covered with L-shaped trominoes.

\begin{proof}

\end{proof}

\subsection{Exercise 35}
Let $m$ and $n$ be any integers that are greater than or equal to 1.

\subsubsection{(a)}
Prove that a necessary condition for an $m \times n$ checkerboard to be completely coverable by L-shaped trominoes is that $mn$ be divisible by 3.

\begin{proof}

\end{proof}

\subsubsection{(b)}
Prove that having $mn$ be divisible by 3 is not a sufficient condition for an $m \times n$ checkerboard to be completely coverable by L-shaped trominoes.

\begin{proof}

\end{proof}

\subsection{Exercise 36}
In a round-robin tournament each team plays every other team exactly once with ties not allowed. If the teams are labeled $T_1, T_2, \ldots, T_n$, then the outcome of such a tournament can be represented by a directed graph, in which the teams are represented as dots and an arrow is drawn from one dot to another if, and only if, the following team represented by the first dot beats the team represented by the second dot. For example, the following directed graph shows one outcome of a round-robin tournament involving five teams, A, B, C, D, and E.

\begin{figure}[ht!]
\centering
\includegraphics[scale=0.5]{../images/5.3.36.png}
\end{figure}

Use mathematical induction to show that in any round-robin tournament involving $n$ teams, where $n \geq 2$, it is possible to label the teams $T_1, T_2, \ldots, T_n$ so that $T_i$ beats $T_i + 1$ for all $i = 1, 2, \ldots, n - 1$. (For instance, one such labeling in the example above is $T_1 = A, T_2 = B, T_3 = C, T_4 = E, T_5 = D$.) (Hint: Given $k+1$ teams, pick one, say $T_9$, and apply the inductive hypothesis to the remaining teams to obtain an ordering $T_1, T_2, \ldots, T_k$. Consider three cases: $T_9$ beats $T_1$, $T_9$ loses to the first $m$ teams (where $1 \leq m \leq k - 1$) and beats the $(m + 1)$st team, and $T_9$ loses to all the other teams.)

\begin{proof}

\end{proof}

\subsection{Exercise 37}
On the outside rim of a circular disk the integers from 1 through 30 are painted in random order. Show that no matter what this order is, there must be three successive integers whose sum is at least 45.

\begin{proof}

\end{proof}

\subsection{Exercise 38}
Suppose that $n$ $a$’s and $n$ $b$’s are distributed around
the outside of a circle. Use mathematical induction to prove that for any integer $n \geq 1$, given any such arrangement, it is possible to find a starting point so that if you travel around the circle in a clockwise direction, the number of $a$’s you pass is never less than the number of $b$’s you have passed. For example, in the diagram shown below, you could start at the $a$ with an asterisk.

\begin{figure}[ht!]
\centering
\includegraphics[scale=0.5]{../images/5.3.38.png}
\end{figure}

\begin{proof}

\end{proof}

\subsection{Exercise 39}

\begin{proof}

\end{proof}

\subsection{Exercise 40}
\subsubsection{(a)}
Prove that in an $8 \times 8$ checkerboard with alternating black and white squares, if the squares in the top right and bottom left corners are removed the remaining board cannot be covered with dominoes. ({\it Hint:} Mathematical induction is not needed for this proof.)

\begin{proof}

\end{proof}

\subsubsection{(b)}
Use mathematical induction to prove that for each positive integer $n$, if a $2n \times 2n$ checkerboard with alternating black and white squares has one white square and one black square removed anywhere on the board, the remaining squares can be covered with dominoes.

\begin{proof}

\end{proof}

\subsection{Exercise 41}
A group of people are positioned so that the distance between any two people is different from the distance between any other two people. Suppose that the group contains an odd number of people and each person sends a message to their nearest neighbor. Use mathematical induction to prove that at least one person does not receive a message from anyone. [This exercise is inspired by the article “Odd Pie Fights” by L. Carmony, The Mathematics Teacher, 72(1), 1979, 61–64.]

\begin{proof}

\end{proof}

\subsection{Exercise 42}
Show that for any even integer $n$, it is possible to find a group of $n$ people who are all positioned so that the distance between any two people is different from the distance between any other two people, so that each person sends a message to their nearest neighbor, and so that every person in the group receives a message from another person in the group.

\begin{proof}

\end{proof}

\subsection{Exercise 43}
Define a game as follows: You begin with an urn that contains a mixture of white and black balls, and during the game you have access to as many additional white and black balls as you might need. In each move you remove two balls from the urn without looking at their colors. If the balls
are the same color, you put in one black ball. If the balls are different colors, you put the white ball back into the urn and keep the black ball out. Because each move reduces the number of balls in the urn by one, the game will end with a single ball in the urn. If you know how many white balls and how many black balls are initially in the urn, can you predict the color of the ball at the end of the game? [This exercise is based on one described in “Why correctness must be a mathematical concern” by E. W. Dijkstra, www.cs.utexas.edu/users /EWD/transcriptions/EWD07xx/EWD720.html.]

\subsubsection{(a)}
Map out all possibilities for playing the game starting with two balls in the urn, then three balls, and then four balls. For each case keep track of the number of white and black balls you start with and the color of the ball at the
end of the game.

\begin{proof}

\end{proof}

\subsubsection{(b)}
Does the number of white balls seem to be predictive? Does the number of black balls seem to be predictive? Make a conjecture about the color of the ball at the end of the game given the numbers of white and black balls at the beginning.

\begin{proof}

\end{proof}

\subsubsection{(c)}
Use mathematical induction to prove the conjecture you made in part (b).

\begin{proof}

\end{proof}

\subsection{Exercise 44}
Let $P(n)$ be the following sentence: Given any graph $G$ with $n$ vertices satisfying the condition that every vertex of $G$ has degree at most $M$, then the vertices of $G$ can be colored with at most $M + 1$ colors in such a way that no two adjacent vertices have the same color. Use mathematical induction to prove this statement is true for every integer $n \geq 1$.

\begin{proof}

\end{proof}

{\bf \cy In order for a proof by mathematical induction to be valid, the basis statement must be true for $n = a$ and the argument of the inductive step must be correct for every integer $k \geq a$. In 45 and 46 find the mistakes in the “proofs” by mathematical induction.}

\subsection{Exercise 45}
{\bf “Theorem:”} For any integer $n \geq 1$, all the numbers in a set of n numbers are equal to each other.

{\bf “Proof (by mathematical induction):} It is obviously true that all the numbers in a set consisting of just one number are equal to each other, so the basis step is true. For the inductive step, let $A = \{a_1, a_2, \ldots, a_k, a_{k+1}\}$ be any set of $k + 1$ numbers. Form two subsets each of size $k$:
\[
B = \{a_1, a_2, a_3, \ldots, a_k\} \text{ and } C = \{a_1, a_3, a_4, \ldots, a_{k+1}\}.
\]
($B$ consists of all the numbers in $A$ except $a_{k+1}$, and $C$ consists of all the numbers in $A$ except $a_2$.) By inductive hypothesis, all the numbers in $B$ equal $a_1$ and all the numbers in $C$ equal $a_1$ (since both sets have only $k$ numbers). But every number in $A$ is in $B$ or $C$, so all the numbers in $A$ equal $a_1$; hence all are equal to each other.”

\begin{proof}

\end{proof}

\subsection{Exercise 46}
{\bf “Theorem:”} For every integer $n \geq 1$, $3^n - 2$ is even. 

{\bf “Proof (by mathematical induction):} Suppose the theorem is true for an integer $k$, where $k \geq 1$. That is, suppose that $3^k - 2$ is even. We must show that $3^{k+1} - 2$ is even. Observe that $3^{k+1} - 2 = 3^k \cdot 3 - 2 = 3^k(1 + 2) - 2 = (3^k - 2) + 3^k \cdot 2$. Now $3^k - 2$ is even by inductive hypothesis and $3^k \cdot 2$ is even by inspection. Hence the sum of the two quantities is even (by Theorem 4.1.1). It follows that $3^{k+1} - 2$ is even, which is what we needed to show.”

\begin{proof}

\end{proof}

\end{document}
